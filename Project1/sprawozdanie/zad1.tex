\chapter{Punkt pracy}
Wartości punktu pracy opisane w sekcji \ref{sec:opis} zostały
zweryfikowane. Weryfikacja polegała na prostym sprawdzeniu na jakiej
wartości wyjścia stabilizuje się obiekt przy zadanym sterowaniu.
Eksperyment potwierdził wcześniej opisane wartości, a jego przebieg
obrazuje wykres \ref{fig:pkt_prac}

\begin{figure}[tb]
\centering
\begin{tikzpicture}
\begin{axis}[
width=0.75\textwidth,
xmin=0,xmax=300,ymin=0,ymax=3,
xlabel={Chwila (k)},
ylabel={Wartość wyjścia/sterowania},
xtick={0, 50, 100, 150, 200, 250, 300},
ytick={0, 0.5, 1, 1.5, 2, 2.5, 3},
legend pos=north east,
/pgf/number format/.cd,
use comma,
1000 sep={}
]
\addplot[blue,semithick] file {wykresy/zad1Y.txt};
\addplot[red,semithick] file {wykresy/zad1U.txt};
\legend{Wyjście, Sterowanie}
\end{axis}
\end{tikzpicture}
\caption{Zachowanie obiektu dla stałej wartości sterowania $U = 0.9$.}
\label{fig:pkt_prac}
\end{figure}
