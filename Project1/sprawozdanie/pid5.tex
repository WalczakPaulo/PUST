\chapter{Dobór nastaw}
Eksperymentalne testy w poszukiwaniu lepszych nastaw regulatora PID rozpoczynamy od tych wyznaczonych za pomocą eksperymentu ZN. Za wzmocnienie przyjmujemy $K = 0,92$, będące połową wartości wzmocnienia, dla którego występują niegasnące oscylacje.
Najpierw zaczniemy regulować nastawę $T_i$. Po kilku testach zauważamy spadek błędu, dla wartości $T_i = 20$. Poprawę oceniamy za pomocą wskaźnika numerycznego wartości błędu,
gdyż ciężko jest ujrzeć zmianę obserwując jedynie przebiegi. Następnie sprawdzamy wartość $T_d$. Po testach uważamy za dobrą wartość $T_d = 4$. Działanie takiego regulatora
obrazują wykresy \ref{fig:inz_pid_out} i \ref{fig:inz_pid_ster}. Wartość wskaźnika błędu dla powyższych nastaw wynosi $E = 8,65$.

\begin{figure}[tb]
\centering
\begin{tikzpicture}
\begin{axis}[
width=0.75\textwidth,
height=0.375\textwidth,
xmin=0,xmax=1000,ymin=2.5,ymax=3.5,
xlabel={Chwila (k)},
ylabel={Wyjście (y)},
xtick={0, 200, 400, 600, 800, 1000},
ytick={2.5, 3, 3.5},
legend pos=south east,
/pgf/number format/.cd,
use comma,
1000 sep={}
]
\addplot[blue,semithick] file {wykresy/pid_inz_y.txt};
\addplot[red,semithick] file {wykresy/pid_inz_yzad.txt};

\legend{Wyjście, Wartość zadana}
\end{axis}
\end{tikzpicture}
\caption{Wyjście obiektu z regulatorem PID z nastawami wyznaczonymi metodą inżynierską.}
\label{fig:inz_pid_out}
\end{figure}

\begin{figure}[tb]
\centering
\begin{tikzpicture}
\begin{axis}[
width=0.75\textwidth,
height=0.375\textwidth,
xmin=0,xmax=1000,ymin=0.6,ymax=1.2,
xlabel={Chwila (k)},
ylabel={Sterowanie (u)},
xtick={0, 200, 400, 600, 800, 1000},
ytick={0.6, 0.8, 1, 1.2},
legend pos=south east,
/pgf/number format/.cd,
use comma,
1000 sep={}
]
\addplot[blue,semithick] file {wykresy/pid_inz_u.txt};

% \legend{Wyjście, Wartość zadana}
\end{axis}
\end{tikzpicture}
\caption{Sterowanie obiektu z regulatorem PID z nastawami wyznaczonymi metodą inżynierską.}
\label{fig:inz_pid_ster}
\end{figure}
