\documentclass[a4paper,titlepage,11pt,twosides,floatssmall]{mwrep}
\usepackage[left=2.5cm,right=2.5cm,top=2.5cm,bottom=2.5cm]{geometry}
\usepackage[OT1]{fontenc}
\usepackage{polski}
\usepackage{amsmath}
\usepackage{amsfonts}
\usepackage{amssymb}
\usepackage{graphicx}
\usepackage{url}
\usepackage{tikz}
\usetikzlibrary{arrows,calc,decorations.markings,math,arrows.meta}
\usepackage{rotating}
\usepackage[percent]{overpic}
\usepackage[utf8]{inputenc}
\usepackage{xcolor}
\usepackage{pgfplots}
\usetikzlibrary{pgfplots.groupplots}
\usepackage{listings}
\usepackage{matlab-prettifier}
\usepackage{siunitx}
\definecolor{szary}{rgb}{0.95,0.95,0.95}
\sisetup{detect-weight,exponent-product=\cdot,output-decimal-marker={,},per-mode=symbol,binary-units=true,range-phrase={-},range-units=single}

%konfiguracje pakietu listings
\lstset{
	backgroundcolor=\color{szary},
	frame=single,
	breaklines=true,
}
\lstdefinestyle{customlatex}{
	basicstyle=\footnotesize\ttfamily,
	%basicstyle=\small\ttfamily,
}
\lstdefinestyle{customc}{
	breaklines=true,
	frame=tb,
	language=C,
	xleftmargin=0pt,
	showstringspaces=false,
	basicstyle=\small\ttfamily,
	keywordstyle=\bfseries\color{green!40!black},
	commentstyle=\itshape\color{purple!40!black},
	identifierstyle=\color{blue},
	stringstyle=\color{orange},
}
\lstdefinestyle{custommatlab}{
	captionpos=t,
	breaklines=true,
	frame=tb,
	xleftmargin=0pt,
	language=matlab,
	showstringspaces=false,
	%basicstyle=\footnotesize\ttfamily,
	basicstyle=\scriptsize\ttfamily,
	keywordstyle=\bfseries\color{green!40!black},
	commentstyle=\itshape\color{purple!40!black},
	identifierstyle=\color{blue},
	stringstyle=\color{orange},
}

%wymiar tekstu (bez �ywej paginy)
\textwidth 160mm \textheight 247mm

%ustawienia pakietu pgfplots
\pgfplotsset{
tick label style={font=\scriptsize},
label style={font=\small},
legend style={font=\small},
title style={font=\small}
}

\def\figurename{Rys.}
\def\tablename{Tab.}

%konfiguracja liczby p�ywaj�cych element�w
\setcounter{topnumber}{0}%2
\setcounter{bottomnumber}{3}%1
\setcounter{totalnumber}{5}%3
\renewcommand{\textfraction}{0.01}%0.2
\renewcommand{\topfraction}{0.95}%0.7
\renewcommand{\bottomfraction}{0.95}%0.3
\renewcommand{\floatpagefraction}{0.35}%0.5

\begin{document}
\frenchspacing
\pagestyle{uheadings}

%strona tytu�owa
\title{\bf Sprawozdanie z projektu i ćwiczenia laboratoryjnego nr 1, zadanie nr 1\vskip 0.1cm}
\author{Kamil Gabryjelski, Paweł Rybak, Paweł Walczak}
\date{2017}

\makeatletter
\renewcommand{\maketitle}{\begin{titlepage}
\begin{center}{\LARGE {\bf
Wydział Elektroniki i Technik Informacyjnych}}\\
\vspace{0.4cm}
{\LARGE {\bf Politechnika Warszawska}}\\
\vspace{0.3cm}
\end{center}
\vspace{5cm}
\begin{center}
{\bf \LARGE Projektowanie układów sterowania\\ (projekt grupowy) \vskip 0.1cm}
\end{center}
\vspace{1cm}
\begin{center}
{\bf \LARGE \@title}
\end{center}
\vspace{2cm}
\begin{center}
{\bf \Large \@author \par}
\end{center}
\vspace*{\stretch{6}}
\begin{center}
\bf{\large{Warszawa, \@date\vskip 0.1cm}}
\end{center}
\end{titlepage}
}
\makeatother

\maketitle

\tableofcontents
\chapter{Opis obiektu}
\label{sec:opis}
Obiekt używany w projekcie jest symulacją obiektu, napisaną w języku MATLAB.
Opisywany jest on wzorem
\begin{equation}
  Y(k) = f(U(k - 7), U(k - 8), Z(k - 3), Z(k - 4), Y(k - 1), Y(k - 2))
\end{equation}
gdzie $k$ jest aktualną chwilą symulacji.
Wartości sygnałów w punkcie pracy mają wartość $u=y=z=0$. Okres próbkowania wynosi $T_p=0,5s$.
\chapter{Opis obiektu}
Badany obiekt jest obiektem o dwóch sygnałach wejściowych ($u_1$, $u_2$), oraz
dwóch sygnałach wyjściowych ($y_1$, $y_2$). Obiekt jest obiektem dyskretnym,
a jego okres próbkowania wynosi $0,5$s. Punktem pracy naszego obiektu, będzie
zerowa wartość obydwu wejść. W takim przypadku obiekt stabilizuje się przy
zerowej wartości wyjść.

\chapter{Odpowiedzi skokowe}
\section{Tor sterowania}
W tym punkcie badane są odpowiedzi skokowe obiektu na różne wartości skoku sygnału sterowania. Założono, że w chwili początkowej obiekt znajduje się w punkcie pracy. W chwili $k=9$ wykonywany jest sterowania do zadanej wartości. Sygnał zakłócenia ma wartość $Z=0$. Wyniki badań przedstawia wykres \ref{fig:z2_yu}.
\begin{figure}[!htb]
	\centering
	\begin{tikzpicture}
	\begin{axis}[
	width=0.9\textwidth,
	width=0.9\textwidth,
	xmin=0,xmax=200,ymin=-3,ymax=3,
	xlabel={$k$},
	ylabel={$Y(k)$},
	xtick={0,50,100,150,200},
	ytick={-3,-2,-1,0,1,2,3},
	/pgf/number format/.cd,
	use comma,
	1000 sep={}
	]
	\addplot[blue,semithick] file {wykresy/zad2_yu_1.txt};
	\addplot[brown,semithick] file {wykresy/zad2_yu_2.txt};
	\addplot[cyan,semithick] file {wykresy/zad2_yu_3.txt};
	\addplot[green,semithick] file {wykresy/zad2_yu_4.txt};
	\addplot[lime,semithick] file {wykresy/zad2_yu_5.txt};
	\addplot[magenta,semithick] file {wykresy/zad2_yu_6.txt};
	\addplot[orange,semithick] file {wykresy/zad2_yu_7.txt};
	\addplot[pink,semithick] file {wykresy/zad2_yu_8.txt};
	\legend{$U=-\num{2}$,$U=-\num{1,5}$,$U=-\num{1}$,$U=-\num{0,5}$,$U=\num{0,5}$,$U=\num{1}$,$U=\num{1,5}$,$U=\num{2}$}
	\end{axis}
	\end{tikzpicture}
	\caption{Odpowiedź $Y(k)$ dla skoków sterowania $U$}
\label{fig:z2_yu}
\end{figure}

\section{Charakterystyka statyczna toru sterowania}
Charakterystyka statyczna toru sterowania wyznaczona została poprzez sprawdzenie, na jakich wartościach stabilizuje się wyjście obiektu dla różnych wartości sygnału $U$. Liniowości charakterystki statycznej dowodzi wykres \ref{fig:z2_y_stat_u}, który (w przybliżeniu) jest liniowy.

Wartość wzmocnienia statycznego można wyznaczyć normalizując odpowiedź skokową. Wynosi ona $s_u=1,1022$.

\begin{figure}[!htb]
	\centering
	\begin{tikzpicture}
	\begin{axis}[
	width=0.9\textwidth,
	width=0.9\textwidth,
	xmin=-1.5,xmax=1.5,ymin=-2,ymax=2,
	xlabel={$U$},
	ylabel={$Y(U)$},
	xtick={-1.5,-1,-0.5,0,0.5,1,1.5},
	ytick={-2,-1.5,-1,-0.5,0,0.5,1,1.5,2},
	/pgf/number format/.cd,
	use comma,
	1000 sep={}
	]
	\addplot[blue,semithick] file {wykresy/zad2_y_stat_u.txt};
	\end{axis}
	\end{tikzpicture}
	\caption{Charakterystyka statyczna $Y(U)$}
\label{fig:z2_y_stat_u}
\end{figure}

\section{Tor zakłócenia}
W tym punkcie badane są odpowiedzi skokowe obiektu na różne wartości skoku sygnału zakłócenia. Założono, że w chwili początkowej obiekt znajduje się w punkcie pracy. W chwili $k=9$ wykonywany jest sterowania do zadanej wartości. Sygnał sterowania ma wartość $U=0$. Wyniki badań przedstawia wykres \ref{fig:z2_yz}.
\begin{figure}[!htb]
	\centering
	\begin{tikzpicture}
	\begin{axis}[
	width=0.9\textwidth,
	width=0.9\textwidth,
	xmin=0,xmax=200,ymin=-1.5,ymax=1.5,
	xlabel={$k$},
	ylabel={$Y(k)$},
	xtick={0,50,100,150,200},
	ytick={-1.5,-1,-0.5,0,0.5,1,1.5},
	/pgf/number format/.cd,
	use comma,
	1000 sep={}
	]
	\addplot[blue,semithick] file {wykresy/zad2_yz_1.txt};
	\addplot[brown,semithick] file {wykresy/zad2_yz_2.txt};
	\addplot[cyan,semithick] file {wykresy/zad2_yz_3.txt};
	\addplot[green,semithick] file {wykresy/zad2_yz_4.txt};
	\addplot[lime,semithick] file {wykresy/zad2_yz_5.txt};
	\addplot[magenta,semithick] file {wykresy/zad2_yz_6.txt};
	\addplot[orange,semithick] file {wykresy/zad2_yz_7.txt};
	\addplot[pink,semithick] file {wykresy/zad2_yz_8.txt};
	\legend{$Z=-\num{2}$,$Z=-\num{1,5}$,$Z=-\num{1}$,$Z=-\num{0,5}$,$Z=\num{0,5}$,$Z=\num{1}$,$Z=\num{1,5}$,$Z=\num{2}$}
	\end{axis}
	\end{tikzpicture}
	\caption{Odpowiedź $Y(k)$ dla skoków zakłócenia $Z$}
\label{fig:z2_yz}
\end{figure}

\section{Charakterystyka statyczna toru zakłócenia}
Charakterystyka statyczna toru zakłócenia wyznaczona została poprzez sprawdzenie, na jakich wartościach stabilizuje się wyjście obiektu dla różnych wartości sygnału $Z$. Liniowości charakterystki statycznej dowodzi wykres \ref{fig:z2_y_stat_z}, który (w przybliżeniu) jest liniowy.

Wartość wzmocnienia statycznego można wyznaczyć normalizując odpowiedź skokową. Wynosi ona $s_z=0,501$.

\begin{figure}[!htb]
	\centering
	\begin{tikzpicture}
	\begin{axis}[
	width=0.9\textwidth,
	width=0.9\textwidth,
	xmin=-1.5,xmax=1.5,ymin=-1.5,ymax=1.5,
	xlabel={$U$},
	ylabel={$Y(Z)$},
	xtick={-1.5,-1,-0.5,0,0.5,1,1.5},
	ytick={-1.5,-1,-0.5,0,0.5,1,1.5},
	/pgf/number format/.cd,
	use comma,
	1000 sep={}
	]
	\addplot[blue,semithick] file {wykresy/zad2_y_stat_z.txt};
	\end{axis}
	\end{tikzpicture}
	\caption{Charakterystyka statyczna $Y(Z)$}
\label{fig:z2_y_stat_z}
\end{figure}
\chapter{Odpowiedź skokowa do DMC}
Celem zadania trzeciego było przygotowanie odpowiedzi skokowych wykorzystywanych
w algorytmie DMC. W naszym przypadku konieczne było zebranie odpowiedzi na skok
sygnału sterującego oraz zakłócającego. Skok wykonujemy z obranego wcześniej punktu
pracy ($Y_{pp} = 32,20$, $U_{pp} = 27$, $Z = 0$). Z powodu nieuniknionych zakłóceń
w pomieszczeniu, szczególnie trudne było ustabilizowanie obiektu w punkcie pracy,
po wykonaniu wcześniejszych eksperymentów. Z tego powodu otrzymane przez nas odpowiedzi
były prawdopodbnie niedokładne. W naszych głowach zrodziła się z początku myśl,
aby skorzystać z odpowiedzi skokowej ( na skok sterowania ) z poprzedniego laboratoium
(pracowaliśmy również na tym stanowisku ), jednakże jak się później okazało
( problemy z algorytmem DMC) zdecydowaliśmy się na ponowne pozyskanie owej odpowiedzi skokowej.
Wykonując skok sterowania, wartość zakłócenia ustawiliśmy na 0 i odpowiednio
wykonując skok zakłócenia, wartość sterowania pozostawiliśmy na taką jak w punkcie
pracy ( $U_{pp} = 27$). Z powodu zakłóceń i problemów ze stabilizacją obiektu uzyskane
przez nas odpowiedzi skokowe nie były dla nas satysfakcjonujące, lecz z powodu braku
czasu nie zdecydowaliśmy się na powtarzanie eksperymentu. Normalizacji dokonaliśmy
poprzez przesunięcie wszystkich wartości wyjścia obiektu o wartość wyjścia w punkcie
pracy oraz podzielenie przez wartość odpowiednio skoku sterowania lub zakłócenia.
Aproksymacja ma postać członu inercyjnego drugiego rzędu z opóźnieniem.
Aproksymacji dokonaliśmy przy użyciu opracowanego przez nas wcześniej skryptu.
Użyliśmy do tego między innymi wbudowanej funkcji GA ( Genetic Algorithm ) w celu
pozyskania aproksymacji minimalizującej funkcję błędu kwadratowego.

\begin{figure}[tb]
\centering
\begin{tikzpicture}
\begin{axis}[
width=0.75\textwidth,
xmin=0,xmax=400,ymin=0,ymax=0.35,
xlabel={Czas (s)},
ylabel={Wartość wyjścia},
xtick={0, 100, 200, 300, 400},
ytick={0, 0.05, 0.1, 0.15, 0.2, 0.25, 0.3, 0.35},
legend pos=north west,
/pgf/number format/.cd,
use comma,
1000 sep={}
]
\addplot[blue,semithick] file {Skrypty/s_approx.txt};
\end{axis}
\end{tikzpicture}
\caption{Przybliżona odpowiedź obiektu na skok sterowania.}
\label{fig:skok_zak}
\end{figure}


\begin{figure}[tb]
\centering
\begin{tikzpicture}
\begin{axis}[
width=0.75\textwidth,
xmin=0,xmax=400,ymin=0,ymax=0.08,
xlabel={Czas (s)},
ylabel={Wartość wyjścia},
xtick={0, 100, 200, 300, 400},
ytick={0, 0.01, 0.02, 0.03, 0.04, 0.05, 0.06, 0.07, 0.08},
legend pos=north west,
/pgf/number format/.cd,
use comma,
1000 sep={}
]
\addplot[blue,semithick] file {Skrypty/odp_zakl.txt};
\end{axis}
\end{tikzpicture}
\caption{Przybliżona odpowiedź obiektu na skok zakłócenia.}
\label{fig:skok_zak}
\end{figure}

% \input{rysunki}
% \input{listingi}
% \input{literatura}
\end{document}
