\chapter{Odpowiedź skokowa na potrzeby DMC}
W tym rozdziale opisany został eksperyment zbierania odpowiedzi skokowej,
przystosowanej do wykorzystania przy projektowaniu regulatora DMC dla obiektu.
Polegał on na przeprowadzeniu skoku z punktu pracy obiektu, czyli przy
wartości sterowania $U = 0,9$, do górnego ograniczenia sygnału sterującego,
czyli $U = 1,2$. Następnie odpowiedź została znormalizowana, czyli przesunięta
o wartość wyjścia w punkcie pracy, oraz podzielona o długość
skoku sterowania.
\begin{equation}
  Y(k) = (Y_{eksperymentu}(k) - Y_{pp})/\Delta U
\end{equation}
Wyniki zostały przedstawione na wykresie \ref{fig:skok_DMC}. Widać na nich
poprawność wyliczonego w rozdziale \ref{sec:zad2} wzmocnienia statycznego.

\begin{figure}[tb]
\centering
\begin{tikzpicture}
\begin{axis}[
width=0.75\textwidth,
xmin=0,xmax=150,ymin=0,ymax=1.8,
xlabel={Chwila (k)},
ylabel={Wartość $s_n$},
xtick={0, 50, 100, 150},
ytick={0, 0.2, 0.4, 0.6, 0.8, 1, 1.2, 1.4, 1.6, 1.8},
% legend pos=north east,
/pgf/number format/.cd,
use comma,
1000 sep={}
]
\addplot[blue,semithick] file {wykresy/zad3.txt};
% \addplot[red,semithick] file {wykresy/zad1U.txt};
% \legend{Wyjście, Sterowanie}
\end{axis}
\end{tikzpicture}
\caption{Odpowiedź skokowa znormalizowana na potrzeby algorytmu DMC.}
\label{fig:skok_DMC}
\end{figure}
