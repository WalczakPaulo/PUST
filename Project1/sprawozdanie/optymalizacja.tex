% /******** Optymalizacja *********/
\chapter{Optymalizacja}
Do wyznaczenia optymalnych parametrów regulatorów PID i DMC użyliśmy funkcji \emph{fmincon} oraz \emph{GA}. Są to funkcje pozwalające wyznaczyć minimum (globalne) zadanej funkcji celu.
W obu przypadkach regulatorów, funkcję celu traktujemy jako suma kwadratów błędów ( między wyjściem obiektu a wartością zadaną ).
Do wyznaczenia optymalnych parametrów regulatora PID użyliśmy funkcji \emph{fmincon}. Optymalizujemy trzy parametry: $K$ (wzmocnienie - $X(1)$), $T_i$ ($X(2)$) oraz $T_d$ ($X(3)$).
Ograniczeniem jakie przyjmujemy są dodatnie wartości parametrów regulatora. Po uruchomieniu \emph{fmincon} otrzymujemy następujące parametry:
\begin{align}
  K &= 1,2253 \nonumber \\
  T_d &= 4,4339 \\
  T_i &= 27,3968 \nonumber
\end{align}
Wyniki zostały zobrazowane na wykresach \ref{fig:optim_pid_out}, oraz \ref{fig:optim_pid_ster}.
Zauważamy, że funkcja celu przyjmuje niższą wartość niż dla parametrów regulatora wyznaczonego metodą inżynierską, co sugeruje prawidłowe działanie funkcji optymalizacji.

\begin{figure}[tb]
\centering
\begin{tikzpicture}
\begin{axis}[
width=0.75\textwidth,
height=0.375\textwidth,
xmin=0,xmax=1000,ymin=2.5,ymax=3.5,
xlabel={Chwila (k)},
ylabel={Wyjście (y)},
xtick={0, 200, 400, 600, 800, 1000},
ytick={2.5, 3, 3.5},
legend pos=south east,
/pgf/number format/.cd,
use comma,
1000 sep={}
]
\addplot[blue,semithick] file {wykresy/pid_optim_y.txt};
\addplot[red,semithick] file {wykresy/pid_optim_yzad.txt};

\legend{Wyjście, Wartość zadana}
\end{axis}
\end{tikzpicture}
\caption{Wyjście obiektu z regulatorem PID z nastawami wyznaczonymi funkcją \emph{fmincon}.}
\label{fig:optim_pid_out}
\end{figure}

\begin{figure}[tb]
\centering
\begin{tikzpicture}
\begin{axis}[
width=0.75\textwidth,
height=0.375\textwidth,
xmin=0,xmax=1000,ymin=0.6,ymax=1.2,
xlabel={Chwila (k)},
ylabel={Sterowanie (u)},
xtick={0, 200, 400, 600, 800, 1000},
ytick={0.6, 0.8, 1, 1.2},
legend pos=south east,
/pgf/number format/.cd,
use comma,
1000 sep={}
]
\addplot[blue,semithick] file {wykresy/pid_optim_u.txt};

% \legend{Wyjście, Wartość zadana}
\end{axis}
\end{tikzpicture}
\caption{Sterowanie obiektu z regulatorem PID z nastawami wyznaczonymi funkcją \emph{fmincon}.}
\label{fig:optim_pid_ster}
\end{figure}

Do wyznaczenia optymalnych parametrów regulatora DMC użyliśmy funkcji \emph{GA}. Jest to algorytm genetyczny umożliwiający znalezienie minimum danej funkcji. Optymalizujemy trzy parametry:
$N$ - horyzont predykcji,
$N_u$ - horyzont sterowania,
$\lambda$ - współczynnik 'kary' za zmiany sterowania.
Użycie algorytmu \emph{GA} motywujemy tym, że w przeciwieństwie do \emph{fmincon}, możemy wprowadzić ograniczenie na N oraz Nu, tak aby ich wartości mogły być dodatnie całkowite.
Po uruchomieniu \emph{GA} otrzymujemy następujące parametry:
\begin{align}
  N &= 150 \nonumber \\
  N_u &= 2 \\
  \lambda &= 3,3089 \nonumber
\end{align}
Wyniki obrazują wykresy \ref{fig:optim_dmc_out} i \ref{fig:optim_dmc_ster}.
Zauważamy, że funkcja celu przyjmuje niższą wartość niż dla parametrów regulatora DMC wyznaczonego na chybił trafił, oraz zdecydowanie lepsze niż parametry optymalne regulatora PID.

\begin{figure}[tb]
\centering
\begin{tikzpicture}
\begin{axis}[
width=0.75\textwidth,
height=0.375\textwidth,
xmin=0,xmax=1000,ymin=2.5,ymax=3.5,
xlabel={Chwila (k)},
ylabel={Wyjście (y)},
xtick={0, 200, 400, 600, 800, 1000},
ytick={2.5, 3, 3.5},
legend pos=south east,
/pgf/number format/.cd,
use comma,
1000 sep={}
]
\addplot[blue,semithick] file {wykresy/dmc_optim_y.txt};
\addplot[red,semithick] file {wykresy/dmc_optim_yzad.txt};

\legend{Wyjście, Wartość zadana}
\end{axis}
\end{tikzpicture}
\caption{Wyjście obiektu z regulatorem DMC z nastawami wyznaczonymi funkcją \emph{GA}.}
\label{fig:optim_dmc_out}
\end{figure}

\begin{figure}[tb]
\centering
\begin{tikzpicture}
\begin{axis}[
width=0.75\textwidth,
height=0.375\textwidth,
xmin=0,xmax=1000,ymin=0.6,ymax=1.2,
xlabel={Chwila (k)},
ylabel={Sterowanie (u)},
xtick={0, 200, 400, 600, 800, 1000},
ytick={0.6, 0.8, 1, 1.2},
legend pos=south east,
/pgf/number format/.cd,
use comma,
1000 sep={}]
\addplot[blue,semithick] file {wykresy/dmc_optim_u.txt};

% \legend{Wyjście, Wartość zadana}
\end{axis}
\end{tikzpicture}
\caption{Sterowanie obiektu z regulatorem DMC z nastawami wyznaczonymi funkcją \emph{GA}.}
\label{fig:optim_dmc_ster}
\end{figure}
