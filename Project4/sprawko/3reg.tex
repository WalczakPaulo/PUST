\section{3 regulatory lokalne}
Dla 3 regulatorów lokalnych użyliśmy sigmoidalnych funkcji przynależności, danych wzorami \ref{eq:mi3_1}, \ref{eq:mi3_2} i \ref{eq:mi3_3}. Parametry mają wartośi: $\alpha = -3$, $c^1 = -0,05$, $c^2=1,3$.
\begin{equation} \label{eq:mi3_1}
\mu_1 = 1 - \frac{1}{1+e^{\alpha * (y(k)-c^1))}}
\end{equation}
\begin{equation} \label{eq:mi3_2}
\mu_2 = \frac{1}{1+e^{\alpha * (y(k)-c^1))}} - \frac{1}{1+e^{\alpha * (y(k)-c^2))}}
\end{equation}
\begin{equation} \label{eq:mi3_3}
\mu_3 = \frac{1}{1+e^{\alpha * (y(k)-c^2))}}
\end{equation}

\begin{figure}[H]
\centering
\begin{tikzpicture}
\begin{axis}[
height=0.5\textwidth,
width=0.75\textwidth,
xmin=-0.15,xmax=4.28,
xlabel={$y(k)$},
ylabel={$\mu_1$},
/pgf/number format/.cd,
use comma,
1000 sep={}
]
\addplot file {wykresy/3_reg_sigm_1.txt};
\end{axis}
\end{tikzpicture}
\caption{Funkcja przynależności regulatora 1}
\label{fig:mi3_1}
\end{figure}

\begin{figure}[H]
\centering
\begin{tikzpicture}
\begin{axis}[
height=0.5\textwidth,
width=0.75\textwidth,
xmin=-0.15,xmax=4.28,
xlabel={$y(k)$},
ylabel={$\mu_2$},
/pgf/number format/.cd,
use comma,
1000 sep={}
]
\addplot file {wykresy/3_reg_sigm_2.txt};
\end{axis}
\end{tikzpicture}
\caption{Funkcja przynależności regulatora 2}
\label{fig:mi3_2}
\end{figure}

\begin{figure}[H]
\centering
\begin{tikzpicture}
\begin{axis}[
height=0.5\textwidth,
width=0.75\textwidth,
xmin=-0.15,xmax=4.28,
xlabel={$y(k)$},
ylabel={$\mu_1$},
/pgf/number format/.cd,
use comma,
1000 sep={}
]
\addplot file {wykresy/3_reg_sigm_3.txt};
\end{axis}
\end{tikzpicture}
\caption{Funkcja przynależności regulatora 3}
\label{fig:mi3_3}
\end{figure}

\subsection{PID}
Parametry algorytmu PID wyznaczone przez algorytm genetyczny są następujące:
\begin{itemize}
\item $K^1 = 0,3096$
\item $T^1_i = 3,3483$
\item $T^1_d = 2,5194$
\\
\item $K^2 = 0,3041$
\item $T^2_i = 2,7366$
\item $T^2_d = 3,3854$
\\
\item $K^3 = 0,1113$
\item $T^3_i = 2,0384$
\item $T^3_d = 2,6278$
\end{itemize}

\begin{figure}[H]
\centering
\begin{tikzpicture}
\begin{axis}[
height=0.5\textwidth,
width=0.75\textwidth,
xmin=0,xmax=1200,
xlabel={$k$},
ylabel={$y(k)$},
/pgf/number format/.cd,
use comma,
1000 sep={}
]
\addplot file {wykresy/3_reg_pid_y.txt};
\addplot file {wykresy/zad4_yzad.txt};
\legend{$y$,$y_{zad}$}
\end{axis}
\end{tikzpicture}
\caption{Wyjście obiektu dla 3 regulatorów lokalnych PID}
\label{fig:3_reg_pid_y}
\end{figure}

\begin{figure}[H]
\centering
\begin{tikzpicture}
\begin{axis}[
height=0.5\textwidth,
width=0.75\textwidth,
xmin=0,xmax=1200,
xlabel={$k$},
ylabel={$u(k)$},
/pgf/number format/.cd,
use comma,
1000 sep={}
]
\addplot file {wykresy/3_reg_pid_u.txt};
\end{axis}
\end{tikzpicture}
\caption{Sterowanie obiektu dla 3 regulatorów lokalnych PID}
\label{fig:3_reg_pid_u}
\end{figure}

Wskaźnik jakości regulacji dla regulatora, którego działanie przedstawiają wykresy \ref{fig:3_reg_pid_y} i \ref{fig:3_reg_pid_u}, ma wartość $E=93,5945$. Poprawa wskaźnika jakości jest niewielka, jednak porównując z wykresem \ref{fig:2_reg_pid_y} można dostrzec znacznie szybsze osiągnięcia wartości zadanej.

\subsection{DMC}
Na początek użyliśmy nastaw $\lambda_1=10$, $\lambda_2=10$ i $\lambda_3=10$. Błąd regulacji wynosił $E=96,7063$. Działanie regulatora przedstawiają wykresy \ref{fig:31_reg_dmc_u} i \ref{fig:31_reg_dmc_y}.

\begin{figure}[H]
\centering
\begin{tikzpicture}
\begin{axis}[
height=0.5\textwidth,
width=0.75\textwidth,
xmin=0,xmax=1200,
xlabel={$k$},
ylabel={$y(k)$},
/pgf/number format/.cd,
use comma,
1000 sep={}
]
\addplot file {wykresy/31_reg_dmc_y.txt};
\addplot file {wykresy/zad4_yzad.txt};
\legend{$y$,$y_{zad}$}
\end{axis}
\end{tikzpicture}
\caption{Wyjście obiektu dla 3 regulatorów lokalnych DMC}
\label{fig:31_reg_dmc_y}
\end{figure}

\begin{figure}[H]
\centering
\begin{tikzpicture}
\begin{axis}[
height=0.5\textwidth,
width=0.75\textwidth,
xmin=0,xmax=1200,
xlabel={$k$},
ylabel={$u(k)$},
/pgf/number format/.cd,
use comma,
1000 sep={}
]
\addplot file {wykresy/31_reg_dmc_u.txt};
\end{axis}
\end{tikzpicture}
\caption{Sterowanie obiektu dla 3 regulatorów lokalnych DMC}
\label{fig:31_reg_dmc_u}
\end{figure}

Metodą eksperymentalną dobraliśmy wartości nastaw $\lambda_1=50$, $\lambda_2=0.1$, $\lambda_3=0.2$. Wskaźnik jakości regulacji wynosił $E=96,1992$. Działanie regulatora przedstawiają wykresy \ref{fig:32_reg_dmc_y} i \ref{fig:32_reg_dmc_u}. Porównując z działaniem regulatora pid (wykresy \ref{fig:3_reg_pid_y} i \ref{fig:3_reg_pid_u}) uznajemy, że regulator DMC działa lepiej ze względu na mniejsze oscylacje i przeregulowania, mimo większego błędu regulaji. Poprawa w stosunku do DMC z 2 regulatorami lokalnymi jest znikoma.

\begin{figure}[H]
\centering
\begin{tikzpicture}
\begin{axis}[
height=0.5\textwidth,
width=0.75\textwidth,
xmin=0,xmax=1200,
xlabel={$k$},
ylabel={$y(k)$},
/pgf/number format/.cd,
use comma,
1000 sep={}
]
\addplot file {wykresy/32_reg_dmc_y.txt};
\addplot file {wykresy/zad4_yzad.txt};
\legend{$y$,$y_{zad}$}
\end{axis}
\end{tikzpicture}
\caption{Wyjście obiektu dla 3 regulatorów lokalnych DMC}
\label{fig:32_reg_dmc_y}
\end{figure}

\begin{figure}[H]
\centering
\begin{tikzpicture}
\begin{axis}[
height=0.5\textwidth,
width=0.75\textwidth,
xmin=0,xmax=1200,
xlabel={$k$},
ylabel={$u(k)$},
/pgf/number format/.cd,
use comma,
1000 sep={}
]
\addplot file {wykresy/32_reg_dmc_u.txt};
\end{axis}
\end{tikzpicture}
\caption{Sterowanie obiektu dla 3 regulatorów lokalnych DMC}
\label{fig:32_reg_dmc_u}
\end{figure}