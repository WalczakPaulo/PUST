\chapter{Regulatory PID i DMC}
Do wyznaczenia dobrych nastaw regulatorów PID i DMC użyjemy metody eksperymentalnej, starając się zminimalizować błąd regulacji dany wzorem:
\begin{equation}
E = \sum_{k=1}^{k_{konc}} (y^{zad}(k)-y(k))^2
\end{equation}
Ponadto jakość regulacji będziemy oceniać optycznie, na podstawie otrzymanych wykresów. \\
Wartościami zadanymi wykorzystywanymi we wszystkich eksperymentach są kolejno: $1,3,2,-0,07$.

\section{Regulator PID}
Badania zaczęliśmy od nastaw:
\begin{itemize}
\item $K=1$
\item $T_i=10$
\item $T_d=2$
\end{itemize}

\begin{figure}[H]
\centering
\begin{tikzpicture}
\begin{axis}[
height=0.5\textwidth,
width=0.75\textwidth,
xmin=0,xmax=1200,ymin=-0.5,ymax=4.5,
xlabel={k},
ylabel={y(k)},
/pgf/number format/.cd,
use comma,
1000 sep={}
]
\addplot file {wykresy/y_pid_1_10_2.txt};
\addplot file {wykresy/zad4_yzad.txt};
\legend{$y$,$y_{zad}$}
\end{axis}
\end{tikzpicture}
\caption{Parametry PID: $K=1$, $T_i=10$, $T_d=2$}
\label{fig:pid1_y}
\end{figure}

\begin{figure}[H]
\centering
\begin{tikzpicture}
\begin{axis}[
height=0.5\textwidth,
width=0.75\textwidth,
xmin=0,xmax=1200,ymin=-1,ymax=1,
xlabel={k},
ylabel={u(k)},
/pgf/number format/.cd,
use comma,
1000 sep={}
]
\addplot file {wykresy/u_pid_1_10_2.txt};
\end{axis}
\end{tikzpicture}
\caption{Parametry PID: $K=1$, $T_i=10$, $T_d=2$}
\label{fig:pid1_u}
\end{figure}

Jak widać na wykresach \ref{fig:pid1_y} i \ref{fig:pid1_u}, występują silne oscylacje na wyjściu, a sterowanie odbija się od ograniczeń. Błąd regulacji wynosi $E=368,0794$. W tym przypadku zdecydowanie należy zmniejszyć wzmocnienie regulatora. Wykresy \ref{fig:pid2_y} i \ref{fig:pid2_u} przedstawiają rezultaty dla wzmocnienia $K=0,3$.

\begin{figure}[H]
\centering
\begin{tikzpicture}
\begin{axis}[
height=0.5\textwidth,
width=0.75\textwidth,
xmin=0,xmax=1200,ymin=-0.5,ymax=4.5,
xlabel={k},
ylabel={y(k)},
/pgf/number format/.cd,
use comma,
1000 sep={}
]
\addplot file {wykresy/y_pid_0.3_10_2.txt};
\addplot file {wykresy/zad4_yzad.txt};
\legend{$y$,$y_{zad}$}
\end{axis}
\end{tikzpicture}
\caption{Parametry PID: $K=0,3$, $T_i=10$, $T_d=2$}
\label{fig:pid2_y}
\end{figure}

\begin{figure}[H]
\centering
\begin{tikzpicture}
\begin{axis}[
height=0.5\textwidth,
width=0.75\textwidth,
xmin=0,xmax=1200,ymin=-1,ymax=1,
xlabel={k},
ylabel={u(k)},
/pgf/number format/.cd,
use comma,
1000 sep={}
]
\addplot file {wykresy/u_pid_0.3_10_2.txt};
\end{axis}
\end{tikzpicture}
\caption{Parametry PID: $K=0,3$, $T_i=10$, $T_d=2$}
\label{fig:pid2_u}
\end{figure}

Udało się wyeliminować odbijanie sterowania od ograniczeń, ale nadal występują przeregulowania i gasnące oscylacje. Błąd regulacji wynosi $E=114,1445$, a więc około trzykrotnie mniej niż w poprzednim przypadku. Kolejnym krokiem będzie zwiększenie wpływu całkowania, a więc zmniejszenie parametru $T_i$. Testowaną wartością będzie $T_i=7$.

\begin{figure}[H]
\centering
\begin{tikzpicture}
\begin{axis}[
height=0.5\textwidth,
width=0.75\textwidth,
xmin=0,xmax=1200,ymin=-0.5,ymax=4.5,
xlabel={k},
ylabel={y(k)},
/pgf/number format/.cd,
use comma,
1000 sep={}
]
\addplot file {wykresy/y_pid_0.3_7_2.txt};
\addplot file {wykresy/zad4_yzad.txt};
\legend{$y$,$y_{zad}$}
\end{axis}
\end{tikzpicture}
\caption{Parametry PID: $K=0,3$, $T_i=7$, $T_d=2$}
\label{fig:pid3_y}
\end{figure}

\begin{figure}[H]
\centering
\begin{tikzpicture}
\begin{axis}[
height=0.5\textwidth,
width=0.75\textwidth,
xmin=0,xmax=1200,ymin=-1,ymax=1,
xlabel={k},
ylabel={u(k)},
/pgf/number format/.cd,
use comma,
1000 sep={}
]
\addplot file {wykresy/u_pid_0.3_7_2.txt};
\end{axis}
\end{tikzpicture}
\caption{Parametry PID: $K=0,3$, $T_i=7$, $T_d=2$}
\label{fig:pid3_u}
\end{figure}

Różnica w stosunku do poprzednich wartości nie jest znaczna, choć oscylacje gasną nieco szybciej, co można zauważyć na wykresach \ref{fig:pid3_y} i \ref{fig:pid3_u}. Poprawił się również wskaźnik jakości regulacji, wynoszący teraz $E=106,9172$. \\W kolejnej próbie poprawy działania regulatora przetestowane zostaną nastawy:
\begin{itemize}
\item $K=0,3$
\item $T_i=6$
\item $T_d=1,5$
\end{itemize}

\begin{figure}[H]
\centering
\begin{tikzpicture}
\begin{axis}[
height=0.5\textwidth,
width=0.75\textwidth,
xmin=0,xmax=1200,ymin=-0.5,ymax=4.5,
xlabel={k},
ylabel={y(k)},
/pgf/number format/.cd,
use comma,
1000 sep={}
]
\addplot file {wykresy/y_pid_0.3_6_1.5.txt};
\addplot file {wykresy/zad4_yzad.txt};
\legend{$y$,$y_{zad}$}
\end{axis}
\end{tikzpicture}
\caption{Parametry PID: $K=0,3$, $T_i=6$, $T_d=1,5$}
\label{fig:pid4_y}
\end{figure}

\begin{figure}[H]
\centering
\begin{tikzpicture}
\begin{axis}[
height=0.5\textwidth,
width=0.75\textwidth,
xmin=0,xmax=1200,ymin=-1,ymax=1,
xlabel={k},
ylabel={u(k)},
/pgf/number format/.cd,
use comma,
1000 sep={}
]
\addplot file {wykresy/u_pid_0.3_6_1.5.txt};
\end{axis}
\end{tikzpicture}
\caption{Parametry PID: $K=0,3$, $T_i=6$, $T_d=1,5$}
\label{fig:pid4_u}
\end{figure}

Zmiany na wykresach \ref{fig:pid4_y} i \ref{fig:pid4_u} w stosunku do poprzednich przypadków są niewielkie, choć nastąpiła poprawa wskaźnika jakości regulacji, którego wartość wynosi $E=105,6223$.

\section{Regulator DMC}
Dobór parametrów DMC metodą eksperymentalną rozpoczęliśmy od parametrów:
\begin{itemize}
\item $N=50$
\item $N_u=50$
\item $\lambda=10$
\end{itemize}

\begin{figure}[H]
\centering
\begin{tikzpicture}
\begin{axis}[
height=0.5\textwidth,
width=0.75\textwidth,
xmin=0,xmax=1200,ymin=-0.5,ymax=4.5,
xlabel={k},
ylabel={y(k)},
/pgf/number format/.cd,
use comma,
1000 sep={}
]
\addplot file {wykresy/y_dmc_50_50_10.txt};
\addplot file {wykresy/zad4_yzad.txt};
\legend{$y$,$y_{zad}$}
\end{axis}
\end{tikzpicture}
\caption{Parametry DMC: $N=50$, $N_u=50$, $\lambda=10$}
\label{fig:dmc1_y}
\end{figure}

\begin{figure}[H]
\centering
\begin{tikzpicture}
\begin{axis}[
height=0.5\textwidth,
width=0.75\textwidth,
xmin=0,xmax=1200,ymin=-1,ymax=1,
xlabel={k},
ylabel={u(k)},
/pgf/number format/.cd,
use comma,
1000 sep={}
]
\addplot file {wykresy/u_dmc_50_50_10.txt};
\end{axis}
\end{tikzpicture}
\caption{Parametry DMC: $N=50$, $N_u=50$, $\lambda=10$}
\label{fig:dmc1_u}
\end{figure}

Jak widać na wykresach \ref{fig:dmc1_y} i \ref{fig:dmc1_u} występują przeregulowania i gasnące oscylacje. Wartość błędu regulacji wynosi $E=121,2675$. Należy zwiększyć wartość parametru $\lambda$, aby "złagodzić" działanie regulatora.

\begin{figure}[H]
\centering
\begin{tikzpicture}
\begin{axis}[
height=0.5\textwidth,
width=0.75\textwidth,
xmin=0,xmax=1200,ymin=-0.5,ymax=4.5,
xlabel={k},
ylabel={y(k)},
/pgf/number format/.cd,
use comma,
1000 sep={}
]
\addplot file {wykresy/y_dmc_50_50_40.txt};
\addplot file {wykresy/zad4_yzad.txt};
\legend{$y$,$y_{zad}$}
\end{axis}
\end{tikzpicture}
\caption{Parametry DMC: $N=50$, $N_u=50$, $\lambda=40$}
\label{fig:dmc2_y}
\end{figure}

\begin{figure}[H]
\centering
\begin{tikzpicture}
\begin{axis}[
height=0.5\textwidth,
width=0.75\textwidth,
xmin=0,xmax=1200,ymin=-1,ymax=1,
xlabel={k},
ylabel={u(k)},
/pgf/number format/.cd,
use comma,
1000 sep={}
]
\addplot file {wykresy/u_dmc_50_50_40.txt};
\end{axis}
\end{tikzpicture}
\caption{Parametry DMC: $N=50$, $N_u=50$, $\lambda=40$}
\label{fig:dmc2_u}
\end{figure}

Dla regulatora, którego działanie przedstawiają wykresy \ref{fig:dmc2_y} i \ref{fig:dmc2_u} parametr $\lambda$ ma wartość 40. Tak duża wartość spowodowała zmniejszenie oscylacji i poprawę wskaźnika wartości regulacji $E=110,1876$. W kolejnych próbach skrócimy horyzonty sterowania i predykcji.

\begin{figure}[H]
\centering
\begin{tikzpicture}
\begin{axis}[
height=0.5\textwidth,
width=0.75\textwidth,
xmin=0,xmax=1200,ymin=-0.5,ymax=4.5,
xlabel={k},
ylabel={y(k)},
/pgf/number format/.cd,
use comma,
1000 sep={}
]
\addplot file {wykresy/y_dmc_25_25_40.txt};
\addplot file {wykresy/zad4_yzad.txt};
\legend{$y$,$y_{zad}$}
\end{axis}
\end{tikzpicture}
\caption{Parametry DMC: $N=25$, $N_u=25$, $\lambda=40$}
\label{fig:dmc3_y}
\end{figure}

\begin{figure}[H]
\centering
\begin{tikzpicture}
\begin{axis}[
height=0.5\textwidth,
width=0.75\textwidth,
xmin=0,xmax=1200,ymin=-1,ymax=1,
xlabel={k},
ylabel={u(k)},
/pgf/number format/.cd,
use comma,
1000 sep={}
]
\addplot file {wykresy/u_dmc_25_25_40.txt};
\end{axis}
\end{tikzpicture}
\caption{Parametry DMC: $N=25$, $N_u=25$, $\lambda=40$}
\label{fig:dmc3_u}
\end{figure}

Jak widać na wykresach \ref{fig:dmc3_y} i \ref{fig:dmc3_u}, skrócenie horyzontów do wartości $N=25$ i $N_u=25$ nie przyniosło poprawy jakości sterowania. Błąd regulacji wynosi $E=110,3305$. Jako że obiekt jest nieliniowy, regulator nie ma do dyspozycji prawidłowego modelu w postaci odpowiedzi skokowej dla części zakresów sterowania. Z tego powodu sprawdzimy działanie DMC dla horyzontu sterowania równego $N_u=1$.

\begin{figure}[H]
\centering
\begin{tikzpicture}
\begin{axis}[
height=0.5\textwidth,
width=0.75\textwidth,
xmin=0,xmax=1200,ymin=-0.5,ymax=4.5,
xlabel={k},
ylabel={y(k)},
/pgf/number format/.cd,
use comma,
1000 sep={}
]
\addplot file {wykresy/y_dmc_25_1_40.txt};
\addplot file {wykresy/zad4_yzad.txt};
\legend{$y$,$y_{zad}$}
\end{axis}
\end{tikzpicture}
\caption{Parametry DMC: $N=25$, $N_u=1$, $\lambda=40$}
\label{fig:dmc4_y}
\end{figure}

\begin{figure}[H]
\centering
\begin{tikzpicture}
\begin{axis}[
height=0.5\textwidth,
width=0.75\textwidth,
xmin=0,xmax=1200,ymin=-1,ymax=1,
xlabel={k},
ylabel={u(k)},
/pgf/number format/.cd,
use comma,
1000 sep={}
]
\addplot file {wykresy/u_dmc_25_1_40.txt};
\end{axis}
\end{tikzpicture}
\caption{Parametry DMC: $N=25$, $N_u=1$, $\lambda=40$}
\label{fig:dmc4_u}
\end{figure}

Jak widać na wykresach \ref{fig:dmc4_y} i \ref{fig:dmc4_u} zmiana horyzontu sterowania na $N_u=1$ przyniosła oczekiwane rezultaty. Niemalże całkowicie zostały wyeliminowane przeregulowania, a wyjście obiektu stabilizuje się szybko. Wskaźnik jakości regulacji wynosi $E=102,3977$, co jest nieco lepszym wynikiem niż dla regulatora PID, którego działanie przedstawiają wykresy \ref{fig:pid4_y} i \ref{fig:pid4_u}.