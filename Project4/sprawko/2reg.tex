\section{2 regulatory lokalne}
Dla 2 regulatorów lokalnych użyliśmy sigmoidalnych funkcji przynależności, danych wzorami \ref{eq:mi1} i \ref{eq:mi2}. Parametry mają wartośi: $\alpha = -3$, $c = 0.3$.
\begin{equation} \label{eq:mi1}
\mu_1 = 1 - \frac{1}{1+e^{\alpha * (y(k)-c))}}
\end{equation}
\begin{equation} \label{eq:mi2}
\mu_2 = \frac{1}{1+e^{\alpha * (y(k)-c))}}
\end{equation}

\begin{figure}[H]
\centering
\begin{tikzpicture}
\begin{axis}[
height=0.5\textwidth,
width=0.75\textwidth,
xmin=-0.15,xmax=4.28,
xlabel={$y(k)$},
ylabel={$\mu_1$},
/pgf/number format/.cd,
use comma,
1000 sep={}
]
\addplot file {wykresy/2_reg_sigm_1.txt};
\end{axis}
\end{tikzpicture}
\caption{Funkcja przynależności regulatora 1}
\label{fig:mi2_1}
\end{figure}

\begin{figure}[H]
\centering
\begin{tikzpicture}
\begin{axis}[
height=0.5\textwidth,
width=0.75\textwidth,
xmin=-0.15,xmax=4.28,
xlabel={$y(k)$},
ylabel={$\mu_2$},
/pgf/number format/.cd,
use comma,
1000 sep={}
]
\addplot file {wykresy/2_reg_sigm_2.txt};
\end{axis}
\end{tikzpicture}
\caption{Funkcja przynależności regulatora 2}
\label{fig:mi2_2}
\end{figure}

\subsection{PID}
Parametry algorytmu PID wyznaczone przez algorytm genetyczny są następujące:
\begin{itemize}
\item $K^1 = 0,24$
\item $T^1_i = 5,37$
\item $T^1_d = 1,05$
\\
\item $K^2 = 0,21$
\item $T^2_i = 5,32$
\item $T^2_d = 2,12$
\end{itemize}

\begin{figure}[H]
\centering
\begin{tikzpicture}
\begin{axis}[
height=0.5\textwidth,
width=0.75\textwidth,
xmin=0,xmax=1200,
xlabel={$k$},
ylabel={$y(k)$},
/pgf/number format/.cd,
use comma,
1000 sep={}
]
\addplot file {wykresy/2_reg_pid_y.txt};
\addplot file {wykresy/zad4_yzad.txt};
\legend{$y$,$y_{zad}$}
\end{axis}
\end{tikzpicture}
\caption{Wyjście obiektu dla 2 regulatorów lokalnych PID}
\label{fig:2_reg_pid_y}
\end{figure}

\begin{figure}[H]
\centering
\begin{tikzpicture}
\begin{axis}[
height=0.5\textwidth,
width=0.75\textwidth,
xmin=0,xmax=1200,
xlabel={$k$},
ylabel={$u(k)$},
/pgf/number format/.cd,
use comma,
1000 sep={}
]
\addplot file {wykresy/2_reg_pid_u.txt};
\end{axis}
\end{tikzpicture}
\caption{Sterowanie obiektu dla 2 regulatorów lokalnych PID}
\label{fig:2_reg_pid_u}
\end{figure}

Wskaźnik jakości regulacji dla regulatora, którego działanie przedstawiają wykresy \ref{fig:2_reg_pid_y} i \ref{fig:2_reg_pid_u}, ma wartość $E=101,5549$. Błąd jest zatem nieznacznie niższy niż dla "klasycznego" regulatora DMC z wykresów \ref{fig:dmc4_y} i \ref{fig:dmc4_u}, choć można zauważyć, że występują większe przeregulowania i oscylacje.

\subsection{DMC}