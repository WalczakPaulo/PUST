\section{5 regulatorów lokalnych}
Dla 3 regulatorów lokalnych użyliśmy sigmoidalnych funkcji przynależności, danych wzorami \ref{eq:mi5_1}, \ref{eq:mi5_2}, \ref{eq:mi5_3}, \ref{eq:mi5_4} i \ref{eq:mi5_5}. Parametry mają wartośi: $\alpha = -3$, $c^1 = -0,05$, $c^2=0,25$, $c^3=0,5$, $c^4=1.3$.
\begin{equation} \label{eq:mi5_1}
\mu_1 = 1 - \frac{1}{1+e^{\alpha * (y(k)-c^1))}}
\end{equation}
\begin{equation} \label{eq:mi5_2}
\mu_2 = \frac{1}{1+e^{\alpha * (y(k)-c^1))}} - \frac{1}{1+e^{\alpha * (y(k)-c^2))}}
\end{equation}
\begin{equation} \label{eq:mi5_3}
\mu_3 = \frac{1}{1+e^{\alpha * (y(k)-c^2))}} - \frac{1}{1+e^{\alpha * (y(k)-c^3))}}
\end{equation}
\begin{equation} \label{eq:mi5_4}
\mu_4 = \frac{1}{1+e^{\alpha * (y(k)-c^3))}} - \frac{1}{1+e^{\alpha * (y(k)-c^4))}}
\end{equation}
\begin{equation} \label{eq:mi5_5}
\mu_5 = \frac{1}{1+e^{\alpha * (y(k)-c^4))}}
\end{equation}

\begin{figure}[H]
\centering
\begin{tikzpicture}
\begin{axis}[
height=0.5\textwidth,
width=0.75\textwidth,
xmin=-0.15,xmax=4.28,
xlabel={$y(k)$},
ylabel={$\mu_1$},
/pgf/number format/.cd,
use comma,
1000 sep={}
]
\addplot file {wykresy/5_reg_sigm_1.txt};
\end{axis}
\end{tikzpicture}
\caption{Funkcja przynależności regulatora 1}
\label{fig:mi5_1}
\end{figure}

\begin{figure}[H]
\centering
\begin{tikzpicture}
\begin{axis}[
height=0.5\textwidth,
width=0.75\textwidth,
xmin=-0.15,xmax=4.28,
xlabel={$y(k)$},
ylabel={$\mu_2$},
/pgf/number format/.cd,
use comma,
1000 sep={}
]
\addplot file {wykresy/5_reg_sigm_2.txt};
\end{axis}
\end{tikzpicture}
\caption{Funkcja przynależności regulatora 2}
\label{fig:mi5_2}
\end{figure}

\begin{figure}[H]
\centering
\begin{tikzpicture}
\begin{axis}[
height=0.5\textwidth,
width=0.75\textwidth,
xmin=-0.15,xmax=4.28,
xlabel={$y(k)$},
ylabel={$\mu_1$},
/pgf/number format/.cd,
use comma,
1000 sep={}
]
\addplot file {wykresy/5_reg_sigm_3.txt};
\end{axis}
\end{tikzpicture}
\caption{Funkcja przynależności regulatora 3}
\label{fig:mi5_3}
\end{figure}

\begin{figure}[H]
\centering
\begin{tikzpicture}
\begin{axis}[
height=0.5\textwidth,
width=0.75\textwidth,
xmin=-0.15,xmax=4.28,
xlabel={$y(k)$},
ylabel={$\mu_1$},
/pgf/number format/.cd,
use comma,
1000 sep={}
]
\addplot file {wykresy/5_reg_sigm_4.txt};
\end{axis}
\end{tikzpicture}
\caption{Funkcja przynależności regulatora 4}
\label{fig:mi5_4}
\end{figure}

\begin{figure}[H]
\centering
\begin{tikzpicture}
\begin{axis}[
height=0.5\textwidth,
width=0.75\textwidth,
xmin=-0.15,xmax=4.28,
xlabel={$y(k)$},
ylabel={$\mu_1$},
/pgf/number format/.cd,
use comma,
1000 sep={}
]
\addplot file {wykresy/5_reg_sigm_5.txt};
\end{axis}
\end{tikzpicture}
\caption{Funkcja przynależności regulatora 5}
\label{fig:mi5_5}
\end{figure}

\subsection{PID}
W przypadku 5 regulatorów lokalnych algorytm genetyczny nie poradził sobie z optymalizacją parametrów. Konieczne okazało się użycie metody eksperymentalnej, co doprowadziło do tego, że ta wersja regulatora działa gorzej niż w przypadku 4 regulatorów lokalnych. Parametry są następujące:
\begin{itemize}
\item $K^1 = 0,68$
\item $T^1_i = 3,1$
\item $T^1_d = 2,8$
\\
\item $K^2 = 0,07$
\item $T^2_i = 2,1$
\item $T^2_d = 1,8$
\\
\item $K^3 = 0,3$
\item $T^3_i = 1,05$
\item $T^3_d = 3,4$
\\
\item $K^4 = 0,21$
\item $T^4_i = 5$
\item $T^4_d = 2,44$
\\
\item $K^5 = 0,17$
\item $T^5_i = 3.41$
\item $T^5_d = 1,6$
\end{itemize}
Błąd regulacji wynosi $E=94,3681$. Stąd wniosek, że algorytm rozmyty PID z 5 regulatorami lokalnymi jest niepraktyczny ze względu na to, że należy dobrać aż 15 parametrów (nie licząc parametrów opisujących funkcje przynależności. Wyniki eksperymentu przedstawiają wykresy \ref{fig:5_reg_pid_y} i \ref{fig:5_reg_pid_u}.
\begin{figure}[H]
\centering
\begin{tikzpicture}
\begin{axis}[
height=0.5\textwidth,
width=0.75\textwidth,
xmin=0,xmax=1200,
xlabel={$k$},
ylabel={$y(k)$},
/pgf/number format/.cd,
use comma,
1000 sep={}
]
\addplot file {wykresy/5_reg_pid_y.txt};
\addplot file {wykresy/zad4_yzad.txt};
\legend{$y$,$y_{zad}$}
\end{axis}
\end{tikzpicture}
\caption{Wyjście obiektu dla 5 regulatorów lokalnych PID}
\label{fig:5_reg_pid_y}
\end{figure}

\begin{figure}[H]
\centering
\begin{tikzpicture}
\begin{axis}[
height=0.5\textwidth,
width=0.75\textwidth,
xmin=0,xmax=1200,
xlabel={$k$},
ylabel={$u(k)$},
/pgf/number format/.cd,
use comma,
1000 sep={}
]
\addplot file {wykresy/5_reg_pid_u.txt};
\end{axis}
\end{tikzpicture}
\caption{Sterowanie obiektu dla 5 regulatorów lokalnych PID}
\label{fig:5_reg_pid_u}
\end{figure}

\subsection{DMC}
Na początek użyliśmy nastaw $\lambda_1=10$, $\lambda_2=10$, $\lambda_3=10$, $\lambda_4=10$ i $\lambda_5=10$. Błąd regulacji wynosił $E=96,6185$. Działanie regulatora przedstawiają wykresy \ref{fig:51_reg_dmc_u} i \ref{fig:51_reg_dmc_y}.

\begin{figure}[H]
\centering
\begin{tikzpicture}
\begin{axis}[
height=0.5\textwidth,
width=0.75\textwidth,
xmin=0,xmax=1200,
xlabel={$k$},
ylabel={$y(k)$},
/pgf/number format/.cd,
use comma,
1000 sep={}
]
\addplot file {wykresy/51_reg_dmc_y.txt};
\addplot file {wykresy/zad4_yzad.txt};
\legend{$y$,$y_{zad}$}
\end{axis}
\end{tikzpicture}
\caption{Wyjście obiektu dla 5 regulatorów lokalnych DMC}
\label{fig:51_reg_dmc_y}
\end{figure}

\begin{figure}[H]
\centering
\begin{tikzpicture}
\begin{axis}[
height=0.5\textwidth,
width=0.75\textwidth,
xmin=0,xmax=1200,
xlabel={$k$},
ylabel={$u(k)$},
/pgf/number format/.cd,
use comma,
1000 sep={}
]
\addplot file {wykresy/51_reg_dmc_u.txt};
\end{axis}
\end{tikzpicture}
\caption{Sterowanie obiektu dla 5 regulatorów lokalnych DMC}
\label{fig:51_reg_dmc_u}
\end{figure}

Metodą eksperymentalną dobraliśmy wartości nastaw $\lambda_1=149,99$, $\lambda_2=24,75$, $\lambda_3=0,11$,  $\lambda_4=0,1085$ i $\lambda_5=0,1078$. Wskaźnik jakości regulacji wynosił $E=93,9931$. Działanie regulatora przedstawiają wykresy \ref{fig:52_reg_dmc_y} i \ref{fig:52_reg_dmc_u}. Porównując z działaniem regulatora pid (wykresy \ref{fig:5_reg_pid_y} i \ref{fig:5_reg_pid_u}) uznajemy, że regulator DMC działa lepiej ze względu na mniejsze oscylacje i przeregulowania, mimo większego błędu regulaji. Poprawa w stosunku do DMC z 4 regulatorami lokalnymi jest znikoma.

\begin{figure}[H]
\centering
\begin{tikzpicture}
\begin{axis}[
height=0.5\textwidth,
width=0.75\textwidth,
xmin=0,xmax=1200,
xlabel={$k$},
ylabel={$y(k)$},
/pgf/number format/.cd,
use comma,
1000 sep={}
]
\addplot file {wykresy/52_reg_dmc_y.txt};
\addplot file {wykresy/zad4_yzad.txt};
\legend{$y$,$y_{zad}$}
\end{axis}
\end{tikzpicture}
\caption{Wyjście obiektu dla 5 regulatorów lokalnych DMC}
\label{fig:52_reg_dmc_y}
\end{figure}

\begin{figure}[H]
\centering
\begin{tikzpicture}
\begin{axis}[
height=0.5\textwidth,
width=0.75\textwidth,
xmin=0,xmax=1200,
xlabel={$k$},
ylabel={$u(k)$},
/pgf/number format/.cd,
use comma,
1000 sep={}
]
\addplot file {wykresy/52_reg_dmc_u.txt};
\end{axis}
\end{tikzpicture}
\caption{Sterowanie obiektu dla 5 regulatorów lokalnych DMC}
\label{fig:52_reg_dmc_u}
\end{figure}