\section{5 regulatory lokalne}
Dla 3 regulatorów lokalnych użyliśmy sigmoidalnych funkcji przynależności, danych wzorami \ref{eq:mi5_1}, \ref{eq:mi5_2}, \ref{eq:mi5_3}, \ref{eq:mi5_4} i \ref{eq:mi5_5}. Parametry mają wartośi: $\alpha = -3$, $c^1 = -0,05$, $c^2=0,25$, $c^3=0,5$, $c^4=1.3$.
\begin{equation} \label{eq:mi5_1}
\mu_1 = 1 - \frac{1}{1+e^{\alpha * (y(k)-c^1))}}
\end{equation}
\begin{equation} \label{eq:mi5_2}
\mu_2 = \frac{1}{1+e^{\alpha * (y(k)-c^1))}} - \frac{1}{1+e^{\alpha * (y(k)-c^2))}}
\end{equation}
\begin{equation} \label{eq:mi5_3}
\mu_3 = \frac{1}{1+e^{\alpha * (y(k)-c^2))}} - \frac{1}{1+e^{\alpha * (y(k)-c^3))}}
\end{equation}
\begin{equation} \label{eq:mi5_4}
\mu_4 = \frac{1}{1+e^{\alpha * (y(k)-c^3))}} - \frac{1}{1+e^{\alpha * (y(k)-c^4))}}
\end{equation}
\begin{equation} \label{eq:mi5_5}
\mu_5 = \frac{1}{1+e^{\alpha * (y(k)-c^4))}}
\end{equation}

\begin{figure}[H]
\centering
\begin{tikzpicture}
\begin{axis}[
height=0.5\textwidth,
width=0.75\textwidth,
xmin=-0.15,xmax=4.28,
xlabel={$y(k)$},
ylabel={$\mu_1$},
/pgf/number format/.cd,
use comma,
1000 sep={}
]
\addplot file {wykresy/5_reg_sigm_1.txt};
\end{axis}
\end{tikzpicture}
\caption{Funkcja przynależności regulatora 1}
\label{fig:mi5_1}
\end{figure}

\begin{figure}[H]
\centering
\begin{tikzpicture}
\begin{axis}[
height=0.5\textwidth,
width=0.75\textwidth,
xmin=-0.15,xmax=4.28,
xlabel={$y(k)$},
ylabel={$\mu_2$},
/pgf/number format/.cd,
use comma,
1000 sep={}
]
\addplot file {wykresy/5_reg_sigm_2.txt};
\end{axis}
\end{tikzpicture}
\caption{Funkcja przynależności regulatora 2}
\label{fig:mi5_2}
\end{figure}

\begin{figure}[H]
\centering
\begin{tikzpicture}
\begin{axis}[
height=0.5\textwidth,
width=0.75\textwidth,
xmin=-0.15,xmax=4.28,
xlabel={$y(k)$},
ylabel={$\mu_1$},
/pgf/number format/.cd,
use comma,
1000 sep={}
]
\addplot file {wykresy/5_reg_sigm_3.txt};
\end{axis}
\end{tikzpicture}
\caption{Funkcja przynależności regulatora 3}
\label{fig:mi5_3}
\end{figure}

\begin{figure}[H]
\centering
\begin{tikzpicture}
\begin{axis}[
height=0.5\textwidth,
width=0.75\textwidth,
xmin=-0.15,xmax=4.28,
xlabel={$y(k)$},
ylabel={$\mu_1$},
/pgf/number format/.cd,
use comma,
1000 sep={}
]
\addplot file {wykresy/5_reg_sigm_4.txt};
\end{axis}
\end{tikzpicture}
\caption{Funkcja przynależności regulatora 4}
\label{fig:mi5_4}
\end{figure}

\begin{figure}[H]
\centering
\begin{tikzpicture}
\begin{axis}[
height=0.5\textwidth,
width=0.75\textwidth,
xmin=-0.15,xmax=4.28,
xlabel={$y(k)$},
ylabel={$\mu_1$},
/pgf/number format/.cd,
use comma,
1000 sep={}
]
\addplot file {wykresy/5_reg_sigm_5.txt};
\end{axis}
\end{tikzpicture}
\caption{Funkcja przynależności regulatora 5}
\label{fig:mi5_5}
\end{figure}

\subsection{PID}
Parametry algorytmu PID wyznaczone przez algorytm genetyczny są następujące:
\begin{itemize}
\item $K^1 = 0,163$
\item $T^1_i = 4,327$
\item $T^1_d = 1,91$
\\
\item $K^2 = 0,297$
\item $T^2_i = 9,87$
\item $T^2_d = 4,624$
\\
\item $K^3 = 0,385$
\item $T^3_i = 5,32$
\item $T^3_d = 0,383$
\\
\item $K^4 = 0,385$
\item $T^4_i = 5,32$
\item $T^4_d = 0,383$
\\
\item $K^5 = 0,385$
\item $T^5_i = 5,32$
\item $T^5_d = 0,383$
\end{itemize}

\begin{figure}[H]
\centering
\begin{tikzpicture}
\begin{axis}[
height=0.5\textwidth,
width=0.75\textwidth,
xmin=0,xmax=1200,
xlabel={$k$},
ylabel={$y(k)$},
/pgf/number format/.cd,
use comma,
1000 sep={}
]
\addplot file {wykresy/5_reg_pid_y.txt};
\addplot file {wykresy/zad4_yzad.txt};
\legend{$y$,$y_{zad}$}
\end{axis}
\end{tikzpicture}
\caption{Wyjście obiektu dla 5 regulatorów lokalnych PID}
\label{fig:5_reg_pid_y}
\end{figure}

\begin{figure}[H]
\centering
\begin{tikzpicture}
\begin{axis}[
height=0.5\textwidth,
width=0.75\textwidth,
xmin=0,xmax=1200,
xlabel={$k$},
ylabel={$u(k)$},
/pgf/number format/.cd,
use comma,
1000 sep={}
]
\addplot file {wykresy/5_reg_pid_u.txt};
\end{axis}
\end{tikzpicture}
\caption{Sterowanie obiektu dla 5 regulatorów lokalnych PID}
\label{fig:5_reg_pid_u}
\end{figure}

Wskaźnik jakości regulacji dla regulatora, którego działanie przedstawiają wykresy \ref{fig:5_reg_pid_y} i \ref{fig:5_reg_pid_u}, ma wartość $E=101,5480$, a więc równy błędowi 4 regulatorów lokalnych. Potwierdza to tezę, że tworzenie kolejnych regulatorów lokalnych nie ma sensu.
\subsection{DMC}