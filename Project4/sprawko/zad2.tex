\chapter{Badanie zachowania obiektu}
\label{sec:zad2}
\section{Odpowiedzi skokowe}
Aby lepiej poznać naturę obiektu przeprowadzone zostały ogólne badania
zachowania obiektu i jego odpowiedzi na różne skoki wartości sterującej.
Eksperyment zakładał, iż na początku obiekt będzie w punkcie pracy
($Y = 0$, $U = 0$), a następnie, w chwili $k = 7$ wykonany zostanie skok do zaplanowanej wcześniej wartości sterowania. Biorąc pod uwagę ograniczenia na wartość sterowania $U^{min} = -1$ i $U^{max} = 1$ wartość sterowania
po skoku mieściła się w owym zakresie. Wyniki eksperymentu zostały zobrazowane
na wykresie \ref{fig:skoki}.


\begin{figure}[tb]
\centering
\begin{tikzpicture}
\begin{axis}[
width=0.75\textwidth,
xmin=0,xmax=200,ymin=-0.5,ymax=4.5,
xlabel={Chwila (k)},
ylabel={Wyjście (y)},
legend pos=north east,
/pgf/number format/.cd,
use comma,
1000 sep={},
cycle list name=color
]
\addplot file {wykresy/zad2_y_1.txt};
\addplot file {wykresy/zad2_y_2.txt};
\addplot file {wykresy/zad2_y_3.txt};
\addplot file {wykresy/zad2_y_4.txt};
\addplot file {wykresy/zad2_y_5.txt};
\addplot file {wykresy/zad2_y_6.txt};
\addplot file {wykresy/zad2_y_7.txt};
\addplot file {wykresy/zad2_y_8.txt};
\addplot file {wykresy/zad2_y_9.txt};
\addplot file {wykresy/zad2_y_10.txt};
\legend{$U = -1$,$U = -0.8$,$U = -0.6$,$U = -0.4$,$U = -0.2$,$U = 0.2$,
		$U = 0.4$, $U = 0.6$, $U = 0.8$, $U = 1$}
\end{axis}
\end{tikzpicture}
\caption{Odpowiedzi skokowe dla różnych wartości sterowania.
(Wartość końcowa sterowania w legendzie)}
\label{fig:skoki}
\end{figure}

\section{Charakterystyka statyczna}
Następnie wyznaczona została charakterystyka statyczna obiektu. Znaleziona
została poprzez sprawdzenie przy jakiej wartości wyjścia obiekt stabilizuje
się dla danej wartości sterowania. Na podstawie tego sporządzony został wykres.
Wyniki zostały zamieszczone na wykresie \ref{fig:char_stat}.
\begin{figure}[H]
\centering
\begin{tikzpicture}
\begin{axis}[
width=0.75\textwidth,
xmin=-1,xmax=1,ymin=-0.5,ymax=5,
xlabel={Sterowanie (u)},
ylabel={Wyjście (y)},
/pgf/number format/.cd,
use comma,
1000 sep={}
]
\addplot file {wykresy/zad2_yu.txt};
\end{axis}
\end{tikzpicture}
\caption{Charakterystyka statyczna obiektu.}
\label{fig:char_stat}
\end{figure}

Otrzymana charakterystyka przedstawiona na wykresie \ref{fig:char_stat} jest nieliniowa. Nie można zatem określić wzmocnienia statycznego. Ponadto należy się spodziewać, że "tradycyjne" regulatory PID i DMC, przystosowane do obiektów liniowych, mogą mieć problemy z prawidłową regulacją.


\section{Charakterystyka dynamiczna}
Charakterystyka dynamiczna obiektu wyznaczana jest na podstawie czasu potrzebnego na osiągnięcie co najmniej $90\%$ wartości końcowej wyjścia. Oś odciętych stanowią wartości sterowania, zaś na osi rzędnych wykresu znajdują się chwile $k$, w których zostało osiągnięte $90\%$ wartości końcowej wyjścia. Wynik eksperymentu przedstawia wykres \ref{fig:char_dyn}.
\begin{figure}[H]
\centering
\begin{tikzpicture}
\begin{axis}[
width=0.75\textwidth,
xmin=-1,xmax=1,ymin=14,ymax=26,
xlabel={$U$},
ylabel={chwila $k$},
/pgf/number format/.cd,
use comma,
1000 sep={}
]
\addplot file {wykresy/zad2_kd.txt};
\end{axis}
\end{tikzpicture}
\caption{Charakterystyka dynamiczna obiektu.}
\label{fig:char_dyn}
\end{figure}

Jak widać na wykresie \ref{fig:char_dyn}, charakterystyka dynamiczna obiektu nie jest liniowa.