\chapter{Regulacja}
Wykorzystując wcześniej zdobyte informacje stworzyliśmy dwa regulatory dla naszego
obiektu --- regulator PID, oraz regulator DMC. Porównaliśmy je dla zadanej ścieżki
wartości zadanych. Została ona dobrana tak, aby uwzględniać zarówno jednoczesne
skoki w przeciwnych kierunkach, oraz zmianę wartości zadanej dla jednego wyjścia
przy stałej wartości zadanej drugiego wyjścia, aby zaobserwować ewentualny wpływ
wartości zadanej wyjścia na sterowanie drugim wyjściem. Przebiegi, wraz z wartościami
zadanymi ilustrują wykresy \ref{fig:reg_pid}, oraz \ref{fig:reg_dmc}. Wskaźniki
jakości regulacji dla tych dwóch przypadków wynoszą $E_{DMC} \approx 2875,4$,
oraz $E_{PID} \approx 3060,3$. Nastawy obydwu regulatorów zostały dobrane
eksperymentalnie. Dla regulatora PID wynoszą one:
\begin{equation}
  K^1 = 1,5 \qquad T^1_i = 20 \qquad T^1_d = 0.1
\end{equation}
\begin{equation}
  K^2 = 1,5 \qquad T^2_i = 20 \qquad T^2_d = 0.1
\end{equation}
Tory sterowania regulatora PID zostały ustalone $y_1$ -- $u_1$, $y_2$ -- $u_2$.
Przy innym zestawieniu regulator w ogóle nie zdawał egzaminu.
Dla regulatora DMC ustawiony został horyzont predykcji i horyzont sterowania
równy horyzontowi dynamiki. Uznaliśmy, iż chcemy osiągnąć jak najlepsze rezultaty,
podczas gdy mocy obliczeniowej mamy aż nadto. Parametr $\lambda$ ustanowiliśmy
taki sam dla obydwu sygnałów sterujących i równy $10$. Celem takiego działania
było ograniczenie zmian jednego wyjścia, przy skoku wartości zadanej
drugiego wyjścia. Zgodnie z powyższym nastawy regulatora DMC prezentują się
następująco:
\begin{equation}
  D = 150 \qquad N = 150 \qquad N_u = 150 \qquad \lambda = 10
\end{equation}

\begin{figure}[tb]
\centering
\begin{tikzpicture}
\begin{groupplot}[group style={group size=1 by 2}, width=0.9\textwidth, height=0.4\textwidth]
\nextgroupplot
[
xmin=0,xmax=1000,ymin=0,ymax=15,
xlabel={$k$},
ylabel={$y_1$},
xtick={0, 200, 400, 600, 800, 1000},
ytick={0, 5, 10, 15},
legend pos=north west,
/pgf/number format/.cd,
use comma,
1000 sep={}
]
\addplot[blue,semithick] file {wykresy/z5_pid_y1.txt};
\addplot[red,semithick] file {wykresy/z5_yzad1.txt};
\legend{$y_1$, $y^{zad}_1$}

\nextgroupplot
[
xmin=0,xmax=1000,ymin=0,ymax=15,
xlabel={$k$},
ylabel={$y_2$},
xtick={0, 200, 400, 600, 800, 1000},
ytick={0, 5, 10, 15},
legend pos=north west,
/pgf/number format/.cd,
use comma,
1000 sep={}
]
\addplot[blue,semithick] file {wykresy/z5_pid_y2.txt};
\addplot[red,semithick] file {wykresy/z5_yzad2.txt};
\legend{$y_2$, $y^{zad}_2$}

\end{groupplot}
\end{tikzpicture}
\caption{Regulacja PID.}
\label{fig:reg_pid}
\end{figure}

\begin{figure}[tb]
\centering
\begin{tikzpicture}
\begin{groupplot}[group style={group size=1 by 2}, width=0.9\textwidth, height=0.4\textwidth]
\nextgroupplot
[
xmin=0,xmax=1000,ymin=0,ymax=15,
xlabel={$k$},
ylabel={$y_1$},
xtick={0, 200, 400, 600, 800, 1000},
ytick={0, 5, 10, 15},
legend pos=north west,
/pgf/number format/.cd,
use comma,
1000 sep={}
]
\addplot[blue,semithick] file {wykresy/z5_dmc_y1.txt};
\addplot[red,semithick] file {wykresy/z5_yzad1.txt};
\legend{$y_1$, $y^{zad}_1$}

\nextgroupplot
[
xmin=0,xmax=1000,ymin=0,ymax=15,
xlabel={$k$},
ylabel={$y_2$},
xtick={0, 200, 400, 600, 800, 1000},
ytick={0, 5, 10, 15},
legend pos=north west,
/pgf/number format/.cd,
use comma,
1000 sep={}
]
\addplot[blue,semithick] file {wykresy/z5_dmc_y2.txt};
\addplot[red,semithick] file {wykresy/z5_yzad2.txt};
\legend{$y_2$, $y^{zad}_2$}

\end{groupplot}
\end{tikzpicture}
\caption{Regulacja DMC.}
\label{fig:reg_dmc}
\end{figure}
