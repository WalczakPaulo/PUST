\chapter{Skok zakłócenia}
Na koniec zbadaliśmy odporność regulatorów na niemierzalny skok zakłócenia
wyjścia obiektu. Eksperyment polegał na modyfikacji stanu obiektu w chwili
750, gdy obiekt jest ustabilizowany i wartość zadana stała na obydwu torach,
tak aby nie nakładać na siebie efektów naszego zakłócenia i bliskiej zmiany
wartości zadanej na którymkolwiek z torów. Skok zakłócenia był jednostkowy
na wyjściu $y_1$. Wyniki przedstawiają wykresy \ref{fig:skok_zakl_dmc} i \ref{fig:skok_zakl_pid}.
Widać na nich, iż obydwa regulatory natychmiast zareagowały. W obydwu przypadkach
również widać przeregulowanie na torze na którym wystąpiło zakłócenie. Wielkość
przerególowania w obydwu przypadkach jest porównywalna, jednak regulator DMC
wolniej wraca do wartości zadanej. Początkowo wydawało nam się, iż powodem tego
jest wysoka wartość parametru $\lambda$, jednak jego zmniejszenie nie daje dużej
poprawy pod tym względem. Uznaliśmy, iż taka jest charakterystyka regulatora.
Regulator PID szybciej znajduje się w okolicy wartości zadanej jednak gdy jest
już blisko dochodzi do niej bardzo powoli. Naszym zdaniem zachowuje się typowo
dla regulatora PID. Oczywiście szybki spadek sterowania, będący efektem zakłócenia,
miał też wpływ na wyjście $y_2$ mimo, iż nie było na nim żadnego dodatkowego zakłócenia.
Trajektoria tego wyjścia w takcie zakłócenia wygląda podobnie do toru z zakłóceniem.
Odejście od wartości zadanej jest odrobine mniejsze i zaczyna się odrobinę później, gdyż
nie jest bezpośrednim efektem zakłócenia, ale sterowania niwelującego zakłócenie.

\begin{figure}[tb]
\centering
\begin{tikzpicture}
\begin{groupplot}[group style={group size=1 by 2}, width=0.9\textwidth, height=0.4\textwidth]
\nextgroupplot
[
xmin=0,xmax=1000,ymin=0,ymax=15,
xlabel={$k$},
ylabel={$y_1$},
xtick={0, 200, 400, 600, 800, 1000},
ytick={0, 5, 10, 15},
legend pos=north west,
/pgf/number format/.cd,
use comma,
1000 sep={}
]
\addplot[blue,semithick] file {wykresy/z7_dmc_y1_tor1_t750_d1.txt};
\addplot[red,semithick] file {wykresy/z7_yzad1.txt};
\legend{$y_1$, $y^{zad}_1$}

\nextgroupplot
[
xmin=0,xmax=1000,ymin=0,ymax=15,
xlabel={$k$},
ylabel={$y_2$},
xtick={0, 200, 400, 600, 800, 1000},
ytick={0, 5, 10, 15},
legend pos=north west,
/pgf/number format/.cd,
use comma,
1000 sep={}
]
\addplot[blue,semithick] file {wykresy/z7_dmc_y2_tor1_t750_d1.txt};
\addplot[red,semithick] file {wykresy/z7_yzad2.txt};
\legend{$y_2$, $y^{zad}_2$}

\end{groupplot}
\end{tikzpicture}
\caption{Regulacja DMC ze skokiem zakłócenia w chwili $750$.}
\label{fig:skok_zakl_dmc}
\end{figure}

\begin{figure}[tb]
\centering
\begin{tikzpicture}
\begin{groupplot}[group style={group size=1 by 2}, width=0.9\textwidth, height=0.4\textwidth]
\nextgroupplot
[
xmin=0,xmax=1000,ymin=0,ymax=15,
xlabel={$k$},
ylabel={$y_1$},
xtick={0, 200, 400, 600, 800, 1000},
ytick={0, 5, 10, 15},
legend pos=north west,
/pgf/number format/.cd,
use comma,
1000 sep={}
]
\addplot[blue,semithick] file {wykresy/z7_pid_y1_tor1_t750_d1.txt};
\addplot[red,semithick] file {wykresy/z7_yzad1.txt};
\legend{$y_1$, $y^{zad}_1$}

\nextgroupplot
[
xmin=0,xmax=1000,ymin=0,ymax=15,
xlabel={$k$},
ylabel={$y_2$},
xtick={0, 200, 400, 600, 800, 1000},
ytick={0, 5, 10, 15},
legend pos=north west,
/pgf/number format/.cd,
use comma,
1000 sep={}
]
\addplot[blue,semithick] file {wykresy/z7_pid_y2_tor1_t750_d1.txt};
\addplot[red,semithick] file {wykresy/z7_yzad2.txt};
\legend{$y_2$, $y^{zad}_2$}

\end{groupplot}
\end{tikzpicture}
\caption{Regulacja PID ze skokiem zakłocenia w chwili $750$.}
\label{fig:skok_zakl_pid}
\end{figure}
