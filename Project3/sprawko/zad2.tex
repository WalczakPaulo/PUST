\chapter{Charakterystyka obiektu}
Następnym punktem naszych badań było zbadanie zachowania wyjść obiektu przy różnych
wartościach wejściowych. Eksperyment był przeprowadzany poprzez skok wartości
sterującej przy drugiej wartości sterowania utrzymywanej w punkcie pracy.
W ten sposób otrzymujemy cztery tory opisujące obiekt. Zachowanie dwóch wyjść
na każdy z dwóch sygnałów wejściowych. Eksperymenty były przeprowadzane zaczynając
z punktu pracy. Zbadane zostały skoki sygnałów steruących do wartości $1$, $3$,
oraz $5$. Wyniki przedstawiają wykresy \ref{fig:skoki_u1}, oraz \ref{fig:skoki_u2}.
Wszystkie te wykresy zostały wyskalowane na takie same wartości, tak aby
dobrze pokazywały zależności między wzmocnieniami sygnałów. Wzmocnienie statyczne
jest w przybliżeniu liniowe. Zostało one opisane dalej zmienną $k^{ij}$, gdzie $i$ oznacza
numer wyjścia, natomiast $j$ numer sygnału sterującego. Przy takim oznaczeniu:
$k^{11} \approx 2,00$, $k^{12} \approx 0,75$ $k^{21} \approx 0,90$, $k^{22} \approx 0,99$.


\begin{figure}[tb]
\centering
\begin{tikzpicture}
\begin{groupplot}[group style={group size=1 by 2}, width=0.9\textwidth, height=0.6\textwidth]
\nextgroupplot
[
xmin=0,xmax=200,ymin=0,ymax=12,
xlabel={$k$},
ylabel={$y_1$},
xtick={0, 50, 100, 150, 200},
ytick={0, 2, 4, 6, 8, 10, 12},
legend pos=north west,
/pgf/number format/.cd,
use comma,
1000 sep={}
]
\addplot[blue,semithick] file {wykresy/z2_y1_u1_1.txt};
\addplot[red,semithick] file {wykresy/z2_y1_u1_3.txt};
\addplot[green,semithick] file {wykresy/z2_y1_u1_5.txt};
\legend{$u_1 = 1$, $u_1 = 3$, $u_1 = 5$}

\nextgroupplot
[
xmin=0,xmax=200,ymin=0,ymax=12,
xlabel={$k$},
ylabel={$y_2$},
xtick={0, 50, 100, 150, 200},
ytick={0, 2, 4, 6, 8, 10, 12},
legend pos=north west,
/pgf/number format/.cd,
use comma,
1000 sep={}
]
\addplot[blue,semithick] file {wykresy/z2_y2_u1_1.txt};
\addplot[red,semithick] file {wykresy/z2_y2_u1_3.txt};
\addplot[green,semithick] file {wykresy/z2_y2_u1_5.txt};
\legend{$u_1 = 1$, $u_1 = 3$, $u_1 = 5$}

\end{groupplot}
\end{tikzpicture}
\caption{Odpowiedzi na skoki sygnału $u_1$}
\label{fig:skoki_u1}
\end{figure}

\begin{figure}[tb]
\centering
\begin{tikzpicture}
\begin{groupplot}[group style={group size=1 by 2}, width=0.9\textwidth, height=0.6\textwidth]
\nextgroupplot
[
xmin=0,xmax=200,ymin=0,ymax=12,
xlabel={$k$},
ylabel={$y_1$},
xtick={0, 50, 100, 150, 200},
ytick={0, 2, 4, 6, 8, 10, 12},
legend pos=north west,
/pgf/number format/.cd,
use comma,
1000 sep={}
]
\addplot[blue,semithick] file {wykresy/z2_y1_u2_1.txt};
\addplot[red,semithick] file {wykresy/z2_y1_u2_3.txt};
\addplot[green,semithick] file {wykresy/z2_y1_u2_5.txt};
\legend{$u_2 = 1$, $u_2 = 3$, $u_2 = 5$}

\nextgroupplot
[
xmin=0,xmax=200,ymin=0,ymax=12,
xlabel={$k$},
ylabel={$y_2$},
xtick={0, 50, 100, 150, 200},
ytick={0, 2, 4, 6, 8, 10, 12},
legend pos=north west,
/pgf/number format/.cd,
use comma,
1000 sep={}
]
\addplot[blue,semithick] file {wykresy/z2_y2_u2_1.txt};
\addplot[red,semithick] file {wykresy/z2_y2_u2_3.txt};
\addplot[green,semithick] file {wykresy/z2_y2_u2_5.txt};
\legend{$u_2 = 1$, $u_2 = 3$, $u_2 = 5$}

\end{groupplot}
\end{tikzpicture}
\caption{Odpowiedzi na skoki sygnału $u_2$}
\label{fig:skoki_u2}
\end{figure}
