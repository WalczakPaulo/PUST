\chapter{Zakłócenia}
W kolejnym kroku została zbadana odporność obydwu algorytmów na zakłócenia pomiarowe.
Zostały one zaimplementowane poprzez proste dodanie do wartości wyjścia podawanej
do algorytmu losowej wartości z zakresu $<-n, n>$. Z każdym krokiem produkowana była
nowa wartość losowa z zakresu. Trajektoria wartości zadanej była taka sama
jak poprzednio, tak aby można było rozróżnić, czy dane zachowanie jest efektem
zakłócenia, czy występowało również bez niego. Eksperymenty zostały przeprowadzone
dla różnych wartości $n$. Za każdym razem jednak szum był dokładnie taki sam dla obydwu regulatorów.
Wykresy \ref{fig:szum_dmc}, oraz \ref{fig:szum_pid} obrazują zachowanie w przypadku $n = 2$.
Przy innych wartościach szum po prostu się skalował. Na wykresach widać, iż
wpływ szumu na regulator DMC jest mniejszy, jednak wynik to jedynie z dobranej
wartości parametru $\lambda$. Regulator DMC mniej radykalnie zmienia sterowanie,
co w tym przypadku wyszło mu na dobre i dało stabilniejsze wyjście.
Sądzę, iż przy regulatoreze PID można osiągnąć podobny efekt ograniczając
zmianę sygnału sterującego.

\begin{figure}[tb]
\centering
\begin{tikzpicture}
\begin{groupplot}[group style={group size=1 by 2}, width=0.9\textwidth, height=0.4\textwidth]
\nextgroupplot
[
xmin=0,xmax=1000,ymin=0,ymax=15,
xlabel={$k$},
ylabel={$y_1$},
xtick={0, 200, 400, 600, 800, 1000},
ytick={0, 5, 10, 15},
legend pos=north west,
/pgf/number format/.cd,
use comma,
1000 sep={}
]
\addplot[blue,semithick] file {wykresy/z6_dmc_y1_z_2.txt};
\addplot[red,semithick] file {wykresy/z5_yzad1.txt};
\legend{$y_1$, $y^{zad}_1$}

\nextgroupplot
[
xmin=0,xmax=1000,ymin=0,ymax=15,
xlabel={$k$},
ylabel={$y_2$},
xtick={0, 200, 400, 600, 800, 1000},
ytick={0, 5, 10, 15},
legend pos=north west,
/pgf/number format/.cd,
use comma,
1000 sep={}
]
\addplot[blue,semithick] file {wykresy/z6_dmc_y2_z_2.txt};
\addplot[red,semithick] file {wykresy/z5_yzad2.txt};
\legend{$y_2$, $y^{zad}_2$}

\end{groupplot}
\end{tikzpicture}
\caption{Regulacja DMC z szumem pomiarowym z zakresu $<-2, 2>$.}
\label{fig:szum_dmc}
\end{figure}

\begin{figure}[tb]
\centering
\begin{tikzpicture}
\begin{groupplot}[group style={group size=1 by 2}, width=0.9\textwidth, height=0.4\textwidth]
\nextgroupplot
[
xmin=0,xmax=1000,ymin=0,ymax=15,
xlabel={$k$},
ylabel={$y_1$},
xtick={0, 200, 400, 600, 800, 1000},
ytick={0, 5, 10, 15},
legend pos=north west,
/pgf/number format/.cd,
use comma,
1000 sep={}
]
\addplot[blue,semithick] file {wykresy/z6_pid_y1_z_2.txt};
\addplot[red,semithick] file {wykresy/z5_yzad1.txt};
\legend{$y_1$, $y^{zad}_1$}

\nextgroupplot
[
xmin=0,xmax=1000,ymin=0,ymax=15,
xlabel={$k$},
ylabel={$y_2$},
xtick={0, 200, 400, 600, 800, 1000},
ytick={0, 5, 10, 15},
legend pos=north west,
/pgf/number format/.cd,
use comma,
1000 sep={}
]
\addplot[blue,semithick] file {wykresy/z6_pid_y2_z_2.txt};
\addplot[red,semithick] file {wykresy/z5_yzad2.txt};
\legend{$y_2$, $y^{zad}_2$}

\end{groupplot}
\end{tikzpicture}
\caption{Regulacja PID z szumem pomiarowym z zakresu $<-2, 2>$.}
\label{fig:szum_pid}
\end{figure}
