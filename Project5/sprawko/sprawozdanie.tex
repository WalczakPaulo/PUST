\documentclass[a4paper,titlepage,11pt,twosides,floatssmall]{mwrep}
\usepackage[left=2.5cm,right=2.5cm,top=2.5cm,bottom=2.5cm]{geometry}
\usepackage[OT1]{fontenc}
\usepackage{polski}
\usepackage{amsmath}
\usepackage{xr}
\usepackage{amsfonts}
\usepackage{amssymb}
\usepackage{graphicx}
\usepackage{url}
\usepackage{float}
\usepackage[section]{placeins}
\usepackage{tikz}
\usetikzlibrary{arrows,calc,decorations.markings,math,arrows.meta}
\usepackage{rotating}
\usepackage[percent]{overpic}
\usepackage[utf8]{inputenc}
\usepackage{xcolor}
\usepackage{pgfplots}
\usetikzlibrary{pgfplots.groupplots}
\usepackage{listings}
\usepackage{matlab-prettifier}
\usepackage{siunitx}
\usepackage{bm}
\definecolor{szary}{rgb}{0.95,0.95,0.95}
\sisetup{detect-weight,exponent-product=\cdot,output-decimal-marker={,},per-mode=symbol,binary-units=true,range-phrase={-},range-units=single}

%konfiguracje pakietu listings
\lstset{
	backgroundcolor=\color{szary},
	frame=single,
	breaklines=true,
}
\lstdefinestyle{customlatex}{
	basicstyle=\footnotesize\ttfamily,
	%basicstyle=\small\ttfamily,
}
\lstdefinestyle{customc}{
	breaklines=true,
	frame=tb,
	language=C,
	xleftmargin=0pt,
	showstringspaces=false,
	basicstyle=\small\ttfamily,
	keywordstyle=\bfseries\color{green!40!black},
	commentstyle=\itshape\color\usepackage{float}{purple!40!black},
	identifierstyle=\color{blue},
	stringstyle=\color{orange},
}
\lstdefinestyle{custommatlab}{
	captionpos=t,
	breaklines=true,
	frame=tb,
	xleftmargin=0pt,
	language=matlab,
	showstringspaces=false,
	%basicstyle=\footnotesize\ttfamily,
	basicstyle=\scriptsize\ttfamily,
	keywordstyle=\bfseries\color{green!40!black},
	commentstyle=\itshape\color{purple!40!black},
	identifierstyle=\color{blue},
	stringstyle=\color{orange},
}

%wymiar tekstu (bez �ywej paginy)
\textwidth 160mm \textheight 247mm

%ustawienia pakietu pgfplots
\pgfplotsset{
tick label style={font=\scriptsize},
label style={font=\small},
legend style={font=\small},
title style={font=\small},
every axis plot post/.append style={semithick,mark=none}
}

\def\figurename{Rys.}
\def\tablename{Tab.}

%konfiguracja liczby p�ywaj�cych element�w
\setcounter{topnumber}{0}%2
\setcounter{bottomnumber}{3}%1
\setcounter{totalnumber}{5}%3
\renewcommand{\textfraction}{0.01}%0.2
\renewcommand{\topfraction}{0.95}%0.7
\renewcommand{\bottomfraction}{0.95}%0.3
\renewcommand{\floatpagefraction}{0.35}%0.5

\begin{document}
\frenchspacing
\pagestyle{uheadings}

%strona tytu�owa
\title{\bf Sprawozdanie z projektu nr 5, zadanie nr 5\vskip 0.1cm}
\author{Kamil Gabryjelski, Paweł Rybak, Paweł Walczak}
\date{2017}

\makeatletter
\renewcommand{\maketitle}{\begin{titlepage}
\begin{center}{\LARGE {\bf
Wydział Elektroniki i Technik Informacyjnych}}\\
\vspace{0.4cm}
{\LARGE {\bf Politechnika Warszawska}}\\
\vspace{0.3cm}
\end{center}
\vspace{5cm}
\begin{center}
{\bf \LARGE Projektowanie układów sterowania\\ (projekt grupowy) \vskip 0.1cm}
\end{center}
\vspace{1cm}
\begin{center}
{\bf \LARGE \@title}
\end{center}
\vspace{2cm}
\begin{center}
{\bf \Large \@author \par}
\end{center}
\vspace*{\stretch{6}}
\begin{center}
\bf{\large{Warszawa, \@date\vskip 0.1cm}}
\end{center}
\end{titlepage}
}
\makeatother

\maketitle
\tableofcontents
\chapter{Opis obiektu}
Badany obiekt jest obiektem o dwóch sygnałach wejściowych ($u_1$, $u_2$), oraz
dwóch sygnałach wyjściowych ($y_1$, $y_2$). Obiekt jest obiektem dyskretnym,
a jego okres próbkowania wynosi $0,5$s. Punktem pracy naszego obiektu, będzie
zerowa wartość obydwu wejść. W takim przypadku obiekt stabilizuje się przy
zerowej wartości wyjść.

\chapter{Odpowiedzi skokowe}
\section{Tor sterowania}
W tym punkcie badane są odpowiedzi skokowe obiektu na różne wartości skoku sygnału sterowania. Założono, że w chwili początkowej obiekt znajduje się w punkcie pracy. W chwili $k=9$ wykonywany jest sterowania do zadanej wartości. Sygnał zakłócenia ma wartość $Z=0$. Wyniki badań przedstawia wykres \ref{fig:z2_yu}.
\begin{figure}[!htb]
	\centering
	\begin{tikzpicture}
	\begin{axis}[
	width=0.9\textwidth,
	width=0.9\textwidth,
	xmin=0,xmax=200,ymin=-3,ymax=3,
	xlabel={$k$},
	ylabel={$Y(k)$},
	xtick={0,50,100,150,200},
	ytick={-3,-2,-1,0,1,2,3},
	/pgf/number format/.cd,
	use comma,
	1000 sep={}
	]
	\addplot[blue,semithick] file {wykresy/zad2_yu_1.txt};
	\addplot[brown,semithick] file {wykresy/zad2_yu_2.txt};
	\addplot[cyan,semithick] file {wykresy/zad2_yu_3.txt};
	\addplot[green,semithick] file {wykresy/zad2_yu_4.txt};
	\addplot[lime,semithick] file {wykresy/zad2_yu_5.txt};
	\addplot[magenta,semithick] file {wykresy/zad2_yu_6.txt};
	\addplot[orange,semithick] file {wykresy/zad2_yu_7.txt};
	\addplot[pink,semithick] file {wykresy/zad2_yu_8.txt};
	\legend{$U=-\num{2}$,$U=-\num{1,5}$,$U=-\num{1}$,$U=-\num{0,5}$,$U=\num{0,5}$,$U=\num{1}$,$U=\num{1,5}$,$U=\num{2}$}
	\end{axis}
	\end{tikzpicture}
	\caption{Odpowiedź $Y(k)$ dla skoków sterowania $U$}
\label{fig:z2_yu}
\end{figure}

\section{Charakterystyka statyczna toru sterowania}
Charakterystyka statyczna toru sterowania wyznaczona została poprzez sprawdzenie, na jakich wartościach stabilizuje się wyjście obiektu dla różnych wartości sygnału $U$. Liniowości charakterystki statycznej dowodzi wykres \ref{fig:z2_y_stat_u}, który (w przybliżeniu) jest liniowy.

Wartość wzmocnienia statycznego można wyznaczyć normalizując odpowiedź skokową. Wynosi ona $s_u=1,1022$.

\begin{figure}[!htb]
	\centering
	\begin{tikzpicture}
	\begin{axis}[
	width=0.9\textwidth,
	width=0.9\textwidth,
	xmin=-1.5,xmax=1.5,ymin=-2,ymax=2,
	xlabel={$U$},
	ylabel={$Y(U)$},
	xtick={-1.5,-1,-0.5,0,0.5,1,1.5},
	ytick={-2,-1.5,-1,-0.5,0,0.5,1,1.5,2},
	/pgf/number format/.cd,
	use comma,
	1000 sep={}
	]
	\addplot[blue,semithick] file {wykresy/zad2_y_stat_u.txt};
	\end{axis}
	\end{tikzpicture}
	\caption{Charakterystyka statyczna $Y(U)$}
\label{fig:z2_y_stat_u}
\end{figure}

\section{Tor zakłócenia}
W tym punkcie badane są odpowiedzi skokowe obiektu na różne wartości skoku sygnału zakłócenia. Założono, że w chwili początkowej obiekt znajduje się w punkcie pracy. W chwili $k=9$ wykonywany jest sterowania do zadanej wartości. Sygnał sterowania ma wartość $U=0$. Wyniki badań przedstawia wykres \ref{fig:z2_yz}.
\begin{figure}[!htb]
	\centering
	\begin{tikzpicture}
	\begin{axis}[
	width=0.9\textwidth,
	width=0.9\textwidth,
	xmin=0,xmax=200,ymin=-1.5,ymax=1.5,
	xlabel={$k$},
	ylabel={$Y(k)$},
	xtick={0,50,100,150,200},
	ytick={-1.5,-1,-0.5,0,0.5,1,1.5},
	/pgf/number format/.cd,
	use comma,
	1000 sep={}
	]
	\addplot[blue,semithick] file {wykresy/zad2_yz_1.txt};
	\addplot[brown,semithick] file {wykresy/zad2_yz_2.txt};
	\addplot[cyan,semithick] file {wykresy/zad2_yz_3.txt};
	\addplot[green,semithick] file {wykresy/zad2_yz_4.txt};
	\addplot[lime,semithick] file {wykresy/zad2_yz_5.txt};
	\addplot[magenta,semithick] file {wykresy/zad2_yz_6.txt};
	\addplot[orange,semithick] file {wykresy/zad2_yz_7.txt};
	\addplot[pink,semithick] file {wykresy/zad2_yz_8.txt};
	\legend{$Z=-\num{2}$,$Z=-\num{1,5}$,$Z=-\num{1}$,$Z=-\num{0,5}$,$Z=\num{0,5}$,$Z=\num{1}$,$Z=\num{1,5}$,$Z=\num{2}$}
	\end{axis}
	\end{tikzpicture}
	\caption{Odpowiedź $Y(k)$ dla skoków zakłócenia $Z$}
\label{fig:z2_yz}
\end{figure}

\section{Charakterystyka statyczna toru zakłócenia}
Charakterystyka statyczna toru zakłócenia wyznaczona została poprzez sprawdzenie, na jakich wartościach stabilizuje się wyjście obiektu dla różnych wartości sygnału $Z$. Liniowości charakterystki statycznej dowodzi wykres \ref{fig:z2_y_stat_z}, który (w przybliżeniu) jest liniowy.

Wartość wzmocnienia statycznego można wyznaczyć normalizując odpowiedź skokową. Wynosi ona $s_z=0,501$.

\begin{figure}[!htb]
	\centering
	\begin{tikzpicture}
	\begin{axis}[
	width=0.9\textwidth,
	width=0.9\textwidth,
	xmin=-1.5,xmax=1.5,ymin=-1.5,ymax=1.5,
	xlabel={$U$},
	ylabel={$Y(Z)$},
	xtick={-1.5,-1,-0.5,0,0.5,1,1.5},
	ytick={-1.5,-1,-0.5,0,0.5,1,1.5},
	/pgf/number format/.cd,
	use comma,
	1000 sep={}
	]
	\addplot[blue,semithick] file {wykresy/zad2_y_stat_z.txt};
	\end{axis}
	\end{tikzpicture}
	\caption{Charakterystyka statyczna $Y(Z)$}
\label{fig:z2_y_stat_z}
\end{figure}
\chapter{PID}
Na początku strojenia wyznaczona została macierz wzmocnień, zawierająca
wzmocnienie każdego z wyjść w zależności od wejścia. Macierz ta jest następująca:
\begin{equation}
  \bm{K} =
  \begin{bmatrix}
    \num{1.9500}  &  \num{1.5000}  &  \num{1.2500} \\
    \num{1.5500}  &  \num{1.2000}  &  \num{0.1500} \\
    \num{0.8500}  &  \num{0.9000}  &  \num{0.3000} \\
    \num{0.1000} &   \num{0.5500}  &  \num{1.2000}
  \end{bmatrix}
\end{equation}
Następnie otrzymujemy z tego cztery macierze $\bm{K}_i$. Każda z nich powstaje poprzez
usunięcie $i$-tego wiersza z macierzy $\bm{K}$. Macierze te są następujące:
\begin{equation}
  \bm{K}_1 =
  \begin{bmatrix}
    \num{1.5500}   & \num{1.2000} &   \num{0.1500} \\
        \num{0.8500}   & \num{0.9000} &   \num{0.3000} \\
        \num{0.1000}   & \num{0.5500}  &  \num{1.2000}
  \end{bmatrix}
\end{equation}
\begin{equation}
  \bm{K}_2 =
  \begin{bmatrix}
    \num{1.9500} &   \num{1.5000} &   \num{1.2500} \\
       \num{0.8500}   & \num{0.9000} &   \num{0.3000} \\
       \num{0.1000}   & \num{0.5500} &   \num{1.2000}
  \end{bmatrix}
\end{equation}
\begin{equation}
  \bm{K}_3 =
  \begin{bmatrix}
    \num{1.9500} &   \num{1.5000} &   \num{1.2500} \\
       \num{1.5500}   & \num{1.2000} &   \num{0.1500} \\
       \num{0.1000}   & \num{0.5500}  &  \num{1.2000}
  \end{bmatrix}
\end{equation}
\begin{equation}
  \bm{K}_4 =
  \begin{bmatrix}
    \num{1.9500}  &  \num{1.5000}  &  \num{1.2500} \\
    \num{1.5500}  &  \num{1.2000}  &  \num{0.1500} \\
    \num{0.8500}  &  \num{0.9000}  &  \num{0.3000}
  \end{bmatrix}
\end{equation}
Następnie obliczane są wskaźniki uwarunkowania każdej z czterech macierzy:
\begin{align}
  \text{cond}(\bm{K}_1) &= \num{23.8517} \\
  \text{cond}(\bm{K}_2) &=  \num{12.4174} \\
  \text{cond}(\bm{K}_3) &= \num{17.5922} \\
  \text{cond}(\bm{K}_4) &= \num{20.1116}
\end{align}
Nastęnie wylicza się macierz $KK_i = K_i * K_i^{-1}$. Z tej macierzy
wybiera się tory sterowania, poprzez wybranie najmniejszych wartości dodatnich
z macierzy $KK_i$, tak aby wybrana była tylko jedna wartość w danym wierszu i kolumnie.
Wartości ujemne są wykluczone. Tory sterowania są wyznaczane poprzez numer kolumny
i wiersza wybranych wartości. Numer kolumny odpowiada wyjściu, a numer wiersza sterowaniu.
Teoretycznie najlepszy wynik będzie osiągnięty dla
macierzy $KK_i$, dla której wskaźnik uwarunkowania $K_i$ był najmniejszy, czyli
w naszym wypadku $K_2$, ale mimo to sprawdzimy wszystkie cztery opcje. Macierze
wychodzą następujące:



\begin{equation}
  \bm{KK}_1 =
  \begin{bmatrix}
   \num{ 4.9438} &  \num{-4.1412}  & \num{ 0.1974} \\
   \num{-4.0222}  & \num{ 5.7882}  & \num{-0.7660} \\
   \num{ 0.0784} &  \num{-0.6471} &  \num{ 1.5686}
  \end{bmatrix}
\end{equation}
\begin{equation}
  \bm{KK}_2 =
  \begin{bmatrix}
   \num{ 2.3138} &  \num{-1.9258}  & \num{ 0.6119} \\
   \num{-1.2263}   & \num{ 2.5852}  & \num{-0.3589} \\
   \num{-0.0875}   &\num{ 0.3406} &  \num{ 0.7470}
  \end{bmatrix}
\end{equation}
\begin{equation}
  \bm{KK}_3 =
  \begin{bmatrix}
   \num{ 3.3287} &  \num{-3.4800}  & \num{ 1.1514} \\
   \num{-2.1683}  & \num{ 3.3423} &  \num{-0.1740} \\
   \num{-0.1603}  & \num{ 1.1377}  & \num{ 0.0226}
  \end{bmatrix}
\end{equation}
\begin{equation}
  \bm{KK}_4 =
  \begin{bmatrix}
   \num{ 1.0935}  & \num{-1.2617}  & \num{ 1.1682} \\
   \num{ 2.6075} &  \num{-1.4280}  & \num{-0.1794} \\
   \num{-2.7009}  & \num{ 3.6897} &  \num{ 0.0112}
  \end{bmatrix}
\end{equation}
Stąd wybieramy cztery opcje torów sterowania. Dla $\bm{KK}_1$:
\begin{itemize}
  \item $y_1$ -- $u_4$
 \item $y_2$ -- $u_3$
 \item $y_3$ -- $u_2$
\end{itemize}
 Dla $\bm{KK}_2$:
\begin{itemize}
  \item $y_1$ -- $u_1$
 \item $y_2$ -- $u_3$
 \item $y_3$ -- $u_4$
\end{itemize}
Zgodnie z wyznaczonymi wskaźnikami uwarunkowania te tory powinny być najlepsze. Dla $\bm{KK}_3$:
\begin{itemize}
  \item $y_1$ -- $u_1$
 \item $y_2$ -- $u_2$
 \item $y_3$ -- $u_4$
\end{itemize}
Dla $\bm{KK}_4$:
\begin{itemize}
  \item $y_1$ -- $u_2$
 \item $y_2$ -- $u_3$
 \item $y_3$ -- $u_1$
\end{itemize}
Mając teoretycznie najlepsze tory sterowania przystąpiliśmy do dobierania nastaw
dla regulatorów. Nasza taktyka polegała na wyłączeniu wszelkich regulatorów, a
następnie znalezieniu wartości wzmocnienia pierwszego regulatora, dla którego
oscylacje są niegasnące. Mając tą wartość wzmocnienia dzielona była ona przez
dwa i dołączany był regulator drugi. Znów szukaliśmy wartości oscylacji niegasnących
i po znalezieniu dzieliliśmy wzmocnienie drugiego regulatora na dwa. Następnie
dołączaliśmy trzeci regulator i postępowaliśmy tak samo. Następnie dobieraliśmy
wartości całkowania, metodą prób i błędów, a na końcu tak samo dobieraliśmy
wartości członów różniczkujących dla regulatorów. Zaskakująco metoda ta okazała się
przynosić zadowalające rezultaty. Dla toru otrzymanego na podstawie macierzy $\bm{KK}_1$
ta metoda dała nastawy:
\begin{equation}
  K_1 = \num{1.3170} \qquad T_{i1} = 5, \qquad T_{d1} = 0 \nonumber
\end{equation}
\begin{equation}
  K_2 = \num{14.8350} \qquad T_{i2} = 10, \qquad T_{d2} = 0
\end{equation}
\begin{equation}
  K_3 = \num{6.2700} \qquad T_{i3} = 8, \qquad T_{d3} = 0 \nonumber
\end{equation}
Wskaźnik jakości regulacji dla takich nastaw wynosił $E_1 = \num{301.8303}$.
Co ciekawe włączenie różniczki nie dawało poprawy wskaźnika jakości, więc z niej
zrezygnowaliśmy.
Wyniki działania takiego regulatora dla tych nastaw przedstawiają wykresy \ref{fig:z3_pid1_y1},
\ref{fig:z3_pid1_y2} oraz \ref{fig:z3_pid1_y3}.

Następnie sprawdzony został tor otrzymany na podstawie macierzy $\bm{KK}_2$.
Po kilku eksperymentach otrzymaliśmy następujące nastawy:
\begin{equation}
  K_1 = \num{5.1325} \qquad T_{i1} = 9, \qquad T_{d1} = 0 \nonumber
\end{equation}
\begin{equation}
  K_2 = \num{3.1150} \qquad T_{i2} = 10, \qquad T_{d2} = 0
\end{equation}
\begin{equation}
  K_3 = \num{7.2400} \qquad T_{i3} = 10, \qquad T_{d3} = 0 \nonumber
\end{equation}
Wskaźnik jakości regulacji dla takich nastaw wynosił $E_2 = \num{134.8370}$, czyli
lepszy niż ten dla pierwszego
toru sterowania. Podobnie jak wcześniej, dodanie różniczkowania nie poprawiało
wyników w sensie wskaźnika jakości, więc z niego zrezygnowaliśmy. Działanie
tych regulatoróœ przedstawiają wykresy \ref{fig:z3_pid2_y1},
\ref{fig:z3_pid2_y2} oraz \ref{fig:z3_pid2_y3}. Dla ułatwienia porównania zastosowano
tą samą skalę co w przypadku poprzedniego zestawu regulatorów. 

Dla torów wynikających z macierzy $\bm{KK}_3$ otrzymaliśmy nastawy:
\begin{equation}
  K_1 = \num{5.1320} \qquad T_{i1} = 10, \qquad T_{d1} = 0 \nonumber
\end{equation}
\begin{equation}
  K_2 = \num{0.6130} \qquad T_{i2} = 8, \qquad T_{d2} = 0
\end{equation}
\begin{equation}
  K_3 = \num{7.2710} \qquad T_{i3} = 9, \qquad T_{d3} = 0 \nonumber
\end{equation}
oraz wskaźnik jakości $E_3 = \num{141.6150}$. Jest to wynik niewiele gorszy, niż
w przypadku macierzy $\bm{KK}_2$, która dała najlepsze rezultaty.
Działanie takich regulatorów przedstawiają wykresy \ref{fig:z3_pid4_y1},
\ref{fig:z3_pid4_y2} oraz \ref{fig:z3_pid4_y3}.

Dla macierzy $\bm{KK}_4$ otrzymane nastawy
były nasępujące:
\begin{equation}
  K_1 = \num{1.3170} \qquad T_{i1} = 4, \qquad T_{d1} = 0 \nonumber
\end{equation}
\begin{equation}
  K_2 = \num{14.8350} \qquad T_{i2} = 9, \qquad T_{d2} = 0
\end{equation}
\begin{equation}
  K_3 = \num{5.6000} \qquad T_{i3} = 6, \qquad T_{d3} = 0 \nonumber
\end{equation}
Wskaźnik jakości wynosił $\num{565.3247}$, czyli był najgorszy ze wszystkich.
Znów, dodanie różniczkowania jedynie pogarszało wyniki. Działanie tych
regulatorów ukazują wykresy \ref{fig:z3_pid3_y1},
\ref{fig:z3_pid3_y2} oraz \ref{fig:z3_pid3_y3}.

%%%%%%%%%%%%%%%%%%%%%%%%%%%%%%%%%%%%%%%%%%%%%%%%%%%%%%%%%%%%%%%%%%%%%%%%%%%%%%%%%%%%%%%%%%%%
\begin{figure}[b]
\centering
\begin{tikzpicture}
\begin{axis}[
width=0.75\textwidth,
height = 0.4\textwidth,
xmin=0,xmax=1100,ymin=-15,ymax=15,
xlabel={Numer próbki},
ylabel={Wyjście},
xtick={0, 200, 400, 600, 800, 1000},
ytick={-15, -10, -5, 0, 5, 10, 15},
legend pos=north east,
/pgf/number format/.cd,
use comma,
1000 sep={}
]

\addplot[blue,semithick] file {wykresy/z3_yzad.txt};
\addplot[red,semithick] file {wykresy/z3_pid1_y1.txt};
\legend{Wartość zadana, Wyjście $y_1$}

\end{axis}
\end{tikzpicture}
\caption{Trajektoria wyjścia $y_1$, dla pierwszego zestawu regulatorów PID}
\label{fig:z3_pid1_y1}
\end{figure}
%%%%%%%%%%%%%%%%%%%%%%%%%%%%%%%%%%%%%%%%%%%%%%%%%%%%%%%%%%%%%%%%%%%%%%%%%%%%%%%%%%%%%%%%%%%%

%%%%%%%%%%%%%%%%%%%%%%%%%%%%%%%%%%%%%%%%%%%%%%%%%%%%%%%%%%%%%%%%%%%%%%%%%%%%%%%%%%%%%%%%%%%%
\begin{figure}[b]
\centering
\begin{tikzpicture}
\begin{axis}[
width=0.75\textwidth,
height = 0.4\textwidth,
xmin=0,xmax=1100,ymin=-5,ymax=5,
xlabel={Numer próbki},
ylabel={Wyjście},
xtick={0, 200, 400, 600, 800, 1000},
ytick={-5, -4, -2, 0, 2, 4, 5},
legend pos=north east,
/pgf/number format/.cd,
use comma,
1000 sep={}
]

\addplot[blue,semithick] file {wykresy/z3_yzad.txt};
\addplot[red,semithick] file {wykresy/z3_pid1_y2.txt};
\legend{Wartość zadana, Wyjście $y_2$}

\end{axis}
\end{tikzpicture}
\caption{Trajektoria wyjścia $y_2$, dla pierwszego zestawu regulatorów PID}
\label{fig:z3_pid1_y2}
\end{figure}
%%%%%%%%%%%%%%%%%%%%%%%%%%%%%%%%%%%%%%%%%%%%%%%%%%%%%%%%%%%%%%%%%%%%%%%%%%%%%%%%%%%%%%%%%%%%

%%%%%%%%%%%%%%%%%%%%%%%%%%%%%%%%%%%%%%%%%%%%%%%%%%%%%%%%%%%%%%%%%%%%%%%%%%%%%%%%%%%%%%%%%%%%
\begin{figure}[b]
\centering
\begin{tikzpicture}
\begin{axis}[
width=0.75\textwidth,
height = 0.4\textwidth,
xmin=0,xmax=1100,ymin=-2.5,ymax=2.5,
xlabel={Numer próbki},
ylabel={Wyjście},
xtick={0, 200, 400, 600, 800, 1000},
ytick={-2.5, -2, -1.5, -1, -.5, 0, .5, 1, 1.5, 2, 2.5},
legend pos=north east,
/pgf/number format/.cd,
use comma,
1000 sep={}
]

\addplot[blue,semithick] file {wykresy/z3_yzad.txt};
\addplot[red,semithick] file {wykresy/z3_pid1_y3.txt};
\legend{Wartość zadana, Wyjście $y_3$}

\end{axis}
\end{tikzpicture}
\caption{Trajektoria wyjścia $y_3$, dla pierwszego zestawu regulatorów PID}
\label{fig:z3_pid1_y3}
\end{figure}
%%%%%%%%%%%%%%%%%%%%%%%%%%%%%%%%%%%%%%%%%%%%%%%%%%%%%%%%%%%%%%%%%%%%%%%%%%%%%%%%%%%%%%%%%%%%

%%%%%%%%%%%%%%%%%%%%%%%%%%%%%%%%%%%%%%%%%%%%%%%%%%%%%%%%%%%%%%%%%%%%%%%%%%%%%%%%%%%%%%%%%%%%
\begin{figure}[b]
\centering
\begin{tikzpicture}
\begin{axis}[
width=0.75\textwidth,
height = 0.4\textwidth,
xmin=0,xmax=1100,ymin=-15,ymax=15,
xlabel={Numer próbki},
ylabel={Wyjście},
xtick={0, 200, 400, 600, 800, 1000},
ytick={-15, -10, -5, 0, 5, 10, 15},
legend pos=north east,
/pgf/number format/.cd,
use comma,
1000 sep={}
]

\addplot[blue,semithick] file {wykresy/z3_yzad.txt};
\addplot[red,semithick] file {wykresy/z3_pid2_y1.txt};
\legend{Wartość zadana, Wyjście $y_1$}

\end{axis}
\end{tikzpicture}
\caption{Trajektoria wyjścia $y_1$, dla drugiego zestawu regulatorów PID}
\label{fig:z3_pid2_y1}
\end{figure}
%%%%%%%%%%%%%%%%%%%%%%%%%%%%%%%%%%%%%%%%%%%%%%%%%%%%%%%%%%%%%%%%%%%%%%%%%%%%%%%%%%%%%%%%%%%%

%%%%%%%%%%%%%%%%%%%%%%%%%%%%%%%%%%%%%%%%%%%%%%%%%%%%%%%%%%%%%%%%%%%%%%%%%%%%%%%%%%%%%%%%%%%%
\begin{figure}[b]
\centering
\begin{tikzpicture}
\begin{axis}[
width=0.75\textwidth,
height = 0.4\textwidth,
xmin=0,xmax=1100,ymin=-5,ymax=5,
xlabel={Numer próbki},
ylabel={Wyjście},
xtick={0, 200, 400, 600, 800, 1000},
ytick={-5, -4, -2, 0, 2, 4, 5},
legend pos=north east,
/pgf/number format/.cd,
use comma,
1000 sep={}
]

\addplot[blue,semithick] file {wykresy/z3_yzad.txt};
\addplot[red,semithick] file {wykresy/z3_pid2_y2.txt};
\legend{Wartość zadana, Wyjście $y_2$}

\end{axis}
\end{tikzpicture}
\caption{Trajektoria wyjścia $y_2$, dla drugiego zestawu regulatorów PID}
\label{fig:z3_pid2_y2}
\end{figure}
%%%%%%%%%%%%%%%%%%%%%%%%%%%%%%%%%%%%%%%%%%%%%%%%%%%%%%%%%%%%%%%%%%%%%%%%%%%%%%%%%%%%%%%%%%%%

%%%%%%%%%%%%%%%%%%%%%%%%%%%%%%%%%%%%%%%%%%%%%%%%%%%%%%%%%%%%%%%%%%%%%%%%%%%%%%%%%%%%%%%%%%%%
\begin{figure}[b]
\centering
\begin{tikzpicture}
\begin{axis}[
width=0.75\textwidth,
height = 0.4\textwidth,
xmin=0,xmax=1100,ymin=-2.5,ymax=2.5,
xlabel={Numer próbki},
ylabel={Wyjście},
xtick={0, 200, 400, 600, 800, 1000},
ytick={-2.5, -2, -1.5, -1, -.5, 0, .5, 1, 1.5, 2, 2.5},
legend pos=north east,
/pgf/number format/.cd,
use comma,
1000 sep={}
]

\addplot[blue,semithick] file {wykresy/z3_yzad.txt};
\addplot[red,semithick] file {wykresy/z3_pid2_y3.txt};
\legend{Wartość zadana, Wyjście $y_3$}

\end{axis}
\end{tikzpicture}
\caption{Trajektoria wyjścia $y_3$, dla drugiego zestawu regulatorów PID}
\label{fig:z3_pid2_y3}
\end{figure}
%%%%%%%%%%%%%%%%%%%%%%%%%%%%%%%%%%%%%%%%%%%%%%%%%%%%%%%%%%%%%%%%%%%%%%%%%%%%%%%%%%%%%%%%%%%%

%%%%%%%%%%%%%%%%%%%%%%%%%%%%%%%%%%%%%%%%%%%%%%%%%%%%%%%%%%%%%%%%%%%%%%%%%%%%%%%%%%%%%%%%%%%%
\begin{figure}[b]
\centering
\begin{tikzpicture}
\begin{axis}[
width=0.75\textwidth,
height = 0.4\textwidth,
xmin=0,xmax=1100,ymin=-15,ymax=15,
xlabel={Numer próbki},
ylabel={Wyjście},
xtick={0, 200, 400, 600, 800, 1000},
ytick={-15, -10, -5, 0, 5, 10, 15},
legend pos=north east,
/pgf/number format/.cd,
use comma,
1000 sep={}
]

\addplot[blue,semithick] file {wykresy/z3_yzad.txt};
\addplot[red,semithick] file {wykresy/z3_pid4_y1.txt};
\legend{Wartość zadana, Wyjście $y_1$}

\end{axis}
\end{tikzpicture}
\caption{Trajektoria wyjścia $y_1$, dla trzeciego zestawu regulatorów PID}
\label{fig:z3_pid4_y1}
\end{figure}
%%%%%%%%%%%%%%%%%%%%%%%%%%%%%%%%%%%%%%%%%%%%%%%%%%%%%%%%%%%%%%%%%%%%%%%%%%%%%%%%%%%%%%%%%%%%

%%%%%%%%%%%%%%%%%%%%%%%%%%%%%%%%%%%%%%%%%%%%%%%%%%%%%%%%%%%%%%%%%%%%%%%%%%%%%%%%%%%%%%%%%%%%
\begin{figure}[b]
\centering
\begin{tikzpicture}
\begin{axis}[
width=0.75\textwidth,
height = 0.4\textwidth,
xmin=0,xmax=1100,ymin=-5,ymax=5,
xlabel={Numer próbki},
ylabel={Wyjście},
xtick={0, 200, 400, 600, 800, 1000},
ytick={-5, -4, -2, 0, 2, 4, 5},
legend pos=north east,
/pgf/number format/.cd,
use comma,
1000 sep={}
]

\addplot[blue,semithick] file {wykresy/z3_yzad.txt};
\addplot[red,semithick] file {wykresy/z3_pid4_y2.txt};
\legend{Wartość zadana, Wyjście $y_2$}

\end{axis}
\end{tikzpicture}
\caption{Trajektoria wyjścia $y_2$, dla trzeciego zestawu regulatorów PID}
\label{fig:z3_pid4_y2}
\end{figure}
%%%%%%%%%%%%%%%%%%%%%%%%%%%%%%%%%%%%%%%%%%%%%%%%%%%%%%%%%%%%%%%%%%%%%%%%%%%%%%%%%%%%%%%%%%%%

%%%%%%%%%%%%%%%%%%%%%%%%%%%%%%%%%%%%%%%%%%%%%%%%%%%%%%%%%%%%%%%%%%%%%%%%%%%%%%%%%%%%%%%%%%%%
\begin{figure}[b]
\centering
\begin{tikzpicture}
\begin{axis}[
width=0.75\textwidth,
height = 0.4\textwidth,
xmin=0,xmax=1100,ymin=-2.5,ymax=2.5,
xlabel={Numer próbki},
ylabel={Wyjście},
xtick={0, 200, 400, 600, 800, 1000},
ytick={-2.5, -2, -1.5, -1, -.5, 0, .5, 1, 1.5, 2, 2.5},
legend pos=north east,
/pgf/number format/.cd,
use comma,
1000 sep={}
]

\addplot[blue,semithick] file {wykresy/z3_yzad.txt};
\addplot[red,semithick] file {wykresy/z3_pid4_y3.txt};
\legend{Wartość zadana, Wyjście $y_3$}

\end{axis}
\end{tikzpicture}
\caption{Trajektoria wyjścia $y_3$, dla trzeciego zestawu regulatorów PID}
\label{fig:z3_pid4_y3}
\end{figure}
%%%%%%%%%%%%%%%%%%%%%%%%%%%%%%%%%%%%%%%%%%%%%%%%%%%%%%%%%%%%%%%%%%%%%%%%%%%%%%%%%%%%%%%%%%%%

%%%%%%%%%%%%%%%%%%%%%%%%%%%%%%%%%%%%%%%%%%%%%%%%%%%%%%%%%%%%%%%%%%%%%%%%%%%%%%%%%%%%%%%%%%%%
\begin{figure}[b]
\centering
\begin{tikzpicture}
\begin{axis}[
width=0.75\textwidth,
height = 0.4\textwidth,
xmin=0,xmax=1100,ymin=-15,ymax=15,
xlabel={Numer próbki},
ylabel={Wyjście},
xtick={0, 200, 400, 600, 800, 1000},
ytick={-15, -10, -5, 0, 5, 10, 15},
legend pos=north east,
/pgf/number format/.cd,
use comma,
1000 sep={}
]

\addplot[blue,semithick] file {wykresy/z3_yzad.txt};
\addplot[red,semithick] file {wykresy/z3_pid3_y1.txt};
\legend{Wartość zadana, Wyjście $y_1$}

\end{axis}
\end{tikzpicture}
\caption{Trajektoria wyjścia $y_1$, dla czwartego zestawu regulatorów PID}
\label{fig:z3_pid3_y1}
\end{figure}
%%%%%%%%%%%%%%%%%%%%%%%%%%%%%%%%%%%%%%%%%%%%%%%%%%%%%%%%%%%%%%%%%%%%%%%%%%%%%%%%%%%%%%%%%%%%

%%%%%%%%%%%%%%%%%%%%%%%%%%%%%%%%%%%%%%%%%%%%%%%%%%%%%%%%%%%%%%%%%%%%%%%%%%%%%%%%%%%%%%%%%%%%
\begin{figure}[b]
\centering
\begin{tikzpicture}
\begin{axis}[
width=0.75\textwidth,
height = 0.4\textwidth,
xmin=0,xmax=1100,ymin=-5,ymax=5,
xlabel={Numer próbki},
ylabel={Wyjście},
xtick={0, 200, 400, 600, 800, 1000},
ytick={-5, -4, -2, 0, 2, 4, 5},
legend pos=north east,
/pgf/number format/.cd,
use comma,
1000 sep={}
]

\addplot[blue,semithick] file {wykresy/z3_yzad.txt};
\addplot[red,semithick] file {wykresy/z3_pid3_y2.txt};
\legend{Wartość zadana, Wyjście $y_2$}

\end{axis}
\end{tikzpicture}
\caption{Trajektoria wyjścia $y_2$, dla czwartego zestawu regulatorów PID}
\label{fig:z3_pid3_y2}
\end{figure}
%%%%%%%%%%%%%%%%%%%%%%%%%%%%%%%%%%%%%%%%%%%%%%%%%%%%%%%%%%%%%%%%%%%%%%%%%%%%%%%%%%%%%%%%%%%%

%%%%%%%%%%%%%%%%%%%%%%%%%%%%%%%%%%%%%%%%%%%%%%%%%%%%%%%%%%%%%%%%%%%%%%%%%%%%%%%%%%%%%%%%%%%%
\begin{figure}[b]
\centering
\begin{tikzpicture}
\begin{axis}[
width=0.75\textwidth,
height = 0.4\textwidth,
xmin=0,xmax=1100,ymin=-2.5,ymax=2.5,
xlabel={Numer próbki},
ylabel={Wyjście},
xtick={0, 200, 400, 600, 800, 1000},
ytick={-2.5, -2, -1.5, -1, -.5, 0, .5, 1, 1.5, 2, 2.5},
legend pos=north east,
/pgf/number format/.cd,
use comma,
1000 sep={}
]

\addplot[blue,semithick] file {wykresy/z3_yzad.txt};
\addplot[red,semithick] file {wykresy/z3_pid3_y3.txt};
\legend{Wartość zadana, Wyjście $y_3$}

\end{axis}
\end{tikzpicture}
\caption{Trajektoria wyjścia $y_3$, dla czwartego zestawu regulatorów PID}
\label{fig:z3_pid3_y3}
\end{figure}
%%%%%%%%%%%%%%%%%%%%%%%%%%%%%%%%%%%%%%%%%%%%%%%%%%%%%%%%%%%%%%%%%%%%%%%%%%%%%%%%%%%%%%%%%%%%

\chapter{DMC}
Na podstawie analizy odpowiedzi skokowych przyjęliśmy horyzont dynamiki $D=80$. Przez $E_i$ oznaczyliśmy wartość wskaźnika błędu dla wyjścia $i$, natomiast $E$ jest sumą błędów dla wszystkich wyjść obiektu.

\section{Dobór horyzontów predykcji i sterowania}
Dobór horyzontów przeprowadzaliśmy korzystając z parametrów $\psi$ i $\lambda$ równymi 1. Rozpoczęliśmy od nastaw $N=N_u=D=80$. Dla tych parametrów błędy wynosiły:
\begin{itemize}
\item $E_1=\num{45,0725}$
\item $E_2=\num{45,9624}$
\item $E_3=\num{24,6561}$
\item $E=\num{115,6910}$
\end{itemize}
Przebieg wyjść obiektu przedstawia wykres \ref{fig:dmc_y_N80_Nu80}, a sterowań wykres \ref{fig:dmc_u_N80_Nu80}.

Postanowiliśmy skrócić horyzonty do wartości $N=N_u=50$. Otrzymane błędy wyniosły:
\begin{itemize}
\item $E_1=\num{45,0726}$
\item $E_2=\num{45,962}$
\item $E_3=\num{24,6562}$
\item $E=\num{115,6908}$
\end{itemize}
Błędy regulacji były więc praktycznie jednakowe jak dla dłuższych horyzontów. Przebiegi wyjść i sterowań przedstawiają wykresy \ref{fig:dmc_y_N50_Nu50} i \ref{fig:dmc_u_N50_Nu50}.

W kolejnym kroku skróciliśmy horyzont predykcji do wartości $N=40$, a sterowania  $N_u=10$. Taka zmiana przyniosła niewielką poprawę wskaźników błędu:
\begin{itemize}
\item $E_1=\num{45,0801}$
\item $E_2=\num{45,933}$
\item $E_3=\num{24,6021}$
\item $E=\num{115,6152}$
\end{itemize}
Przebieg wyjść obiektu przedstawiają wykresy \ref{fig:dmc_y_N40_Nu10} i \ref{fig:dmc_u_N40_Nu10}.

Jak się okazało, dalsze skracanie horyzontu sterowania przyniosło znacznie bardziej wymierne rezultaty - dla $N_u=5$  wskaźniki błędów zmalały do wartości:
\begin{itemize}
\item $E_1=\num{44,4289}$
\item $E_2=\num{44,1988}$
\item $E_3=\num{23,0761}$
\item $E=\num{111,7038}$
\end{itemize}
Przebieg wyjść obiektu przedstawiają wykresy \ref{fig:dmc_y_N40_Nu5} i \ref{fig:dmc_u_N40_Nu5}.

Dalsze skracanie horyzontu predykcji nie przyniosło pozytywnych rezultatów. Dla $N_u=2$ wskaźniki błędów wyniosły:
\begin{itemize}
\item $E_1=\num{46,6684}$
\item $E_2=\num{50,4711}$
\item $E_3=\num{27,023}$
\item $E=\num{124,1625}$
\end{itemize}
Można więc przypuszczać, że jeszcze mniejsze wartości horyzontu sterowania przyniosłyby pogorszenie jakości regulacji. Przebieg wyjść obiektu przedstawiają wykresy \ref{fig:dmc_y_N40_Nu2} i \ref{fig:dmc_u_N40_Nu2}.

W kolejnych zadaniach używane będą horyzonty $N=40$ i $N_u=5$.

\begin{figure}
	\centering
	\begin{tikzpicture}
	\begin{groupplot}[group style={group size=1 by 3,vertical sep={1.5 cm}},
	width=0.9\textwidth,height=0.3\textwidth,xmin=0]
	\nextgroupplot
	[
	xlabel={$k$},
	ylabel={$y_1$},
	y tick label style={/pgf/number format/1000 sep=},
	]
	\addplot[blue,semithick] file {wykresy/dmc_y1_N80_Nu80.txt};
	\addplot[red,semithick,densely dashed] file {wykresy/yzad1.txt};
	\legend{$y_1$,$y_{zad}$}
	\nextgroupplot
	[
	xlabel={$k$},
	ylabel={$y_2$},
	y tick label style={/pgf/number format/1000 sep=},
	]
	\addplot[blue,semithick] file {wykresy/dmc_y2_N80_Nu80.txt};
	\addplot[red,semithick,densely dashed] file {wykresy/yzad2.txt};
	\legend{$y_2$,$y_{zad}$}
	\nextgroupplot
	[
	xlabel={$k$},
	ylabel={$y_3$},
	y tick label style={/pgf/number format/1000 sep=},
	]
	\addplot[blue,semithick] file {wykresy/dmc_y3_N80_Nu80.txt};
	\addplot[red,semithick,densely dashed] file {wykresy/yzad3.txt};
	\legend{$y_3$,$y_{zad}$}
	\end{groupplot}
	\end{tikzpicture}
	\caption{Przebiegi wyjść obiektu dla horyzontów predykcji i sterowania $N=80$, $N_u=80$.}
	\label{fig:dmc_y_N80_Nu80}
\end{figure}

\begin{figure}
	\centering
	\begin{tikzpicture}
	\begin{groupplot}[group style={group size=1 by 4,vertical sep={1.5 cm}},
	width=0.9\textwidth,height=0.25\textwidth,xmin=0]
	\nextgroupplot
	[
	xlabel={$k$},
	ylabel={$u_1$},
	y tick label style={/pgf/number format/1000 sep=},
	]
	\addplot[blue,semithick] file {wykresy/dmc_y1_N80_Nu80.txt};

	\nextgroupplot
	[
	xlabel={$k$},
	ylabel={$u_2$},
	y tick label style={/pgf/number format/1000 sep=},
	]
	\addplot[blue,semithick] file {wykresy/dmc_u2_N80_Nu80.txt};
	
	\nextgroupplot
	[
	xlabel={$k$},
	ylabel={$u_3$},
	y tick label style={/pgf/number format/1000 sep=},
	]
	\addplot[blue,semithick] file {wykresy/dmc_u3_N80_Nu80.txt};
	
	\nextgroupplot
	[
	xlabel={$k$},
	ylabel={$u_4$},
	y tick label style={/pgf/number format/1000 sep=},
	]
	\addplot[blue,semithick] file {wykresy/dmc_u4_N80_Nu80.txt};
	\end{groupplot}
	\end{tikzpicture}
	\caption{Przebiegi sterowań obiektu dla horyzontów predykcji i sterowania $N=80$, $N_u=80$.}
	\label{fig:dmc_u_N80_Nu80}
\end{figure}

\begin{figure}
	\centering
	\begin{tikzpicture}
	\begin{groupplot}[group style={group size=1 by 3,vertical sep={1.5 cm}},
	width=0.9\textwidth,height=0.3\textwidth,xmin=0]
	\nextgroupplot
	[
	xlabel={$k$},
	ylabel={$y_1$},
	y tick label style={/pgf/number format/1000 sep=},
	]
	\addplot[blue,semithick] file {wykresy/dmc_y1_N50_Nu50.txt};
	\addplot[red,semithick,densely dashed] file {wykresy/yzad1.txt};
	\legend{$y_1$,$y_{zad}$}
	\nextgroupplot
	[
	xlabel={$k$},
	ylabel={$y_2$},
	y tick label style={/pgf/number format/1000 sep=},
	]
	\addplot[blue,semithick] file {wykresy/dmc_y2_N50_Nu50.txt};
	\addplot[red,semithick,densely dashed] file {wykresy/yzad2.txt};
	\legend{$y_2$,$y_{zad}$}
	\nextgroupplot
	[
	xlabel={$k$},
	ylabel={$y_3$},
	y tick label style={/pgf/number format/1000 sep=},
	]
	\addplot[blue,semithick] file {wykresy/dmc_y3_N50_Nu50.txt};
	\addplot[red,semithick,densely dashed] file {wykresy/yzad3.txt};
	\legend{$y_3$,$y_{zad}$}
	\end{groupplot}
	\end{tikzpicture}
	\caption{Przebiegi wyjść obiektu dla horyzontów predykcji i sterowania $N=50$, $N_u=50$.}
	\label{fig:dmc_y_N50_Nu50}
\end{figure}

\begin{figure}
	\centering
	\begin{tikzpicture}
	\begin{groupplot}[group style={group size=1 by 4,vertical sep={1.5 cm}},
	width=0.9\textwidth,height=0.25\textwidth,xmin=0]
	\nextgroupplot
	[
	xlabel={$k$},
	ylabel={$u_1$},
	y tick label style={/pgf/number format/1000 sep=},
	]
	\addplot[blue,semithick] file {wykresy/dmc_y1_N50_Nu50.txt};

	\nextgroupplot
	[
	xlabel={$k$},
	ylabel={$u_2$},
	y tick label style={/pgf/number format/1000 sep=},
	]
	\addplot[blue,semithick] file {wykresy/dmc_u2_N50_Nu50.txt};
	
	\nextgroupplot
	[
	xlabel={$k$},
	ylabel={$u_3$},
	y tick label style={/pgf/number format/1000 sep=},
	]
	\addplot[blue,semithick] file {wykresy/dmc_u3_N50_Nu50.txt};
	
	\nextgroupplot
	[
	xlabel={$k$},
	ylabel={$u_4$},
	y tick label style={/pgf/number format/1000 sep=},
	]
	\addplot[blue,semithick] file {wykresy/dmc_u4_N50_Nu50.txt};
	\end{groupplot}
	\end{tikzpicture}
	\caption{Przebiegi sterowań obiektu dla horyzontów predykcji i sterowania $N=50$, $N_u=50$.}
	\label{fig:dmc_u_N50_Nu50}
\end{figure}

\begin{figure}
	\centering
	\begin{tikzpicture}
	\begin{groupplot}[group style={group size=1 by 3,vertical sep={1.5 cm}},
	width=0.9\textwidth,height=0.3\textwidth,xmin=0]
	\nextgroupplot
	[
	xlabel={$k$},
	ylabel={$y_1$},
	y tick label style={/pgf/number format/1000 sep=},
	]
	\addplot[blue,semithick] file {wykresy/dmc_y1_N40_Nu10.txt};
	\addplot[red,semithick,densely dashed] file {wykresy/yzad1.txt};
	\legend{$y_1$,$y_{zad}$}
	\nextgroupplot
	[
	xlabel={$k$},
	ylabel={$y_2$},
	y tick label style={/pgf/number format/1000 sep=},
	]
	\addplot[blue,semithick] file {wykresy/dmc_y2_N40_Nu10.txt};
	\addplot[red,semithick,densely dashed] file {wykresy/yzad2.txt};
	\legend{$y_2$,$y_{zad}$}
	\nextgroupplot
	[
	xlabel={$k$},
	ylabel={$y_3$},
	y tick label style={/pgf/number format/1000 sep=},
	]
	\addplot[blue,semithick] file {wykresy/dmc_y3_N40_Nu10.txt};
	\addplot[red,semithick,densely dashed] file {wykresy/yzad3.txt};
	\legend{$y_3$,$y_{zad}$}
	\end{groupplot}
	\end{tikzpicture}
	\caption{Przebiegi wyjść obiektu dla horyzontów predykcji i sterowania $N=40$, $N_u=10$.}
	\label{fig:dmc_y_N40_Nu10}
\end{figure}

\begin{figure}
	\centering
	\begin{tikzpicture}
	\begin{groupplot}[group style={group size=1 by 4,vertical sep={1.5 cm}},
	width=0.9\textwidth,height=0.25\textwidth,xmin=0]
	\nextgroupplot
	[
	xlabel={$k$},
	ylabel={$u_1$},
	y tick label style={/pgf/number format/1000 sep=},
	]
	\addplot[blue,semithick] file {wykresy/dmc_y1_N40_Nu10.txt};

	\nextgroupplot
	[
	xlabel={$k$},
	ylabel={$u_2$},
	y tick label style={/pgf/number format/1000 sep=},
	]
	\addplot[blue,semithick] file {wykresy/dmc_u2_N40_Nu10.txt};
	
	\nextgroupplot
	[
	xlabel={$k$},
	ylabel={$u_3$},
	y tick label style={/pgf/number format/1000 sep=},
	]
	\addplot[blue,semithick] file {wykresy/dmc_u3_N40_Nu10.txt};
	
	\nextgroupplot
	[
	xlabel={$k$},
	ylabel={$u_4$},
	y tick label style={/pgf/number format/1000 sep=},
	]
	\addplot[blue,semithick] file {wykresy/dmc_u4_N40_Nu10.txt};
	\end{groupplot}
	\end{tikzpicture}
	\caption{Przebiegi sterowań obiektu dla horyzontów predykcji i sterowania $N=40$, $N_u=10$.}
	\label{fig:dmc_u_N40_Nu10}
\end{figure}


\begin{figure}
	\centering
	\begin{tikzpicture}
	\begin{groupplot}[group style={group size=1 by 3,vertical sep={1.5 cm}},
	width=0.9\textwidth,height=0.3\textwidth,xmin=0]
	\nextgroupplot
	[
	xlabel={$k$},
	ylabel={$y_1$},
	y tick label style={/pgf/number format/1000 sep=},
	]
	\addplot[blue,semithick] file {wykresy/dmc_y1_N40_Nu5.txt};
	\addplot[red,semithick,densely dashed] file {wykresy/yzad1.txt};
	\legend{$y_1$,$y_{zad}$}
	\nextgroupplot
	[
	xlabel={$k$},
	ylabel={$y_2$},
	y tick label style={/pgf/number format/1000 sep=},
	]
	\addplot[blue,semithick] file {wykresy/dmc_y2_N40_Nu5.txt};
	\addplot[red,semithick,densely dashed] file {wykresy/yzad2.txt};
	\legend{$y_2$,$y_{zad}$}
	\nextgroupplot
	[
	xlabel={$k$},
	ylabel={$y_3$},
	y tick label style={/pgf/number format/1000 sep=},
	]
	\addplot[blue,semithick] file {wykresy/dmc_y3_N40_Nu5.txt};
	\addplot[red,semithick,densely dashed] file {wykresy/yzad3.txt};
	\legend{$y_3$,$y_{zad}$}
	\end{groupplot}
	\end{tikzpicture}
	\caption{Przebiegi wyjść obiektu dla horyzontów predykcji i sterowania $N=40$, $N_u=5$.}
	\label{fig:dmc_y_N40_Nu5}
\end{figure}

\begin{figure}
	\centering
	\begin{tikzpicture}
	\begin{groupplot}[group style={group size=1 by 4,vertical sep={1.5 cm}},
	width=0.9\textwidth,height=0.25\textwidth,xmin=0]
	\nextgroupplot
	[
	xlabel={$k$},
	ylabel={$u_1$},
	y tick label style={/pgf/number format/1000 sep=},
	]
	\addplot[blue,semithick] file {wykresy/dmc_y1_N40_Nu5.txt};

	\nextgroupplot
	[
	xlabel={$k$},
	ylabel={$u_2$},
	y tick label style={/pgf/number format/1000 sep=},
	]
	\addplot[blue,semithick] file {wykresy/dmc_u2_N40_Nu5.txt};
	
	\nextgroupplot
	[
	xlabel={$k$},
	ylabel={$u_3$},
	y tick label style={/pgf/number format/1000 sep=},
	]
	\addplot[blue,semithick] file {wykresy/dmc_u3_N40_Nu5.txt};
	
	\nextgroupplot
	[
	xlabel={$k$},
	ylabel={$u_4$},
	y tick label style={/pgf/number format/1000 sep=},
	]
	\addplot[blue,semithick] file {wykresy/dmc_u4_N40_Nu5.txt};
	\end{groupplot}
	\end{tikzpicture}
	\caption{Przebiegi sterowań obiektu dla horyzontów predykcji i sterowania $N=40$, $N_u=5$.}
	\label{fig:dmc_u_N40_Nu5}
\end{figure}

\begin{figure}
	\centering
	\begin{tikzpicture}
	\begin{groupplot}[group style={group size=1 by 3,vertical sep={1.5 cm}},
	width=0.9\textwidth,height=0.3\textwidth,xmin=0]
	\nextgroupplot
	[
	xlabel={$k$},
	ylabel={$y_1$},
	y tick label style={/pgf/number format/1000 sep=},
	]
	\addplot[blue,semithick] file {wykresy/dmc_y1_N40_Nu2.txt};
	\addplot[red,semithick,densely dashed] file {wykresy/yzad1.txt};
	\legend{$y_1$,$y_{zad}$}
	\nextgroupplot
	[
	xlabel={$k$},
	ylabel={$y_2$},
	y tick label style={/pgf/number format/1000 sep=},
	]
	\addplot[blue,semithick] file {wykresy/dmc_y2_N40_Nu2.txt};
	\addplot[red,semithick,densely dashed] file {wykresy/yzad2.txt};
	\legend{$y_2$,$y_{zad}$}
	\nextgroupplot
	[
	xlabel={$k$},
	ylabel={$y_3$},
	y tick label style={/pgf/number format/1000 sep=},
	]
	\addplot[blue,semithick] file {wykresy/dmc_y3_N40_Nu2.txt};
	\addplot[red,semithick,densely dashed] file {wykresy/yzad3.txt};
	\legend{$y_3$,$y_{zad}$}
	\end{groupplot}
	\end{tikzpicture}
	\caption{Przebiegi wyjść obiektu dla horyzontów predykcji i sterowania $N=40$, $N_u=5$.}
	\label{fig:dmc_y_N40_Nu2}
\end{figure}

\begin{figure}
	\centering
	\begin{tikzpicture}
	\begin{groupplot}[group style={group size=1 by 4,vertical sep={1.5 cm}},
	width=0.9\textwidth,height=0.25\textwidth,xmin=0]
	\nextgroupplot
	[
	xlabel={$k$},
	ylabel={$u_1$},
	y tick label style={/pgf/number format/1000 sep=},
	]
	\addplot[blue,semithick] file {wykresy/dmc_y1_N40_Nu2.txt};

	\nextgroupplot
	[
	xlabel={$k$},
	ylabel={$u_2$},
	y tick label style={/pgf/number format/1000 sep=},
	]
	\addplot[blue,semithick] file {wykresy/dmc_u2_N40_Nu2.txt};
	
	\nextgroupplot
	[
	xlabel={$k$},
	ylabel={$u_3$},
	y tick label style={/pgf/number format/1000 sep=},
	]
	\addplot[blue,semithick] file {wykresy/dmc_u3_N40_Nu2.txt};
	
	\nextgroupplot
	[
	xlabel={$k$},
	ylabel={$u_4$},
	y tick label style={/pgf/number format/1000 sep=},
	]
	\addplot[blue,semithick] file {wykresy/dmc_u4_N40_Nu2.txt};
	\end{groupplot}
	\end{tikzpicture}
	\caption{Przebiegi sterowań obiektu dla horyzontów predykcji i sterowania $N=40$, $N_u=2$.}
	\label{fig:dmc_u_N40_Nu2}
\end{figure}
\chapter{Parametry $\lambda$ i $\psi$}
\section{Parametr $\lambda$}
W wyniku testowania różnych wartości współczynnika $\lambda$ zaobserwowaliśmy, że błąd regulacji jest najmniejszy dla bardzo małych wartości $\lambda$. Trzeba jednak zauważyć, że niskie wartości parametru powodują, że przebieg sterowania jest znacznie "ostrzejszy", występują duże i nagłe skoki $u$ przy zmianach wartości zadanej. W przypadku rzeczywistego obiektu, zjawisko to mogłoby mieć negatywny efekt, na przykład uszkodzenie części sterujących. Staraliśmy się więc doprowadzić do kompromisu między niskim wskaźnikiem błędu a łagodnym przebiegiem sterowania.

Testując różne wartości parametru $\lambda$, przyjęliśmy długości horyzontów $N=40$ i $N_u=5$, a parametry $\psi = 1$.

Próba zwiększenia wartości parametrów $\lambda$ do wartości 2 okazała się przynosić znacznie wyższe współczynniki błędu.
\begin{itemize}
\item $E_1=\num{49,4821}$
\item $E_2=\num{49,361}$
\item $E_3=\num{28,0998}$
\item $E=\num{126,9428}$
\end{itemize}
Zdecydowaliśmy więc w kolejnych testach skupić się na parametrach $\lambda$ poniżej 1. Przebiegi wyjść i sterowań przedstawiają wykresy \ref{fig:dmc_y_l_2_2_2_2_psi_1_1_1} i \ref{fig:dmc_u_l_2_2_2_2_psi_1_1_1}.

Ustawienie parametrów na wartości $\lambda_1=\lambda_2=\lambda_3=\lambda_4=\num{0,2}$ dało w rezultacie bardzo dużą poprawę wskaźników błędu regulacji.
\begin{itemize}
\item $E_1=\num{37,0136}$
\item $E_2=\num{37,5058}$
\item $E_3=\num{14,446}$
\item $E=\num{88,9654}$
\end{itemize}
Charakterystykę sterowania uznaliśmy za akceptowalną. Przebiegi wyjść i sterowań przedstawiają wykresy \ref{fig:dmc_y_l_02_02_02_02_psi_1_1_1} i \ref{fig:dmc_u_l_02_02_02_02_psi_1_1_1}.

 Można dostrzec, że skoki sterowania na torach 1 i 4 osiągają znacznie większe wartości, niż na torach 2 i 3. Tor sterowania 3 natomiast ma łagodniejszy przebieg niż pozostałe. Z tego powodu przetestujemy, jak zachowuje się obiekt w przypadku, gdy parametry $\lambda_1$ i $\lambda_4$ mają wyższe wartości niż $\lambda_2$, a $\lambda_3$ ma niższą wartość. W ten sposób tory sterowania 1 i 4 powinny zostać złagodzone, a tor 3 przyspieszony. 

Przyjęliśmy parametry o następujących wartościach:
\begin{itemize}
\item $\lambda_1=\num{0,3}$
\item $\lambda_2=\num{0,2}$
\item $\lambda_3=\num{0,1}$
\item $\lambda_4=\num{0,3}$
\end{itemize}
Błędy regulacji:
\begin{itemize}
\item $E_1=\num{36,357}$
\item $E_2=\num{38,3743}$
\item $E_3=\num{16,2287}$
\item $E=\num{90,9600}$
\end{itemize}
Jak widać odnotowaliśmy nieznaczne pogorszenie jakości regulacji. Można jednak zaobserwować na wykresie sterowań \ref{fig:dmc_u_l_03_02_01_03_psi_1_1_1}, że tory 1 i 4 mają łagodniejsze przebiegi. Uznaliśmy więc, że te wartości $\lambda$ są w naszym przypadku optymalne. Przebiegi wyjść przedstawia wykres \ref{fig:dmc_y_l_03_02_01_03_psi_1_1_1}.

W kolejnych testach używane będą parametry $\lambda$ o wartościach:
\begin{itemize}
\item $\lambda_1=\num{0,3}$
\item $\lambda_2=\num{0,2}$
\item $\lambda_3=\num{0,1}$
\item $\lambda_4=\num{0,3}$
\end{itemize}

\begin{figure}
	\centering
	\begin{tikzpicture}
	\begin{groupplot}[group style={group size=1 by 3,vertical sep={1.5 cm}},
	width=0.9\textwidth,height=0.3\textwidth,xmin=0]
	\nextgroupplot
	[
	xlabel={$k$},
	ylabel={$y_1$},
	y tick label style={/pgf/number format/1000 sep=},
	]
	\addplot[blue,semithick] file {wykresy/dmc_y1_l_2_2_2_2_psi_1_1_1.txt};
	\addplot[red,semithick,densely dashed] file {wykresy/yzad1.txt};
	\legend{$y_1$,$y_{zad}$}
	\nextgroupplot
	[
	xlabel={$k$},
	ylabel={$y_2$},
	y tick label style={/pgf/number format/1000 sep=},
	]
	\addplot[blue,semithick] file {wykresy/dmc_y2_l_2_2_2_2_psi_1_1_1.txt};
	\addplot[red,semithick,densely dashed] file {wykresy/yzad2.txt};
	\legend{$y_2$,$y_{zad}$}
	\nextgroupplot
	[
	xlabel={$k$},
	ylabel={$y_3$},
	y tick label style={/pgf/number format/1000 sep=},
	]
	\addplot[blue,semithick] file {wykresy/dmc_y3_l_2_2_2_2_psi_1_1_1.txt};
	\addplot[red,semithick,densely dashed] file {wykresy/yzad3.txt};
	\legend{$y_3$,$y_{zad}$}
	\end{groupplot}
	\end{tikzpicture}
	\caption{Przebiegi wyjść obiektu dla $\lambda_1=2$, $\lambda_2=2$, $\lambda_3=2$ i $\lambda_4=2$.}
	\label{fig:dmc_y_l_2_2_2_2_psi_1_1_1}
\end{figure}

\begin{figure}
	\centering
	\begin{tikzpicture}
	\begin{groupplot}[group style={group size=1 by 4,vertical sep={1.5 cm}},
	width=0.9\textwidth,height=0.25\textwidth,xmin=0]
	\nextgroupplot
	[
	xlabel={$k$},
	ylabel={$u_1$},
	y tick label style={/pgf/number format/1000 sep=},
	]
	\addplot[blue,semithick] file {wykresy/dmc_u1_l_2_2_2_2_psi_1_1_1.txt};

	\nextgroupplot
	[
	xlabel={$k$},
	ylabel={$u_2$},
	y tick label style={/pgf/number format/1000 sep=},
	]
	\addplot[blue,semithick] file {wykresy/dmc_u2_l_2_2_2_2_psi_1_1_1.txt};
	
	\nextgroupplot
	[
	xlabel={$k$},
	ylabel={$u_3$},
	y tick label style={/pgf/number format/1000 sep=},
	]
	\addplot[blue,semithick] file {wykresy/dmc_u3_l_2_2_2_2_psi_1_1_1.txt};
	
	\nextgroupplot
	[
	xlabel={$k$},
	ylabel={$u_4$},
	y tick label style={/pgf/number format/1000 sep=},
	]
	\addplot[blue,semithick] file {wykresy/dmc_u4_l_2_2_2_2_psi_1_1_1.txt};
	\end{groupplot}
	\end{tikzpicture}
	\caption{Przebiegi sterowań obiektu dla $\lambda_1=2$, $\lambda_2=2$, $\lambda_3=2$ i $\lambda_4=2$.}
	\label{fig:dmc_u_l_2_2_2_2_psi_1_1_1}
\end{figure}


\begin{figure}
	\centering
	\begin{tikzpicture}
	\begin{groupplot}[group style={group size=1 by 3,vertical sep={1.5 cm}},
	width=0.9\textwidth,height=0.3\textwidth,xmin=0]
	\nextgroupplot
	[
	xlabel={$k$},
	ylabel={$y_1$},
	y tick label style={/pgf/number format/1000 sep=},
	]
	\addplot[blue,semithick] file {wykresy/dmc_y1_l_02_02_02_02_psi_1_1_1.txt};
	\addplot[red,semithick,densely dashed] file {wykresy/yzad1.txt};
	\legend{$y_1$,$y_{zad}$}
	\nextgroupplot
	[
	xlabel={$k$},
	ylabel={$y_2$},
	y tick label style={/pgf/number format/1000 sep=},
	]
	\addplot[blue,semithick] file {wykresy/dmc_y2_l_02_02_02_02_psi_1_1_1.txt};
	\addplot[red,semithick,densely dashed] file {wykresy/yzad2.txt};
	\legend{$y_2$,$y_{zad}$}
	\nextgroupplot
	[
	xlabel={$k$},
	ylabel={$y_3$},
	y tick label style={/pgf/number format/1000 sep=},
	]
	\addplot[blue,semithick] file {wykresy/dmc_y3_l_02_02_02_02_psi_1_1_1.txt};
	\addplot[red,semithick,densely dashed] file {wykresy/yzad3.txt};
	\legend{$y_3$,$y_{zad}$}
	\end{groupplot}
	\end{tikzpicture}
	\caption{Przebiegi wyjść obiektu dla $\lambda_1=\num{0,2}$, $\lambda_2=\num{0,2}$, $\lambda_3=\num{0,2}$ i $\lambda_4=\num{0,2}$.}
	\label{fig:dmc_y_l_02_02_02_02_psi_1_1_1}
\end{figure}

\begin{figure}
	\centering
	\begin{tikzpicture}
	\begin{groupplot}[group style={group size=1 by 4,vertical sep={1.5 cm}},
	width=0.9\textwidth,height=0.25\textwidth,xmin=0]
	\nextgroupplot
	[
	xlabel={$k$},
	ylabel={$u_1$},
	y tick label style={/pgf/number format/1000 sep=},
	]
	\addplot[blue,semithick] file {wykresy/dmc_u1_l_02_02_02_02_psi_1_1_1.txt};

	\nextgroupplot
	[
	xlabel={$k$},
	ylabel={$u_2$},
	y tick label style={/pgf/number format/1000 sep=},
	]
	\addplot[blue,semithick] file {wykresy/dmc_u2_l_02_02_02_02_psi_1_1_1.txt};
	
	\nextgroupplot
	[
	xlabel={$k$},
	ylabel={$u_3$},
	y tick label style={/pgf/number format/1000 sep=},
	]
	\addplot[blue,semithick] file {wykresy/dmc_u3_l_02_02_02_02_psi_1_1_1.txt};
	
	\nextgroupplot
	[
	xlabel={$k$},
	ylabel={$u_4$},
	y tick label style={/pgf/number format/1000 sep=},
	]
	\addplot[blue,semithick] file {wykresy/dmc_u4_l_02_02_02_02_psi_1_1_1.txt};
	\end{groupplot}
	\end{tikzpicture}
	\caption{Przebiegi sterowań obiektu dla $\lambda_1=\num{0,2}$, $\lambda_2=\num{0,2}$, $\lambda_3=\num{0,2}$ i $\lambda_4=\num{0,2}$.}
	\label{fig:dmc_u_l_02_02_02_02_psi_1_1_1}
\end{figure}

\begin{figure}
	\centering
	\begin{tikzpicture}
	\begin{groupplot}[group style={group size=1 by 3,vertical sep={1.5 cm}},
	width=0.9\textwidth,height=0.3\textwidth,xmin=0]
	\nextgroupplot
	[
	xlabel={$k$},
	ylabel={$y_1$},
	y tick label style={/pgf/number format/1000 sep=},
	]
	\addplot[blue,semithick] file {wykresy/dmc_y1_l_03_02_01_03_psi_1_1_1.txt};
	\addplot[red,semithick,densely dashed] file {wykresy/yzad1.txt};
	\legend{$y_1$,$y_{zad}$}
	\nextgroupplot
	[
	xlabel={$k$},
	ylabel={$y_2$},
	y tick label style={/pgf/number format/1000 sep=},
	]
	\addplot[blue,semithick] file {wykresy/dmc_y2_l_03_02_01_03_psi_1_1_1.txt};
	\addplot[red,semithick,densely dashed] file {wykresy/yzad2.txt};
	\legend{$y_2$,$y_{zad}$}
	\nextgroupplot
	[
	xlabel={$k$},
	ylabel={$y_3$},
	y tick label style={/pgf/number format/1000 sep=},
	]
	\addplot[blue,semithick] file {wykresy/dmc_y3_l_03_02_01_03_psi_1_1_1.txt};
	\addplot[red,semithick,densely dashed] file {wykresy/yzad3.txt};
	\legend{$y_3$,$y_{zad}$}
	\end{groupplot}
	\end{tikzpicture}
	\caption{Przebiegi wyjść obiektu dla $\lambda_1=\num{0,3}$, $\lambda_2=\num{0,2}$, $\lambda_3=\num{0,1}$ i $\lambda_4=\num{0,3}$.}
	\label{fig:dmc_y_l_03_02_01_03_psi_1_1_1}
\end{figure}

\begin{figure}
	\centering
	\begin{tikzpicture}
	\begin{groupplot}[group style={group size=1 by 4,vertical sep={1.5 cm}},
	width=0.9\textwidth,height=0.25\textwidth,xmin=0]
	\nextgroupplot
	[
	xlabel={$k$},
	ylabel={$u_1$},
	y tick label style={/pgf/number format/1000 sep=},
	]
	\addplot[blue,semithick] file {wykresy/dmc_u1_l_03_02_01_03_psi_1_1_1.txt};

	\nextgroupplot
	[
	xlabel={$k$},
	ylabel={$u_2$},
	y tick label style={/pgf/number format/1000 sep=},
	]
	\addplot[blue,semithick] file {wykresy/dmc_u2_l_03_02_01_03_psi_1_1_1.txt};
	
	\nextgroupplot
	[
	xlabel={$k$},
	ylabel={$u_3$},
	y tick label style={/pgf/number format/1000 sep=},
	]
	\addplot[blue,semithick] file {wykresy/dmc_u3_l_03_02_01_03_psi_1_1_1.txt};
	
	\nextgroupplot
	[
	xlabel={$k$},
	ylabel={$u_4$},
	y tick label style={/pgf/number format/1000 sep=},
	]
	\addplot[blue,semithick] file {wykresy/dmc_u4_l_03_02_01_03_psi_1_1_1.txt};
	\end{groupplot}
	\end{tikzpicture}
	\caption{Przebiegi sterowań obiektu dla $\lambda_1=\num{0,3}$, $\lambda_2=\num{0,2}$, $\lambda_3=\num{0,1}$ i $\lambda_4=\num{0,3}$.}
	\label{fig:dmc_u_l_03_02_01_03_psi_1_1_1}
\end{figure}
\section{Parametr $\psi$}
W wyniku testowania różnych wartości współczynnika $\psi$ zaobserwowaliśmy, że błąd regulacji jest najmniejszy dla duzych wartości $\psi$. Podobnie jednak jak w przypadku dobierania $\lambda$ zauważamy, że wysokie wartości parametru powodują, że przebieg sterowania jest znacznie "ostrzejszy", występują duże i nagłe skoki $u$ przy zmianach wartości zadanej. W przypadku rzeczywistego obiektu, zjawisko to mogłoby mieć negatywny efekt, na przykład uszkodzenie części sterujących. Staraliśmy się więc doprowadzić do kompromisu między niskim wskaźnikiem błędu a łagodnym przebiegiem sterowania.

Testując różne wartości parametru $psi$, przyjęliśmy długości horyzontów $N=40$ i $N_u=5$ oraz współczynniki $\lambda_1=\num{0,3}$, $\lambda_2=\num{0,2}$, $\lambda_3=\num{0,1}$, $\lambda_4=\num{0,3}$.

Próba ustawienia parametrów $\psi$ na wartość poniżej 1 dała w rezultacie wyższe błędy regulacji.
Próba zmniejszenia wartości parametrów $psi$ do $\num{0,8}$ okazała się przynosić wyższe współczynniki błędu.
\begin{itemize}
\item $E_1=\num{36,9114}$
\item $E_2=\num{39,0784}$
\item $E_3=\num{17,3408}$
\item $E=\num{93,3306}$
\end{itemize}
Zdecydowaliśmy więc, że kolejne testy przeprowadzane będą na wartościach $\psi$ powyżej 1. Przebiegi wyjść i sterowań przedstawiają wykresy \ref{fig:dmc_y_l_03_02_01_03_psi_08_08_08} i \ref{fig:dmc_y_l_03_02_01_03_psi_08_08_08}.

Zwiększenie współczynników $\psi$ do wartości 5 dało w rezultacie bardzo dużą poprawę błędu regulacji.
\begin{itemize}
\item $E_1=\num{34,6604}$
\item $E_2=\num{34,131}$
\item $E_3=\num{9,0736}$
\item $E=\num{78,8650}$
\end{itemize}
Należy jednak odnotować, że przebieg sterowania jest teraz znacznie ostrzejszy, co jest szczególnie widoczne na torze sterowania 4 (wykres \ref{fig:dmc_u_l_03_02_01_03_psi_5_5_5}). Zmiana $\psi$ nie miała dużego wpływu na pozostałe tory. Spróbujemy więc, manipulując parametrami $\psi$, złagodzić sterowanie na torze 4, zachowując jednocześnie poprawę błędu regulacji. Wyjścia obiektu przedstawia wykres \ref{fig:dmc_y_l_03_02_01_03_psi_5_5_5}.

W wyniku eksperymentów dowiedzieliśmy się, że najbardziej na sterowanie na torze czwartym wpływa parametr $\psi_3$. Postanowiliśmy więc zmniejszyć $\psi_3$, jednocześnie zwiększająć $\psi_1$ i $\psi_2$. Przetestowaliśmy działanie obiektu na wartościach $\psi_1=\num{6,5}$, $\psi_2=7$, $\psi_3=2$. Jak widać na wykresie \ref{fig:dmc_u_l_03_02_01_03_psi_65_7_2}, sterowanie zostało nieco złagodzone, choć w rezultacie nieznacznie pogorszył się wskaźnik błędu regulacji.
\begin{itemize}
\item $E_1=\num{34,584}$
\item $E_2=\num{35,2223}$
\item $E_3=\num{10,3634}$
\item $E=\num{80,1697}$
\end{itemize}
Uznaliśmy jednak, że takie nastawy dają dobry kompromis między jakością regulacji a łagodnym sterowaniem. Przebiegi wyjść obiektu przedstawia wykres \ref{fig:dmc_y_l_03_02_01_03_psi_65_7_2}.

\begin{figure}
	\centering
	\begin{tikzpicture}
	\begin{groupplot}[group style={group size=1 by 3,vertical sep={1.5 cm}},
	width=0.9\textwidth,height=0.3\textwidth,xmin=0]
	\nextgroupplot
	[
	xlabel={$k$},
	ylabel={$y_1$},
	y tick label style={/pgf/number format/1000 sep=},
	]
	\addplot[blue,semithick] file {wykresy/dmc_y1_l_03_02_01_03_psi_08_08_08.txt};
	\addplot[red,semithick,densely dashed] file {wykresy/yzad1.txt};
	\legend{$y_1$,$y_{zad}$}
	\nextgroupplot
	[
	xlabel={$k$},
	ylabel={$y_2$},
	y tick label style={/pgf/number format/1000 sep=},
	]
	\addplot[blue,semithick] file {wykresy/dmc_y2_l_03_02_01_03_psi_08_08_08.txt};
	\addplot[red,semithick,densely dashed] file {wykresy/yzad2.txt};
	\legend{$y_2$,$y_{zad}$}
	\nextgroupplot
	[
	xlabel={$k$},
	ylabel={$y_3$},
	y tick label style={/pgf/number format/1000 sep=},
	]
	\addplot[blue,semithick] file {wykresy/dmc_y3_l_03_02_01_03_psi_08_08_08.txt};
	\addplot[red,semithick,densely dashed] file {wykresy/yzad3.txt};
	\legend{$y_3$,$y_{zad}$}
	\end{groupplot}
	\end{tikzpicture}
	\caption{Przebiegi wyjść obiektu dla $\psi_1=\num{0,8}$, $\psi_2=\num{0,8}$, $\psi_3=\num{0,8}$.}
	\label{fig:dmc_y_l_03_02_01_03_psi_08_08_08}
\end{figure}

\begin{figure}
	\centering
	\begin{tikzpicture}
	\begin{groupplot}[group style={group size=1 by 4,vertical sep={1.5 cm}},
	width=0.9\textwidth,height=0.25\textwidth,xmin=0]
	\nextgroupplot
	[
	xlabel={$k$},
	ylabel={$u_1$},
	y tick label style={/pgf/number format/1000 sep=},
	]
	\addplot[blue,semithick] file {wykresy/dmc_y1_l_03_02_01_03_psi_08_08_08.txt};

	\nextgroupplot
	[
	xlabel={$k$},
	ylabel={$u_2$},
	y tick label style={/pgf/number format/1000 sep=},
	]
	\addplot[blue,semithick] file {wykresy/dmc_u2_l_03_02_01_03_psi_08_08_08.txt};
	
	\nextgroupplot
	[
	xlabel={$k$},
	ylabel={$u_3$},
	y tick label style={/pgf/number format/1000 sep=},
	]
	\addplot[blue,semithick] file {wykresy/dmc_u3_l_03_02_01_03_psi_08_08_08.txt};
	
	\nextgroupplot
	[
	xlabel={$k$},
	ylabel={$u_4$},
	y tick label style={/pgf/number format/1000 sep=},
	]
	\addplot[blue,semithick] file {wykresy/dmc_u4_l_03_02_01_03_psi_08_08_08.txt};
	\end{groupplot}
	\end{tikzpicture}
	\caption{Przebiegi sterowań obiektu dla $\psi_1=\num{0,8}$, $\psi_2=\num{0,8}$, $\psi_3=\num{0,8}$.}
	\label{fig:dmc_u_l_03_02_01_03_psi_08_08_08}
\end{figure}


\begin{figure}
	\centering
	\begin{tikzpicture}
	\begin{groupplot}[group style={group size=1 by 3,vertical sep={1.5 cm}},
	width=0.9\textwidth,height=0.3\textwidth,xmin=0]
	\nextgroupplot
	[
	xlabel={$k$},
	ylabel={$y_1$},
	y tick label style={/pgf/number format/1000 sep=},
	]
	\addplot[blue,semithick] file {wykresy/dmc_y1_l_03_02_01_03_psi_5_5_5.txt};
	\addplot[red,semithick,densely dashed] file {wykresy/yzad1.txt};
	\legend{$y_1$,$y_{zad}$}
	\nextgroupplot
	[
	xlabel={$k$},
	ylabel={$y_2$},
	y tick label style={/pgf/number format/1000 sep=},
	]
	\addplot[blue,semithick] file {wykresy/dmc_y2_l_03_02_01_03_psi_5_5_5.txt};
	\addplot[red,semithick,densely dashed] file {wykresy/yzad2.txt};
	\legend{$y_2$,$y_{zad}$}
	\nextgroupplot
	[
	xlabel={$k$},
	ylabel={$y_3$},
	y tick label style={/pgf/number format/1000 sep=},
	]
	\addplot[blue,semithick] file {wykresy/dmc_y3_l_03_02_01_03_psi_5_5_5.txt};
	\addplot[red,semithick,densely dashed] file {wykresy/yzad3.txt};
	\legend{$y_3$,$y_{zad}$}
	\end{groupplot}
	\end{tikzpicture}
	\caption{Przebiegi wyjść obiektu dla $\psi_1=5$, $\psi_2=5$, $\psi_3=5$.}
	\label{fig:dmc_y_l_03_02_01_03_psi_5_5_5}
\end{figure}

\begin{figure}
	\centering
	\begin{tikzpicture}
	\begin{groupplot}[group style={group size=1 by 4,vertical sep={1.5 cm}},
	width=0.9\textwidth,height=0.25\textwidth,xmin=0]
	\nextgroupplot
	[
	xlabel={$k$},
	ylabel={$u_1$},
	y tick label style={/pgf/number format/1000 sep=},
	]
	\addplot[blue,semithick] file {wykresy/dmc_y1_l_03_02_01_03_psi_5_5_5.txt};

	\nextgroupplot
	[
	xlabel={$k$},
	ylabel={$u_2$},
	y tick label style={/pgf/number format/1000 sep=},
	]
	\addplot[blue,semithick] file {wykresy/dmc_u2_l_03_02_01_03_psi_5_5_5.txt};
	
	\nextgroupplot
	[
	xlabel={$k$},
	ylabel={$u_3$},
	y tick label style={/pgf/number format/1000 sep=},
	]
	\addplot[blue,semithick] file {wykresy/dmc_u3_l_03_02_01_03_psi_5_5_5.txt};
	
	\nextgroupplot
	[
	xlabel={$k$},
	ylabel={$u_4$},
	y tick label style={/pgf/number format/1000 sep=},
	]
	\addplot[blue,semithick] file {wykresy/dmc_u4_l_03_02_01_03_psi_5_5_5.txt};
	\end{groupplot}
	\end{tikzpicture}
	\caption{Przebiegi sterowań obiektu dla $\psi_1=5$, $\psi_2=5$, $\psi_3=5$.}
	\label{fig:dmc_u_l_03_02_01_03_psi_5_5_5}
\end{figure}


\begin{figure}
	\centering
	\begin{tikzpicture}
	\begin{groupplot}[group style={group size=1 by 3,vertical sep={1.5 cm}},
	width=0.9\textwidth,height=0.3\textwidth,xmin=0]
	\nextgroupplot
	[
	xlabel={$k$},
	ylabel={$y_1$},
	y tick label style={/pgf/number format/1000 sep=},
	]
	\addplot[blue,semithick] file {wykresy/dmc_y1_l_03_02_01_03_psi_65_7_2.txt};
	\addplot[red,semithick,densely dashed] file {wykresy/yzad1.txt};
	\legend{$y_1$,$y_{zad}$}
	\nextgroupplot
	[
	xlabel={$k$},
	ylabel={$y_2$},
	y tick label style={/pgf/number format/1000 sep=},
	]
	\addplot[blue,semithick] file {wykresy/dmc_y2_l_03_02_01_03_psi_65_7_2.txt};
	\addplot[red,semithick,densely dashed] file {wykresy/yzad2.txt};
	\legend{$y_2$,$y_{zad}$}
	\nextgroupplot
	[
	xlabel={$k$},
	ylabel={$y_3$},
	y tick label style={/pgf/number format/1000 sep=},
	]
	\addplot[blue,semithick] file {wykresy/dmc_y3_l_03_02_01_03_psi_65_7_2.txt};
	\addplot[red,semithick,densely dashed] file {wykresy/yzad3.txt};
	\legend{$y_3$,$y_{zad}$}
	\end{groupplot}
	\end{tikzpicture}
	\caption{Przebiegi wyjść obiektu dla $\psi_1=\num{6,5}$, $\psi_2=7$, $\psi_3=2$.}
	\label{fig:dmc_y_l_03_02_01_03_psi_65_7_2}
\end{figure}

\begin{figure}
	\centering
	\begin{tikzpicture}
	\begin{groupplot}[group style={group size=1 by 4,vertical sep={1.5 cm}},
	width=0.9\textwidth,height=0.25\textwidth,xmin=0]
	\nextgroupplot
	[
	xlabel={$k$},
	ylabel={$u_1$},
	y tick label style={/pgf/number format/1000 sep=},
	]
	\addplot[blue,semithick] file {wykresy/dmc_y1_l_03_02_01_03_psi_65_7_2.txt};

	\nextgroupplot
	[
	xlabel={$k$},
	ylabel={$u_2$},
	y tick label style={/pgf/number format/1000 sep=},
	]
	\addplot[blue,semithick] file {wykresy/dmc_u2_l_03_02_01_03_psi_65_7_2.txt};
	
	\nextgroupplot
	[
	xlabel={$k$},
	ylabel={$u_3$},
	y tick label style={/pgf/number format/1000 sep=},
	]
	\addplot[blue,semithick] file {wykresy/dmc_u3_l_03_02_01_03_psi_65_7_2.txt};
	
	\nextgroupplot
	[
	xlabel={$k$},
	ylabel={$u_4$},
	y tick label style={/pgf/number format/1000 sep=},
	]
	\addplot[blue,semithick] file {wykresy/dmc_u4_l_03_02_01_03_psi_65_7_2.txt};
	\end{groupplot}
	\end{tikzpicture}
	\caption{Przebiegi sterowań obiektu dla $\psi_1=\num{6,5}$, $\psi_2=7$, $\psi_3=2$.}
	\label{fig:dmc_u_l_03_02_01_03_psi_65_7_2}
\end{figure}
\chapter{PID/DMC Optymalizacja}
Zadanie piąte polegało na znalezieniu optymalnych wartości nastaw dla regulatora PID oraz DMC, wykorzystując do optymalizacji ilościowy wskaźnik błędu regulacji E.
Stworzyliśmy do tego skrypt, który używa funkcji $fmincon$ w celu dobierania nastaw dla regulatorów.


Rozpoczniemy od optymalizacji nastaw regulatora PID. Jako cele optymalizacji przyjmujemy parametry $K$ oraz $T_i$. Zdecydowaliśmy się odgórnie narzucić zerowe wartości
$T_d$ ze względu na wyniki badań jakie przeprowadziliśmy w poprzednim punkcie sprawozdania.
Jako ograniczenia przyjmujemy zgodnie z logiką, wartości nie mniejsze niż 0 oraz mniejsze od nieskonczoności.
Jako punkt startowy wybieramy wartości wzmocnień równe $1$ a  czasy zdwojenia jako $1000$.
Optymalizacji poddajemy trzy regulatory PID w czterech konfiguracjach, tzn.
rozważamy cztery rózne konfiguracje torów sterowania wyznaczone w poprzednim punkcie sprawozdania.

Dla toru:
\begin{itemize}
  \item $y_1$ -- $u_4$
 \item $y_2$ -- $u_3$
 \item $y_3$ -- $u_2$
\end{itemize}

otrzymujemy nastawy:
\begin{equation}
  K_1 = \num{1.9724} \qquad T_{i1} = \num{210008}, \qquad T_{d1} = 0 \nonumber
\end{equation}
\begin{equation}
  K_2 = \num{1.3646} \qquad T_{i2} = \num{7.9930}, \qquad T_{d2} = 0
\end{equation}
\begin{equation}
  K_3 = \num{0.1696} \qquad T_{i3} = \num{300220}, \qquad T_{d3} = 0 \nonumber
\end{equation}
Takie nastawy dają błąd regulacji \num{486.6761}. Jest to dziwnie trudny przypadek,
gdyż nie udało nam się znaleźć nastaw dla niego ręcznie, a lekka modyfikacja nastaw
wygenerowanych funkcją \texttt{fmincon} daje ogromny skok błędu. Działanie
przedstawiają wykresy \ref{fig:pid1_fmincon_y1}, \ref{fig:pid1_fmincon_y2} i \ref{fig:pid1_fmincon_y3}.


Dla toru:
\begin{itemize}
  \item $y_1$ -- $u_1$
 \item $y_2$ -- $u_3$
 \item $y_3$ -- $u_4$
\end{itemize}

otrzymujemy nastawy:
\begin{equation}
  K_1 = \num{2.7249} \qquad T_{i1} = \num{3.9641}, \qquad T_{d1} = 0 \nonumber
\end{equation}
\begin{equation}
  K_2 = \num{2.9122} \qquad T_{i2} = \num{3.1237}, \qquad T_{d2} = 0
\end{equation}
\begin{equation}
  K_3 = \num{5.5929} \qquad T_{i3} = \num{9.8384}, \qquad T_{d3} = 0 \nonumber
\end{equation}

Błąd regulacji wynosi: \num{104.4858} . Jest to najlepszy ogólnie otrzymany wynik.
Działanie
przedstawiają wykresy \ref{fig:pid2_fmincon_y1}, \ref{fig:pid2_fmincon_y2} i \ref{fig:pid2_fmincon_y3}.

Dla toru:
\begin{itemize}
  \item $y_1$ -- $u_1$
 \item $y_2$ -- $u_2$
 \item $y_3$ -- $u_4$
\end{itemize}

otrzymujemy nastawy:
\begin{equation}
  K_1 = \num{2.9883} \qquad T_{i1} = \num{4.4246}, \qquad T_{d1} = 0 \nonumber
\end{equation}
\begin{equation}
  K_2 = \num{0.6972} \qquad T_{i2} = \num{8824.7}, \qquad T_{d2} = 0
\end{equation}
\begin{equation}
  K_3 = \num{5.6629} \qquad T_{i3} = \num{11.6449}, \qquad T_{d3} = 0 \nonumber
\end{equation}

Błąd regulacji wynosi: \num{115.6349} .
Jest to bardzo dobry wynik, jednakże nieco gorszy niż dla poprzedniego toru.
Zauważmy, że wyznaczone nastawy dla pierwszego i trzeciego regulatora są bardzo podobne do tych dla poprzedniego toru.
Wynika to z faktu, że zmienił się tu jedynie tor dla drugiego regulatora, gdzie teraz wpływa na wyjscie drugie regulator drugi.
Działanie
przedstawiają wykresy \ref{fig:pid3_fmincon_y1}, \ref{fig:pid3_fmincon_y2} i \ref{fig:pid3_fmincon_y3}.

Dla toru:
\begin{itemize}
  \item $y_1$ -- $u_2$
 \item $y_2$ -- $u_3$
 \item $y_3$ -- $u_1$
\end{itemize}

otrzymujemy nastawy:
\begin{equation}
  K_1 = \num{1.1001} \qquad T_{i1} = \num{1.7044}, \qquad T_{d1} = 0 \nonumber
\end{equation}
\begin{equation}
  K_2 = \num{2.2444} \qquad T_{i2} = \num{3.7464}, \qquad T_{d2} = 0
\end{equation}
\begin{equation}
  K_3 = \num{4.0054} \qquad T_{i3} = \num{15.7112}, \qquad T_{d3} = 0 \nonumber
\end{equation}

Błąd regulacji wynosi: \num{207.1231} .
Jest to wynik nieco gorszej jakości niż dla poprzednich dwóch torów.
Działanie
przedstawiają wykresy \ref{fig:pid4_fmincon_y1}, \ref{fig:pid4_fmincon_y2} i \ref{fig:pid4_fmincon_y3}.

Analizując otrzymane wyniki, zauważamy że dla wszystkich przypadków funkcja optymalizacji
znalazła lepsze wyniki niż dla regulatorów wyznaczonych metodą inżynierską. Jest to jak najbardziej normalne zjawisko.
Również potwierdziła się metoda Multiple Gain Array. Najlepsze wyniki otrzymaliśmy dla regulatora o konfiguracji
torów odpowiadającej najmniejszej wartości współczynnika uwarunkowania macierzy, zaś najgorsze wyniki
dla torów odpowiadających największej wartości współczynnika uwarunkowania macierzy.

%%%%%%%%%%%%%%%%%%%%%%%%%%%%%%%%%%%%%%%%%%%%%%%%%%%%%%%%%%%%%%%%%%%%%%%%%%%%%%%%%%%%%%%%%%%%
\begin{figure}[b]
\centering
\begin{tikzpicture}
\begin{axis}[
width=0.75\textwidth,
height = 0.4\textwidth,
xmin=0,xmax=1100,ymin=-15,ymax=15,
xlabel={Numer próbki},
ylabel={Wyjście},
xtick={0, 200, 400, 600, 800, 1000},
ytick={-15, -10, -5, 0, 5, 10, 15},
legend pos=north east,
/pgf/number format/.cd,
use comma,
1000 sep={}
]

\addplot[blue,semithick] file {wykresy/z3_yzad.txt};
\addplot[red,semithick] file {wykresy/pid1_fmincon_y1.txt};
\legend{Wartość zadana, Wyjście $y_1$}

\end{axis}
\end{tikzpicture}
\caption{Trajektoria wyjścia $y_1$, dla pierwszego zestawu regulatorów PID, dostrojonych funkcją optymalizacyjną}
\label{fig:pid1_fmincon_y1}
\end{figure}
%%%%%%%%%%%%%%%%%%%%%%%%%%%%%%%%%%%%%%%%%%%%%%%%%%%%%%%%%%%%%%%%%%%%%%%%%%%%%%%%%%%%%%%%%%%%

%%%%%%%%%%%%%%%%%%%%%%%%%%%%%%%%%%%%%%%%%%%%%%%%%%%%%%%%%%%%%%%%%%%%%%%%%%%%%%%%%%%%%%%%%%%%
\begin{figure}[b]
\centering
\begin{tikzpicture}
\begin{axis}[
width=0.75\textwidth,
height = 0.4\textwidth,
xmin=0,xmax=1100,ymin=-5,ymax=5,
xlabel={Numer próbki},
ylabel={Wyjście},
xtick={0, 200, 400, 600, 800, 1000},
ytick={-5, -4, -2, 0, 2, 4, 5},
legend pos=north east,
/pgf/number format/.cd,
use comma,
1000 sep={}
]

\addplot[blue,semithick] file {wykresy/z3_yzad.txt};
\addplot[red,semithick] file {wykresy/pid1_fmincon_y2.txt};
\legend{Wartość zadana, Wyjście $y_2$}

\end{axis}
\end{tikzpicture}
\caption{Trajektoria wyjścia $y_2$, dla pierwszego zestawu regulatorów PID, dostrojonych funkcją optymalizacyjną}
\label{fig:pid1_fmincon_y2}
\end{figure}
%%%%%%%%%%%%%%%%%%%%%%%%%%%%%%%%%%%%%%%%%%%%%%%%%%%%%%%%%%%%%%%%%%%%%%%%%%%%%%%%%%%%%%%%%%%%

%%%%%%%%%%%%%%%%%%%%%%%%%%%%%%%%%%%%%%%%%%%%%%%%%%%%%%%%%%%%%%%%%%%%%%%%%%%%%%%%%%%%%%%%%%%%
\begin{figure}[b]
\centering
\begin{tikzpicture}
\begin{axis}[
width=0.75\textwidth,
height = 0.4\textwidth,
xmin=0,xmax=1100,ymin=-2.5,ymax=2.5,
xlabel={Numer próbki},
ylabel={Wyjście},
xtick={0, 200, 400, 600, 800, 1000},
ytick={-2.5, -2, -1.5, -1, -.5, 0, .5, 1, 1.5, 2, 2.5},
legend pos=north east,
/pgf/number format/.cd,
use comma,
1000 sep={}
]

\addplot[blue,semithick] file {wykresy/z3_yzad.txt};
\addplot[red,semithick] file {wykresy/pid1_fmincon_y3.txt};
\legend{Wartość zadana, Wyjście $y_3$}

\end{axis}
\end{tikzpicture}
\caption{Trajektoria wyjścia $y_3$, dla pierwszego zestawu regulatorów PID, dostrojonych funkcją optymalizacyjną}
\label{fig:pid1_fmincon_y3}
\end{figure}
%%%%%%%%%%%%%%%%%%%%%%%%%%%%%%%%%%%%%%%%%%%%%%%%%%%%%%%%%%%%%%%%%%%%%%%%%%%%%%%%%%%%%%%%%%%%

%%%%%%%%%%%%%%%%%%%%%%%%%%%%%%%%%%%%%%%%%%%%%%%%%%%%%%%%%%%%%%%%%%%%%%%%%%%%%%%%%%%%%%%%%%%%
\begin{figure}[b]
\centering
\begin{tikzpicture}
\begin{axis}[
width=0.75\textwidth,
height = 0.4\textwidth,
xmin=0,xmax=1100,ymin=-15,ymax=15,
xlabel={Numer próbki},
ylabel={Wyjście},
xtick={0, 200, 400, 600, 800, 1000},
ytick={-15, -10, -5, 0, 5, 10, 15},
legend pos=north east,
/pgf/number format/.cd,
use comma,
1000 sep={}
]

\addplot[blue,semithick] file {wykresy/z3_yzad.txt};
\addplot[red,semithick] file {wykresy/pid2_fmincon_y1.txt};
\legend{Wartość zadana, Wyjście $y_1$}

\end{axis}
\end{tikzpicture}
\caption{Trajektoria wyjścia $y_1$, dla drugiego zestawu regulatorów PID, dostrojonych funkcją optymalizacyjną}
\label{fig:pid2_fmincon_y1}
\end{figure}
%%%%%%%%%%%%%%%%%%%%%%%%%%%%%%%%%%%%%%%%%%%%%%%%%%%%%%%%%%%%%%%%%%%%%%%%%%%%%%%%%%%%%%%%%%%%

%%%%%%%%%%%%%%%%%%%%%%%%%%%%%%%%%%%%%%%%%%%%%%%%%%%%%%%%%%%%%%%%%%%%%%%%%%%%%%%%%%%%%%%%%%%%
\begin{figure}[b]
\centering
\begin{tikzpicture}
\begin{axis}[
width=0.75\textwidth,
height = 0.4\textwidth,
xmin=0,xmax=1100,ymin=-5,ymax=5,
xlabel={Numer próbki},
ylabel={Wyjście},
xtick={0, 200, 400, 600, 800, 1000},
ytick={-5, -4, -2, 0, 2, 4, 5},
legend pos=north east,
/pgf/number format/.cd,
use comma,
1000 sep={}
]

\addplot[blue,semithick] file {wykresy/z3_yzad.txt};
\addplot[red,semithick] file {wykresy/pid2_fmincon_y2.txt};
\legend{Wartość zadana, Wyjście $y_2$}

\end{axis}
\end{tikzpicture}
\caption{Trajektoria wyjścia $y_2$, dla drugiego zestawu regulatorów PID, dostrojonych funkcją optymalizacyjną}
\label{fig:pid2_fmincon_y2}
\end{figure}
%%%%%%%%%%%%%%%%%%%%%%%%%%%%%%%%%%%%%%%%%%%%%%%%%%%%%%%%%%%%%%%%%%%%%%%%%%%%%%%%%%%%%%%%%%%%

%%%%%%%%%%%%%%%%%%%%%%%%%%%%%%%%%%%%%%%%%%%%%%%%%%%%%%%%%%%%%%%%%%%%%%%%%%%%%%%%%%%%%%%%%%%%
\begin{figure}[b]
\centering
\begin{tikzpicture}
\begin{axis}[
width=0.75\textwidth,
height = 0.4\textwidth,
xmin=0,xmax=1100,ymin=-2.5,ymax=2.5,
xlabel={Numer próbki},
ylabel={Wyjście},
xtick={0, 200, 400, 600, 800, 1000},
ytick={-2.5, -2, -1.5, -1, -.5, 0, .5, 1, 1.5, 2, 2.5},
legend pos=north east,
/pgf/number format/.cd,
use comma,
1000 sep={}
]

\addplot[blue,semithick] file {wykresy/z3_yzad.txt};
\addplot[red,semithick] file {wykresy/pid2_fmincon_y3.txt};
\legend{Wartość zadana, Wyjście $y_3$}

\end{axis}
\end{tikzpicture}
\caption{Trajektoria wyjścia $y_3$, dla drugiego zestawu regulatorów PID, dostrojonych funkcją optymalizacyjną}
\label{fig:pid2_fmincon_y3}
\end{figure}
%%%%%%%%%%%%%%%%%%%%%%%%%%%%%%%%%%%%%%%%%%%%%%%%%%%%%%%%%%%%%%%%%%%%%%%%%%%%%%%%%%%%%%%%%%%%

%%%%%%%%%%%%%%%%%%%%%%%%%%%%%%%%%%%%%%%%%%%%%%%%%%%%%%%%%%%%%%%%%%%%%%%%%%%%%%%%%%%%%%%%%%%%
\begin{figure}[b]
\centering
\begin{tikzpicture}
\begin{axis}[
width=0.75\textwidth,
height = 0.4\textwidth,
xmin=0,xmax=1100,ymin=-15,ymax=15,
xlabel={Numer próbki},
ylabel={Wyjście},
xtick={0, 200, 400, 600, 800, 1000},
ytick={-15, -10, -5, 0, 5, 10, 15},
legend pos=north east,
/pgf/number format/.cd,
use comma,
1000 sep={}
]

\addplot[blue,semithick] file {wykresy/z3_yzad.txt};
\addplot[red,semithick] file {wykresy/pid3_fmincon_y1.txt};
\legend{Wartość zadana, Wyjście $y_1$}

\end{axis}
\end{tikzpicture}
\caption{Trajektoria wyjścia $y_1$, dla trzeciego zestawu regulatorów PID, dostrojonych funkcją optymalizacyjną}
\label{fig:pid3_fmincon_y1}
\end{figure}
%%%%%%%%%%%%%%%%%%%%%%%%%%%%%%%%%%%%%%%%%%%%%%%%%%%%%%%%%%%%%%%%%%%%%%%%%%%%%%%%%%%%%%%%%%%%

%%%%%%%%%%%%%%%%%%%%%%%%%%%%%%%%%%%%%%%%%%%%%%%%%%%%%%%%%%%%%%%%%%%%%%%%%%%%%%%%%%%%%%%%%%%%
\begin{figure}[b]
\centering
\begin{tikzpicture}
\begin{axis}[
width=0.75\textwidth,
height = 0.4\textwidth,
xmin=0,xmax=1100,ymin=-5,ymax=5,
xlabel={Numer próbki},
ylabel={Wyjście},
xtick={0, 200, 400, 600, 800, 1000},
ytick={-5, -4, -2, 0, 2, 4, 5},
legend pos=north east,
/pgf/number format/.cd,
use comma,
1000 sep={}
]

\addplot[blue,semithick] file {wykresy/z3_yzad.txt};
\addplot[red,semithick] file {wykresy/pid3_fmincon_y2.txt};
\legend{Wartość zadana, Wyjście $y_2$}

\end{axis}
\end{tikzpicture}
\caption{Trajektoria wyjścia $y_2$, dla trzeciego zestawu regulatorów PID, dostrojonych funkcją optymalizacyjną}
\label{fig:pid3_fmincon_y2}
\end{figure}
%%%%%%%%%%%%%%%%%%%%%%%%%%%%%%%%%%%%%%%%%%%%%%%%%%%%%%%%%%%%%%%%%%%%%%%%%%%%%%%%%%%%%%%%%%%%

%%%%%%%%%%%%%%%%%%%%%%%%%%%%%%%%%%%%%%%%%%%%%%%%%%%%%%%%%%%%%%%%%%%%%%%%%%%%%%%%%%%%%%%%%%%%
\begin{figure}[b]
\centering
\begin{tikzpicture}
\begin{axis}[
width=0.75\textwidth,
height = 0.4\textwidth,
xmin=0,xmax=1100,ymin=-2.5,ymax=2.5,
xlabel={Numer próbki},
ylabel={Wyjście},
xtick={0, 200, 400, 600, 800, 1000},
ytick={-2.5, -2, -1.5, -1, -.5, 0, .5, 1, 1.5, 2, 2.5},
legend pos=north east,
/pgf/number format/.cd,
use comma,
1000 sep={}
]

\addplot[blue,semithick] file {wykresy/z3_yzad.txt};
\addplot[red,semithick] file {wykresy/pid3_fmincon_y3.txt};
\legend{Wartość zadana, Wyjście $y_3$}

\end{axis}
\end{tikzpicture}
\caption{Trajektoria wyjścia $y_3$, dla trzeciego zestawu regulatorów PID, dostrojonych funkcją optymalizacyjną}
\label{fig:pid3_fmincon_y3}
\end{figure}
%%%%%%%%%%%%%%%%%%%%%%%%%%%%%%%%%%%%%%%%%%%%%%%%%%%%%%%%%%%%%%%%%%%%%%%%%%%%%%%%%%%%%%%%%%%%

%%%%%%%%%%%%%%%%%%%%%%%%%%%%%%%%%%%%%%%%%%%%%%%%%%%%%%%%%%%%%%%%%%%%%%%%%%%%%%%%%%%%%%%%%%%%
\begin{figure}[b]
\centering
\begin{tikzpicture}
\begin{axis}[
width=0.75\textwidth,
height = 0.4\textwidth,
xmin=0,xmax=1100,ymin=-15,ymax=15,
xlabel={Numer próbki},
ylabel={Wyjście},
xtick={0, 200, 400, 600, 800, 1000},
ytick={-15, -10, -5, 0, 5, 10, 15},
legend pos=north east,
/pgf/number format/.cd,
use comma,
1000 sep={}
]

\addplot[blue,semithick] file {wykresy/z3_yzad.txt};
\addplot[red,semithick] file {wykresy/pid4_fmincon_y1.txt};
\legend{Wartość zadana, Wyjście $y_1$}

\end{axis}
\end{tikzpicture}
\caption{Trajektoria wyjścia $y_1$, dla czwartego zestawu regulatorów PID, dostrojonych funkcją optymalizacyjną}
\label{fig:pid4_fmincon_y1}
\end{figure}
%%%%%%%%%%%%%%%%%%%%%%%%%%%%%%%%%%%%%%%%%%%%%%%%%%%%%%%%%%%%%%%%%%%%%%%%%%%%%%%%%%%%%%%%%%%%

%%%%%%%%%%%%%%%%%%%%%%%%%%%%%%%%%%%%%%%%%%%%%%%%%%%%%%%%%%%%%%%%%%%%%%%%%%%%%%%%%%%%%%%%%%%%
\begin{figure}[b]
\centering
\begin{tikzpicture}
\begin{axis}[
width=0.75\textwidth,
height = 0.4\textwidth,
xmin=0,xmax=1100,ymin=-5,ymax=5,
xlabel={Numer próbki},
ylabel={Wyjście},
xtick={0, 200, 400, 600, 800, 1000},
ytick={-5, -4, -2, 0, 2, 4, 5},
legend pos=north east,
/pgf/number format/.cd,
use comma,
1000 sep={}
]

\addplot[blue,semithick] file {wykresy/z3_yzad.txt};
\addplot[red,semithick] file {wykresy/pid4_fmincon_y2.txt};
\legend{Wartość zadana, Wyjście $y_2$}

\end{axis}
\end{tikzpicture}
\caption{Trajektoria wyjścia $y_2$, dla czwartego zestawu regulatorów PID, dostrojonych funkcją optymalizacyjną}
\label{fig:pid4_fmincon_y2}
\end{figure}
%%%%%%%%%%%%%%%%%%%%%%%%%%%%%%%%%%%%%%%%%%%%%%%%%%%%%%%%%%%%%%%%%%%%%%%%%%%%%%%%%%%%%%%%%%%%

%%%%%%%%%%%%%%%%%%%%%%%%%%%%%%%%%%%%%%%%%%%%%%%%%%%%%%%%%%%%%%%%%%%%%%%%%%%%%%%%%%%%%%%%%%%%
\begin{figure}[b]
\centering
\begin{tikzpicture}
\begin{axis}[
width=0.75\textwidth,
height = 0.4\textwidth,
xmin=0,xmax=1100,ymin=-2.5,ymax=2.5,
xlabel={Numer próbki},
ylabel={Wyjście},
xtick={0, 200, 400, 600, 800, 1000},
ytick={-2.5, -2, -1.5, -1, -.5, 0, .5, 1, 1.5, 2, 2.5},
legend pos=north east,
/pgf/number format/.cd,
use comma,
1000 sep={}
]

\addplot[blue,semithick] file {wykresy/z3_yzad.txt};
\addplot[red,semithick] file {wykresy/pid4_fmincon_y3.txt};
\legend{Wartość zadana, Wyjście $y_3$}

\end{axis}
\end{tikzpicture}
\caption{Trajektoria wyjścia $y_3$, dla czwartego zestawu regulatorów PID, dostrojonych funkcją optymalizacyjną}
\label{fig:pid4_fmincon_y3}
\end{figure}
%%%%%%%%%%%%%%%%%%%%%%%%%%%%%%%%%%%%%%%%%%%%%%%%%%%%%%%%%%%%%%%%%%%%%%%%%%%%%%%%%%%%%%%%%%%%


\end{document}
