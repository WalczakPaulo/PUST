\documentclass[a4paper,titlepage,11pt,twosides,floatssmall]{mwrep}
\usepackage[left=2.5cm,right=2.5cm,top=2.5cm,bottom=2.5cm]{geometry}
\usepackage[OT1]{fontenc}
\usepackage{polski}
\usepackage{amsmath}
\usepackage{xr}
\usepackage{amsfonts}
\usepackage{amssymb}
\usepackage{graphicx}
\usepackage{url}
\usepackage{float}
\usepackage[section]{placeins}
\usepackage{tikz}
\usetikzlibrary{arrows,calc,decorations.markings,math,arrows.meta}
\usepackage{rotating}
\usepackage[percent]{overpic}
\usepackage[utf8]{inputenc}
\usepackage{xcolor}
\usepackage{pgfplots}
\usetikzlibrary{pgfplots.groupplots}
\usepackage{listings}
\usepackage{matlab-prettifier}
\usepackage{siunitx}
\definecolor{szary}{rgb}{0.95,0.95,0.95}
\sisetup{detect-weight,exponent-product=\cdot,output-decimal-marker={,},per-mode=symbol,binary-units=true,range-phrase={-},range-units=single}

%konfiguracje pakietu listings
\lstset{
	backgroundcolor=\color{szary},
	frame=single,
	breaklines=true,
}
\lstdefinestyle{customlatex}{
	basicstyle=\footnotesize\ttfamily,
	%basicstyle=\small\ttfamily,
}
\lstdefinestyle{customc}{
	breaklines=true,
	frame=tb,
	language=C,
	xleftmargin=0pt,
	showstringspaces=false,
	basicstyle=\small\ttfamily,
	keywordstyle=\bfseries\color{green!40!black},
	commentstyle=\itshape\color\usepackage{float}{purple!40!black},
	identifierstyle=\color{blue},
	stringstyle=\color{orange},
}
\lstdefinestyle{custommatlab}{
	captionpos=t,
	breaklines=true,
	frame=tb,
	xleftmargin=0pt,
	language=matlab,
	showstringspaces=false,
	%basicstyle=\footnotesize\ttfamily,
	basicstyle=\scriptsize\ttfamily,
	keywordstyle=\bfseries\color{green!40!black},
	commentstyle=\itshape\color{purple!40!black},
	identifierstyle=\color{blue},
	stringstyle=\color{orange},
}

%wymiar tekstu (bez �ywej paginy)
\textwidth 160mm \textheight 247mm

%ustawienia pakietu pgfplots
\pgfplotsset{
tick label style={font=\scriptsize},
label style={font=\small},
legend style={font=\small},
title style={font=\small},
every axis plot post/.append style={semithick,mark=none}
}

\def\figurename{Rys.}
\def\tablename{Tab.}

%konfiguracja liczby p�ywaj�cych element�w
\setcounter{topnumber}{0}%2
\setcounter{bottomnumber}{3}%1
\setcounter{totalnumber}{5}%3
\renewcommand{\textfraction}{0.01}%0.2
\renewcommand{\topfraction}{0.95}%0.7
\renewcommand{\bottomfraction}{0.95}%0.3
\renewcommand{\floatpagefraction}{0.35}%0.5

\begin{document}
\frenchspacing
\pagestyle{uheadings}

%strona tytu�owa
\title{\bf Sprawozdanie z projektu nr 5, zadanie nr 5\vskip 0.1cm}
\author{Kamil Gabryjelski, Paweł Rybak, Paweł Walczak}
\date{2017}

\makeatletter
\renewcommand{\maketitle}{\begin{titlepage}
\begin{center}{\LARGE {\bf
Wydział Elektroniki i Technik Informacyjnych}}\\
\vspace{0.4cm}
{\LARGE {\bf Politechnika Warszawska}}\\
\vspace{0.3cm}
\end{center}
\vspace{5cm}
\begin{center}
{\bf \LARGE Projektowanie układów sterowania\\ (projekt grupowy) \vskip 0.1cm}
\end{center}
\vspace{1cm}
\begin{center}
{\bf \LARGE \@title}
\end{center}
\vspace{2cm}
\begin{center}
{\bf \Large \@author \par}
\end{center}
\vspace*{\stretch{6}}
\begin{center}
\bf{\large{Warszawa, \@date\vskip 0.1cm}}
\end{center}
\end{titlepage}
}
\makeatother

\maketitle
\tableofcontents
\chapter{DMC}
Na podstawie analizy odpowiedzi skokowych przyjęliśmy horyzont dynamiki $D=80$. Przez $E_i$ oznaczyliśmy wartość wskaźnika błędu dla wyjścia $i$, natomiast $E$ jest sumą błędów dla wszystkich wyjść obiektu.

\section{Dobór horyzontów predykcji i sterowania}
Dobór horyzontów przeprowadzaliśmy korzystając z parametrów $\psi$ i $\lambda$ równymi 1. Rozpoczęliśmy od nastaw $N=N_u=D=80$. Dla tych parametrów błędy wynosiły:
\begin{itemize}
\item $E_1=\num{45,0725}$
\item $E_2=\num{45,9624}$
\item $E_3=\num{24,6561}$
\item $E=\num{115,6910}$
\end{itemize}
Przebieg wyjść obiektu przedstawia wykres \ref{fig:dmc_y_N80_Nu80}, a sterowań wykres \ref{fig:dmc_u_N80_Nu80}.

Postanowiliśmy skrócić horyzonty do wartości $N=N_u=50$. Otrzymane błędy wyniosły:
\begin{itemize}
\item $E_1=\num{45,0726}$
\item $E_2=\num{45,962}$
\item $E_3=\num{24,6562}$
\item $E=\num{115,6908}$
\end{itemize}
Błędy regulacji były więc praktycznie jednakowe jak dla dłuższych horyzontów. Przebiegi wyjść i sterowań przedstawiają wykresy \ref{fig:dmc_y_N50_Nu50} i \ref{fig:dmc_u_N50_Nu50}.

W kolejnym kroku skróciliśmy horyzont predykcji do wartości $N=40$, a sterowania  $N_u=10$. Taka zmiana przyniosła niewielką poprawę wskaźników błędu:
\begin{itemize}
\item $E_1=\num{45,0801}$
\item $E_2=\num{45,933}$
\item $E_3=\num{24,6021}$
\item $E=\num{115,6152}$
\end{itemize}
Przebieg wyjść obiektu przedstawiają wykresy \ref{fig:dmc_y_N40_Nu10} i \ref{fig:dmc_u_N40_Nu10}.

Jak się okazało, dalsze skracanie horyzontu sterowania przyniosło znacznie bardziej wymierne rezultaty - dla $N_u=5$  wskaźniki błędów zmalały do wartości:
\begin{itemize}
\item $E_1=\num{44,4289}$
\item $E_2=\num{44,1988}$
\item $E_3=\num{23,0761}$
\item $E=\num{111,7038}$
\end{itemize}
Przebieg wyjść obiektu przedstawiają wykresy \ref{fig:dmc_y_N40_Nu5} i \ref{fig:dmc_u_N40_Nu5}.

Dalsze skracanie horyzontu predykcji nie przyniosło pozytywnych rezultatów. Dla $N_u=2$ wskaźniki błędów wyniosły:
\begin{itemize}
\item $E_1=\num{46,6684}$
\item $E_2=\num{50,4711}$
\item $E_3=\num{27,023}$
\item $E=\num{124,1625}$
\end{itemize}
Można więc przypuszczać, że jeszcze mniejsze wartości horyzontu sterowania przyniosłyby pogorszenie jakości regulacji. Przebieg wyjść obiektu przedstawiają wykresy \ref{fig:dmc_y_N40_Nu2} i \ref{fig:dmc_u_N40_Nu2}.

W kolejnych zadaniach używane będą horyzonty $N=40$ i $N_u=5$.

\begin{figure}
	\centering
	\begin{tikzpicture}
	\begin{groupplot}[group style={group size=1 by 3,vertical sep={1.5 cm}},
	width=0.9\textwidth,height=0.3\textwidth,xmin=0]
	\nextgroupplot
	[
	xlabel={$k$},
	ylabel={$y_1$},
	y tick label style={/pgf/number format/1000 sep=},
	]
	\addplot[blue,semithick] file {wykresy/dmc_y1_N80_Nu80.txt};
	\addplot[red,semithick,densely dashed] file {wykresy/yzad1.txt};
	\legend{$y_1$,$y_{zad}$}
	\nextgroupplot
	[
	xlabel={$k$},
	ylabel={$y_2$},
	y tick label style={/pgf/number format/1000 sep=},
	]
	\addplot[blue,semithick] file {wykresy/dmc_y2_N80_Nu80.txt};
	\addplot[red,semithick,densely dashed] file {wykresy/yzad2.txt};
	\legend{$y_2$,$y_{zad}$}
	\nextgroupplot
	[
	xlabel={$k$},
	ylabel={$y_3$},
	y tick label style={/pgf/number format/1000 sep=},
	]
	\addplot[blue,semithick] file {wykresy/dmc_y3_N80_Nu80.txt};
	\addplot[red,semithick,densely dashed] file {wykresy/yzad3.txt};
	\legend{$y_3$,$y_{zad}$}
	\end{groupplot}
	\end{tikzpicture}
	\caption{Przebiegi wyjść obiektu dla horyzontów predykcji i sterowania $N=80$, $N_u=80$.}
	\label{fig:dmc_y_N80_Nu80}
\end{figure}

\begin{figure}
	\centering
	\begin{tikzpicture}
	\begin{groupplot}[group style={group size=1 by 4,vertical sep={1.5 cm}},
	width=0.9\textwidth,height=0.25\textwidth,xmin=0]
	\nextgroupplot
	[
	xlabel={$k$},
	ylabel={$u_1$},
	y tick label style={/pgf/number format/1000 sep=},
	]
	\addplot[blue,semithick] file {wykresy/dmc_y1_N80_Nu80.txt};

	\nextgroupplot
	[
	xlabel={$k$},
	ylabel={$u_2$},
	y tick label style={/pgf/number format/1000 sep=},
	]
	\addplot[blue,semithick] file {wykresy/dmc_u2_N80_Nu80.txt};
	
	\nextgroupplot
	[
	xlabel={$k$},
	ylabel={$u_3$},
	y tick label style={/pgf/number format/1000 sep=},
	]
	\addplot[blue,semithick] file {wykresy/dmc_u3_N80_Nu80.txt};
	
	\nextgroupplot
	[
	xlabel={$k$},
	ylabel={$u_4$},
	y tick label style={/pgf/number format/1000 sep=},
	]
	\addplot[blue,semithick] file {wykresy/dmc_u4_N80_Nu80.txt};
	\end{groupplot}
	\end{tikzpicture}
	\caption{Przebiegi sterowań obiektu dla horyzontów predykcji i sterowania $N=80$, $N_u=80$.}
	\label{fig:dmc_u_N80_Nu80}
\end{figure}

\begin{figure}
	\centering
	\begin{tikzpicture}
	\begin{groupplot}[group style={group size=1 by 3,vertical sep={1.5 cm}},
	width=0.9\textwidth,height=0.3\textwidth,xmin=0]
	\nextgroupplot
	[
	xlabel={$k$},
	ylabel={$y_1$},
	y tick label style={/pgf/number format/1000 sep=},
	]
	\addplot[blue,semithick] file {wykresy/dmc_y1_N50_Nu50.txt};
	\addplot[red,semithick,densely dashed] file {wykresy/yzad1.txt};
	\legend{$y_1$,$y_{zad}$}
	\nextgroupplot
	[
	xlabel={$k$},
	ylabel={$y_2$},
	y tick label style={/pgf/number format/1000 sep=},
	]
	\addplot[blue,semithick] file {wykresy/dmc_y2_N50_Nu50.txt};
	\addplot[red,semithick,densely dashed] file {wykresy/yzad2.txt};
	\legend{$y_2$,$y_{zad}$}
	\nextgroupplot
	[
	xlabel={$k$},
	ylabel={$y_3$},
	y tick label style={/pgf/number format/1000 sep=},
	]
	\addplot[blue,semithick] file {wykresy/dmc_y3_N50_Nu50.txt};
	\addplot[red,semithick,densely dashed] file {wykresy/yzad3.txt};
	\legend{$y_3$,$y_{zad}$}
	\end{groupplot}
	\end{tikzpicture}
	\caption{Przebiegi wyjść obiektu dla horyzontów predykcji i sterowania $N=50$, $N_u=50$.}
	\label{fig:dmc_y_N50_Nu50}
\end{figure}

\begin{figure}
	\centering
	\begin{tikzpicture}
	\begin{groupplot}[group style={group size=1 by 4,vertical sep={1.5 cm}},
	width=0.9\textwidth,height=0.25\textwidth,xmin=0]
	\nextgroupplot
	[
	xlabel={$k$},
	ylabel={$u_1$},
	y tick label style={/pgf/number format/1000 sep=},
	]
	\addplot[blue,semithick] file {wykresy/dmc_y1_N50_Nu50.txt};

	\nextgroupplot
	[
	xlabel={$k$},
	ylabel={$u_2$},
	y tick label style={/pgf/number format/1000 sep=},
	]
	\addplot[blue,semithick] file {wykresy/dmc_u2_N50_Nu50.txt};
	
	\nextgroupplot
	[
	xlabel={$k$},
	ylabel={$u_3$},
	y tick label style={/pgf/number format/1000 sep=},
	]
	\addplot[blue,semithick] file {wykresy/dmc_u3_N50_Nu50.txt};
	
	\nextgroupplot
	[
	xlabel={$k$},
	ylabel={$u_4$},
	y tick label style={/pgf/number format/1000 sep=},
	]
	\addplot[blue,semithick] file {wykresy/dmc_u4_N50_Nu50.txt};
	\end{groupplot}
	\end{tikzpicture}
	\caption{Przebiegi sterowań obiektu dla horyzontów predykcji i sterowania $N=50$, $N_u=50$.}
	\label{fig:dmc_u_N50_Nu50}
\end{figure}

\begin{figure}
	\centering
	\begin{tikzpicture}
	\begin{groupplot}[group style={group size=1 by 3,vertical sep={1.5 cm}},
	width=0.9\textwidth,height=0.3\textwidth,xmin=0]
	\nextgroupplot
	[
	xlabel={$k$},
	ylabel={$y_1$},
	y tick label style={/pgf/number format/1000 sep=},
	]
	\addplot[blue,semithick] file {wykresy/dmc_y1_N40_Nu10.txt};
	\addplot[red,semithick,densely dashed] file {wykresy/yzad1.txt};
	\legend{$y_1$,$y_{zad}$}
	\nextgroupplot
	[
	xlabel={$k$},
	ylabel={$y_2$},
	y tick label style={/pgf/number format/1000 sep=},
	]
	\addplot[blue,semithick] file {wykresy/dmc_y2_N40_Nu10.txt};
	\addplot[red,semithick,densely dashed] file {wykresy/yzad2.txt};
	\legend{$y_2$,$y_{zad}$}
	\nextgroupplot
	[
	xlabel={$k$},
	ylabel={$y_3$},
	y tick label style={/pgf/number format/1000 sep=},
	]
	\addplot[blue,semithick] file {wykresy/dmc_y3_N40_Nu10.txt};
	\addplot[red,semithick,densely dashed] file {wykresy/yzad3.txt};
	\legend{$y_3$,$y_{zad}$}
	\end{groupplot}
	\end{tikzpicture}
	\caption{Przebiegi wyjść obiektu dla horyzontów predykcji i sterowania $N=40$, $N_u=10$.}
	\label{fig:dmc_y_N40_Nu10}
\end{figure}

\begin{figure}
	\centering
	\begin{tikzpicture}
	\begin{groupplot}[group style={group size=1 by 4,vertical sep={1.5 cm}},
	width=0.9\textwidth,height=0.25\textwidth,xmin=0]
	\nextgroupplot
	[
	xlabel={$k$},
	ylabel={$u_1$},
	y tick label style={/pgf/number format/1000 sep=},
	]
	\addplot[blue,semithick] file {wykresy/dmc_y1_N40_Nu10.txt};

	\nextgroupplot
	[
	xlabel={$k$},
	ylabel={$u_2$},
	y tick label style={/pgf/number format/1000 sep=},
	]
	\addplot[blue,semithick] file {wykresy/dmc_u2_N40_Nu10.txt};
	
	\nextgroupplot
	[
	xlabel={$k$},
	ylabel={$u_3$},
	y tick label style={/pgf/number format/1000 sep=},
	]
	\addplot[blue,semithick] file {wykresy/dmc_u3_N40_Nu10.txt};
	
	\nextgroupplot
	[
	xlabel={$k$},
	ylabel={$u_4$},
	y tick label style={/pgf/number format/1000 sep=},
	]
	\addplot[blue,semithick] file {wykresy/dmc_u4_N40_Nu10.txt};
	\end{groupplot}
	\end{tikzpicture}
	\caption{Przebiegi sterowań obiektu dla horyzontów predykcji i sterowania $N=40$, $N_u=10$.}
	\label{fig:dmc_u_N40_Nu10}
\end{figure}


\begin{figure}
	\centering
	\begin{tikzpicture}
	\begin{groupplot}[group style={group size=1 by 3,vertical sep={1.5 cm}},
	width=0.9\textwidth,height=0.3\textwidth,xmin=0]
	\nextgroupplot
	[
	xlabel={$k$},
	ylabel={$y_1$},
	y tick label style={/pgf/number format/1000 sep=},
	]
	\addplot[blue,semithick] file {wykresy/dmc_y1_N40_Nu5.txt};
	\addplot[red,semithick,densely dashed] file {wykresy/yzad1.txt};
	\legend{$y_1$,$y_{zad}$}
	\nextgroupplot
	[
	xlabel={$k$},
	ylabel={$y_2$},
	y tick label style={/pgf/number format/1000 sep=},
	]
	\addplot[blue,semithick] file {wykresy/dmc_y2_N40_Nu5.txt};
	\addplot[red,semithick,densely dashed] file {wykresy/yzad2.txt};
	\legend{$y_2$,$y_{zad}$}
	\nextgroupplot
	[
	xlabel={$k$},
	ylabel={$y_3$},
	y tick label style={/pgf/number format/1000 sep=},
	]
	\addplot[blue,semithick] file {wykresy/dmc_y3_N40_Nu5.txt};
	\addplot[red,semithick,densely dashed] file {wykresy/yzad3.txt};
	\legend{$y_3$,$y_{zad}$}
	\end{groupplot}
	\end{tikzpicture}
	\caption{Przebiegi wyjść obiektu dla horyzontów predykcji i sterowania $N=40$, $N_u=5$.}
	\label{fig:dmc_y_N40_Nu5}
\end{figure}

\begin{figure}
	\centering
	\begin{tikzpicture}
	\begin{groupplot}[group style={group size=1 by 4,vertical sep={1.5 cm}},
	width=0.9\textwidth,height=0.25\textwidth,xmin=0]
	\nextgroupplot
	[
	xlabel={$k$},
	ylabel={$u_1$},
	y tick label style={/pgf/number format/1000 sep=},
	]
	\addplot[blue,semithick] file {wykresy/dmc_y1_N40_Nu5.txt};

	\nextgroupplot
	[
	xlabel={$k$},
	ylabel={$u_2$},
	y tick label style={/pgf/number format/1000 sep=},
	]
	\addplot[blue,semithick] file {wykresy/dmc_u2_N40_Nu5.txt};
	
	\nextgroupplot
	[
	xlabel={$k$},
	ylabel={$u_3$},
	y tick label style={/pgf/number format/1000 sep=},
	]
	\addplot[blue,semithick] file {wykresy/dmc_u3_N40_Nu5.txt};
	
	\nextgroupplot
	[
	xlabel={$k$},
	ylabel={$u_4$},
	y tick label style={/pgf/number format/1000 sep=},
	]
	\addplot[blue,semithick] file {wykresy/dmc_u4_N40_Nu5.txt};
	\end{groupplot}
	\end{tikzpicture}
	\caption{Przebiegi sterowań obiektu dla horyzontów predykcji i sterowania $N=40$, $N_u=5$.}
	\label{fig:dmc_u_N40_Nu5}
\end{figure}

\begin{figure}
	\centering
	\begin{tikzpicture}
	\begin{groupplot}[group style={group size=1 by 3,vertical sep={1.5 cm}},
	width=0.9\textwidth,height=0.3\textwidth,xmin=0]
	\nextgroupplot
	[
	xlabel={$k$},
	ylabel={$y_1$},
	y tick label style={/pgf/number format/1000 sep=},
	]
	\addplot[blue,semithick] file {wykresy/dmc_y1_N40_Nu2.txt};
	\addplot[red,semithick,densely dashed] file {wykresy/yzad1.txt};
	\legend{$y_1$,$y_{zad}$}
	\nextgroupplot
	[
	xlabel={$k$},
	ylabel={$y_2$},
	y tick label style={/pgf/number format/1000 sep=},
	]
	\addplot[blue,semithick] file {wykresy/dmc_y2_N40_Nu2.txt};
	\addplot[red,semithick,densely dashed] file {wykresy/yzad2.txt};
	\legend{$y_2$,$y_{zad}$}
	\nextgroupplot
	[
	xlabel={$k$},
	ylabel={$y_3$},
	y tick label style={/pgf/number format/1000 sep=},
	]
	\addplot[blue,semithick] file {wykresy/dmc_y3_N40_Nu2.txt};
	\addplot[red,semithick,densely dashed] file {wykresy/yzad3.txt};
	\legend{$y_3$,$y_{zad}$}
	\end{groupplot}
	\end{tikzpicture}
	\caption{Przebiegi wyjść obiektu dla horyzontów predykcji i sterowania $N=40$, $N_u=5$.}
	\label{fig:dmc_y_N40_Nu2}
\end{figure}

\begin{figure}
	\centering
	\begin{tikzpicture}
	\begin{groupplot}[group style={group size=1 by 4,vertical sep={1.5 cm}},
	width=0.9\textwidth,height=0.25\textwidth,xmin=0]
	\nextgroupplot
	[
	xlabel={$k$},
	ylabel={$u_1$},
	y tick label style={/pgf/number format/1000 sep=},
	]
	\addplot[blue,semithick] file {wykresy/dmc_y1_N40_Nu2.txt};

	\nextgroupplot
	[
	xlabel={$k$},
	ylabel={$u_2$},
	y tick label style={/pgf/number format/1000 sep=},
	]
	\addplot[blue,semithick] file {wykresy/dmc_u2_N40_Nu2.txt};
	
	\nextgroupplot
	[
	xlabel={$k$},
	ylabel={$u_3$},
	y tick label style={/pgf/number format/1000 sep=},
	]
	\addplot[blue,semithick] file {wykresy/dmc_u3_N40_Nu2.txt};
	
	\nextgroupplot
	[
	xlabel={$k$},
	ylabel={$u_4$},
	y tick label style={/pgf/number format/1000 sep=},
	]
	\addplot[blue,semithick] file {wykresy/dmc_u4_N40_Nu2.txt};
	\end{groupplot}
	\end{tikzpicture}
	\caption{Przebiegi sterowań obiektu dla horyzontów predykcji i sterowania $N=40$, $N_u=2$.}
	\label{fig:dmc_u_N40_Nu2}
\end{figure}
\chapter{Parametry $\lambda$ i $\psi$}
\section{Parametr $\lambda$}
W wyniku testowania różnych wartości współczynnika $\lambda$ zaobserwowaliśmy, że błąd regulacji jest najmniejszy dla bardzo małych wartości $\lambda$. Trzeba jednak zauważyć, że niskie wartości parametru powodują, że przebieg sterowania jest znacznie "ostrzejszy", występują duże i nagłe skoki $u$ przy zmianach wartości zadanej. W przypadku rzeczywistego obiektu, zjawisko to mogłoby mieć negatywny efekt, na przykład uszkodzenie części sterujących. Staraliśmy się więc doprowadzić do kompromisu między niskim wskaźnikiem błędu a łagodnym przebiegiem sterowania.

Testując różne wartości parametru $\lambda$, przyjęliśmy długości horyzontów $N=40$ i $N_u=5$, a parametry $\psi = 1$.

Próba zwiększenia wartości parametrów $\lambda$ do wartości 2 okazała się przynosić znacznie wyższe współczynniki błędu.
\begin{itemize}
\item $E_1=\num{49,4821}$
\item $E_2=\num{49,361}$
\item $E_3=\num{28,0998}$
\item $E=\num{126,9428}$
\end{itemize}
Zdecydowaliśmy więc w kolejnych testach skupić się na parametrach $\lambda$ poniżej 1. Przebiegi wyjść i sterowań przedstawiają wykresy \ref{fig:dmc_y_l_2_2_2_2_psi_1_1_1} i \ref{fig:dmc_u_l_2_2_2_2_psi_1_1_1}.

Ustawienie parametrów na wartości $\lambda_1=\lambda_2=\lambda_3=\lambda_4=\num{0,2}$ dało w rezultacie bardzo dużą poprawę wskaźników błędu regulacji.
\begin{itemize}
\item $E_1=\num{37,0136}$
\item $E_2=\num{37,5058}$
\item $E_3=\num{14,446}$
\item $E=\num{88,9654}$
\end{itemize}
Charakterystykę sterowania uznaliśmy za akceptowalną. Przebiegi wyjść i sterowań przedstawiają wykresy \ref{fig:dmc_y_l_02_02_02_02_psi_1_1_1} i \ref{fig:dmc_u_l_02_02_02_02_psi_1_1_1}.

 Można dostrzec, że skoki sterowania na torach 1 i 4 osiągają znacznie większe wartości, niż na torach 2 i 3. Tor sterowania 3 natomiast ma łagodniejszy przebieg niż pozostałe. Z tego powodu przetestujemy, jak zachowuje się obiekt w przypadku, gdy parametry $\lambda_1$ i $\lambda_4$ mają wyższe wartości niż $\lambda_2$, a $\lambda_3$ ma niższą wartość. W ten sposób tory sterowania 1 i 4 powinny zostać złagodzone, a tor 3 przyspieszony. 

Przyjęliśmy parametry o następujących wartościach:
\begin{itemize}
\item $\lambda_1=\num{0,3}$
\item $\lambda_2=\num{0,2}$
\item $\lambda_3=\num{0,1}$
\item $\lambda_4=\num{0,3}$
\end{itemize}
Błędy regulacji:
\begin{itemize}
\item $E_1=\num{36,357}$
\item $E_2=\num{38,3743}$
\item $E_3=\num{16,2287}$
\item $E=\num{90,9600}$
\end{itemize}
Jak widać odnotowaliśmy nieznaczne pogorszenie jakości regulacji. Można jednak zaobserwować na wykresie sterowań \ref{fig:dmc_u_l_03_02_01_03_psi_1_1_1}, że tory 1 i 4 mają łagodniejsze przebiegi. Uznaliśmy więc, że te wartości $\lambda$ są w naszym przypadku optymalne. Przebiegi wyjść przedstawia wykres \ref{fig:dmc_y_l_03_02_01_03_psi_1_1_1}.

W kolejnych testach używane będą parametry $\lambda$ o wartościach:
\begin{itemize}
\item $\lambda_1=\num{0,3}$
\item $\lambda_2=\num{0,2}$
\item $\lambda_3=\num{0,1}$
\item $\lambda_4=\num{0,3}$
\end{itemize}

\begin{figure}
	\centering
	\begin{tikzpicture}
	\begin{groupplot}[group style={group size=1 by 3,vertical sep={1.5 cm}},
	width=0.9\textwidth,height=0.3\textwidth,xmin=0]
	\nextgroupplot
	[
	xlabel={$k$},
	ylabel={$y_1$},
	y tick label style={/pgf/number format/1000 sep=},
	]
	\addplot[blue,semithick] file {wykresy/dmc_y1_l_2_2_2_2_psi_1_1_1.txt};
	\addplot[red,semithick,densely dashed] file {wykresy/yzad1.txt};
	\legend{$y_1$,$y_{zad}$}
	\nextgroupplot
	[
	xlabel={$k$},
	ylabel={$y_2$},
	y tick label style={/pgf/number format/1000 sep=},
	]
	\addplot[blue,semithick] file {wykresy/dmc_y2_l_2_2_2_2_psi_1_1_1.txt};
	\addplot[red,semithick,densely dashed] file {wykresy/yzad2.txt};
	\legend{$y_2$,$y_{zad}$}
	\nextgroupplot
	[
	xlabel={$k$},
	ylabel={$y_3$},
	y tick label style={/pgf/number format/1000 sep=},
	]
	\addplot[blue,semithick] file {wykresy/dmc_y3_l_2_2_2_2_psi_1_1_1.txt};
	\addplot[red,semithick,densely dashed] file {wykresy/yzad3.txt};
	\legend{$y_3$,$y_{zad}$}
	\end{groupplot}
	\end{tikzpicture}
	\caption{Przebiegi wyjść obiektu dla $\lambda_1=2$, $\lambda_2=2$, $\lambda_3=2$ i $\lambda_4=2$.}
	\label{fig:dmc_y_l_2_2_2_2_psi_1_1_1}
\end{figure}

\begin{figure}
	\centering
	\begin{tikzpicture}
	\begin{groupplot}[group style={group size=1 by 4,vertical sep={1.5 cm}},
	width=0.9\textwidth,height=0.25\textwidth,xmin=0]
	\nextgroupplot
	[
	xlabel={$k$},
	ylabel={$u_1$},
	y tick label style={/pgf/number format/1000 sep=},
	]
	\addplot[blue,semithick] file {wykresy/dmc_u1_l_2_2_2_2_psi_1_1_1.txt};

	\nextgroupplot
	[
	xlabel={$k$},
	ylabel={$u_2$},
	y tick label style={/pgf/number format/1000 sep=},
	]
	\addplot[blue,semithick] file {wykresy/dmc_u2_l_2_2_2_2_psi_1_1_1.txt};
	
	\nextgroupplot
	[
	xlabel={$k$},
	ylabel={$u_3$},
	y tick label style={/pgf/number format/1000 sep=},
	]
	\addplot[blue,semithick] file {wykresy/dmc_u3_l_2_2_2_2_psi_1_1_1.txt};
	
	\nextgroupplot
	[
	xlabel={$k$},
	ylabel={$u_4$},
	y tick label style={/pgf/number format/1000 sep=},
	]
	\addplot[blue,semithick] file {wykresy/dmc_u4_l_2_2_2_2_psi_1_1_1.txt};
	\end{groupplot}
	\end{tikzpicture}
	\caption{Przebiegi sterowań obiektu dla $\lambda_1=2$, $\lambda_2=2$, $\lambda_3=2$ i $\lambda_4=2$.}
	\label{fig:dmc_u_l_2_2_2_2_psi_1_1_1}
\end{figure}


\begin{figure}
	\centering
	\begin{tikzpicture}
	\begin{groupplot}[group style={group size=1 by 3,vertical sep={1.5 cm}},
	width=0.9\textwidth,height=0.3\textwidth,xmin=0]
	\nextgroupplot
	[
	xlabel={$k$},
	ylabel={$y_1$},
	y tick label style={/pgf/number format/1000 sep=},
	]
	\addplot[blue,semithick] file {wykresy/dmc_y1_l_02_02_02_02_psi_1_1_1.txt};
	\addplot[red,semithick,densely dashed] file {wykresy/yzad1.txt};
	\legend{$y_1$,$y_{zad}$}
	\nextgroupplot
	[
	xlabel={$k$},
	ylabel={$y_2$},
	y tick label style={/pgf/number format/1000 sep=},
	]
	\addplot[blue,semithick] file {wykresy/dmc_y2_l_02_02_02_02_psi_1_1_1.txt};
	\addplot[red,semithick,densely dashed] file {wykresy/yzad2.txt};
	\legend{$y_2$,$y_{zad}$}
	\nextgroupplot
	[
	xlabel={$k$},
	ylabel={$y_3$},
	y tick label style={/pgf/number format/1000 sep=},
	]
	\addplot[blue,semithick] file {wykresy/dmc_y3_l_02_02_02_02_psi_1_1_1.txt};
	\addplot[red,semithick,densely dashed] file {wykresy/yzad3.txt};
	\legend{$y_3$,$y_{zad}$}
	\end{groupplot}
	\end{tikzpicture}
	\caption{Przebiegi wyjść obiektu dla $\lambda_1=\num{0,2}$, $\lambda_2=\num{0,2}$, $\lambda_3=\num{0,2}$ i $\lambda_4=\num{0,2}$.}
	\label{fig:dmc_y_l_02_02_02_02_psi_1_1_1}
\end{figure}

\begin{figure}
	\centering
	\begin{tikzpicture}
	\begin{groupplot}[group style={group size=1 by 4,vertical sep={1.5 cm}},
	width=0.9\textwidth,height=0.25\textwidth,xmin=0]
	\nextgroupplot
	[
	xlabel={$k$},
	ylabel={$u_1$},
	y tick label style={/pgf/number format/1000 sep=},
	]
	\addplot[blue,semithick] file {wykresy/dmc_u1_l_02_02_02_02_psi_1_1_1.txt};

	\nextgroupplot
	[
	xlabel={$k$},
	ylabel={$u_2$},
	y tick label style={/pgf/number format/1000 sep=},
	]
	\addplot[blue,semithick] file {wykresy/dmc_u2_l_02_02_02_02_psi_1_1_1.txt};
	
	\nextgroupplot
	[
	xlabel={$k$},
	ylabel={$u_3$},
	y tick label style={/pgf/number format/1000 sep=},
	]
	\addplot[blue,semithick] file {wykresy/dmc_u3_l_02_02_02_02_psi_1_1_1.txt};
	
	\nextgroupplot
	[
	xlabel={$k$},
	ylabel={$u_4$},
	y tick label style={/pgf/number format/1000 sep=},
	]
	\addplot[blue,semithick] file {wykresy/dmc_u4_l_02_02_02_02_psi_1_1_1.txt};
	\end{groupplot}
	\end{tikzpicture}
	\caption{Przebiegi sterowań obiektu dla $\lambda_1=\num{0,2}$, $\lambda_2=\num{0,2}$, $\lambda_3=\num{0,2}$ i $\lambda_4=\num{0,2}$.}
	\label{fig:dmc_u_l_02_02_02_02_psi_1_1_1}
\end{figure}

\begin{figure}
	\centering
	\begin{tikzpicture}
	\begin{groupplot}[group style={group size=1 by 3,vertical sep={1.5 cm}},
	width=0.9\textwidth,height=0.3\textwidth,xmin=0]
	\nextgroupplot
	[
	xlabel={$k$},
	ylabel={$y_1$},
	y tick label style={/pgf/number format/1000 sep=},
	]
	\addplot[blue,semithick] file {wykresy/dmc_y1_l_03_02_01_03_psi_1_1_1.txt};
	\addplot[red,semithick,densely dashed] file {wykresy/yzad1.txt};
	\legend{$y_1$,$y_{zad}$}
	\nextgroupplot
	[
	xlabel={$k$},
	ylabel={$y_2$},
	y tick label style={/pgf/number format/1000 sep=},
	]
	\addplot[blue,semithick] file {wykresy/dmc_y2_l_03_02_01_03_psi_1_1_1.txt};
	\addplot[red,semithick,densely dashed] file {wykresy/yzad2.txt};
	\legend{$y_2$,$y_{zad}$}
	\nextgroupplot
	[
	xlabel={$k$},
	ylabel={$y_3$},
	y tick label style={/pgf/number format/1000 sep=},
	]
	\addplot[blue,semithick] file {wykresy/dmc_y3_l_03_02_01_03_psi_1_1_1.txt};
	\addplot[red,semithick,densely dashed] file {wykresy/yzad3.txt};
	\legend{$y_3$,$y_{zad}$}
	\end{groupplot}
	\end{tikzpicture}
	\caption{Przebiegi wyjść obiektu dla $\lambda_1=\num{0,3}$, $\lambda_2=\num{0,2}$, $\lambda_3=\num{0,1}$ i $\lambda_4=\num{0,3}$.}
	\label{fig:dmc_y_l_03_02_01_03_psi_1_1_1}
\end{figure}

\begin{figure}
	\centering
	\begin{tikzpicture}
	\begin{groupplot}[group style={group size=1 by 4,vertical sep={1.5 cm}},
	width=0.9\textwidth,height=0.25\textwidth,xmin=0]
	\nextgroupplot
	[
	xlabel={$k$},
	ylabel={$u_1$},
	y tick label style={/pgf/number format/1000 sep=},
	]
	\addplot[blue,semithick] file {wykresy/dmc_u1_l_03_02_01_03_psi_1_1_1.txt};

	\nextgroupplot
	[
	xlabel={$k$},
	ylabel={$u_2$},
	y tick label style={/pgf/number format/1000 sep=},
	]
	\addplot[blue,semithick] file {wykresy/dmc_u2_l_03_02_01_03_psi_1_1_1.txt};
	
	\nextgroupplot
	[
	xlabel={$k$},
	ylabel={$u_3$},
	y tick label style={/pgf/number format/1000 sep=},
	]
	\addplot[blue,semithick] file {wykresy/dmc_u3_l_03_02_01_03_psi_1_1_1.txt};
	
	\nextgroupplot
	[
	xlabel={$k$},
	ylabel={$u_4$},
	y tick label style={/pgf/number format/1000 sep=},
	]
	\addplot[blue,semithick] file {wykresy/dmc_u4_l_03_02_01_03_psi_1_1_1.txt};
	\end{groupplot}
	\end{tikzpicture}
	\caption{Przebiegi sterowań obiektu dla $\lambda_1=\num{0,3}$, $\lambda_2=\num{0,2}$, $\lambda_3=\num{0,1}$ i $\lambda_4=\num{0,3}$.}
	\label{fig:dmc_u_l_03_02_01_03_psi_1_1_1}
\end{figure}
\section{Parametr $\psi$}
W wyniku testowania różnych wartości współczynnika $\psi$ zaobserwowaliśmy, że błąd regulacji jest najmniejszy dla duzych wartości $\psi$. Podobnie jednak jak w przypadku dobierania $\lambda$ zauważamy, że wysokie wartości parametru powodują, że przebieg sterowania jest znacznie "ostrzejszy", występują duże i nagłe skoki $u$ przy zmianach wartości zadanej. W przypadku rzeczywistego obiektu, zjawisko to mogłoby mieć negatywny efekt, na przykład uszkodzenie części sterujących. Staraliśmy się więc doprowadzić do kompromisu między niskim wskaźnikiem błędu a łagodnym przebiegiem sterowania.

Testując różne wartości parametru $psi$, przyjęliśmy długości horyzontów $N=40$ i $N_u=5$ oraz współczynniki $\lambda_1=\num{0,3}$, $\lambda_2=\num{0,2}$, $\lambda_3=\num{0,1}$, $\lambda_4=\num{0,3}$.

Próba ustawienia parametrów $\psi$ na wartość poniżej 1 dała w rezultacie wyższe błędy regulacji.
Próba zmniejszenia wartości parametrów $psi$ do $\num{0,8}$ okazała się przynosić wyższe współczynniki błędu.
\begin{itemize}
\item $E_1=\num{36,9114}$
\item $E_2=\num{39,0784}$
\item $E_3=\num{17,3408}$
\item $E=\num{93,3306}$
\end{itemize}
Zdecydowaliśmy więc, że kolejne testy przeprowadzane będą na wartościach $\psi$ powyżej 1. Przebiegi wyjść i sterowań przedstawiają wykresy \ref{fig:dmc_y_l_03_02_01_03_psi_08_08_08} i \ref{fig:dmc_y_l_03_02_01_03_psi_08_08_08}.

Zwiększenie współczynników $\psi$ do wartości 5 dało w rezultacie bardzo dużą poprawę błędu regulacji.
\begin{itemize}
\item $E_1=\num{34,6604}$
\item $E_2=\num{34,131}$
\item $E_3=\num{9,0736}$
\item $E=\num{78,8650}$
\end{itemize}
Należy jednak odnotować, że przebieg sterowania jest teraz znacznie ostrzejszy, co jest szczególnie widoczne na torze sterowania 4 (wykres \ref{fig:dmc_u_l_03_02_01_03_psi_5_5_5}). Zmiana $\psi$ nie miała dużego wpływu na pozostałe tory. Spróbujemy więc, manipulując parametrami $\psi$, złagodzić sterowanie na torze 4, zachowując jednocześnie poprawę błędu regulacji. Wyjścia obiektu przedstawia wykres \ref{fig:dmc_y_l_03_02_01_03_psi_5_5_5}.

W wyniku eksperymentów dowiedzieliśmy się, że najbardziej na sterowanie na torze czwartym wpływa parametr $\psi_3$. Postanowiliśmy więc zmniejszyć $\psi_3$, jednocześnie zwiększająć $\psi_1$ i $\psi_2$. Przetestowaliśmy działanie obiektu na wartościach $\psi_1=\num{6,5}$, $\psi_2=7$, $\psi_3=2$. Jak widać na wykresie \ref{fig:dmc_u_l_03_02_01_03_psi_65_7_2}, sterowanie zostało nieco złagodzone, choć w rezultacie nieznacznie pogorszył się wskaźnik błędu regulacji.
\begin{itemize}
\item $E_1=\num{34,584}$
\item $E_2=\num{35,2223}$
\item $E_3=\num{10,3634}$
\item $E=\num{80,1697}$
\end{itemize}
Uznaliśmy jednak, że takie nastawy dają dobry kompromis między jakością regulacji a łagodnym sterowaniem. Przebiegi wyjść obiektu przedstawia wykres \ref{fig:dmc_y_l_03_02_01_03_psi_65_7_2}.

\begin{figure}
	\centering
	\begin{tikzpicture}
	\begin{groupplot}[group style={group size=1 by 3,vertical sep={1.5 cm}},
	width=0.9\textwidth,height=0.3\textwidth,xmin=0]
	\nextgroupplot
	[
	xlabel={$k$},
	ylabel={$y_1$},
	y tick label style={/pgf/number format/1000 sep=},
	]
	\addplot[blue,semithick] file {wykresy/dmc_y1_l_03_02_01_03_psi_08_08_08.txt};
	\addplot[red,semithick,densely dashed] file {wykresy/yzad1.txt};
	\legend{$y_1$,$y_{zad}$}
	\nextgroupplot
	[
	xlabel={$k$},
	ylabel={$y_2$},
	y tick label style={/pgf/number format/1000 sep=},
	]
	\addplot[blue,semithick] file {wykresy/dmc_y2_l_03_02_01_03_psi_08_08_08.txt};
	\addplot[red,semithick,densely dashed] file {wykresy/yzad2.txt};
	\legend{$y_2$,$y_{zad}$}
	\nextgroupplot
	[
	xlabel={$k$},
	ylabel={$y_3$},
	y tick label style={/pgf/number format/1000 sep=},
	]
	\addplot[blue,semithick] file {wykresy/dmc_y3_l_03_02_01_03_psi_08_08_08.txt};
	\addplot[red,semithick,densely dashed] file {wykresy/yzad3.txt};
	\legend{$y_3$,$y_{zad}$}
	\end{groupplot}
	\end{tikzpicture}
	\caption{Przebiegi wyjść obiektu dla $\psi_1=\num{0,8}$, $\psi_2=\num{0,8}$, $\psi_3=\num{0,8}$.}
	\label{fig:dmc_y_l_03_02_01_03_psi_08_08_08}
\end{figure}

\begin{figure}
	\centering
	\begin{tikzpicture}
	\begin{groupplot}[group style={group size=1 by 4,vertical sep={1.5 cm}},
	width=0.9\textwidth,height=0.25\textwidth,xmin=0]
	\nextgroupplot
	[
	xlabel={$k$},
	ylabel={$u_1$},
	y tick label style={/pgf/number format/1000 sep=},
	]
	\addplot[blue,semithick] file {wykresy/dmc_y1_l_03_02_01_03_psi_08_08_08.txt};

	\nextgroupplot
	[
	xlabel={$k$},
	ylabel={$u_2$},
	y tick label style={/pgf/number format/1000 sep=},
	]
	\addplot[blue,semithick] file {wykresy/dmc_u2_l_03_02_01_03_psi_08_08_08.txt};
	
	\nextgroupplot
	[
	xlabel={$k$},
	ylabel={$u_3$},
	y tick label style={/pgf/number format/1000 sep=},
	]
	\addplot[blue,semithick] file {wykresy/dmc_u3_l_03_02_01_03_psi_08_08_08.txt};
	
	\nextgroupplot
	[
	xlabel={$k$},
	ylabel={$u_4$},
	y tick label style={/pgf/number format/1000 sep=},
	]
	\addplot[blue,semithick] file {wykresy/dmc_u4_l_03_02_01_03_psi_08_08_08.txt};
	\end{groupplot}
	\end{tikzpicture}
	\caption{Przebiegi sterowań obiektu dla $\psi_1=\num{0,8}$, $\psi_2=\num{0,8}$, $\psi_3=\num{0,8}$.}
	\label{fig:dmc_u_l_03_02_01_03_psi_08_08_08}
\end{figure}


\begin{figure}
	\centering
	\begin{tikzpicture}
	\begin{groupplot}[group style={group size=1 by 3,vertical sep={1.5 cm}},
	width=0.9\textwidth,height=0.3\textwidth,xmin=0]
	\nextgroupplot
	[
	xlabel={$k$},
	ylabel={$y_1$},
	y tick label style={/pgf/number format/1000 sep=},
	]
	\addplot[blue,semithick] file {wykresy/dmc_y1_l_03_02_01_03_psi_5_5_5.txt};
	\addplot[red,semithick,densely dashed] file {wykresy/yzad1.txt};
	\legend{$y_1$,$y_{zad}$}
	\nextgroupplot
	[
	xlabel={$k$},
	ylabel={$y_2$},
	y tick label style={/pgf/number format/1000 sep=},
	]
	\addplot[blue,semithick] file {wykresy/dmc_y2_l_03_02_01_03_psi_5_5_5.txt};
	\addplot[red,semithick,densely dashed] file {wykresy/yzad2.txt};
	\legend{$y_2$,$y_{zad}$}
	\nextgroupplot
	[
	xlabel={$k$},
	ylabel={$y_3$},
	y tick label style={/pgf/number format/1000 sep=},
	]
	\addplot[blue,semithick] file {wykresy/dmc_y3_l_03_02_01_03_psi_5_5_5.txt};
	\addplot[red,semithick,densely dashed] file {wykresy/yzad3.txt};
	\legend{$y_3$,$y_{zad}$}
	\end{groupplot}
	\end{tikzpicture}
	\caption{Przebiegi wyjść obiektu dla $\psi_1=5$, $\psi_2=5$, $\psi_3=5$.}
	\label{fig:dmc_y_l_03_02_01_03_psi_5_5_5}
\end{figure}

\begin{figure}
	\centering
	\begin{tikzpicture}
	\begin{groupplot}[group style={group size=1 by 4,vertical sep={1.5 cm}},
	width=0.9\textwidth,height=0.25\textwidth,xmin=0]
	\nextgroupplot
	[
	xlabel={$k$},
	ylabel={$u_1$},
	y tick label style={/pgf/number format/1000 sep=},
	]
	\addplot[blue,semithick] file {wykresy/dmc_y1_l_03_02_01_03_psi_5_5_5.txt};

	\nextgroupplot
	[
	xlabel={$k$},
	ylabel={$u_2$},
	y tick label style={/pgf/number format/1000 sep=},
	]
	\addplot[blue,semithick] file {wykresy/dmc_u2_l_03_02_01_03_psi_5_5_5.txt};
	
	\nextgroupplot
	[
	xlabel={$k$},
	ylabel={$u_3$},
	y tick label style={/pgf/number format/1000 sep=},
	]
	\addplot[blue,semithick] file {wykresy/dmc_u3_l_03_02_01_03_psi_5_5_5.txt};
	
	\nextgroupplot
	[
	xlabel={$k$},
	ylabel={$u_4$},
	y tick label style={/pgf/number format/1000 sep=},
	]
	\addplot[blue,semithick] file {wykresy/dmc_u4_l_03_02_01_03_psi_5_5_5.txt};
	\end{groupplot}
	\end{tikzpicture}
	\caption{Przebiegi sterowań obiektu dla $\psi_1=5$, $\psi_2=5$, $\psi_3=5$.}
	\label{fig:dmc_u_l_03_02_01_03_psi_5_5_5}
\end{figure}


\begin{figure}
	\centering
	\begin{tikzpicture}
	\begin{groupplot}[group style={group size=1 by 3,vertical sep={1.5 cm}},
	width=0.9\textwidth,height=0.3\textwidth,xmin=0]
	\nextgroupplot
	[
	xlabel={$k$},
	ylabel={$y_1$},
	y tick label style={/pgf/number format/1000 sep=},
	]
	\addplot[blue,semithick] file {wykresy/dmc_y1_l_03_02_01_03_psi_65_7_2.txt};
	\addplot[red,semithick,densely dashed] file {wykresy/yzad1.txt};
	\legend{$y_1$,$y_{zad}$}
	\nextgroupplot
	[
	xlabel={$k$},
	ylabel={$y_2$},
	y tick label style={/pgf/number format/1000 sep=},
	]
	\addplot[blue,semithick] file {wykresy/dmc_y2_l_03_02_01_03_psi_65_7_2.txt};
	\addplot[red,semithick,densely dashed] file {wykresy/yzad2.txt};
	\legend{$y_2$,$y_{zad}$}
	\nextgroupplot
	[
	xlabel={$k$},
	ylabel={$y_3$},
	y tick label style={/pgf/number format/1000 sep=},
	]
	\addplot[blue,semithick] file {wykresy/dmc_y3_l_03_02_01_03_psi_65_7_2.txt};
	\addplot[red,semithick,densely dashed] file {wykresy/yzad3.txt};
	\legend{$y_3$,$y_{zad}$}
	\end{groupplot}
	\end{tikzpicture}
	\caption{Przebiegi wyjść obiektu dla $\psi_1=\num{6,5}$, $\psi_2=7$, $\psi_3=2$.}
	\label{fig:dmc_y_l_03_02_01_03_psi_65_7_2}
\end{figure}

\begin{figure}
	\centering
	\begin{tikzpicture}
	\begin{groupplot}[group style={group size=1 by 4,vertical sep={1.5 cm}},
	width=0.9\textwidth,height=0.25\textwidth,xmin=0]
	\nextgroupplot
	[
	xlabel={$k$},
	ylabel={$u_1$},
	y tick label style={/pgf/number format/1000 sep=},
	]
	\addplot[blue,semithick] file {wykresy/dmc_y1_l_03_02_01_03_psi_65_7_2.txt};

	\nextgroupplot
	[
	xlabel={$k$},
	ylabel={$u_2$},
	y tick label style={/pgf/number format/1000 sep=},
	]
	\addplot[blue,semithick] file {wykresy/dmc_u2_l_03_02_01_03_psi_65_7_2.txt};
	
	\nextgroupplot
	[
	xlabel={$k$},
	ylabel={$u_3$},
	y tick label style={/pgf/number format/1000 sep=},
	]
	\addplot[blue,semithick] file {wykresy/dmc_u3_l_03_02_01_03_psi_65_7_2.txt};
	
	\nextgroupplot
	[
	xlabel={$k$},
	ylabel={$u_4$},
	y tick label style={/pgf/number format/1000 sep=},
	]
	\addplot[blue,semithick] file {wykresy/dmc_u4_l_03_02_01_03_psi_65_7_2.txt};
	\end{groupplot}
	\end{tikzpicture}
	\caption{Przebiegi sterowań obiektu dla $\psi_1=\num{6,5}$, $\psi_2=7$, $\psi_3=2$.}
	\label{fig:dmc_u_l_03_02_01_03_psi_65_7_2}
\end{figure}

\end{document}
