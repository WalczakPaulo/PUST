\chapter{Optymalizacja}
Zadanie piąte polegało na znalezieniu optymalnych wartości nastaw dla regulatora PID oraz DMC, wykorzystując do optymalizacji ilościowy wskaźnik błędu regulacji E.
Stworzyliśmy do tego skrypt, który używa funkcji $fmincon$ w celu dobierania nastaw dla regulatorów.

\section{PID}
Rozpoczniemy od optymalizacji nastaw regulatora PID. Jako cele optymalizacji przyjmujemy parametry $K$ oraz $T_i$. Zdecydowaliśmy się odgórnie narzucić zerowe wartości
$T_d$ ze względu na wyniki badań jakie przeprowadziliśmy w poprzednim punkcie sprawozdania.
Jako ograniczenia przyjmujemy zgodnie z logiką, wartości nie mniejsze niż 0 oraz mniejsze od nieskonczoności.
Jako punkt startowy wybieramy wartości wzmocnień równe $1$ a  czasy zdwojenia jako $1000$.
Optymalizacji poddajemy trzy regulatory PID w czterech konfiguracjach, tzn.
rozważamy cztery rózne konfiguracje torów sterowania wyznaczone w poprzednim punkcie sprawozdania.

Dla toru:
\begin{itemize}
  \item $y_1$ -- $u_4$
 \item $y_2$ -- $u_3$
 \item $y_3$ -- $u_2$
\end{itemize}

otrzymujemy nastawy:
\begin{equation}
  K_1 = \num{1.9724} \qquad T_{i1} = \num{210008}, \qquad T_{d1} = 0 \nonumber
\end{equation}
\begin{equation}
  K_2 = \num{1.3646} \qquad T_{i2} = \num{7.9930}, \qquad T_{d2} = 0
\end{equation}
\begin{equation}
  K_3 = \num{0.1696} \qquad T_{i3} = \num{300220}, \qquad T_{d3} = 0 \nonumber
\end{equation}

Takie nastawy dają błąd regulacji \num{486,6761}. Jest to dziwnie trudny przypadek,
gdyż nie udało nam się znaleźć nastaw dla niego ręcznie, a lekka modyfikacja nastaw
wygenerowanych funkcją \texttt{fmincon} daje ogromny skok błędu. Działanie
przedstawiają wykresy \ref{fig:pid1_fmincon_y1}, \ref{fig:pid1_fmincon_y2} i \ref{fig:pid1_fmincon_y3}.

Dla toru:
\begin{itemize}
  \item $y_1$ -- $u_1$
 \item $y_2$ -- $u_3$
 \item $y_3$ -- $u_4$
\end{itemize}

otrzymujemy nastawy:
\begin{equation}
  K_1 = \num{2,7249} \qquad T_{i1} = \num{3,9641}, \qquad T_{d1} = 0 \nonumber
\end{equation}
\begin{equation}
  K_2 = \num{2,9122} \qquad T_{i2} = \num{3,1237}, \qquad T_{d2} = 0
\end{equation}
\begin{equation}
  K_3 = \num{5,5929} \qquad T_{i3} = \num{9,8384}, \qquad T_{d3} = 0 \nonumber
\end{equation}

Błąd regulacji wynosi: \num{104,4858} . Jest to najlepszy ogólnie otrzymany wynik.
Działanie
przedstawiają wykresy \ref{fig:pid2_fmincon_y1}, \ref{fig:pid2_fmincon_y2} i \ref{fig:pid2_fmincon_y3}.

Dla toru:
\begin{itemize}
  \item $y_1$ -- $u_1$
 \item $y_2$ -- $u_2$
 \item $y_3$ -- $u_4$
\end{itemize}

otrzymujemy nastawy:
\begin{equation}
  K_1 = \num{2,9883} \qquad T_{i1} = \num{4,4246}, \qquad T_{d1} = 0 \nonumber
\end{equation}
\begin{equation}
  K_2 = \num{0,6972} \qquad T_{i2} = \num{8824,7}, \qquad T_{d2} = 0
\end{equation}
\begin{equation}
  K_3 = \num{5,6629} \qquad T_{i3} = \num{11,6449}, \qquad T_{d3} = 0 \nonumber
\end{equation}

Błąd regulacji wynosi: $\num{115,6349}$ .
Jest to bardzo dobry wynik, jednakże nieco gorszy niż dla poprzedniego toru.
Zauważmy, że wyznaczone nastawy dla pierwszego i trzeciego regulatora są bardzo podobne do tych dla poprzedniego toru.
Wynika to z faktu, że zmienił się tu jedynie tor dla drugiego regulatora, gdzie teraz wpływa na wyjscie drugie regulator drugi.
Działanie
przedstawiają wykresy \ref{fig:pid3_fmincon_y1}, \ref{fig:pid3_fmincon_y2} i \ref{fig:pid3_fmincon_y3}.

Dla toru:
\begin{itemize}
  \item $y_1$ -- $u_2$
 \item $y_2$ -- $u_3$
 \item $y_3$ -- $u_1$
\end{itemize}

otrzymujemy nastawy:
\begin{equation}
  K_1 = \num{1,1001} \qquad T_{i1} = \num{1,7044}, \qquad T_{d1} = 0 \nonumber
\end{equation}
\begin{equation}
  K_2 = \num{2,2444} \qquad T_{i2} = \num{3,7464}, \qquad T_{d2} = 0
\end{equation}
\begin{equation}
  K_3 = \num{4,0054} \qquad T_{i3} = \num{15,7112}, \qquad T_{d3} = 0 \nonumber
\end{equation}

Błąd regulacji wynosi: $\num{207,1231}$ .
Jest to wynik nieco gorszej jakości niż dla poprzednich dwóch torów.
Działanie
przedstawiają wykresy \ref{fig:pid4_fmincon_y1}, \ref{fig:pid4_fmincon_y2} i \ref{fig:pid4_fmincon_y3}.

Analizując otrzymane wyniki, zauważamy że dla wszystkich przypadków funkcja optymalizacji
znalazła lepsze wyniki niż dla regulatorów wyznaczonych metodą inżynierską. Jest to jak najbardziej normalne zjawisko.
Również potwierdziła się metoda Multiple Gain Array. Najlepsze wyniki otrzymaliśmy dla regulatora o konfiguracji
torów odpowiadającej najmniejszej wartości współczynnika uwarunkowania macierzy, zaś najgorsze wyniki
dla torów odpowiadających największej wartości współczynnika uwarunkowania macierzy.

<<<<<<< HEAD
\section{DMC}
Optymalizacja nastaw algorytmu DMC dała bardzo zaskakujące rezultaty. Wyznaczone parametry $\lambda$ i $\psi$ to:
\begin{equation}
\lambda_1=\num{50200} \qquad \lambda_2 = 2390 \qquad \lambda_3 = 546370 \lambda_4 = 140
\end{equation}
\begin{equation}
\psi_1 = 120050 \qquad \psi_2 = 234440 \qquad \psi_3 = 43840
\end{equation}

Nastawy te mają bardzo duże wartości, odbiegające od wyznaczonych przez nas. Błędy regulacji są następujące:
\begin{itemize}
\item $E_1 = \num{34,6239}$
\item $E_2 = \num{34,5739}$
\item $E_3 = \num{0,57272}$
\item $E = \num{69,7705}$
\end{itemize}
Błąd regulacji jest zatem znacznie mniejszy od błędy zwracanego przez nasz najlepszy regulator. Na wykresie \ref{fig:dmc_u_fmincon} widać jednak, że przebieg sterowania na torze 4 osiąga bardzo duże wartości i byłby nieakceptowalny na rzeczywistym obiekcie. Przebieg wyjść obiektu przedstawia wykres \ref{fig:dmc_y_fmincon}.

%%%%%%%%%%%%%%%%%%%%%%%%%%%%%%%%%%%%%%%%%%%%%%%%%%%%%%%%%%%%%%%%%%%%%%%%%%%%%%%%%%%%%%%%%%%%
\begin{figure}[b]
\centering
\begin{tikzpicture}
\begin{axis}[
width=0.75\textwidth,
height = 0.4\textwidth,
xmin=0,xmax=1100,ymin=-15,ymax=15,
xlabel={Numer próbki},
ylabel={Wyjście},
xtick={0, 200, 400, 600, 800, 1000},
ytick={-15, -10, -5, 0, 5, 10, 15},
legend pos=north east,
/pgf/number format/.cd,
use comma,
1000 sep={}
]

\addplot[blue,semithick] file {wykresy/z3_yzad.txt};
\addplot[red,semithick] file {wykresy/pid1_fmincon_y1.txt};
\legend{Wartość zadana, Wyjście $y_1$}

\end{axis}
\end{tikzpicture}
\caption{Trajektoria wyjścia $y_1$, dla pierwszego zestawu regulatorów PID, dostrojonych funkcją optymalizacyjną}
\label{fig:pid1_fmincon_y1}
\end{figure}
%%%%%%%%%%%%%%%%%%%%%%%%%%%%%%%%%%%%%%%%%%%%%%%%%%%%%%%%%%%%%%%%%%%%%%%%%%%%%%%%%%%%%%%%%%%%

%%%%%%%%%%%%%%%%%%%%%%%%%%%%%%%%%%%%%%%%%%%%%%%%%%%%%%%%%%%%%%%%%%%%%%%%%%%%%%%%%%%%%%%%%%%%
\begin{figure}[b]
\centering
\begin{tikzpicture}
\begin{axis}[
width=0.75\textwidth,
height = 0.4\textwidth,
xmin=0,xmax=1100,ymin=-5,ymax=5,
xlabel={Numer próbki},
ylabel={Wyjście},
xtick={0, 200, 400, 600, 800, 1000},
ytick={-5, -4, -2, 0, 2, 4, 5},
legend pos=north east,
/pgf/number format/.cd,
use comma,
1000 sep={}
]

\addplot[blue,semithick] file {wykresy/z3_yzad.txt};
\addplot[red,semithick] file {wykresy/pid1_fmincon_y2.txt};
\legend{Wartość zadana, Wyjście $y_2$}

\end{axis}
\end{tikzpicture}
\caption{Trajektoria wyjścia $y_2$, dla pierwszego zestawu regulatorów PID, dostrojonych funkcją optymalizacyjną}
\label{fig:pid1_fmincon_y2}
\end{figure}
%%%%%%%%%%%%%%%%%%%%%%%%%%%%%%%%%%%%%%%%%%%%%%%%%%%%%%%%%%%%%%%%%%%%%%%%%%%%%%%%%%%%%%%%%%%%

%%%%%%%%%%%%%%%%%%%%%%%%%%%%%%%%%%%%%%%%%%%%%%%%%%%%%%%%%%%%%%%%%%%%%%%%%%%%%%%%%%%%%%%%%%%%
\begin{figure}[b]
\centering
\begin{tikzpicture}
\begin{axis}[
width=0.75\textwidth,
height = 0.4\textwidth,
xmin=0,xmax=1100,ymin=-2.5,ymax=2.5,
xlabel={Numer próbki},
ylabel={Wyjście},
xtick={0, 200, 400, 600, 800, 1000},
ytick={-2.5, -2, -1.5, -1, -.5, 0, .5, 1, 1.5, 2, 2.5},
legend pos=north east,
/pgf/number format/.cd,
use comma,
1000 sep={}
]

\addplot[blue,semithick] file {wykresy/z3_yzad.txt};
\addplot[red,semithick] file {wykresy/pid1_fmincon_y3.txt};
\legend{Wartość zadana, Wyjście $y_3$}

\end{axis}
\end{tikzpicture}
\caption{Trajektoria wyjścia $y_3$, dla pierwszego zestawu regulatorów PID, dostrojonych funkcją optymalizacyjną}
\label{fig:pid1_fmincon_y3}
\end{figure}
%%%%%%%%%%%%%%%%%%%%%%%%%%%%%%%%%%%%%%%%%%%%%%%%%%%%%%%%%%%%%%%%%%%%%%%%%%%%%%%%%%%%%%%%%%%%

%%%%%%%%%%%%%%%%%%%%%%%%%%%%%%%%%%%%%%%%%%%%%%%%%%%%%%%%%%%%%%%%%%%%%%%%%%%%%%%%%%%%%%%%%%%%
\begin{figure}[b]
\centering
\begin{tikzpicture}
\begin{axis}[
width=0.75\textwidth,
height = 0.4\textwidth,
xmin=0,xmax=1100,ymin=-15,ymax=15,
xlabel={Numer próbki},
ylabel={Wyjście},
xtick={0, 200, 400, 600, 800, 1000},
ytick={-15, -10, -5, 0, 5, 10, 15},
legend pos=north east,
/pgf/number format/.cd,
use comma,
1000 sep={}
]

\addplot[blue,semithick] file {wykresy/z3_yzad.txt};
\addplot[red,semithick] file {wykresy/pid2_fmincon_y1.txt};
\legend{Wartość zadana, Wyjście $y_1$}

\end{axis}
\end{tikzpicture}
\caption{Trajektoria wyjścia $y_1$, dla drugiego zestawu regulatorów PID, dostrojonych funkcją optymalizacyjną}
\label{fig:pid2_fmincon_y1}
\end{figure}
%%%%%%%%%%%%%%%%%%%%%%%%%%%%%%%%%%%%%%%%%%%%%%%%%%%%%%%%%%%%%%%%%%%%%%%%%%%%%%%%%%%%%%%%%%%%

%%%%%%%%%%%%%%%%%%%%%%%%%%%%%%%%%%%%%%%%%%%%%%%%%%%%%%%%%%%%%%%%%%%%%%%%%%%%%%%%%%%%%%%%%%%%
\begin{figure}[b]
\centering
\begin{tikzpicture}
\begin{axis}[
width=0.75\textwidth,
height = 0.4\textwidth,
xmin=0,xmax=1100,ymin=-5,ymax=5,
xlabel={Numer próbki},
ylabel={Wyjście},
xtick={0, 200, 400, 600, 800, 1000},
ytick={-5, -4, -2, 0, 2, 4, 5},
legend pos=north east,
/pgf/number format/.cd,
use comma,
1000 sep={}
]

\addplot[blue,semithick] file {wykresy/z3_yzad.txt};
\addplot[red,semithick] file {wykresy/pid2_fmincon_y2.txt};
\legend{Wartość zadana, Wyjście $y_2$}

\end{axis}
\end{tikzpicture}
\caption{Trajektoria wyjścia $y_2$, dla drugiego zestawu regulatorów PID, dostrojonych funkcją optymalizacyjną}
\label{fig:pid2_fmincon_y2}
\end{figure}
%%%%%%%%%%%%%%%%%%%%%%%%%%%%%%%%%%%%%%%%%%%%%%%%%%%%%%%%%%%%%%%%%%%%%%%%%%%%%%%%%%%%%%%%%%%%

%%%%%%%%%%%%%%%%%%%%%%%%%%%%%%%%%%%%%%%%%%%%%%%%%%%%%%%%%%%%%%%%%%%%%%%%%%%%%%%%%%%%%%%%%%%%
\begin{figure}[b]
\centering
\begin{tikzpicture}
\begin{axis}[
width=0.75\textwidth,
height = 0.4\textwidth,
xmin=0,xmax=1100,ymin=-2.5,ymax=2.5,
xlabel={Numer próbki},
ylabel={Wyjście},
xtick={0, 200, 400, 600, 800, 1000},
ytick={-2.5, -2, -1.5, -1, -.5, 0, .5, 1, 1.5, 2, 2.5},
legend pos=north east,
/pgf/number format/.cd,
use comma,
1000 sep={}
]

\addplot[blue,semithick] file {wykresy/z3_yzad.txt};
\addplot[red,semithick] file {wykresy/pid2_fmincon_y3.txt};
\legend{Wartość zadana, Wyjście $y_3$}

\end{axis}
\end{tikzpicture}
\caption{Trajektoria wyjścia $y_3$, dla drugiego zestawu regulatorów PID, dostrojonych funkcją optymalizacyjną}
\label{fig:pid2_fmincon_y3}
\end{figure}
%%%%%%%%%%%%%%%%%%%%%%%%%%%%%%%%%%%%%%%%%%%%%%%%%%%%%%%%%%%%%%%%%%%%%%%%%%%%%%%%%%%%%%%%%%%%

%%%%%%%%%%%%%%%%%%%%%%%%%%%%%%%%%%%%%%%%%%%%%%%%%%%%%%%%%%%%%%%%%%%%%%%%%%%%%%%%%%%%%%%%%%%%
\begin{figure}[b]
\centering
\begin{tikzpicture}
\begin{axis}[
width=0.75\textwidth,
height = 0.4\textwidth,
xmin=0,xmax=1100,ymin=-15,ymax=15,
xlabel={Numer próbki},
ylabel={Wyjście},
xtick={0, 200, 400, 600, 800, 1000},
ytick={-15, -10, -5, 0, 5, 10, 15},
legend pos=north east,
/pgf/number format/.cd,
use comma,
1000 sep={}
]

\addplot[blue,semithick] file {wykresy/z3_yzad.txt};
\addplot[red,semithick] file {wykresy/pid3_fmincon_y1.txt};
\legend{Wartość zadana, Wyjście $y_1$}

\end{axis}
\end{tikzpicture}
\caption{Trajektoria wyjścia $y_1$, dla trzeciego zestawu regulatorów PID, dostrojonych funkcją optymalizacyjną}
\label{fig:pid3_fmincon_y1}
\end{figure}
%%%%%%%%%%%%%%%%%%%%%%%%%%%%%%%%%%%%%%%%%%%%%%%%%%%%%%%%%%%%%%%%%%%%%%%%%%%%%%%%%%%%%%%%%%%%

%%%%%%%%%%%%%%%%%%%%%%%%%%%%%%%%%%%%%%%%%%%%%%%%%%%%%%%%%%%%%%%%%%%%%%%%%%%%%%%%%%%%%%%%%%%%
\begin{figure}[b]
\centering
\begin{tikzpicture}
\begin{axis}[
width=0.75\textwidth,
height = 0.4\textwidth,
xmin=0,xmax=1100,ymin=-5,ymax=5,
xlabel={Numer próbki},
ylabel={Wyjście},
xtick={0, 200, 400, 600, 800, 1000},
ytick={-5, -4, -2, 0, 2, 4, 5},
legend pos=north east,
/pgf/number format/.cd,
use comma,
1000 sep={}
]

\addplot[blue,semithick] file {wykresy/z3_yzad.txt};
\addplot[red,semithick] file {wykresy/pid3_fmincon_y2.txt};
\legend{Wartość zadana, Wyjście $y_2$}

\end{axis}
\end{tikzpicture}
\caption{Trajektoria wyjścia $y_2$, dla trzeciego zestawu regulatorów PID, dostrojonych funkcją optymalizacyjną}
\label{fig:pid3_fmincon_y2}
\end{figure}
%%%%%%%%%%%%%%%%%%%%%%%%%%%%%%%%%%%%%%%%%%%%%%%%%%%%%%%%%%%%%%%%%%%%%%%%%%%%%%%%%%%%%%%%%%%%

%%%%%%%%%%%%%%%%%%%%%%%%%%%%%%%%%%%%%%%%%%%%%%%%%%%%%%%%%%%%%%%%%%%%%%%%%%%%%%%%%%%%%%%%%%%%
\begin{figure}[b]
\centering
\begin{tikzpicture}
\begin{axis}[
width=0.75\textwidth,
height = 0.4\textwidth,
xmin=0,xmax=1100,ymin=-2.5,ymax=2.5,
xlabel={Numer próbki},
ylabel={Wyjście},
xtick={0, 200, 400, 600, 800, 1000},
ytick={-2.5, -2, -1.5, -1, -.5, 0, .5, 1, 1.5, 2, 2.5},
legend pos=north east,
/pgf/number format/.cd,
use comma,
1000 sep={}
]

\addplot[blue,semithick] file {wykresy/z3_yzad.txt};
\addplot[red,semithick] file {wykresy/pid3_fmincon_y3.txt};
\legend{Wartość zadana, Wyjście $y_3$}

\end{axis}
\end{tikzpicture}
\caption{Trajektoria wyjścia $y_3$, dla trzeciego zestawu regulatorów PID, dostrojonych funkcją optymalizacyjną}
\label{fig:pid3_fmincon_y3}
\end{figure}
%%%%%%%%%%%%%%%%%%%%%%%%%%%%%%%%%%%%%%%%%%%%%%%%%%%%%%%%%%%%%%%%%%%%%%%%%%%%%%%%%%%%%%%%%%%%

%%%%%%%%%%%%%%%%%%%%%%%%%%%%%%%%%%%%%%%%%%%%%%%%%%%%%%%%%%%%%%%%%%%%%%%%%%%%%%%%%%%%%%%%%%%%
\begin{figure}[b]
\centering
\begin{tikzpicture}
\begin{axis}[
width=0.75\textwidth,
height = 0.4\textwidth,
xmin=0,xmax=1100,ymin=-15,ymax=15,
xlabel={Numer próbki},
ylabel={Wyjście},
xtick={0, 200, 400, 600, 800, 1000},
ytick={-15, -10, -5, 0, 5, 10, 15},
legend pos=north east,
/pgf/number format/.cd,
use comma,
1000 sep={}
]

\addplot[blue,semithick] file {wykresy/z3_yzad.txt};
\addplot[red,semithick] file {wykresy/pid4_fmincon_y1.txt};
\legend{Wartość zadana, Wyjście $y_1$}

\end{axis}
\end{tikzpicture}
\caption{Trajektoria wyjścia $y_1$, dla czwartego zestawu regulatorów PID, dostrojonych funkcją optymalizacyjną}
\label{fig:pid4_fmincon_y1}
\end{figure}
%%%%%%%%%%%%%%%%%%%%%%%%%%%%%%%%%%%%%%%%%%%%%%%%%%%%%%%%%%%%%%%%%%%%%%%%%%%%%%%%%%%%%%%%%%%%

%%%%%%%%%%%%%%%%%%%%%%%%%%%%%%%%%%%%%%%%%%%%%%%%%%%%%%%%%%%%%%%%%%%%%%%%%%%%%%%%%%%%%%%%%%%%
\begin{figure}[b]
\centering
\begin{tikzpicture}
\begin{axis}[
width=0.75\textwidth,
height = 0.4\textwidth,
xmin=0,xmax=1100,ymin=-5,ymax=5,
xlabel={Numer próbki},
ylabel={Wyjście},
xtick={0, 200, 400, 600, 800, 1000},
ytick={-5, -4, -2, 0, 2, 4, 5},
legend pos=north east,
/pgf/number format/.cd,
use comma,
1000 sep={}
]

\addplot[blue,semithick] file {wykresy/z3_yzad.txt};
\addplot[red,semithick] file {wykresy/pid4_fmincon_y2.txt};
\legend{Wartość zadana, Wyjście $y_2$}

\end{axis}
\end{tikzpicture}
\caption{Trajektoria wyjścia $y_2$, dla czwartego zestawu regulatorów PID, dostrojonych funkcją optymalizacyjną}
\label{fig:pid4_fmincon_y2}
\end{figure}
%%%%%%%%%%%%%%%%%%%%%%%%%%%%%%%%%%%%%%%%%%%%%%%%%%%%%%%%%%%%%%%%%%%%%%%%%%%%%%%%%%%%%%%%%%%%

%%%%%%%%%%%%%%%%%%%%%%%%%%%%%%%%%%%%%%%%%%%%%%%%%%%%%%%%%%%%%%%%%%%%%%%%%%%%%%%%%%%%%%%%%%%%
\begin{figure}[b]
\centering
\begin{tikzpicture}
\begin{axis}[
width=0.75\textwidth,
height = 0.4\textwidth,
xmin=0,xmax=1100,ymin=-2.5,ymax=2.5,
xlabel={Numer próbki},
ylabel={Wyjście},
xtick={0, 200, 400, 600, 800, 1000},
ytick={-2.5, -2, -1.5, -1, -.5, 0, .5, 1, 1.5, 2, 2.5},
legend pos=north east,
/pgf/number format/.cd,
use comma,
1000 sep={}
]

\addplot[blue,semithick] file {wykresy/z3_yzad.txt};
\addplot[red,semithick] file {wykresy/pid4_fmincon_y3.txt};
\legend{Wartość zadana, Wyjście $y_3$}

\end{axis}
\end{tikzpicture}
\caption{Trajektoria wyjścia $y_3$, dla czwartego zestawu regulatorów PID, dostrojonych funkcją optymalizacyjną}
\label{fig:pid4_fmincon_y3}
\end{figure}
%%%%%%%%%%%%%%%%%%%%%%%%%%%%%%%%%%%%%%%%%%%%%%%%%%%%%%%%%%%%%%%%%%%%%%%%%%%%%%%%%%%%%%%%%%%%

\begin{figure}
	\centering
	\begin{tikzpicture}
	\begin{groupplot}[group style={group size=1 by 3,vertical sep={1.5 cm}},
	width=0.9\textwidth,height=0.3\textwidth,xmin=0]
	\nextgroupplot
	[
	xlabel={$k$},
	ylabel={$y_1$},
	y tick label style={/pgf/number format/1000 sep=},
	]
	\addplot[blue,semithick] file {wykresy/dmc_y1_fmincon.txt};
	\addplot[red,semithick,densely dashed] file {wykresy/yzad1.txt};
	\legend{$y_1$,$y_{zad}$}
	\nextgroupplot
	[
	xlabel={$k$},
	ylabel={$y_2$},
	y tick label style={/pgf/number format/1000 sep=},
	]
	\addplot[blue,semithick] file {wykresy/dmc_y2_fmincon.txt};
	\addplot[red,semithick,densely dashed] file {wykresy/yzad2.txt};
	\legend{$y_2$,$y_{zad}$}
	\nextgroupplot
	[
	xlabel={$k$},
	ylabel={$y_3$},
	y tick label style={/pgf/number format/1000 sep=},
	]
	\addplot[blue,semithick] file {wykresy/dmc_y3_fmincon.txt};
	\addplot[red,semithick,densely dashed] file {wykresy/yzad3.txt};
	\legend{$y_3$,$y_{zad}$}
	\end{groupplot}
	\end{tikzpicture}
	\caption{Przebiegi wyjść obiektu dla nastaw wyznaczonych narzędziem fmincon}
	\label{fig:dmc_y_fmincon}
\end{figure}

\begin{figure}
	\centering
	\begin{tikzpicture}
	\begin{groupplot}[group style={group size=1 by 4,vertical sep={1.5 cm}},
	width=0.9\textwidth,height=0.25\textwidth,xmin=0]
	\nextgroupplot
	[
	xlabel={$k$},
	ylabel={$u_1$},
	y tick label style={/pgf/number format/1000 sep=},
	]
	\addplot[blue,semithick] file {wykresy/dmc_u1_fmincon.txt};

	\nextgroupplot
	[
	xlabel={$k$},
	ylabel={$u_2$},
	y tick label style={/pgf/number format/1000 sep=},
	]
	\addplot[blue,semithick] file {wykresy/dmc_u2_fmincon.txt};

	\nextgroupplot
	[
	xlabel={$k$},
	ylabel={$u_3$},
	y tick label style={/pgf/number format/1000 sep=},
	]
	\addplot[blue,semithick] file {wykresy/dmc_u3_fmincon.txt};

	\nextgroupplot
	[
	xlabel={$k$},
	ylabel={$u_4$},
	y tick label style={/pgf/number format/1000 sep=},
	]
	\addplot[blue,semithick] file {wykresy/dmc_u4_fmincon.txt};
	\end{groupplot}
	\end{tikzpicture}
	\caption{Przebiegi sterowań obiektu dla nastaw wyznaczonych narzędziem fmincon}
	\label{fig:dmc_u_fmincon}
\end{figure}
