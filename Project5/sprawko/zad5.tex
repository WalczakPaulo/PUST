\chapter{PID/DMC Optymalizacja}
Zadanie piąte polegało na znalezieniu optymalnych wartości nastaw dla regulatora PID oraz DMC, wykorzystując do optymalizacji ilościowy wskaźnik błędu regulacji E.
Stworzyliśmy do tego skrypt, który używa funkcji $fmincon$ w celu dobierania nastaw dla regulatorów. 


Rozpoczniemy od optymalizacji nastaw regulatora PID. Jako cele optymalizacji przyjmujemy parametry $K$ oraz $T_i$. Zdecydowaliśmy się odgórnie narzucić zerowe wartości
$T_d$ ze względu na wyniki badań jakie przeprowadziliśmy w poprzednim punkcie sprawozdania. 
Jako ograniczenia przyjmujemy zgodnie z logiką, wartości nie mniejsze niż 0 oraz mniejsze od nieskonczoności.
Jako punkt startowy wybieramy wartości wzmocnień równe $1$ a  czasy zdwojenia jako $1000$. 
Optymalizacji poddajemy trzy regulatory PID w czterech konfiguracjach, tzn. 
rozważamy cztery rózne konfiguracje torów sterowania wyznaczone w poprzednim punkcie sprawozdania. 

Dla toru:
\begin{itemize}
  \item $y_1$ -- $u_4$
 \item $y_2$ -- $u_3$
 \item $y_3$ -- $u_2$
\end{itemize}

otrzymujemy nastawy:
\begin{equation}
  K_1 = \num{1.9724} \qquad T_{i1} = \num{210008}, \qquad T_{d1} = 0 \nonumber
\end{equation}
\begin{equation}
  K_2 = \num{1.3646} \qquad T_{i2} = \num{7.9930}, \qquad T_{d2} = 0
\end{equation}
\begin{equation}
  K_3 = \num{0.1696} \qquad T_{i3} = \num{300220}, \qquad T_{d3} = 0 \nonumber
\end{equation}
Takie nastawy dają błąd regulacji \num{486.6761}. Jest to o tyle dziwny wynik, że jest gorszy niż
dla naszych regulatorów wyznaczonych metodą inżynierską dla tej konfiguracji torów. Możliwą przyczyną
takie sytuacji jest trudne dla $fmincon$ uwarunkowanie funkcji celu. 


Dla toru:
\begin{itemize}
  \item $y_1$ -- $u_1$
 \item $y_2$ -- $u_3$
 \item $y_3$ -- $u_4$
\end{itemize}

otrzymujemy nastawy:
\begin{equation}
  K_1 = \num{2.7249} \qquad T_{i1} = \num{3.9641}, \qquad T_{d1} = 0 \nonumber
\end{equation}
\begin{equation}
  K_2 = \num{2.9122} \qquad T_{i2} = \num{3.1237}, \qquad T_{d2} = 0
\end{equation}
\begin{equation}
  K_3 = \num{5.5929} \qquad T_{i3} = \num{9.8384}, \qquad T_{d3} = 0 \nonumber
\end{equation}

Błąd regulacji wynosi: \num{104.4858} . Jest to najlepszy ogólnie otrzymany wynik.

Dla toru:
\begin{itemize}
  \item $y_1$ -- $u_1$
 \item $y_2$ -- $u_2$
 \item $y_3$ -- $u_4$
\end{itemize}

otrzymujemy nastawy:
\begin{equation}
  K_1 = \num{2.9883} \qquad T_{i1} = \num{4.4246}, \qquad T_{d1} = 0 \nonumber
\end{equation}
\begin{equation}
  K_2 = \num{0.6972} \qquad T_{i2} = \num{8824.7}, \qquad T_{d2} = 0
\end{equation}
\begin{equation}
  K_3 = \num{5.6629} \qquad T_{i3} = \num{11.6449}, \qquad T_{d3} = 0 \nonumber
\end{equation}

Błąd regulacji wynosi: \num{115.6349} .
Jest to bardzo dobry wynik, jednakże nieco gorszy niż dla poprzedniego toru.
Zauważmy, że wyznaczone nastawy dla pierwszego i trzeciego regulatora są bardzo podobne do tych dla poprzedniego toru.
Wynika to z faktu, że zmienił się tu jedynie tor dla drugiego regulatora, gdzie teraz wpływa na wyjscie drugie regulator drugi.

Dla toru:
\begin{itemize}
  \item $y_1$ -- $u_2$
 \item $y_2$ -- $u_3$
 \item $y_3$ -- $u_1$
\end{itemize}

otrzymujemy nastawy:
\begin{equation}
  K_1 = \num{1.1001} \qquad T_{i1} = \num{1.7044}, \qquad T_{d1} = 0 \nonumber
\end{equation}
\begin{equation}
  K_2 = \num{2.2444} \qquad T_{i2} = \num{3.7464}, \qquad T_{d2} = 0
\end{equation}
\begin{equation}
  K_3 = \num{4.0054} \qquad T_{i3} = \num{15.7112}, \qquad T_{d3} = 0 \nonumber
\end{equation}

Błąd regulacji wynosi: \num{207.1231} .
Jest to wynik nieco gorszej jakości niż dla poprzednich dwóch torów.

Analizując otrzymane wyniki, zauważamy że dla trzech z czterech przypadków funkcja optymalizacji
znalazła lepsze wyniki niż dla regulatorów wyznaczonych metodą inżynierską. Jest to jak najbardziej normalne zjawisko.
Również potwierdziła się metoda Multiple Gain Array. Najlepsze wyniki otrzymaliśmy dla regulatora o konfiguracji
torów odpowiadającej najmniejszej wartości współczynnika uwarunkowania macierzy, zaś najgorsze wyniki
dla torów odpowiadających największej wartości współczynnika uwarunkowania macierzy.