\chapter{Odpowiedzi skokowe}
Celem zadania drugiego było symulacyjne wyznaczenie wszystkich odpowiedzi skokowych (każdego toru).
Założyliśmy, że obiekt będzie znajdował się przed wykonaniem skoku w wyznaczonym w zadaniu 1. punkcie pracy,
a skok będzie jednostkowy.
Na podanych niżej wykresach znajdują się wykresy odpowiedzi skokowych dla każdego toru.

\begin{figure}[tb]
\centering
\begin{tikzpicture}
\begin{axis}[
width=0.75\textwidth,
height = 0.6\textwidth,
xmin=0,xmax=200,ymin=0,ymax=2,
xlabel={Numer próbki},
ylabel={Wyjście znormalizowane},
xtick={0, 50, 100, 150, 200},
ytick={0, .5, 1, 1.5, 2},
legend pos=south east,
/pgf/number format/.cd,
use comma,
1000 sep={}
]

\addplot[blue,semithick] file {wykresy/z2_y1u1.txt};
\addplot[red,semithick] file {wykresy/z2_y1u2.txt};
\addplot[orange,semithick] file {wykresy/z2_y1u3.txt};
\addplot[violet,semithick] file {wykresy/z2_y1u4.txt};
\legend{Odpowiedź na skok $u_1$, Odpowiedź na skok $u_2$, Odpowiedź na skok $u_3$, Odpowiedź na skok $u_4$,}

\end{axis}
\end{tikzpicture}
\caption{Odpowiedzi skokowe wyjścia $y_1$}
\label{fig:z2_y1}
\end{figure}
%%%%%%%%%%%%%%%%%%%%%%%%%%%%%%%%%%%%%%%%%%%%%%%%%%%%%%%%%%%%%%%%%%%%%%%%%%%%%%%%%%%%%%%%%%%%

\begin{figure}[tb]
\centering
\begin{tikzpicture}
\begin{axis}[
width=0.75\textwidth,
height = 0.6\textwidth,
xmin=0,xmax=200,ymin=0,ymax=2,
xlabel={Numer próbki},
ylabel={Wyjście znormalizowane},
xtick={0, 50, 100, 150, 200},
ytick={0, .5, 1, 1.5, 2},
legend pos=south east,
/pgf/number format/.cd,
use comma,
1000 sep={}
]

\addplot[blue,semithick] file {wykresy/z2_y2u1.txt};
\addplot[red,semithick] file {wykresy/z2_y2u2.txt};
\addplot[orange,semithick] file {wykresy/z2_y2u3.txt};
\addplot[violet,semithick] file {wykresy/z2_y2u4.txt};
\legend{Odpowiedź na skok $u_1$, Odpowiedź na skok $u_2$, Odpowiedź na skok $u_3$, Odpowiedź na skok $u_4$,}

\end{axis}
\end{tikzpicture}
\caption{Odpowiedzi skokowe wyjścia $y_2$}
\label{fig:z2_y2}
\end{figure}
%%%%%%%%%%%%%%%%%%%%%%%%%%%%%%%%%%%%%%%%%%%%%%%%%%%%%%%%%%%%%%%%%%%%%%%%%%%%%%%%%%%%%%%%%%%%

\begin{figure}[tb]
\centering
\begin{tikzpicture}
\begin{axis}[
width=0.75\textwidth,
height = 0.6\textwidth,
xmin=0,xmax=200,ymin=0,ymax=2,
xlabel={Numer próbki},
ylabel={Wyjście znormalizowane},
xtick={0, 50, 100, 150, 200},
ytick={0, .5, 1, 1.5, 2},
legend pos=south east,
/pgf/number format/.cd,
use comma,
1000 sep={}
]

\addplot[blue,semithick] file {wykresy/z2_y3u1.txt};
\addplot[red,semithick] file {wykresy/z2_y3u2.txt};
\addplot[orange,semithick] file {wykresy/z2_y3u3.txt};
\addplot[violet,semithick] file {wykresy/z2_y3u4.txt};
\legend{Odpowiedź na skok $u_1$, Odpowiedź na skok $u_2$, Odpowiedź na skok $u_3$, Odpowiedź na skok $u_4$,}

\end{axis}
\end{tikzpicture}
\caption{Odpowiedzi skokowe wyjścia $y_3$}
\label{fig:z2_y3}
\end{figure}
%%%%%%%%%%%%%%%%%%%%%%%%%%%%%%%%%%%%%%%%%%%%%%%%%%%%%%%%%%%%%%%%%%%%%%%%%%%%%%%%%%%%%%%%%%%%
