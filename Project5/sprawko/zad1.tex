\chapter{Punkt pracy}
Celem zadania pierwszego było zweryfikowanie poprawności punktu pracy procesu. Podany punkt pracy to
$u1=u2=u3=u4=y1=y2=y3=0$.
Poprawność sprawdzimy, poprzez podanie na wejście obiektu $u1=u2=u3=u4=0$ i sprawdzenie czy wyjścia obiektu stabilizują
się na wartości 0. Eksperyment wykazał, że rzeczywiście obiekt stabilizuje się na wartościach wyjść
$y1=y2=y3=0$, a pokazują to wykresy \ref{fig:z1_y1}, \ref{fig:z1_y2} oraz \ref{fig:z1_y3}.

\begin{figure}[tb]
\centering
\begin{tikzpicture}
\begin{axis}[
width=0.75\textwidth,
height = 0.4\textwidth,
xmin=0,xmax=150,ymin=-1,ymax=1,
xlabel={Numer próbki},
ylabel={Wyjście},
xtick={0, 50, 100, 150},
ytick={-1, -0.5, 0, 0.5, 1},
legend pos=south east,
/pgf/number format/.cd,
use comma,
1000 sep={}
]

\addplot[blue,semithick] file {wykresy/z1_y1.txt};
% \legend{Znormalizowane wyjście obiektu, Aproksymacja odpowiedzi skokowej}

\end{axis}
\end{tikzpicture}
\caption{Wyjście $y_1$}
\label{fig:z1_y1}
\end{figure}

\begin{figure}[tb]
\centering
\begin{tikzpicture}
\begin{axis}[
width=0.75\textwidth,
height = 0.4\textwidth,
xmin=0,xmax=150,ymin=-1,ymax=1,
xlabel={Numer próbki},
ylabel={Wyjście},
xtick={0, 50, 100, 150},
ytick={-1, -0.5, 0, 0.5, 1},
legend pos=south east,
/pgf/number format/.cd,
use comma,
1000 sep={}
]

\addplot[blue,semithick] file {wykresy/z1_y2.txt};
% \legend{Znormalizowane wyjście obiektu, Aproksymacja odpowiedzi skokowej}

\end{axis}
\end{tikzpicture}
\caption{Wyjście $y_2$}
\label{fig:z1_y2}
\end{figure}

\begin{figure}[tb]
\centering
\begin{tikzpicture}
\begin{axis}[
width=0.75\textwidth,
height = 0.4\textwidth,
xmin=0,xmax=150,ymin=-1,ymax=1,
xlabel={Numer próbki},
ylabel={Wyjście},
xtick={0, 50, 100, 150},
ytick={-1, -0.5, 0, 0.5, 1},
legend pos=south east,
/pgf/number format/.cd,
use comma,
1000 sep={}
]

\addplot[blue,semithick] file {wykresy/z1_y3.txt};
% \legend{Znormalizowane wyjście obiektu, Aproksymacja odpowiedzi skokowej}

\end{axis}
\end{tikzpicture}
\caption{Wyjście $y_3$}
\label{fig:z1_y3}
\end{figure}
