\chapter{Zadanie 3: Algorytmy PID i DMC}
\section{Cyfrowy algorytm PID}

Celem zadania trzeciego było zaimplementowanie regulatorów $PID$ i $DMC$ odpowiednich do podanego procesu. Należy zwrócić uwagę, że dany dla projektu procesu
cechuje się większą liczbą wejść ( $4$ ) niż wyjść ( $3$ ), co wpływa na realizację implementacji regulatorów.

\section{Cyfrowy algorytm PID}

Regulator $PID$ jest opisany danymi parametrami: $K$ - wzmocnienie, $T_p$ - czas próbkowania, $T_d$ - czas wyprzedzenia, $T_i$ - czas zdwojenia, $N_u$ - ilość sterowań oraz $N_y$ - ilość wyjść.

W implementacji regulatora będziemy wykorzystywać współczynniki:

\begin{equation}
{r_0}^j=K^j*(1+T_p/(2*{T_i}^j)+{T_d}^j/T_p) \quad \forall j \in <1,n_u>
\label{r0}
\end{equation}


\begin{equation}
{r_1}^j=K^j*(T_p/(2*{T_i}^j)-2*{T_d}^j/T_p-1) \quad \forall j \in <1,n_u>
\label{r1}
\end{equation}

\begin{equation}
{r_2}^j=K*{T_d}^j/T_p \quad \forall j \in <1,n_u>
\label{r2}
\end{equation}

Po każdej iteracji obliczany jest nowy uchyb każdego wyjscia, jako różnica wartości zadanej i aktualnej wartości wyjścia.
Sterowania również są obliczane tak jak w klasycznym regulatorze PID.

\begin{equation}
U(k)^j = {r_2}^j*e(k-2)^i + {r_1}^j*e(k-1)^i + {r_0}^j*e(k)^i + u(k-1)^j \quad gdzie \quad j \in <1,n_u>, \quad i \in <1,n_y>
\label{Uk}
\end{equation}

Należy podkreślić, że w przypadku naszego procesu, każdemu wyjściu odpowiada jedno wejście, co oznacza że jedno z wejść nie będzie używane.


\section{Algorytm DMC}
W algorytmie wykonujemy następujące obliczenia:

\begin{equation}
\boldsymbol{y}^{\mathrm{zad}}(k)=\left[
\begin{array}{c}
y^{\mathrm{zad}}_1(k)\\
\vdots\\
y^{\mathrm{zad}}_{ny}(k)
\end{array}
\right]_{\mathrm{n_yx1}}
\label{yzadm}
\end{equation}

\begin{equation}
\boldsymbol{y}(k)=\left[
\begin{array}{c}
y_1(k)\\
\vdots\\
y_{ny}(k)
\end{array}
\right]_{\mathrm{n_yx1}}
\label{ym}
\end{equation}

\begin{equation}
\boldsymbol{u}(k)=\left[
\begin{array}{c}
u_1(k)\\
\vdots\\
u_{n_u}(k)
\end{array}
\right]_{\mathrm{n_ux1}}
\label{Um}
\end{equation}

\begin{equation}
\triangle\boldsymbol{u}(k)=\left[
\begin{array}{c}
\triangle u_1(k)\\
\vdots\\
\triangle u_{n_u}(k)
\end{array}
\right]_{\mathrm{n_ux1}}
\label{dUm}
\end{equation}

\begin{equation}
\boldsymbol{Y}(k)=\left[
\begin{array}{c}
y(k|k)\\
\vdots\\
y(k|k)
\end{array}
\right]_{\mathrm{N*n_yx1}}
\label{Y}
\end{equation}

\begin{equation}
\boldsymbol{Y}^{\mathrm{zad}}(k)=\left[
\begin{array}{c}
y^{\mathrm{zad}}(k|k)\\
\vdots\\
y^{\mathrm{zad}}(k|k)
\end{array}
\right]_{\mathrm{N*n_yx1}}
\label{Yzad}
\end{equation}

\begin{equation}
\triangle\boldsymbol{U}(k)=\left[
\begin{array}{c}
\triangle u(k|k)\\
\vdots\\
\triangle u(k+N_u-1|k)
\end{array}
\right]_{\mathrm{N_u*n_ux1}}
\label{dUms}
\end{equation}

\begin{equation}
\triangle\boldsymbol{U^P}(k)=\left[
\begin{array}{c}
\triangle u(k-1)\\
\vdots\\
\triangle u(k-(D-1))
\end{array}
\right]_{\mathrm{(D-1)*n_ux1}}
\label{dUPm}
\end{equation}

\begin{equation}
\boldsymbol{S_l}=\left[
\begin{array}
{cccc}
s^{11}_l & s^{12}_l & \ldots & s^{1n_u}_l\\
s^{21}_l & s^{22}_l & \ldots & s^{2n_u}_l\\
\vdots & \vdots & \ddots & \vdots\\
s^{n_y1}_l& s^{n_y2}_l & \ldots &  s^{n_yn_u}_l
\end{array}
\right]_{\mathrm{n_yxn_u}}
\label{S}
, l=1,...,D.
\end{equation}

\begin{equation}
\boldsymbol{M}=\left[
\begin{array}
{cccc}
S_{1} & 0 & \ldots & 0\\
S_{2} & S_{1} & \ldots & 0\\
\vdots & \vdots & \ddots & \vdots\\
S_{N} & S_{N-1} & \ldots &  S_{N-N_{\mathrm{u}}+1}
\end{array}
\right]_{\mathrm{(N*n_y)x(N_u*n_u)}}
\label{Mm}
\end{equation}

\begin{equation}
\boldsymbol{M^P}=\left[
\begin{array}
{cccc}
S_{2}-S_{1} & S_{3}-S_{2} & \ldots & S_{D}-S_{D-1}\\
S_{3}-S_{1} & S_{4}-S_{2} & \ldots & S_{D+1}-S_{D-1}\\
\vdots & \vdots & \ddots & \vdots\\
S_{N+1}-S_{1} & S_{N+2}-S_{2} & \ldots &  S_{N+D-1}-S_{D-1}
\end{array}
\right]_{\mathrm{(N*n_y)x((D-1)*n_u)}}
\label{MPm}
\end{equation}

\begin{equation}
\boldsymbol{\Psi}=\left[
\begin{array}
{cccccccccc}
\psi_{1,1} & & & & & & & & & \\
 & \ddots & & & & & & & & \\
 & & \psi_{1,n_y} & & & & & & & \\
 & & & \psi_{2,1} & & & & & & \\
 & & & & \ddots & & & & & \\
 & & & & & \psi_{2,n_y} & & & & \\
 & & & & & & \ddots & & & \\
 & & & & & & & \psi_{N,1} & & \\
 & & & & & & & & \ddots & \\
 & & & & & & & & & \psi_{N,n_y} \\
\end{array}
\right]_{(N*n_y)x(N*n_y)}
\end{equation}

\begin{equation}
\boldsymbol{\Lambda}=\left[
\begin{array}
{cccccccccc}
\lambda_{0,1} & & & & & & & & & \\
& \ddots & & & & & & & & \\
& & \lambda_{0,n_u} & & & & & & & \\
& & & \lambda_{1,1} & & & & & & \\
& & & & \ddots & & & & & \\
& & & & & \lambda_{1,n_u} & & & & \\
& & & & & & \ddots & & & \\
& & & & & & & \lambda_{N_u-1,1} & & \\
& & & & & & & & \ddots & \\
& & & & & & & & & \lambda_{N_u-1,n_u} \\
\end{array}
\right]_{(N_u*n_u)x(N_u*n_u)}
\end{equation}

\begin{equation}
Y^0(k)=Y(k)+M^P
\triangle U^P(k)
\label{Y0}
\end{equation}

\begin{equation}
K=(M^T\Psi M+\Lambda)^{-1}M^T\Psi
\label{K}
\end{equation}

\begin{equation}
\triangle U(k)=K(Y^{zad}(k)-Y^0(k))
\label{dU1}
\end{equation}

W naszym zadaniu używamy tylko pierwszy element $\triangle U(k)$ czyli inaczej $\triangle u(k|k)$. W tym celu macierz K przekształcamy do postaci:

\begin{equation}
	\boldsymbol{K}=\left[
	\begin{array}{c}
	K_1\\
	K_2\\
	\vdots\\
	K_{N_u}
	\end{array}
	\right]=\left[
	\begin{array}
	{cccc}
	k_{1,1} & k_{1,2} & \ldots & k_{1,N}\\
	k_{2,1} & k_{2,2} & \ldots & k_{2,N}\\
	\vdots & \vdots & \ddots & \vdots\\
	k_{N_u,1} & k_{N_u,2} & \ldots & k_{N_u,N}
	\end{array}
	\right]
\end{equation}
gdzie $k_{i,j}$ to macierz o wymiarach $n_u$ x $n_y$, a więc $K_i$ to macierz o wymiarach $n_u$ x (N*$n_y$). Prawo regulacji wygląda następująco:

\begin{equation}
	\triangle u(k)=K_1(Y^{zad}(k)-Y^0(k))=K_1(Y^{zad}(k)-Y(k)-M^P\triangle U^P(k))
\end{equation}


Aktualne sterowanie obliczamy jako suma sterowania poprzedniego i aktualnie wyliczonego $\triangle u(k|k)$.


% \end{equation}
