\chapter{PID}
Na początku strojenia wyznaczona została macierz wzmocnień, zawierająca
wzmocnienie każdego z wyjść w zależności od wejścia. Macierz ta jest następująca:
\begin{equation}
  \bm{K} =
  \begin{bmatrix}
    \num{1,9500}  &  \num{1,5000}  &  \num{1,2500} \\
    \num{1,5500}  &  \num{1,2000}  &  \num{0,1500} \\
    \num{0,8500}  &  \num{0,9000}  &  \num{0,3000} \\
    \num{0,1000} &   \num{0,5500}  &  \num{1,2000}
  \end{bmatrix}
\end{equation}
Następnie otrzymujemy z tego cztery macierze $\bm{K}_i$. Każda z nich powstaje poprzez
usunięcie $i$-tego wiersza z macierzy $\bm{K}$. Macierze te są następujące:
\begin{equation}
  \bm{K}_1 =
  \begin{bmatrix}
    \num{1,5500}   & \num{1,2000} &   \num{0,1500} \\
        \num{0,8500}   & \num{0,9000} &   \num{0,3000} \\
        \num{0,1000}   & \num{0,5500}  &  \num{1,2000}
  \end{bmatrix}
\end{equation}
\begin{equation}
  \bm{K}_2 =
  \begin{bmatrix}
    \num{1,9500} &   \num{1,5000} &   \num{1,2500} \\
       \num{0,8500}   & \num{0,9000} &   \num{0,3000} \\
       \num{0,1000}   & \num{0,5500} &   \num{1,2000}
  \end{bmatrix}
\end{equation}
\begin{equation}
  \bm{K}_3 =
  \begin{bmatrix}
    \num{1,9500} &   \num{1,5000} &   \num{1,2500} \\
       \num{1,5500}   & \num{1,2000} &   \num{0,1500} \\
       \num{0,1000}   & \num{0,5500}  &  \num{1,2000}
  \end{bmatrix}
\end{equation}
\begin{equation}
  \bm{K}_4 =
  \begin{bmatrix}
    \num{1,9500}  &  \num{1,5000}  &  \num{1,2500} \\
    \num{1,5500}  &  \num{1,2000}  &  \num{0,1500} \\
    \num{0,8500}  &  \num{0,9000}  &  \num{0,3000}
  \end{bmatrix}
\end{equation}
Następnie obliczane są wskaźniki uwarunkowania każdej z czterech macierzy:
\begin{align}
  \text{cond}(\bm{K}_1) &= \num{23,8517} \\
  \text{cond}(\bm{K}_2) &=  \num{12,4174} \\
  \text{cond}(\bm{K}_3) &= \num{17,5922} \\
  \text{cond}(\bm{K}_4) &= \num{20,1116}
\end{align}
Nastęnie wylicza się macierz $KK_i = K_i * K_i^{-1}$. Z tej macierzy
wybiera się tory sterowania, poprzez wybranie najmniejszych wartości dodatnich
z macierzy $KK_i$, tak aby wybrana była tylko jedna wartość w danym wierszu i kolumnie.
Wartości ujemne są wykluczone. Tory sterowania są wyznaczane poprzez numer kolumny
i wiersza wybranych wartości. Numer kolumny odpowiada wyjściu, a numer wiersza sterowaniu.
Teoretycznie najlepszy wynik będzie osiągnięty dla
macierzy $KK_i$, dla której wskaźnik uwarunkowania $K_i$ był najmniejszy, czyli
w naszym wypadku $K_2$, ale mimo to sprawdzimy wszystkie cztery opcje. Macierze
wychodzą następujące:



\begin{equation}
  \bm{KK}_1 =
  \begin{bmatrix}
   \num{ 4,9438} &  \num{-4,1412}  & \num{ 0,1974} \\
   \num{-4,0222}  & \num{ 5,7882}  & \num{-0,7660} \\
   \num{ 0,0784} &  \num{-0,6471} &  \num{ 1,5686}
  \end{bmatrix}
\end{equation}
\begin{equation}
  \bm{KK}_2 =
  \begin{bmatrix}
   \num{ 2,3138} &  \num{-1,9258}  & \num{ 0,6119} \\
   \num{-1,2263}   & \num{ 2,5852}  & \num{-0,3589} \\
   \num{-0,0875}   &\num{ 0,3406} &  \num{ 0,7470}
  \end{bmatrix}
\end{equation}
\begin{equation}
  \bm{KK}_3 =
  \begin{bmatrix}
   \num{ 3,3287} &  \num{-3,4800}  & \num{ 1,1514} \\
   \num{-2,1683}  & \num{ 3,3423} &  \num{-0,1740} \\
   \num{-0,1603}  & \num{ 1,1377}  & \num{ 0,0226}
  \end{bmatrix}
\end{equation}
\begin{equation}
  \bm{KK}_4 =
  \begin{bmatrix}
   \num{ 1,0935}  & \num{-1,2617}  & \num{ 1,1682} \\
   \num{ 2,6075} &  \num{-1,4280}  & \num{-0,1794} \\
   \num{-2,7009}  & \num{ 3,6897} &  \num{ 0,0112}
  \end{bmatrix}
\end{equation}
Stąd wybieramy cztery opcje torów sterowania. Dla $\bm{KK}_1$:
\begin{itemize}
  \item $y_1$ -- $u_4$
 \item $y_2$ -- $u_3$
 \item $y_3$ -- $u_2$
\end{itemize}
 Dla $\bm{KK}_2$:
\begin{itemize}
  \item $y_1$ -- $u_1$
 \item $y_2$ -- $u_3$
 \item $y_3$ -- $u_4$
\end{itemize}
Zgodnie z wyznaczonymi wskaźnikami uwarunkowania te tory powinny być najlepsze. Dla $\bm{KK}_3$:
\begin{itemize}
  \item $y_1$ -- $u_1$
 \item $y_2$ -- $u_2$
 \item $y_3$ -- $u_4$
\end{itemize}
Dla $\bm{KK}_4$:
\begin{itemize}
  \item $y_1$ -- $u_2$
 \item $y_2$ -- $u_3$
 \item $y_3$ -- $u_1$
\end{itemize}
Mając teoretycznie najlepsze tory sterowania przystąpiliśmy do dobierania nastaw
dla regulatorów. Nasza taktyka polegała na wyłączeniu wszelkich regulatorów, a
następnie znalezieniu wartości wzmocnienia pierwszego regulatora, dla którego
oscylacje są niegasnące. Mając tą wartość wzmocnienia dzielona była ona przez
dwa i dołączany był regulator drugi. Znów szukaliśmy wartości oscylacji niegasnących
i po znalezieniu dzieliliśmy wzmocnienie drugiego regulatora na dwa. Następnie
dołączaliśmy trzeci regulator i postępowaliśmy tak samo. Następnie dobieraliśmy
wartości całkowania, metodą prób i błędów, a na końcu tak samo dobieraliśmy
wartości członów różniczkujących dla regulatorów. Zaskakująco metoda ta okazała się
przynosić zadowalające rezultaty. Dla toru otrzymanego na podstawie macierzy $\bm{KK}_1$
nie udało nam się wyznaczyć ręcznie nastaw. W tym przypadku zadanie było niezwykle
trudne. Dopiero funkcja minimalizacji dała nam nastawy, dla których obiekt działał.
Następnie sprawdzony został tor otrzymany na podstawie macierzy $\bm{KK}_2$.
Po kilku eksperymentach otrzymaliśmy następujące nastawy:
\begin{equation}
  K_1 = \num{5.1325} \qquad T_{i1} = 9, \qquad T_{d1} = 0 \nonumber
\end{equation}
\begin{equation}
  K_2 = \num{3.1150} \qquad T_{i2} = 10, \qquad T_{d2} = 0
\end{equation}
\begin{equation}
  K_3 = \num{7.2400} \qquad T_{i3} = 10, \qquad T_{d3} = 0 \nonumber
\end{equation}
Wskaźnik jakości regulacji dla takich nastaw wynosił $E_2 = \num{134.8370}$, czyli
lepszy niż ten dla pierwszego
toru sterowania. Podobnie jak wcześniej, dodanie różniczkowania nie poprawiało
wyników w sensie wskaźnika jakości, więc z niego zrezygnowaliśmy. Działanie
tych regulatoróœ przedstawiają wykresy \ref{fig:z3_pid2_y1},
\ref{fig:z3_pid2_y2} oraz \ref{fig:z3_pid2_y3}. Dla ułatwienia porównania zastosowano
tą samą skalę co w przypadku poprzedniego zestawu regulatorów.

Dla torów wynikających z macierzy $\bm{KK}_3$ otrzymaliśmy nastawy:
\begin{equation}
  K_1 = \num{5.1320} \qquad T_{i1} = 10, \qquad T_{d1} = 0 \nonumber
\end{equation}
\begin{equation}
  K_2 = \num{0.6130} \qquad T_{i2} = 8, \qquad T_{d2} = 0
\end{equation}
\begin{equation}
  K_3 = \num{7.2710} \qquad T_{i3} = 9, \qquad T_{d3} = 0 \nonumber
\end{equation}
oraz wskaźnik jakości $E_3 = \num{141.6150}$. Jest to wynik niewiele gorszy, niż
w przypadku macierzy $\bm{KK}_2$, która dała najlepsze rezultaty.
Działanie takich regulatorów przedstawiają wykresy \ref{fig:z3_pid4_y1},
\ref{fig:z3_pid4_y2} oraz \ref{fig:z3_pid4_y3}.

Dla macierzy $\bm{KK}_4$ otrzymane nastawy
były nasępujące:
\begin{equation}
  K_1 = \num{1.3170} \qquad T_{i1} = 4, \qquad T_{d1} = 0 \nonumber
\end{equation}
\begin{equation}
  K_2 = \num{14.8350} \qquad T_{i2} = 9, \qquad T_{d2} = 0
\end{equation}
\begin{equation}
  K_3 = \num{5.6000} \qquad T_{i3} = 6, \qquad T_{d3} = 0 \nonumber
\end{equation}
Wskaźnik jakości wynosił $\num{565.3247}$, czyli był najgorszy ze wszystkich.
Znów, dodanie różniczkowania jedynie pogarszało wyniki. Działanie tych
regulatorów ukazują wykresy \ref{fig:z3_pid3_y1},
\ref{fig:z3_pid3_y2} oraz \ref{fig:z3_pid3_y3}.

%%%%%%%%%%%%%%%%%%%%%%%%%%%%%%%%%%%%%%%%%%%%%%%%%%%%%%%%%%%%%%%%%%%%%%%%%%%%%%%%%%%%%%%%%%%%
\begin{figure}[b]
\centering
\begin{tikzpicture}
\begin{axis}[
width=0.75\textwidth,
height = 0.4\textwidth,
xmin=0,xmax=1100,ymin=-15,ymax=15,
xlabel={Numer próbki},
ylabel={Wyjście},
xtick={0, 200, 400, 600, 800, 1000},
ytick={-15, -10, -5, 0, 5, 10, 15},
legend pos=north east,
/pgf/number format/.cd,
use comma,
1000 sep={}
]

\addplot[blue,semithick] file {wykresy/z3_yzad.txt};
\addplot[red,semithick] file {wykresy/z3_pid2_y1.txt};
\legend{Wartość zadana, Wyjście $y_1$}

\end{axis}
\end{tikzpicture}
\caption{Trajektoria wyjścia $y_1$, dla drugiego zestawu regulatorów PID}
\label{fig:z3_pid2_y1}
\end{figure}
%%%%%%%%%%%%%%%%%%%%%%%%%%%%%%%%%%%%%%%%%%%%%%%%%%%%%%%%%%%%%%%%%%%%%%%%%%%%%%%%%%%%%%%%%%%%

%%%%%%%%%%%%%%%%%%%%%%%%%%%%%%%%%%%%%%%%%%%%%%%%%%%%%%%%%%%%%%%%%%%%%%%%%%%%%%%%%%%%%%%%%%%%
\begin{figure}[b]
\centering
\begin{tikzpicture}
\begin{axis}[
width=0.75\textwidth,
height = 0.4\textwidth,
xmin=0,xmax=1100,ymin=-5,ymax=5,
xlabel={Numer próbki},
ylabel={Wyjście},
xtick={0, 200, 400, 600, 800, 1000},
ytick={-5, -4, -2, 0, 2, 4, 5},
legend pos=north east,
/pgf/number format/.cd,
use comma,
1000 sep={}
]

\addplot[blue,semithick] file {wykresy/z3_yzad.txt};
\addplot[red,semithick] file {wykresy/z3_pid2_y2.txt};
\legend{Wartość zadana, Wyjście $y_2$}

\end{axis}
\end{tikzpicture}
\caption{Trajektoria wyjścia $y_2$, dla drugiego zestawu regulatorów PID}
\label{fig:z3_pid2_y2}
\end{figure}
%%%%%%%%%%%%%%%%%%%%%%%%%%%%%%%%%%%%%%%%%%%%%%%%%%%%%%%%%%%%%%%%%%%%%%%%%%%%%%%%%%%%%%%%%%%%

%%%%%%%%%%%%%%%%%%%%%%%%%%%%%%%%%%%%%%%%%%%%%%%%%%%%%%%%%%%%%%%%%%%%%%%%%%%%%%%%%%%%%%%%%%%%
\begin{figure}[b]
\centering
\begin{tikzpicture}
\begin{axis}[
width=0.75\textwidth,
height = 0.4\textwidth,
xmin=0,xmax=1100,ymin=-2.5,ymax=2.5,
xlabel={Numer próbki},
ylabel={Wyjście},
xtick={0, 200, 400, 600, 800, 1000},
ytick={-2.5, -2, -1.5, -1, -.5, 0, .5, 1, 1.5, 2, 2.5},
legend pos=north east,
/pgf/number format/.cd,
use comma,
1000 sep={}
]

\addplot[blue,semithick] file {wykresy/z3_yzad.txt};
\addplot[red,semithick] file {wykresy/z3_pid2_y3.txt};
\legend{Wartość zadana, Wyjście $y_3$}

\end{axis}
\end{tikzpicture}
\caption{Trajektoria wyjścia $y_3$, dla drugiego zestawu regulatorów PID}
\label{fig:z3_pid2_y3}
\end{figure}
%%%%%%%%%%%%%%%%%%%%%%%%%%%%%%%%%%%%%%%%%%%%%%%%%%%%%%%%%%%%%%%%%%%%%%%%%%%%%%%%%%%%%%%%%%%%

%%%%%%%%%%%%%%%%%%%%%%%%%%%%%%%%%%%%%%%%%%%%%%%%%%%%%%%%%%%%%%%%%%%%%%%%%%%%%%%%%%%%%%%%%%%%
\begin{figure}[b]
\centering
\begin{tikzpicture}
\begin{axis}[
width=0.75\textwidth,
height = 0.4\textwidth,
xmin=0,xmax=1100,ymin=-15,ymax=15,
xlabel={Numer próbki},
ylabel={Wyjście},
xtick={0, 200, 400, 600, 800, 1000},
ytick={-15, -10, -5, 0, 5, 10, 15},
legend pos=north east,
/pgf/number format/.cd,
use comma,
1000 sep={}
]

\addplot[blue,semithick] file {wykresy/z3_yzad.txt};
\addplot[red,semithick] file {wykresy/z3_pid4_y1.txt};
\legend{Wartość zadana, Wyjście $y_1$}

\end{axis}
\end{tikzpicture}
\caption{Trajektoria wyjścia $y_1$, dla trzeciego zestawu regulatorów PID}
\label{fig:z3_pid4_y1}
\end{figure}
%%%%%%%%%%%%%%%%%%%%%%%%%%%%%%%%%%%%%%%%%%%%%%%%%%%%%%%%%%%%%%%%%%%%%%%%%%%%%%%%%%%%%%%%%%%%

%%%%%%%%%%%%%%%%%%%%%%%%%%%%%%%%%%%%%%%%%%%%%%%%%%%%%%%%%%%%%%%%%%%%%%%%%%%%%%%%%%%%%%%%%%%%
\begin{figure}[b]
\centering
\begin{tikzpicture}
\begin{axis}[
width=0.75\textwidth,
height = 0.4\textwidth,
xmin=0,xmax=1100,ymin=-5,ymax=5,
xlabel={Numer próbki},
ylabel={Wyjście},
xtick={0, 200, 400, 600, 800, 1000},
ytick={-5, -4, -2, 0, 2, 4, 5},
legend pos=north east,
/pgf/number format/.cd,
use comma,
1000 sep={}
]

\addplot[blue,semithick] file {wykresy/z3_yzad.txt};
\addplot[red,semithick] file {wykresy/z3_pid4_y2.txt};
\legend{Wartość zadana, Wyjście $y_2$}

\end{axis}
\end{tikzpicture}
\caption{Trajektoria wyjścia $y_2$, dla trzeciego zestawu regulatorów PID}
\label{fig:z3_pid4_y2}
\end{figure}
%%%%%%%%%%%%%%%%%%%%%%%%%%%%%%%%%%%%%%%%%%%%%%%%%%%%%%%%%%%%%%%%%%%%%%%%%%%%%%%%%%%%%%%%%%%%

%%%%%%%%%%%%%%%%%%%%%%%%%%%%%%%%%%%%%%%%%%%%%%%%%%%%%%%%%%%%%%%%%%%%%%%%%%%%%%%%%%%%%%%%%%%%
\begin{figure}[b]
\centering
\begin{tikzpicture}
\begin{axis}[
width=0.75\textwidth,
height = 0.4\textwidth,
xmin=0,xmax=1100,ymin=-2.5,ymax=2.5,
xlabel={Numer próbki},
ylabel={Wyjście},
xtick={0, 200, 400, 600, 800, 1000},
ytick={-2.5, -2, -1.5, -1, -.5, 0, .5, 1, 1.5, 2, 2.5},
legend pos=north east,
/pgf/number format/.cd,
use comma,
1000 sep={}
]

\addplot[blue,semithick] file {wykresy/z3_yzad.txt};
\addplot[red,semithick] file {wykresy/z3_pid4_y3.txt};
\legend{Wartość zadana, Wyjście $y_3$}

\end{axis}
\end{tikzpicture}
\caption{Trajektoria wyjścia $y_3$, dla trzeciego zestawu regulatorów PID}
\label{fig:z3_pid4_y3}
\end{figure}
%%%%%%%%%%%%%%%%%%%%%%%%%%%%%%%%%%%%%%%%%%%%%%%%%%%%%%%%%%%%%%%%%%%%%%%%%%%%%%%%%%%%%%%%%%%%

%%%%%%%%%%%%%%%%%%%%%%%%%%%%%%%%%%%%%%%%%%%%%%%%%%%%%%%%%%%%%%%%%%%%%%%%%%%%%%%%%%%%%%%%%%%%
\begin{figure}[b]
\centering
\begin{tikzpicture}
\begin{axis}[
width=0.75\textwidth,
height = 0.4\textwidth,
xmin=0,xmax=1100,ymin=-15,ymax=15,
xlabel={Numer próbki},
ylabel={Wyjście},
xtick={0, 200, 400, 600, 800, 1000},
ytick={-15, -10, -5, 0, 5, 10, 15},
legend pos=north east,
/pgf/number format/.cd,
use comma,
1000 sep={}
]

\addplot[blue,semithick] file {wykresy/z3_yzad.txt};
\addplot[red,semithick] file {wykresy/z3_pid3_y1.txt};
\legend{Wartość zadana, Wyjście $y_1$}

\end{axis}
\end{tikzpicture}
\caption{Trajektoria wyjścia $y_1$, dla czwartego zestawu regulatorów PID}
\label{fig:z3_pid3_y1}
\end{figure}
%%%%%%%%%%%%%%%%%%%%%%%%%%%%%%%%%%%%%%%%%%%%%%%%%%%%%%%%%%%%%%%%%%%%%%%%%%%%%%%%%%%%%%%%%%%%

%%%%%%%%%%%%%%%%%%%%%%%%%%%%%%%%%%%%%%%%%%%%%%%%%%%%%%%%%%%%%%%%%%%%%%%%%%%%%%%%%%%%%%%%%%%%
\begin{figure}[b]
\centering
\begin{tikzpicture}
\begin{axis}[
width=0.75\textwidth,
height = 0.4\textwidth,
xmin=0,xmax=1100,ymin=-5,ymax=5,
xlabel={Numer próbki},
ylabel={Wyjście},
xtick={0, 200, 400, 600, 800, 1000},
ytick={-5, -4, -2, 0, 2, 4, 5},
legend pos=north east,
/pgf/number format/.cd,
use comma,
1000 sep={}
]

\addplot[blue,semithick] file {wykresy/z3_yzad.txt};
\addplot[red,semithick] file {wykresy/z3_pid3_y2.txt};
\legend{Wartość zadana, Wyjście $y_2$}

\end{axis}
\end{tikzpicture}
\caption{Trajektoria wyjścia $y_2$, dla czwartego zestawu regulatorów PID}
\label{fig:z3_pid3_y2}
\end{figure}
%%%%%%%%%%%%%%%%%%%%%%%%%%%%%%%%%%%%%%%%%%%%%%%%%%%%%%%%%%%%%%%%%%%%%%%%%%%%%%%%%%%%%%%%%%%%

%%%%%%%%%%%%%%%%%%%%%%%%%%%%%%%%%%%%%%%%%%%%%%%%%%%%%%%%%%%%%%%%%%%%%%%%%%%%%%%%%%%%%%%%%%%%
\begin{figure}[b]
\centering
\begin{tikzpicture}
\begin{axis}[
width=0.75\textwidth,
height = 0.4\textwidth,
xmin=0,xmax=1100,ymin=-2.5,ymax=2.5,
xlabel={Numer próbki},
ylabel={Wyjście},
xtick={0, 200, 400, 600, 800, 1000},
ytick={-2.5, -2, -1.5, -1, -.5, 0, .5, 1, 1.5, 2, 2.5},
legend pos=north east,
/pgf/number format/.cd,
use comma,
1000 sep={}
]

\addplot[blue,semithick] file {wykresy/z3_yzad.txt};
\addplot[red,semithick] file {wykresy/z3_pid3_y3.txt};
\legend{Wartość zadana, Wyjście $y_3$}

\end{axis}
\end{tikzpicture}
\caption{Trajektoria wyjścia $y_3$, dla czwartego zestawu regulatorów PID}
\label{fig:z3_pid3_y3}
\end{figure}
%%%%%%%%%%%%%%%%%%%%%%%%%%%%%%%%%%%%%%%%%%%%%%%%%%%%%%%%%%%%%%%%%%%%%%%%%%%%%%%%%%%%%%%%%%%%
