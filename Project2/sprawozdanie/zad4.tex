
\externaldocument{zad3}
\chapter{Algorytm DMC}
\section{Wersja analityczna}
Do realizacji analitycznej wersji algorytmu DMC wykorzystano następujące wzory:

\begin{equation}
\boldsymbol{Y}(k)=\left[
\begin{array}{c}
y(k)\\
\vdots\\
y(k)
\end{array}
\right]_{\mathrm{Nx1}}
\label{ym}
\end{equation}

\begin{equation}
\boldsymbol{Y}^{\mathrm{zad}}(k)=\left[
\begin{array}{c}
Y^{\mathrm{zad}}(k)\\
\vdots\\
Y^{\mathrm{zad}}(k)
\end{array}
\right]_{\mathrm{Nx1}}
\label{yzadm}
\end{equation}

\begin{equation}
\triangle\boldsymbol{U}(k)=\left[
\begin{array}{c}
\triangle u(k|k)\\
\vdots\\
\triangle u(k+N_u -1 |k)
\end{array}
\right]_{\mathrm{N_ux1}}
\label{dUm}
\end{equation}

\begin{equation}
\triangle\boldsymbol{U^P}(k)=\left[
\begin{array}{c}
\triangle u(k-1)\\
\vdots\\
\triangle u(k-(D-1))
\end{array}
\right]_{\mathrm{(D-1)x1}}
\label{dUPm}
\end{equation}

\begin{equation}
\boldsymbol{M}=\left[
\begin{array}
{cccc}
s_{1} & 0 & \ldots & 0\\
s_{2} & s_{1} & \ldots & 0\\
\vdots & \vdots & \ddots & \vdots\\
s_{N} & s_{N-1} & \ldots &  s_{N-N_{\mathrm{u}}+1}
\end{array}
\right]_{\mathrm{NxN_u}}
\label{Mm}
\end{equation}

\begin{equation}
\boldsymbol{M^P}=\left[
\begin{array}
{cccc}
s_{2}-s_{1} & s_{3}-s_{2} & \ldots & s_{D}-s_{D-1}\\
s_{3}-s_{1} & s_{4}-s_{2} & \ldots & s_{D+1}-s_{D-1}\\
\vdots & \vdots & \ddots & \vdots\\
s_{N+1}-s_{1} & s_{N+2}-s_{2} & \ldots &  s_{N+D-1}-S_{D-1}
\end{array}
\right]_{\mathrm{NxD-1}}
\label{MPm}
\end{equation}

\begin{equation}
Y^0(k)=Y(k)+M^P
\triangle U^P(k)
\label{Y0}
\end{equation}

\begin{equation}
K=(M^TM+\lambda*I)^{-1}M^T
\label{K}
\end{equation}

\begin{equation}
\triangle U(k)=K(Y^{zad}(k)-Y^0(k))
\label{dU1}
\end{equation}

\section{Dobieranie nastaw}
Długość horyzontu dynamiki została wyznaczona na podstawie obserwacji odpowiedzi skokowej wyznaczonej w poprzednim zadaniu. Na wykresie \ref{fig:s} można zaobserwować, że jej wartość ustala się w okolicach chwili $k=100$. Dlatego też przyjęto, że wykorzystywaną dalej wartością horyzontu dynamiki jest $D=100$. Parametr ten nie podlega optymalizacji.

Jako początkowe nastawy regulatora przyjęto parametry $D=N=N_u=100$ oraz $\lambda=1$.

\subsection{Horyzont predykcji}
 Pierwszym optymalizacji nastaw jest badanie zachowania regulatora przy zmniejszaniu horyzontu predykcji $N$. Jak się okazało, wysoka wartość $N$ nie jest konieczna dla prawidłowego działania algorytmu. Dla $N=20$ spadek jakości regulacji jest niewielki i to właśnie ta wartość będzie używana w kolejnych etapach doboru nastaw. Przebiegi sterowania i odpowiedzi obiektu podczas regulacji do wartości zadanej $Y_{zad}=1$ dla różnych horyzontów predykcji przedstawia wykres \ref{fig:z4_y}.

\begin{figure}[!htb]
\centering
\begin{tikzpicture}
\begin{groupplot}[group style={group size=1 by 3}, width=0.9\textwidth, height=0.75\textwidth]
\nextgroupplot
[
xmin=0,xmax=300,ymin=0,ymax=1.4,
xlabel={$k$},
	ylabel={$Y(k)$},
	xtick={0,50,100,150,200,250,300},
	ytick={0,0.2,0.4,0.6,0.8,1,1.2,1.4,1.6},
	/pgf/number format/.cd,
	use comma,
	1000 sep={}
]
	\addplot[blue,semithick] file {wykresy/zad4_y_100.txt};
	\addplot[brown,semithick] file {wykresy/zad4_y_50.txt};
	\addplot[purple,semithick] file {wykresy/zad4_y_20.txt};
	\addplot[green,semithick] file {wykresy/zad4_y_10.txt};
	\addplot[red,semithick] file {wykresy/zad4_y_8.txt};

\nextgroupplot
[
xmin=0,xmax=300,ymin=0,ymax=2.5,
xlabel={$k$},
ylabel={$U(k)$},
xtick={0, 50, 100, 150, 200, 250, 300},
ytick={0, 0.5, 1, 1.5, 2, 2.5},
/pgf/number format/.cd,
use comma,
1000 sep={}
]
	\addplot[blue,semithick] file {wykresy/zad4_u_100.txt};
	\addplot[brown,semithick] file {wykresy/zad4_u_50.txt};
	\addplot[purple,semithick] file {wykresy/zad4_u_20.txt};
	\addplot[green,semithick] file {wykresy/zad4_u_10.txt};
	\addplot[red,semithick] file {wykresy/zad4_u_8.txt};
	\legend{$N=\num{100}$,$N=\num{50}$,$N=\num{20}$,$N=\num{10}$,$N=\num{8}$}
\end{groupplot}
\end{tikzpicture}
\caption{Działanie regulatora DMC dla różnych horyzontów dynamiki}
\label{fig:z4_y}
\end{figure}

Wskaźniki jakości regulacji dla różnych horyzontów predykcji:
\begin{itemize}
	\item $N=100$, $E=9,9973$
	\item $N=50$, $E=9,9973$
	\item $N=20$, $E=10,0289$
	\item $N=10$, $E=12,2101$
	\item $N=8$, $E=19,8924$
\end{itemize}
\FloatBarrier
\subsection{Horyzont sterowania}
Kolejnym krokiem po doborze parametru $N$ jest wybór horyzontu sterowania $N_u$.
Wskaźniki jakości regulacji dla różnych horyzontów sterowania:
\begin{itemize}
	\item $N_u=20$, $E=10,0289$
	\item $N_u=10$, $E=10,03$
	\item $N_u=5$, $E=10,1213$
	\item $N_u=2$, $E=12,1356$
	\item $N_u=1$, $E=9,8556$
\end{itemize}
Dobrym kompromisem między jakością regulacji a przebiegiem sygnału sterowania wydaje się wartość $N_u=5$. Taki horyzont sterowania będzie wykorzystywany w kolejnych etapach projektu. 

\begin{figure}[!htb]
\centering
\begin{tikzpicture}
\begin{groupplot}[group style={group size=1 by 3}, width=0.9\textwidth, height=0.75\textwidth]
\nextgroupplot
[
xmin=0,xmax=300,ymin=0,ymax=1.4,
xlabel={$k$},
	ylabel={$Y(k)$},
	xtick={0,50,100,150,200,250,300},
	ytick={0,0.2,0.4,0.6,0.8,1,1.2,1.4,1.6},
	/pgf/number format/.cd,
	use comma,
	1000 sep={}
]
	\addplot[blue,semithick] file {wykresy/zad4_y_nu20.txt};
	\addplot[brown,semithick] file {wykresy/zad4_y_nu10.txt};
	\addplot[purple,semithick] file {wykresy/zad4_y_nu5.txt};
	\addplot[green,semithick] file {wykresy/zad4_y_nu2.txt};
	\addplot[red,semithick] file {wykresy/zad4_y_nu1.txt};

\nextgroupplot
[
xmin=0,xmax=300,ymin=0,ymax=2.5,
xlabel={$k$},
ylabel={$U(k)$},
xtick={0, 50, 100, 150, 200, 250, 300},
ytick={0, 0.5, 1, 1.5, 2, 2.5},
/pgf/number format/.cd,
use comma,
1000 sep={}
]
	\addplot[blue,semithick] file {wykresy/zad4_u_nu20.txt};
	\addplot[brown,semithick] file {wykresy/zad4_u_nu10.txt};
	\addplot[purple,semithick] file {wykresy/zad4_u_nu5.txt};
	\addplot[green,semithick] file {wykresy/zad4_u_nu2.txt};
	\addplot[red,semithick] file {wykresy/zad4_u_nu1.txt};
	\legend{$N_u=\num{20}$,$N_u=\num{10}$,$N_u=\num{5}$,$N_u=\num{2}$,$N_u=\num{1}$}
\end{groupplot}
\end{tikzpicture}
\caption{Działanie regulatora DMC dla różnych horyzontów dynamiki}
\label{fig:z4_nu}
\end{figure}
\FloatBarrier
\subsection{Parametr lambda}
Ostatnim krokiem optymalizacji nastaw regulatora DMC jest dobranie parametru $\lambda$.

Wskaźniki jakości regulacji dla różnych wartości $\lambda$:
\begin{itemize}
	\item $\lambda=10$, $E=12,9323$
	\item $\lambda=5$, $E=11,7987$
	\item $\lambda=2$, $E=10,7840$
	\item $\lambda=0,5$, $E=9,4717$
	\item $\lambda=0,2$, $E=8,7045$
	\item $\lambda=0,1$, $E=8,2595$

\end{itemize}
\begin{figure}[!htb]
\centering
\begin{tikzpicture}
\begin{groupplot}[group style={group size=1 by 3}, width=0.9\textwidth, height=0.75\textwidth]
\nextgroupplot
[
xmin=0,xmax=300,ymin=0,ymax=1.4,
xlabel={$k$},
	ylabel={$Y(k)$},
	xtick={0,50,100,150,200,250,300},
	ytick={0,0.2,0.4,0.6,0.8,1,1.2,1.4,1.6},
	/pgf/number format/.cd,
	use comma,
	1000 sep={}
]
	\addplot[blue,semithick] file {wykresy/zad4_y_lambda100.txt};
	\addplot[brown,semithick] file {wykresy/zad4_y_lambda50.txt};
	\addplot[purple,semithick] file {wykresy/zad4_y_lambda20.txt};
	\addplot[green,semithick] file {wykresy/zad4_y_lambda5.txt};
	\addplot[red,semithick] file {wykresy/zad4_y_lambda2.txt};
	\addplot[red,semithick] file {wykresy/zad4_y_lambda1.txt};

\nextgroupplot
[
xmin=0,xmax=300,ymin=0,ymax=3,
xlabel={$k$},
ylabel={$U(k)$},
xtick={0, 50, 100, 150, 200, 250, 300},
ytick={0, 0.5, 1, 1.5, 2, 2.5,3},
/pgf/number format/.cd,
use comma,
1000 sep={}
]
	\addplot[blue,semithick] file {wykresy/zad4_u_lambda100.txt};
	\addplot[brown,semithick] file {wykresy/zad4_u_lambda50.txt};
	\addplot[purple,semithick] file {wykresy/zad4_u_lambda20.txt};
	\addplot[green,semithick] file {wykresy/zad4_u_lambda5.txt};
	\addplot[red,semithick] file {wykresy/zad4_u_lambda2.txt};
	\addplot[red,semithick] file {wykresy/zad4_u_lambda1.txt};

	\legend{$\lambda=\num{10}$,$\lambda=\num{5}$,$\lambda=\num{2}$,$\lambda=\num{0,5}$,$\lambda=\num{0,2}$,$\lambda=\num{0,1}$}
\end{groupplot}
\end{tikzpicture}
\caption{Działanie regulatora DMC dla różnych horyzontów dynamiki}
\label{fig:z4_nu}
\end{figure}