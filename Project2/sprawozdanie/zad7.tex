\chapter{Odporność algorytmu na błędy pomiaru zakłócenia}
W poprzednich zadaniach zakładano, że pomiar zakłócenia jest precyzyjny. W rzeczywistych zastosowaniach często występuje jednak szum pomiarowy. W celu symulacji błędu pomiaru do sygnału zakłócenia dodawany jest losowy szum (generowany przy pomocy funkcji rand()). Przebiegi wyjścia obiektu i sterowania dla różnych zakresów szumu przedstawia wykres \ref{fig:z7}. Zakresy szumu oznaczone są symbolem $z_s$.

Wskaźniki jakości dla różnych zakresów szumu:
\begin{itemize}
	\item $z_s=0$, $E=10,1652$
	\item $z_s=0,1$, $E=10,1696$
	\item $z_s=0,3$, $E=10,2004$
	\item $z_s=0,7$, $E=10,5454$
	\item $z_s=1$, $E=10,6552$
\end{itemize}
Dla niskich zakresów szumu pogorszenie jakości regulacji jest pomijalne. Dla wyższych wartości $z_s$ występują jednak coraz większe błędy regulacji, czego należało się spodziewać, gdyż dla skoku zakłócenia $Z=1$ szumy osiągają nawet $50\%$ wartości sygnału zakłócenia.

\begin{figure}[!htb]
\centering
\begin{tikzpicture}
\begin{groupplot}[group style={group size=1 by 3}, width=0.9\textwidth, height=0.5\textwidth]
\nextgroupplot
[
xmin=0,xmax=300,ymin=0,ymax=1.6,
xlabel={$k$},
	ylabel={$Y(k)$},
	xtick={0,50,100,150,200,250,300},
	ytick={0,0.2,0.4,0.6,0.8,1,1.2,1.4,1.6},
	/pgf/number format/.cd,
	use comma,
	1000 sep={}
]
	\addplot[blue,semithick] file {wykresy/zad7_y_1.txt};
	\addplot[green,semithick] file {wykresy/zad7_y_3.txt};
	\addplot[brown,semithick] file {wykresy/zad7_y_7.txt};
	\addplot[red,semithick] file {wykresy/zad7_y_10.txt};
	\addplot[magenta,dashed] file {wykresy/zad7_yzad.txt};
\legend{$z_s=\num{0,1}$,$z_s=\num{0,3}$,$z_s=\num{0,7}$,$z_s=1$}

\nextgroupplot
[
xmin=0,xmax=300,ymin=-0.5,ymax=2.5,
xlabel={$k$},
ylabel={$U(k)$},
xtick={0, 50, 100, 150, 200, 250, 300},
ytick={-0.5, 0, 0.5, 1, 1.5, 2, 2.5},
/pgf/number format/.cd,
use comma,
1000 sep={}
]
	\addplot[blue,semithick] file {wykresy/zad7_u_1.txt};
	\addplot[green,semithick] file {wykresy/zad7_u_3.txt};
	\addplot[brown,semithick] file {wykresy/zad7_u_7.txt};
	\addplot[red,semithick] file {wykresy/zad7_u_10.txt};
\legend{$z_s=\num{0,1}$,$z_s=\num{0,3}$,$z_s=\num{0,7}$,$z_s=1$}

\nextgroupplot
[
xmin=0,xmax=200,ymin=-1,ymax=1.5,
xlabel={$k$},
ylabel={$Z(k)$},
xtick={0, 50, 100, 150, 200},
ytick={-1,-0.5,0, 0.5, 1, 1.5},
/pgf/number format/.cd,
use comma,
1000 sep={}
]
\addplot[blue,semithick] file {wykresy/zad7_z_1.txt};
\addplot[green,semithick] file {wykresy/zad7_z_3.txt};
\addplot[brown,semithick] file {wykresy/zad7_z_7.txt};
\addplot[red,semithick] file {wykresy/zad7_z_10.txt};
\legend{$z_s=\num{0,1}$,$z_s=\num{0,3}$,$z_s=\num{0,7}$,$z_s=1$}

\end{groupplot}
\end{tikzpicture}
\caption{Działanie regulatora DMC przy różnych błędach pomiaru zakłóceń}
\label{fig:z7}
\end{figure}