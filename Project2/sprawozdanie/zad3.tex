\chapter{Odpowiedzi skokowe dla algorytmu DMC}
Do poprawnego działania algorytmu DMC wymagana jest znajomość odpowiedzi obiektu na skok jednostkowy. Często jednak dla niskich wartości sterowania odpowiedź jest zbyt zaszumiona, by można ją uznać za poprawną. W takich wypadkach stosuje się wyższą wartość sygnału sterowania, a otrzymaną odpowiedź normalizuje. Aby znormalizować odpowiedź skokową, należy od jej wartości odjąć punkt pracy obiektu, a otrzymany wynik podzielić przez wartość skoku sterowania (dla $s_u$) lub zakłócenia (dla $s_z$). Zależność tę określają wzory \ref{eq:su} i \ref{eq:sz}. Przebiegi znormalizowanych odpowiedzi skokowych przedstawia wykres \ref{fig:s}.

\begin{equation} \label{eq:su}
s_u=\frac{Y-Y_{pp}}{dU}
\end{equation}

\begin{equation} \label{eq:sz}
s_z=\frac{Y-Y_{pp}}{dZ}
\end{equation}

\begin{figure}[tb]
\centering
\begin{tikzpicture}
\begin{groupplot}[group style={group size=1 by 3}, width=0.9\textwidth, height=0.6\textwidth]
\nextgroupplot
[
xmin=0,xmax=200,ymin=0,ymax=1.4,
xlabel={$k$},
ylabel={$s_u$},
xtick={0, 50, 100, 150, 200},
ytick={0, 0.2, 0.4, 0.6, 0.8, 1, 1.2, 1.4},
/pgf/number format/.cd,
use comma,
1000 sep={}
]
\addplot[blue,semithick] file {wykresy/zad3_su.txt};

\nextgroupplot
[
xmin=0,xmax=200,ymin=0,ymax=1,
xlabel={$k$},
ylabel={$s_z$},
xtick={0, 50, 100, 150, 200},
ytick={0, 0.2, 0.4, 0.6, 0.8, 1},
/pgf/number format/.cd,
use comma,
1000 sep={}
]
\addplot[red,semithick] file {wykresy/zad3_sz.txt};
\end{groupplot}
\end{tikzpicture}
\caption{Znormalizowane odpowiedzi skokowe na skoki sygnałów sterowania i zakłócenia}
\label{fig:s}
\end{figure}