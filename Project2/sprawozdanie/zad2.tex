\chapter{Odpowiedzi skokowe}
\section{Tor sterowania}
W tym punkcie badane są odpowiedzi skokowe obiektu na różne wartości skoku sygnału sterowania. Założono, że w chwili początkowej obiekt znajduje się w punkcie pracy. W chwili $k=9$ wykonywany jest sterowania do zadanej wartości. Sygnał zakłócenia ma wartość $Z=0$. Wyniki badań przedstawia wykres \ref{fig:z2_yu}.
\begin{figure}[!htb]
	\centering
	\begin{tikzpicture}
	\begin{axis}[
	width=0.9\textwidth,
	height=0.9\textwidth,
	xmin=0,xmax=200,ymin=-3,ymax=3,
	xlabel={$k$},
	ylabel={$Y(k)$},
	xtick={0,50,100,150,200},
	ytick={-3,-2,-1,0,1,2,3},
	/pgf/number format/.cd,
	use comma,
	1000 sep={}
	]
	\addplot[blue,semithick] file {wykresy/zad2_yu_1.txt};
	\addplot[brown,semithick] file {wykresy/zad2_yu_2.txt};
	\addplot[cyan,semithick] file {wykresy/zad2_yu_3.txt};
	\addplot[green,semithick] file {wykresy/zad2_yu_4.txt};
	\addplot[lime,semithick] file {wykresy/zad2_yu_5.txt};
	\addplot[magenta,semithick] file {wykresy/zad2_yu_6.txt};
	\addplot[orange,semithick] file {wykresy/zad2_yu_7.txt};
	\addplot[pink,semithick] file {wykresy/zad2_yu_8.txt};
	\legend{$U=-\num{2}$,$U=-\num{1,5}$,$U=-\num{1}$,$U=-\num{0,5}$,$U=\num{0,5}$,$U=\num{1}$,$U=\num{1,5}$,$U=\num{2}$}
	\end{axis}
	\end{tikzpicture}
	\caption{Odpowiedź $Y(k)$ dla skoków sterowania $U$}
\label{fig:z2_yu}
\end{figure}

\subsection{Charakterystyka statyczna toru sterowania}
Charakterystyka statyczna toru sterowania wyznaczona została poprzez sprawdzenie, na jakich wartościach stabilizuje się wyjście obiektu dla różnych wartości sygnału $U$. Liniowości charakterystki statycznej dowodzi wykres \ref{fig:z2_y_stat_u}, który (w przybliżeniu) jest liniowy.

Wartość wzmocnienia statycznego można wyznaczyć normalizując odpowiedź skokową. Wynosi ona $s_u=1,1022$.

\begin{figure}[!htb]
	\centering
	\begin{tikzpicture}
	\begin{axis}[
	width=0.9\textwidth,
	height=0.9\textwidth,
	xmin=-1.5,xmax=1.5,ymin=-2,ymax=2,
	xlabel={$U$},
	ylabel={$Y(U)$},
	xtick={-1.5,-1,-0.5,0,0.5,1,1.5},
	ytick={-2,-1.5,-1,-0.5,0,0.5,1,1.5,2},
	/pgf/number format/.cd,
	use comma,
	1000 sep={}
	]
	\addplot[blue,semithick] file {wykresy/zad2_y_stat_u.txt};
	\end{axis}
	\end{tikzpicture}
	\caption{Charakterystyka statyczna $Y(U)$}
\label{fig:z2_y_stat_u}
\end{figure}

\newpage
\subsection{Charakterystyka dynamiczna sygnału sterowania}
Charakterystyka dynamiczna sygnału sterowania określa, w którym kroku symulacji od momentu skoku wartości sygnału wyjście obiektu osiągnęło co najmniej $90\%$ wartości zadanej. Symulacja została przeprowadzona dla różnych skoków sygnału $U$. Jej wynik przedstawia wykres \ref{fig:z2_dyn_u}.
\begin{figure}[!htb]
	\centering
	\begin{tikzpicture}
	\begin{axis}[
	width=0.9\textwidth,
	height=0.9\textwidth,
	xmin=0.01,xmax=0.5,ymin=30,ymax=50,
	xlabel={$dU$},
	ylabel={$k$},
	xtick={0,0.1,0.2,0.3,0.4,0.5},
	ytick={30,40,50},
	/pgf/number format/.cd,
	use comma,
	1000 sep={}
	]
	\addplot[blue,semithick] file {wykresy/zad2_u_dyn.txt};
	\end{axis}
	\end{tikzpicture}
	\caption{Charakterystyka dynamiczna $k(dU)$}
\label{fig:z2_dyn_u}
\end{figure}

\newpage
\section{Tor zakłócenia}
W tym punkcie badane są odpowiedzi skokowe obiektu na różne wartości skoku sygnału zakłócenia. Założono, że w chwili początkowej obiekt znajduje się w punkcie pracy. W chwili $k=9$ wykonywany jest sterowania do zadanej wartości. Sygnał sterowania ma wartość $U=0$. Wyniki badań przedstawia wykres \ref{fig:z2_yz}.
\begin{figure}[!htb]
	\centering
	\begin{tikzpicture}
	\begin{axis}[
	width=0.9\textwidth,
	height=0.9\textwidth,
	xmin=0,xmax=200,ymin=-1.5,ymax=1.5,
	xlabel={$k$},
	ylabel={$Y(k)$},
	xtick={0,50,100,150,200},
	ytick={-1.5,-1,-0.5,0,0.5,1,1.5},
	/pgf/number format/.cd,
	use comma,
	1000 sep={}
	]
	\addplot[blue,semithick] file {wykresy/zad2_yz_1.txt};
	\addplot[brown,semithick] file {wykresy/zad2_yz_2.txt};
	\addplot[cyan,semithick] file {wykresy/zad2_yz_3.txt};
	\addplot[green,semithick] file {wykresy/zad2_yz_4.txt};
	\addplot[lime,semithick] file {wykresy/zad2_yz_5.txt};
	\addplot[magenta,semithick] file {wykresy/zad2_yz_6.txt};
	\addplot[orange,semithick] file {wykresy/zad2_yz_7.txt};
	\addplot[pink,semithick] file {wykresy/zad2_yz_8.txt};
	\legend{$Z=-\num{2}$,$Z=-\num{1,5}$,$Z=-\num{1}$,$Z=-\num{0,5}$,$Z=\num{0,5}$,$Z=\num{1}$,$Z=\num{1,5}$,$Z=\num{2}$}
	\end{axis}
	\end{tikzpicture}
	\caption{Odpowiedź $Y(k)$ dla skoków zakłócenia $Z$}
\label{fig:z2_yz}
\end{figure}

\subsection{Charakterystyka statyczna toru zakłócenia}
Charakterystyka statyczna toru zakłócenia wyznaczona została poprzez sprawdzenie, na jakich wartościach stabilizuje się wyjście obiektu dla różnych wartości sygnału $Z$. Liniowości charakterystki statycznej dowodzi wykres \ref{fig:z2_y_stat_z}, który (w przybliżeniu) jest liniowy.

Wartość wzmocnienia statycznego można wyznaczyć normalizując odpowiedź skokową. Wynosi ona $s_z=0,501$.
\begin{figure}[!htb]
	\centering
	\begin{tikzpicture}
	\begin{axis}[
	width=0.9\textwidth,
	height=0.9\textwidth,
	xmin=-1.5,xmax=1.5,ymin=-1.5,ymax=1.5,
	xlabel={$U$},
	ylabel={$Y(Z)$},
	xtick={-1.5,-1,-0.5,0,0.5,1,1.5},
	ytick={-1.5,-1,-0.5,0,0.5,1,1.5},
	/pgf/number format/.cd,
	use comma,
	1000 sep={}
	]
	\addplot[blue,semithick] file {wykresy/zad2_y_stat_z.txt};
	\end{axis}
	\end{tikzpicture}
	\caption{Charakterystyka statyczna $Y(Z)$}
\label{fig:z2_y_stat_z}
\end{figure}

\newpage
\subsection{Charakterystyka dynamiczna sygnału zakłócenia}
Charakterystyka dynamiczna sygnału zakłócenia określa, w którym kroku symulacji od momentu skoku wartości sygnału wyjście obiektu osiągnęło co najmniej $90\%$ wartości zadanej. Symulacja została przeprowadzona dla różnych skoków sygnału $Z$. Jej wynik przedstawia wykres \ref{fig:z2_dyn_z}.
\begin{figure}[!htb]
	\centering
	\begin{tikzpicture}
	\begin{axis}[
	width=0.9\textwidth,
	height=0.9\textwidth,
	xmin=0.01,xmax=0.5,ymin=0,ymax=20,
	xlabel={$dZ$},
	ylabel={$k$},
	xtick={0,0.1,0.2,0.3,0.4,0.5},
	ytick={0,10,20},
	/pgf/number format/.cd,
	use comma,
	1000 sep={}
	]
	\addplot[blue,semithick] file {wykresy/zad2_z_dyn.txt};
	\end{axis}
	\end{tikzpicture}
	\caption{Charakterystyka dynamiczna $k(dZ)$}
\label{fig:z2_dyn_z}
\end{figure}

\section{Charakterystyka statyczna $Y(U,Z)$}
Charakterystyka statyczna $Y(U,Z)$ ma kształt płaszczyzny, co dowodzi jej liniowości. Wyznaczona została na podstawie symulacji obiektu dla różnych skoków sygnałów sterowania i zakłóceń. Założono, że dla każdego ze skoków w chwili $k=200$ obiekt jest w stanie ustalonym i pobierano wartości wyjścia właśnie w chwili $k=200$. Przebieg charakterystyki przedstawiony jest na wykresie \ref{fig:yuz}.
\begin{figure}[tb]
	\centering
	\begin{tikzpicture}
	\begin{axis}[
	width=0.9\textwidth,
	xlabel={$U$},
	ylabel={$Z$},
	zlabel={$Y$},
	xtick={-1,-0.5,0,0.5,1},
	ytick={-1,-0.5,0,0.5,1},
	ztick={-5,0,5},
	]
	\addplot3[mesh, mesh/cols=21] file {wykresy/zad2_y_uz.txt};
	\end{axis}
	\end{tikzpicture}
	\caption{Charakterystyka statyczna $Y(U,Z)$}
	\label{fig:yuz}
\end{figure}
