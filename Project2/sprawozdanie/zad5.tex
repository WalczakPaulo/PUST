\externaldocument{zad3}
\externaldocument{zad4}
\chapter{Uwzględnianie zakłóceń}
Aby uwzględnić sygnał zakłócenia w algorytmie DMC, konieczne jest obliczenie macierzy $M^{zP}$.

\begin{equation}
\boldsymbol{M^{zP}}=\left[
\begin{array}
{ccccc}
sz_{1} & sz_{2}-sz_{1} & sz_{3}-sz_{2} & \ldots & sz_{D_z}-sz_{D_z-1}\\
sz_{2} & sz_{3}-sz_{1} & sz_{4}-sz_{2} & \ldots & sz_{D_z+1}-sz_{D_z-1}\\
\vdots & \vdots & \vdots & \ddots & \vdots\\
sz_{N} & sz_{N+1}-sz_{1} & sz_{N+2}-sz_{2} & \ldots &  sz_{N+D_z-1}-sz_{D_z-1}
\end{array}
\right]_{\mathrm{NxDz}}
\label{Mzp}
\end{equation}

Posłuży ona do wyznaczenia wektora $k_z$ danego wzorem:
\begin{equation}
k_z=kM^{zP}
\label{kz}
\end{equation}

Długość horyzontu dynamiki zakłócenia została wyznaczona na podstawie obserwacji odpowiedzi skokowej na sygnał zakłócenia wyznaczonej w zadaniu 3. Na wykresie \ref{fig:s} można zaobserwować, że jej wartość ustala się w okolicach chwili $k=40$. Dlatego też przyjęto, że wykorzystywaną dalej wartością horyzontu dynamiki jest $D^z=40$.

Nastawy regulatora używanego w tej części projektu mają wartości:
\begin{itemize}
	\item $D=100$
	\item $D^z=40$
	\item $N=20$
	\item $N_u=5$
	\item $\lambda=0,5$
\end{itemize}


Niebieskie linie na wykresie \ref{fig:z5_zakl} przedstawiają działanie regulatora DMC z uwzględnianiem zakłóceń, czerwone natomiast bez uwzględniania zakłóceń. Wartość zadana wynosi $Y_{zad}=1$. Skok sygnału wartości zakłócenia z $Z=0$ do $Z=1$ nastąpił w chwili $k=100$.

Jak widać, w obu przypadkach zakłócenie spowodowało chwilową zmianę wartości wyjścia obiektu. W przypadku wersji z uwzględnianiem zakłóceń zmiana ta była znacznie mniejsza i została szybciej zniwelowana. Kosztem poprawy szybkości regulacji jest jednak ostrzejsze sterowanie.

Wskaźnik jakości regulacji w przypadku wersji regulatora z uwzględnianiem zakłóceń ma wartość $E=10,1652$, a więc nieznacznie wyższą niż w przypadku regulacji bez zakłóceń. Dla regulatora bez uwzględniania zakłóceń wskaźnik ten ma wartość $E=11.2922$. Jak widać, poprawa jakości regulacji jest znaczna dla regulatora z uwzględnianiem zakłóceń.

\begin{figure}[!htb]
\centering
\begin{tikzpicture}
\begin{groupplot}[group style={group size=1 by 3}, width=0.9\textwidth, height=0.5\textwidth]
\nextgroupplot
[
xmin=0,xmax=300,ymin=0,ymax=1.6,
xlabel={$k$},
	ylabel={$Y(k)$},
	xtick={0,50,100,150,200,250,300},
	ytick={0,0.2,0.4,0.6,0.8,1,1.2,1.4,1.6},
	/pgf/number format/.cd,
	use comma,
	1000 sep={}
]
	\addplot[blue,semithick] file {wykresy/zad5_z_y.txt};
	\addplot[red,semithick] file {wykresy/zad5_bez_y.txt};

	\addplot[magenta,dashed] file {wykresy/zad5_z_yzad.txt};
\legend{$Y^z(k)$,$Y(k)$,$Y_{zad}(k)$}

\nextgroupplot
[
xmin=0,xmax=300,ymin=-0.5,ymax=2.5,
xlabel={$k$},
ylabel={$U(k)$},
xtick={0, 50, 100, 150, 200, 250, 300},
ytick={-0.5, 0, 0.5, 1, 1.5, 2, 2.5},
/pgf/number format/.cd,
use comma,
1000 sep={}
]
	\addplot[blue,semithick] file {wykresy/zad5_z_u.txt};
	\addplot[red,semithick] file {wykresy/zad5_bez_u.txt};
\legend{$U^z(k)$,$U(k)$}

\nextgroupplot
[
xmin=0,xmax=200,ymin=0,ymax=1.5,
xlabel={$k$},
ylabel={$Z(k)$},
xtick={0, 50, 100, 150, 200},
ytick={0, 0.5, 1, 1.5},
/pgf/number format/.cd,
use comma,
1000 sep={}
]
\addplot[red,semithick] file {wykresy/zad5_z_z.txt};

\end{groupplot}
\end{tikzpicture}
\caption{Działanie regulatora DMC z uwzględnianiem zakłóceń}
\label{fig:z5_zakl}
\end{figure}