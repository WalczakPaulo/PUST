\chapter{Punkt pracy}
Celem zadania było sprawdzenie poprawności punktu pracy opisanego w sekcji \ref{sec:opis}. W celu weryfikacji zbadano odpowiedź obiektu na sygnał sterowania równy $U=0$ oraz sygnał zakłócenia równy $Z=0$. Zgodnie z oczekiwaniem, wyjście obiektu miało wartość $Y=0$. Stąd podany punkt pracy $u=y=z=0$ jest poprawny. Przebieg symulacji przedstawiony jest na wykresie \ref{fig:pkt_prac}.

\begin{figure}[tb]
\centering
\begin{tikzpicture}
\begin{groupplot}[group style={group size=1 by 3}, width=0.75\textwidth, height=0.3\textwidth]
\nextgroupplot
[
xmin=0,xmax=200,ymin=-1,ymax=1,
xlabel={$k$},
ylabel={$U$},
xtick={0, 50, 100, 150, 200},
ytick={-1, -0.5, 0, 0.5, 1},
/pgf/number format/.cd,
use comma,
1000 sep={}
]
\addplot[blue,semithick] file {wykresy/zad1u.txt};

\nextgroupplot
[
xmin=0,xmax=200,ymin=-1,ymax=1,
xlabel={$k$},
ylabel={$Z$},
xtick={0, 50, 100, 150, 200},
ytick={-1, -0.5, 0, 0.5, 1},
/pgf/number format/.cd,
use comma,
1000 sep={}
]
\addplot[red,semithick] file {wykresy/zad1z.txt};

\nextgroupplot
[
xmin=0,xmax=200,ymin=-1,ymax=1,
xlabel={k},
ylabel={Y},
xtick={0, 50, 100, 150, 200},
ytick={-1, -0.5, 0, 0.5, 1},
/pgf/number format/.cd,
use comma,
1000 sep={}
]
\addplot[red,semithick] file {wykresy/zad1y.txt};
\end{groupplot}
\end{tikzpicture}
\caption{Przebiegi sygnałów $U(k)$, $Z(k)$, $Y(k)$ w punkcie pracy.}
\label{fig:pkt_prac}
\end{figure}
