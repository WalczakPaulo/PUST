\chapter{Zadanie 3}
Celem zadania było przekształcenie wybranych odpowiedzi skokowych w taki sposób, aby otrzymać odpowiedzi skokowe wykorzystywane w algorytmie DMC. Wybraliśmy odpowiedzi skokowe, o największych skokach sterowania
co pozwoliło na zniwelowanie względnego wpływu zakłóceń na wyjście obiektu. Do aproksymacji użyliśmy napisanych przez nas skryptów, wykorzystanych i przetestowanych na poprzednich projektach oraz
laboratoriach. Opierał się on na wykorzystaniu wbudowanego w Matlabie algorytmu GA (\texttt{Genetic Algorithm}). Użyto w nich członu inercyjnego drugiego rzędu z opóźnieniem.
Ciekawostką jest, że algorytm dawał różne aproksymacje dla tej samej odpowiedzi skokowej. Wynika to z niedeterministycznego charakteru algorytmów ewolucyjnych.
Niestety zebrane odpowiedzi skokowe okazały się być zbyt krótkie, więc pozwoliliśmy sobie wydłużyć zaproksymowane
funkcje
Uważamy, że przedstawione poniżej aproksymacje, dobrze przybliżają kształt odpowiedzi skokowych. Spodziewamy się
zatem mniejszych problemów z regulacją niż na poprzednim laboratorium.

\begin{figure}[tb]
\centering
\begin{tikzpicture}
\begin{groupplot}[group style={group size=1 by 2}, width=0.9\textwidth, height=0.4\textwidth]
\nextgroupplot
[
xmin=0,xmax=500,ymin=0,ymax=0.3,
xlabel={$k$},
ylabel={$s_k^{11}$},
xtick={0, 100, 200, 300, 400, 500},
ytick={0, 0.1, 0.2, 0.3},
legend pos=north west,
/pgf/number format/.cd,
use comma,
1000 sep={}
]
\addplot[blue,semithick] file {wykresy/s11.txt};
\addplot[red,semithick] file {wykresy/approx_s11.txt};
\legend{Faktyczny skok, Aproksymacja}

\nextgroupplot
[
xmin=0,xmax=500,ymin=0,ymax=0.1,
xlabel={$k$},
ylabel={$s_k^{12}$},
xtick={0, 100, 200, 300, 400, 500},
ytick={0, 0.02, 0.04, 0.06, 0.08, 0.1},
legend pos=north west,
/pgf/number format/.cd,
use comma,
1000 sep={}
]
\addplot[blue,semithick] file {wykresy/s12.txt};
\addplot[red,semithick] file {wykresy/approx_s12.txt};
\legend{Faktyczny skok, Aproksymacja}

\end{groupplot}
\end{tikzpicture}
\caption{Znormalizowane odpowiedzi na skok sterowania $u_1$.}
\label{fig:dmc1}
\end{figure}

\begin{figure}[tb]
\centering
\begin{tikzpicture}
\begin{groupplot}[group style={group size=1 by 2}, width=0.9\textwidth, height=0.4\textwidth]
\nextgroupplot
[
xmin=0,xmax=500,ymin=0,ymax=0.1,
xlabel={$k$},
ylabel={$s_k^{21}$},
xtick={0, 100, 200, 300, 400, 500},
ytick={0, 0.02, 0.04, 0.06, 0.08, 0.1},
legend pos=north west,
/pgf/number format/.cd,
use comma,
1000 sep={}
]
\addplot[blue,semithick] file {wykresy/s21.txt};
\addplot[red,semithick] file {wykresy/approx_s21.txt};
\legend{Faktyczny skok, Aproksymacja}

\nextgroupplot
[
xmin=0,xmax=500,ymin=0,ymax=0.4,
xlabel={$k$},
ylabel={$s_k^{22}$},
xtick={0, 100, 200, 300, 400, 500},
ytick={0, 0.1, 0.2, 0.3, 0.4},
legend pos=north west,
/pgf/number format/.cd,
use comma,
1000 sep={}
]
\addplot[blue,semithick] file {wykresy/s22.txt};
\addplot[red,semithick] file {wykresy/approx_s22.txt};
\legend{Faktyczny skok, Aproksymacja}

\end{groupplot}
\end{tikzpicture}
\caption{Znormalizowane odpowiedzi na skok sterowania $u_2$.}
\label{fig:dmc2}
\end{figure}
