\documentclass[a4paper,titlepage,11pt,twosides,floatssmall]{mwrep}
\usepackage[left=2.5cm,right=2.5cm,top=2.5cm,bottom=2.5cm]{geometry}
\usepackage[OT1]{fontenc}
\usepackage{polski}
\usepackage{amsmath}
\usepackage{xr}
\usepackage{amsfonts}
\usepackage{amssymb}
\usepackage{graphicx}
\usepackage{url}
\usepackage[section]{placeins}
\usepackage{tikz}
\usetikzlibrary{arrows,calc,decorations.markings,math,arrows.meta}
\usepackage{rotating}
\usepackage[percent]{overpic}
\usepackage[utf8]{inputenc}
\usepackage{xcolor}
\usepackage{pgfplots}
\usetikzlibrary{pgfplots.groupplots}
\usepackage{listings}
\usepackage{matlab-prettifier}
\usepackage{siunitx}
\definecolor{szary}{rgb}{0.95,0.95,0.95}
\sisetup{detect-weight,exponent-product=\cdot,output-decimal-marker={,},per-mode=symbol,binary-units=true,range-phrase={-},range-units=single}

%konfiguracje pakietu listings
\lstset{
	backgroundcolor=\color{szary},
	frame=single,
	breaklines=true,
}
\lstdefinestyle{customlatex}{
	basicstyle=\footnotesize\ttfamily,
	%basicstyle=\small\ttfamily,
}
\lstdefinestyle{customc}{
	breaklines=true,
	frame=tb,
	language=C,
	xleftmargin=0pt,
	showstringspaces=false,
	basicstyle=\small\ttfamily,
	keywordstyle=\bfseries\color{green!40!black},
	commentstyle=\itshape\color{purple!40!black},
	identifierstyle=\color{blue},
	stringstyle=\color{orange},
}
\lstdefinestyle{custommatlab}{
	captionpos=t,
	breaklines=true,
	frame=tb,
	xleftmargin=0pt,
	language=matlab,
	showstringspaces=false,
	%basicstyle=\footnotesize\ttfamily,
	basicstyle=\scriptsize\ttfamily,
	keywordstyle=\bfseries\color{green!40!black},
	commentstyle=\itshape\color{purple!40!black},
	identifierstyle=\color{blue},
	stringstyle=\color{orange},
}

%wymiar tekstu (bez �ywej paginy)
\textwidth 160mm \textheight 247mm

%ustawienia pakietu pgfplots
\pgfplotsset{
tick label style={font=\scriptsize},
label style={font=\small},
legend style={font=\small},
title style={font=\small}
}

\def\figurename{Rys.}
\def\tablename{Tab.}

%konfiguracja liczby p�ywaj�cych element�w
\setcounter{topnumber}{0}%2
\setcounter{bottomnumber}{3}%1
\setcounter{totalnumber}{5}%3
\renewcommand{\textfraction}{0.01}%0.2
\renewcommand{\topfraction}{0.95}%0.7
\renewcommand{\bottomfraction}{0.95}%0.3
\renewcommand{\floatpagefraction}{0.35}%0.5

\begin{document}
\frenchspacing
\pagestyle{uheadings}

%strona tytu�owa
\title{\bf Sprawozdanie z laboratorium 3\vskip 0.1cm}
\author{Kamil Gabryjelski, Paweł Rybak, Paweł Walczak}
\date{2017}

\makeatletter
\renewcommand{\maketitle}{\begin{titlepage}
\begin{center}{\LARGE {\bf
Wydział Elektroniki i Technik Informacyjnych}}\\
\vspace{0.4cm}
{\LARGE {\bf Politechnika Warszawska}}\\
\vspace{0.3cm}
\end{center}
\vspace{5cm}
\begin{center}
{\bf \LARGE Projektowanie układów sterowania\\ (projekt grupowy) \vskip 0.1cm}
\end{center}
\vspace{1cm}
\begin{center}
{\bf \LARGE \@title}
\end{center}
\vspace{2cm}
\begin{center}
{\bf \Large \@author \par}
\end{center}
\vspace*{\stretch{6}}
\begin{center}
\bf{\large{Warszawa, \@date\vskip 0.1cm}}
\end{center}
\end{titlepage}
}
\makeatother

\maketitle

\tableofcontents
\chapter{Wstęp}
Celem laboratorium była implementacja, weyfikacja poprawności działania i dobór
parametrów algorytmów regulacji jednowymiarowego procesu laboratoryjnego
z pomiarem zakłócenia. Obiekt na którym pracowaliśmy, składał się z grzałki G1,
wentylatora W1 oraz czujnika temperatury T1. Sygnałem sterującym jest moc grzania ( w zakresie 0 - 100)
grzałki G1, sygnałem wyjściowym jest pomiar wskazywany przez czujnik temperatury T1.
Moc wentylatora W1 musi wynosić $50\%$ ( w celu większej responsywności stanowiska ).
Dodatkowo, jako sygnał zakłócający $Z$ zostanie wykorzystana także grzałka G1. 

\chapter{Opis obiektu}
Badany obiekt jest obiektem o dwóch sygnałach wejściowych ($u_1$, $u_2$), oraz
dwóch sygnałach wyjściowych ($y_1$, $y_2$). Obiekt jest obiektem dyskretnym,
a jego okres próbkowania wynosi $0,5$s. Punktem pracy naszego obiektu, będzie
zerowa wartość obydwu wejść. W takim przypadku obiekt stabilizuje się przy
zerowej wartości wyjść.

\chapter{Odpowiedzi skokowe}
\section{Tor sterowania}
W tym punkcie badane są odpowiedzi skokowe obiektu na różne wartości skoku sygnału sterowania. Założono, że w chwili początkowej obiekt znajduje się w punkcie pracy. W chwili $k=9$ wykonywany jest sterowania do zadanej wartości. Sygnał zakłócenia ma wartość $Z=0$. Wyniki badań przedstawia wykres \ref{fig:z2_yu}.
\begin{figure}[!htb]
	\centering
	\begin{tikzpicture}
	\begin{axis}[
	width=0.9\textwidth,
	width=0.9\textwidth,
	xmin=0,xmax=200,ymin=-3,ymax=3,
	xlabel={$k$},
	ylabel={$Y(k)$},
	xtick={0,50,100,150,200},
	ytick={-3,-2,-1,0,1,2,3},
	/pgf/number format/.cd,
	use comma,
	1000 sep={}
	]
	\addplot[blue,semithick] file {wykresy/zad2_yu_1.txt};
	\addplot[brown,semithick] file {wykresy/zad2_yu_2.txt};
	\addplot[cyan,semithick] file {wykresy/zad2_yu_3.txt};
	\addplot[green,semithick] file {wykresy/zad2_yu_4.txt};
	\addplot[lime,semithick] file {wykresy/zad2_yu_5.txt};
	\addplot[magenta,semithick] file {wykresy/zad2_yu_6.txt};
	\addplot[orange,semithick] file {wykresy/zad2_yu_7.txt};
	\addplot[pink,semithick] file {wykresy/zad2_yu_8.txt};
	\legend{$U=-\num{2}$,$U=-\num{1,5}$,$U=-\num{1}$,$U=-\num{0,5}$,$U=\num{0,5}$,$U=\num{1}$,$U=\num{1,5}$,$U=\num{2}$}
	\end{axis}
	\end{tikzpicture}
	\caption{Odpowiedź $Y(k)$ dla skoków sterowania $U$}
\label{fig:z2_yu}
\end{figure}

\section{Charakterystyka statyczna toru sterowania}
Charakterystyka statyczna toru sterowania wyznaczona została poprzez sprawdzenie, na jakich wartościach stabilizuje się wyjście obiektu dla różnych wartości sygnału $U$. Liniowości charakterystki statycznej dowodzi wykres \ref{fig:z2_y_stat_u}, który (w przybliżeniu) jest liniowy.

Wartość wzmocnienia statycznego można wyznaczyć normalizując odpowiedź skokową. Wynosi ona $s_u=1,1022$.

\begin{figure}[!htb]
	\centering
	\begin{tikzpicture}
	\begin{axis}[
	width=0.9\textwidth,
	width=0.9\textwidth,
	xmin=-1.5,xmax=1.5,ymin=-2,ymax=2,
	xlabel={$U$},
	ylabel={$Y(U)$},
	xtick={-1.5,-1,-0.5,0,0.5,1,1.5},
	ytick={-2,-1.5,-1,-0.5,0,0.5,1,1.5,2},
	/pgf/number format/.cd,
	use comma,
	1000 sep={}
	]
	\addplot[blue,semithick] file {wykresy/zad2_y_stat_u.txt};
	\end{axis}
	\end{tikzpicture}
	\caption{Charakterystyka statyczna $Y(U)$}
\label{fig:z2_y_stat_u}
\end{figure}

\section{Tor zakłócenia}
W tym punkcie badane są odpowiedzi skokowe obiektu na różne wartości skoku sygnału zakłócenia. Założono, że w chwili początkowej obiekt znajduje się w punkcie pracy. W chwili $k=9$ wykonywany jest sterowania do zadanej wartości. Sygnał sterowania ma wartość $U=0$. Wyniki badań przedstawia wykres \ref{fig:z2_yz}.
\begin{figure}[!htb]
	\centering
	\begin{tikzpicture}
	\begin{axis}[
	width=0.9\textwidth,
	width=0.9\textwidth,
	xmin=0,xmax=200,ymin=-1.5,ymax=1.5,
	xlabel={$k$},
	ylabel={$Y(k)$},
	xtick={0,50,100,150,200},
	ytick={-1.5,-1,-0.5,0,0.5,1,1.5},
	/pgf/number format/.cd,
	use comma,
	1000 sep={}
	]
	\addplot[blue,semithick] file {wykresy/zad2_yz_1.txt};
	\addplot[brown,semithick] file {wykresy/zad2_yz_2.txt};
	\addplot[cyan,semithick] file {wykresy/zad2_yz_3.txt};
	\addplot[green,semithick] file {wykresy/zad2_yz_4.txt};
	\addplot[lime,semithick] file {wykresy/zad2_yz_5.txt};
	\addplot[magenta,semithick] file {wykresy/zad2_yz_6.txt};
	\addplot[orange,semithick] file {wykresy/zad2_yz_7.txt};
	\addplot[pink,semithick] file {wykresy/zad2_yz_8.txt};
	\legend{$Z=-\num{2}$,$Z=-\num{1,5}$,$Z=-\num{1}$,$Z=-\num{0,5}$,$Z=\num{0,5}$,$Z=\num{1}$,$Z=\num{1,5}$,$Z=\num{2}$}
	\end{axis}
	\end{tikzpicture}
	\caption{Odpowiedź $Y(k)$ dla skoków zakłócenia $Z$}
\label{fig:z2_yz}
\end{figure}

\section{Charakterystyka statyczna toru zakłócenia}
Charakterystyka statyczna toru zakłócenia wyznaczona została poprzez sprawdzenie, na jakich wartościach stabilizuje się wyjście obiektu dla różnych wartości sygnału $Z$. Liniowości charakterystki statycznej dowodzi wykres \ref{fig:z2_y_stat_z}, który (w przybliżeniu) jest liniowy.

Wartość wzmocnienia statycznego można wyznaczyć normalizując odpowiedź skokową. Wynosi ona $s_z=0,501$.

\begin{figure}[!htb]
	\centering
	\begin{tikzpicture}
	\begin{axis}[
	width=0.9\textwidth,
	width=0.9\textwidth,
	xmin=-1.5,xmax=1.5,ymin=-1.5,ymax=1.5,
	xlabel={$U$},
	ylabel={$Y(Z)$},
	xtick={-1.5,-1,-0.5,0,0.5,1,1.5},
	ytick={-1.5,-1,-0.5,0,0.5,1,1.5},
	/pgf/number format/.cd,
	use comma,
	1000 sep={}
	]
	\addplot[blue,semithick] file {wykresy/zad2_y_stat_z.txt};
	\end{axis}
	\end{tikzpicture}
	\caption{Charakterystyka statyczna $Y(Z)$}
\label{fig:z2_y_stat_z}
\end{figure}
\chapter{Odpowiedź skokowa do DMC}
Celem zadania trzeciego było przygotowanie odpowiedzi skokowych wykorzystywanych
w algorytmie DMC. W naszym przypadku konieczne było zebranie odpowiedzi na skok
sygnału sterującego oraz zakłócającego. Skok wykonujemy z obranego wcześniej punktu
pracy ($Y_{pp} = 32,20$, $U_{pp} = 27$, $Z = 0$). Z powodu nieuniknionych zakłóceń
w pomieszczeniu, szczególnie trudne było ustabilizowanie obiektu w punkcie pracy,
po wykonaniu wcześniejszych eksperymentów. Z tego powodu otrzymane przez nas odpowiedzi
były prawdopodbnie niedokładne. W naszych głowach zrodziła się z początku myśl,
aby skorzystać z odpowiedzi skokowej ( na skok sterowania ) z poprzedniego laboratoium
(pracowaliśmy również na tym stanowisku ), jednakże jak się później okazało
( problemy z algorytmem DMC) zdecydowaliśmy się na ponowne pozyskanie owej odpowiedzi skokowej.
Wykonując skok sterowania, wartość zakłócenia ustawiliśmy na 0 i odpowiednio
wykonując skok zakłócenia, wartość sterowania pozostawiliśmy na taką jak w punkcie
pracy ( $U_{pp} = 27$). Z powodu zakłóceń i problemów ze stabilizacją obiektu uzyskane
przez nas odpowiedzi skokowe nie były dla nas satysfakcjonujące, lecz z powodu braku
czasu nie zdecydowaliśmy się na powtarzanie eksperymentu. Normalizacji dokonaliśmy
poprzez przesunięcie wszystkich wartości wyjścia obiektu o wartość wyjścia w punkcie
pracy oraz podzielenie przez wartość odpowiednio skoku sterowania lub zakłócenia.
Aproksymacja ma postać członu inercyjnego drugiego rzędu z opóźnieniem.
Aproksymacji dokonaliśmy przy użyciu opracowanego przez nas wcześniej skryptu.
Użyliśmy do tego między innymi wbudowanej funkcji GA ( Genetic Algorithm ) w celu
pozyskania aproksymacji minimalizującej funkcję błędu kwadratowego.

\begin{figure}[tb]
\centering
\begin{tikzpicture}
\begin{axis}[
width=0.75\textwidth,
xmin=0,xmax=400,ymin=0,ymax=0.35,
xlabel={Czas (s)},
ylabel={Wartość wyjścia},
xtick={0, 100, 200, 300, 400},
ytick={0, 0.05, 0.1, 0.15, 0.2, 0.25, 0.3, 0.35},
legend pos=north west,
/pgf/number format/.cd,
use comma,
1000 sep={}
]
\addplot[blue,semithick] file {Skrypty/s_approx.txt};
\end{axis}
\end{tikzpicture}
\caption{Przybliżona odpowiedź obiektu na skok sterowania.}
\label{fig:skok_zak}
\end{figure}


\begin{figure}[tb]
\centering
\begin{tikzpicture}
\begin{axis}[
width=0.75\textwidth,
xmin=0,xmax=400,ymin=0,ymax=0.08,
xlabel={Czas (s)},
ylabel={Wartość wyjścia},
xtick={0, 100, 200, 300, 400},
ytick={0, 0.01, 0.02, 0.03, 0.04, 0.05, 0.06, 0.07, 0.08},
legend pos=north west,
/pgf/number format/.cd,
use comma,
1000 sep={}
]
\addplot[blue,semithick] file {Skrypty/odp_zakl.txt};
\end{axis}
\end{tikzpicture}
\caption{Przybliżona odpowiedź obiektu na skok zakłócenia.}
\label{fig:skok_zak}
\end{figure}


\externaldocument{zad3}
\chapter{Algorytm DMC}
\section{Wersja analityczna}
Do realizacji analitycznej wersji algorytmu DMC wykorzystano następujące wzory:

\begin{equation}
\boldsymbol{Y}(k)=\left[
\begin{array}{c}
y(k)\\
\vdots\\
y(k)
\end{array}
\right]_{\mathrm{Nx1}}
\label{ym}
\end{equation}

\begin{equation}
\boldsymbol{Y}^{\mathrm{zad}}(k)=\left[
\begin{array}{c}
Y^{\mathrm{zad}}(k)\\
\vdots\\
Y^{\mathrm{zad}}(k)
\end{array}
\right]_{\mathrm{Nx1}}
\label{yzadm}
\end{equation}

\begin{equation}
\triangle\boldsymbol{U}(k)=\left[
\begin{array}{c}
\triangle u(k|k)\\
\vdots\\
\triangle u(k+N_u -1 |k)
\end{array}
\right]_{\mathrm{N_ux1}}
\label{dUm}
\end{equation}

\begin{equation}
\triangle\boldsymbol{U^P}(k)=\left[
\begin{array}{c}
\triangle u(k-1)\\
\vdots\\
\triangle u(k-(D-1))
\end{array}
\right]_{\mathrm{(D-1)x1}}
\label{dUPm}
\end{equation}

\begin{equation}
\boldsymbol{M}=\left[
\begin{array}
{cccc}
s_{1} & 0 & \ldots & 0\\
s_{2} & s_{1} & \ldots & 0\\
\vdots & \vdots & \ddots & \vdots\\
s_{N} & s_{N-1} & \ldots &  s_{N-N_{\mathrm{u}}+1}
\end{array}
\right]_{\mathrm{NxN_u}}
\label{Mm}
\end{equation}

\begin{equation}
\boldsymbol{M^P}=\left[
\begin{array}
{cccc}
s_{2}-s_{1} & s_{3}-s_{2} & \ldots & s_{D}-s_{D-1}\\
s_{3}-s_{1} & s_{4}-s_{2} & \ldots & s_{D+1}-s_{D-1}\\
\vdots & \vdots & \ddots & \vdots\\
s_{N+1}-s_{1} & s_{N+2}-s_{2} & \ldots &  s_{N+D-1}-S_{D-1}
\end{array}
\right]_{\mathrm{NxD-1}}
\label{MPm}
\end{equation}

\begin{equation}
Y^0(k)=Y(k)+M^P
\triangle U^P(k)
\label{Y0}
\end{equation}

\begin{equation}
K=(M^TM+\lambda*I)^{-1}M^T
\label{K}
\end{equation}

\begin{equation}
\triangle U(k)=K(Y^{zad}(k)-Y^0(k))
\label{dU1}
\end{equation}

\section{Dobieranie nastaw}
Długość horyzontu dynamiki została wyznaczona na podstawie obserwacji odpowiedzi skokowej wyznaczonej w poprzednim zadaniu. Na wykresie \ref{fig:s} można zaobserwować, że jej wartość ustala się w okolicach chwili $k=100$. Dlatego też przyjęto, że wykorzystywaną dalej wartością horyzontu dynamiki jest $D=100$. Parametr ten nie podlega optymalizacji.

Jako początkowe nastawy regulatora przyjęto parametry $D=N=N_u=100$ oraz $\lambda=1$.

\subsection{Horyzont predykcji}
 Pierwszym optymalizacji nastaw jest badanie zachowania regulatora przy zmniejszaniu horyzontu predykcji $N$. Jak się okazało, wysoka wartość $N$ nie jest konieczna dla prawidłowego działania algorytmu. Dla $N=20$ spadek jakości regulacji jest niewielki i to właśnie ta wartość będzie używana w kolejnych etapach doboru nastaw. Przebiegi sterowania i odpowiedzi obiektu podczas regulacji do wartości zadanej $Y_{zad}=1$ dla różnych horyzontów predykcji przedstawia wykres \ref{fig:z4_y}.

\begin{figure}[!htb]
\centering
\begin{tikzpicture}
\begin{groupplot}[group style={group size=1 by 3}, width=0.9\textwidth, height=0.75\textwidth]
\nextgroupplot
[
xmin=0,xmax=300,ymin=0,ymax=1.4,
xlabel={$k$},
	ylabel={$Y(k)$},
	xtick={0,50,100,150,200,250,300},
	ytick={0,0.2,0.4,0.6,0.8,1,1.2,1.4,1.6},
	/pgf/number format/.cd,
	use comma,
	1000 sep={}
]
	\addplot[blue,semithick] file {wykresy/zad4_y_100.txt};
	\addplot[brown,semithick] file {wykresy/zad4_y_50.txt};
	\addplot[purple,semithick] file {wykresy/zad4_y_20.txt};
	\addplot[green,semithick] file {wykresy/zad4_y_10.txt};
	\addplot[red,semithick] file {wykresy/zad4_y_8.txt};

\nextgroupplot
[
xmin=0,xmax=300,ymin=0,ymax=2.5,
xlabel={$k$},
ylabel={$U(k)$},
xtick={0, 50, 100, 150, 200, 250, 300},
ytick={0, 0.5, 1, 1.5, 2, 2.5},
/pgf/number format/.cd,
use comma,
1000 sep={}
]
	\addplot[blue,semithick] file {wykresy/zad4_u_100.txt};
	\addplot[brown,semithick] file {wykresy/zad4_u_50.txt};
	\addplot[purple,semithick] file {wykresy/zad4_u_20.txt};
	\addplot[green,semithick] file {wykresy/zad4_u_10.txt};
	\addplot[red,semithick] file {wykresy/zad4_u_8.txt};
	\legend{$N=\num{100}$,$N=\num{50}$,$N=\num{20}$,$N=\num{10}$,$N=\num{8}$}
\end{groupplot}
\end{tikzpicture}
\caption{Działanie regulatora DMC dla różnych horyzontów dynamiki}
\label{fig:z4_y}
\end{figure}

Wskaźniki jakości regulacji dla różnych horyzontów predykcji:
\begin{itemize}
	\item $N=100$, $E=9,9973$
	\item $N=50$, $E=9,9973$
	\item $N=20$, $E=10,0289$
	\item $N=10$, $E=12,2101$
	\item $N=8$, $E=19,8924$
\end{itemize}
\FloatBarrier
\subsection{Horyzont sterowania}
Kolejnym krokiem po doborze parametru $N$ jest wybór horyzontu sterowania $N_u$.
Wskaźniki jakości regulacji dla różnych horyzontów sterowania:
\begin{itemize}
	\item $N_u=20$, $E=10,0289$
	\item $N_u=10$, $E=10,03$
	\item $N_u=5$, $E=10,1213$
	\item $N_u=2$, $E=12,1356$
	\item $N_u=1$, $E=9,8556$
\end{itemize}
Dobrym kompromisem między jakością regulacji a przebiegiem sygnału sterowania wydaje się wartość $N_u=5$. Taki horyzont sterowania będzie wykorzystywany w kolejnych etapach projektu. 

\begin{figure}[!htb]
\centering
\begin{tikzpicture}
\begin{groupplot}[group style={group size=1 by 3}, width=0.9\textwidth, height=0.75\textwidth]
\nextgroupplot
[
xmin=0,xmax=300,ymin=0,ymax=1.4,
xlabel={$k$},
	ylabel={$Y(k)$},
	xtick={0,50,100,150,200,250,300},
	ytick={0,0.2,0.4,0.6,0.8,1,1.2,1.4,1.6},
	/pgf/number format/.cd,
	use comma,
	1000 sep={}
]
	\addplot[blue,semithick] file {wykresy/zad4_y_nu20.txt};
	\addplot[brown,semithick] file {wykresy/zad4_y_nu10.txt};
	\addplot[purple,semithick] file {wykresy/zad4_y_nu5.txt};
	\addplot[green,semithick] file {wykresy/zad4_y_nu2.txt};
	\addplot[red,semithick] file {wykresy/zad4_y_nu1.txt};

\nextgroupplot
[
xmin=0,xmax=300,ymin=0,ymax=2.5,
xlabel={$k$},
ylabel={$U(k)$},
xtick={0, 50, 100, 150, 200, 250, 300},
ytick={0, 0.5, 1, 1.5, 2, 2.5},
/pgf/number format/.cd,
use comma,
1000 sep={}
]
	\addplot[blue,semithick] file {wykresy/zad4_u_nu20.txt};
	\addplot[brown,semithick] file {wykresy/zad4_u_nu10.txt};
	\addplot[purple,semithick] file {wykresy/zad4_u_nu5.txt};
	\addplot[green,semithick] file {wykresy/zad4_u_nu2.txt};
	\addplot[red,semithick] file {wykresy/zad4_u_nu1.txt};
	\legend{$N_u=\num{20}$,$N_u=\num{10}$,$N_u=\num{5}$,$N_u=\num{2}$,$N_u=\num{1}$}
\end{groupplot}
\end{tikzpicture}
\caption{Działanie regulatora DMC dla różnych horyzontów dynamiki}
\label{fig:z4_nu}
\end{figure}
\FloatBarrier
\subsection{Parametr lambda}
Ostatnim krokiem optymalizacji nastaw regulatora DMC jest dobranie parametru $\lambda$.

Wskaźniki jakości regulacji dla różnych wartości $\lambda$:
\begin{itemize}
	\item $\lambda=10$, $E=12,9323$
	\item $\lambda=5$, $E=11,7987$
	\item $\lambda=2$, $E=10,7840$
	\item $\lambda=0,5$, $E=9,4717$
	\item $\lambda=0,2$, $E=8,7045$
	\item $\lambda=0,1$, $E=8,2595$

\end{itemize}
\begin{figure}[!htb]
\centering
\begin{tikzpicture}
\begin{groupplot}[group style={group size=1 by 3}, width=0.9\textwidth, height=0.75\textwidth]
\nextgroupplot
[
xmin=0,xmax=300,ymin=0,ymax=1.4,
xlabel={$k$},
	ylabel={$Y(k)$},
	xtick={0,50,100,150,200,250,300},
	ytick={0,0.2,0.4,0.6,0.8,1,1.2,1.4,1.6},
	/pgf/number format/.cd,
	use comma,
	1000 sep={}
]
	\addplot[blue,semithick] file {wykresy/zad4_y_lambda100.txt};
	\addplot[brown,semithick] file {wykresy/zad4_y_lambda50.txt};
	\addplot[purple,semithick] file {wykresy/zad4_y_lambda20.txt};
	\addplot[green,semithick] file {wykresy/zad4_y_lambda5.txt};
	\addplot[red,semithick] file {wykresy/zad4_y_lambda2.txt};
	\addplot[red,semithick] file {wykresy/zad4_y_lambda1.txt};

\nextgroupplot
[
xmin=0,xmax=300,ymin=0,ymax=3,
xlabel={$k$},
ylabel={$U(k)$},
xtick={0, 50, 100, 150, 200, 250, 300},
ytick={0, 0.5, 1, 1.5, 2, 2.5,3},
/pgf/number format/.cd,
use comma,
1000 sep={}
]
	\addplot[blue,semithick] file {wykresy/zad4_u_lambda100.txt};
	\addplot[brown,semithick] file {wykresy/zad4_u_lambda50.txt};
	\addplot[purple,semithick] file {wykresy/zad4_u_lambda20.txt};
	\addplot[green,semithick] file {wykresy/zad4_u_lambda5.txt};
	\addplot[red,semithick] file {wykresy/zad4_u_lambda2.txt};
	\addplot[red,semithick] file {wykresy/zad4_u_lambda1.txt};

	\legend{$\lambda=\num{10}$,$\lambda=\num{5}$,$\lambda=\num{2}$,$\lambda=\num{0,5}$,$\lambda=\num{0,2}$,$\lambda=\num{0,1}$}
\end{groupplot}
\end{tikzpicture}
\caption{Działanie regulatora DMC dla różnych horyzontów dynamiki}
\label{fig:z4_nu}
\end{figure}
\chapter{PID/DMC Optymalizacja}
Zadanie piąte polegało na znalezieniu optymalnych wartości nastaw dla regulatora PID oraz DMC, wykorzystując do optymalizacji ilościowy wskaźnik błędu regulacji E.
Stworzyliśmy do tego skrypt, który używa funkcji $fmincon$ w celu dobierania nastaw dla regulatorów.


Rozpoczniemy od optymalizacji nastaw regulatora PID. Jako cele optymalizacji przyjmujemy parametry $K$ oraz $T_i$. Zdecydowaliśmy się odgórnie narzucić zerowe wartości
$T_d$ ze względu na wyniki badań jakie przeprowadziliśmy w poprzednim punkcie sprawozdania.
Jako ograniczenia przyjmujemy zgodnie z logiką, wartości nie mniejsze niż 0 oraz mniejsze od nieskonczoności.
Jako punkt startowy wybieramy wartości wzmocnień równe $1$ a  czasy zdwojenia jako $1000$.
Optymalizacji poddajemy trzy regulatory PID w czterech konfiguracjach, tzn.
rozważamy cztery rózne konfiguracje torów sterowania wyznaczone w poprzednim punkcie sprawozdania.

Dla toru:
\begin{itemize}
  \item $y_1$ -- $u_4$
 \item $y_2$ -- $u_3$
 \item $y_3$ -- $u_2$
\end{itemize}

otrzymujemy nastawy:
\begin{equation}
  K_1 = \num{1.9724} \qquad T_{i1} = \num{210008}, \qquad T_{d1} = 0 \nonumber
\end{equation}
\begin{equation}
  K_2 = \num{1.3646} \qquad T_{i2} = \num{7.9930}, \qquad T_{d2} = 0
\end{equation}
\begin{equation}
  K_3 = \num{0.1696} \qquad T_{i3} = \num{300220}, \qquad T_{d3} = 0 \nonumber
\end{equation}
Takie nastawy dają błąd regulacji \num{486.6761}. Jest to dziwnie trudny przypadek,
gdyż nie udało nam się znaleźć nastaw dla niego ręcznie, a lekka modyfikacja nastaw
wygenerowanych funkcją \texttt{fmincon} daje ogromny skok błędu. Działanie
przedstawiają wykresy \ref{fig:pid1_fmincon_y1}, \ref{fig:pid1_fmincon_y2} i \ref{fig:pid1_fmincon_y3}.


Dla toru:
\begin{itemize}
  \item $y_1$ -- $u_1$
 \item $y_2$ -- $u_3$
 \item $y_3$ -- $u_4$
\end{itemize}

otrzymujemy nastawy:
\begin{equation}
  K_1 = \num{2.7249} \qquad T_{i1} = \num{3.9641}, \qquad T_{d1} = 0 \nonumber
\end{equation}
\begin{equation}
  K_2 = \num{2.9122} \qquad T_{i2} = \num{3.1237}, \qquad T_{d2} = 0
\end{equation}
\begin{equation}
  K_3 = \num{5.5929} \qquad T_{i3} = \num{9.8384}, \qquad T_{d3} = 0 \nonumber
\end{equation}

Błąd regulacji wynosi: \num{104.4858} . Jest to najlepszy ogólnie otrzymany wynik.
Działanie
przedstawiają wykresy \ref{fig:pid2_fmincon_y1}, \ref{fig:pid2_fmincon_y2} i \ref{fig:pid2_fmincon_y3}.

Dla toru:
\begin{itemize}
  \item $y_1$ -- $u_1$
 \item $y_2$ -- $u_2$
 \item $y_3$ -- $u_4$
\end{itemize}

otrzymujemy nastawy:
\begin{equation}
  K_1 = \num{2.9883} \qquad T_{i1} = \num{4.4246}, \qquad T_{d1} = 0 \nonumber
\end{equation}
\begin{equation}
  K_2 = \num{0.6972} \qquad T_{i2} = \num{8824.7}, \qquad T_{d2} = 0
\end{equation}
\begin{equation}
  K_3 = \num{5.6629} \qquad T_{i3} = \num{11.6449}, \qquad T_{d3} = 0 \nonumber
\end{equation}

Błąd regulacji wynosi: \num{115.6349} .
Jest to bardzo dobry wynik, jednakże nieco gorszy niż dla poprzedniego toru.
Zauważmy, że wyznaczone nastawy dla pierwszego i trzeciego regulatora są bardzo podobne do tych dla poprzedniego toru.
Wynika to z faktu, że zmienił się tu jedynie tor dla drugiego regulatora, gdzie teraz wpływa na wyjscie drugie regulator drugi.
Działanie
przedstawiają wykresy \ref{fig:pid3_fmincon_y1}, \ref{fig:pid3_fmincon_y2} i \ref{fig:pid3_fmincon_y3}.

Dla toru:
\begin{itemize}
  \item $y_1$ -- $u_2$
 \item $y_2$ -- $u_3$
 \item $y_3$ -- $u_1$
\end{itemize}

otrzymujemy nastawy:
\begin{equation}
  K_1 = \num{1.1001} \qquad T_{i1} = \num{1.7044}, \qquad T_{d1} = 0 \nonumber
\end{equation}
\begin{equation}
  K_2 = \num{2.2444} \qquad T_{i2} = \num{3.7464}, \qquad T_{d2} = 0
\end{equation}
\begin{equation}
  K_3 = \num{4.0054} \qquad T_{i3} = \num{15.7112}, \qquad T_{d3} = 0 \nonumber
\end{equation}

Błąd regulacji wynosi: \num{207.1231} .
Jest to wynik nieco gorszej jakości niż dla poprzednich dwóch torów.
Działanie
przedstawiają wykresy \ref{fig:pid4_fmincon_y1}, \ref{fig:pid4_fmincon_y2} i \ref{fig:pid4_fmincon_y3}.

Analizując otrzymane wyniki, zauważamy że dla wszystkich przypadków funkcja optymalizacji
znalazła lepsze wyniki niż dla regulatorów wyznaczonych metodą inżynierską. Jest to jak najbardziej normalne zjawisko.
Również potwierdziła się metoda Multiple Gain Array. Najlepsze wyniki otrzymaliśmy dla regulatora o konfiguracji
torów odpowiadającej najmniejszej wartości współczynnika uwarunkowania macierzy, zaś najgorsze wyniki
dla torów odpowiadających największej wartości współczynnika uwarunkowania macierzy.

%%%%%%%%%%%%%%%%%%%%%%%%%%%%%%%%%%%%%%%%%%%%%%%%%%%%%%%%%%%%%%%%%%%%%%%%%%%%%%%%%%%%%%%%%%%%
\begin{figure}[b]
\centering
\begin{tikzpicture}
\begin{axis}[
width=0.75\textwidth,
height = 0.4\textwidth,
xmin=0,xmax=1100,ymin=-15,ymax=15,
xlabel={Numer próbki},
ylabel={Wyjście},
xtick={0, 200, 400, 600, 800, 1000},
ytick={-15, -10, -5, 0, 5, 10, 15},
legend pos=north east,
/pgf/number format/.cd,
use comma,
1000 sep={}
]

\addplot[blue,semithick] file {wykresy/z3_yzad.txt};
\addplot[red,semithick] file {wykresy/pid1_fmincon_y1.txt};
\legend{Wartość zadana, Wyjście $y_1$}

\end{axis}
\end{tikzpicture}
\caption{Trajektoria wyjścia $y_1$, dla pierwszego zestawu regulatorów PID, dostrojonych funkcją optymalizacyjną}
\label{fig:pid1_fmincon_y1}
\end{figure}
%%%%%%%%%%%%%%%%%%%%%%%%%%%%%%%%%%%%%%%%%%%%%%%%%%%%%%%%%%%%%%%%%%%%%%%%%%%%%%%%%%%%%%%%%%%%

%%%%%%%%%%%%%%%%%%%%%%%%%%%%%%%%%%%%%%%%%%%%%%%%%%%%%%%%%%%%%%%%%%%%%%%%%%%%%%%%%%%%%%%%%%%%
\begin{figure}[b]
\centering
\begin{tikzpicture}
\begin{axis}[
width=0.75\textwidth,
height = 0.4\textwidth,
xmin=0,xmax=1100,ymin=-5,ymax=5,
xlabel={Numer próbki},
ylabel={Wyjście},
xtick={0, 200, 400, 600, 800, 1000},
ytick={-5, -4, -2, 0, 2, 4, 5},
legend pos=north east,
/pgf/number format/.cd,
use comma,
1000 sep={}
]

\addplot[blue,semithick] file {wykresy/z3_yzad.txt};
\addplot[red,semithick] file {wykresy/pid1_fmincon_y2.txt};
\legend{Wartość zadana, Wyjście $y_2$}

\end{axis}
\end{tikzpicture}
\caption{Trajektoria wyjścia $y_2$, dla pierwszego zestawu regulatorów PID, dostrojonych funkcją optymalizacyjną}
\label{fig:pid1_fmincon_y2}
\end{figure}
%%%%%%%%%%%%%%%%%%%%%%%%%%%%%%%%%%%%%%%%%%%%%%%%%%%%%%%%%%%%%%%%%%%%%%%%%%%%%%%%%%%%%%%%%%%%

%%%%%%%%%%%%%%%%%%%%%%%%%%%%%%%%%%%%%%%%%%%%%%%%%%%%%%%%%%%%%%%%%%%%%%%%%%%%%%%%%%%%%%%%%%%%
\begin{figure}[b]
\centering
\begin{tikzpicture}
\begin{axis}[
width=0.75\textwidth,
height = 0.4\textwidth,
xmin=0,xmax=1100,ymin=-2.5,ymax=2.5,
xlabel={Numer próbki},
ylabel={Wyjście},
xtick={0, 200, 400, 600, 800, 1000},
ytick={-2.5, -2, -1.5, -1, -.5, 0, .5, 1, 1.5, 2, 2.5},
legend pos=north east,
/pgf/number format/.cd,
use comma,
1000 sep={}
]

\addplot[blue,semithick] file {wykresy/z3_yzad.txt};
\addplot[red,semithick] file {wykresy/pid1_fmincon_y3.txt};
\legend{Wartość zadana, Wyjście $y_3$}

\end{axis}
\end{tikzpicture}
\caption{Trajektoria wyjścia $y_3$, dla pierwszego zestawu regulatorów PID, dostrojonych funkcją optymalizacyjną}
\label{fig:pid1_fmincon_y3}
\end{figure}
%%%%%%%%%%%%%%%%%%%%%%%%%%%%%%%%%%%%%%%%%%%%%%%%%%%%%%%%%%%%%%%%%%%%%%%%%%%%%%%%%%%%%%%%%%%%

%%%%%%%%%%%%%%%%%%%%%%%%%%%%%%%%%%%%%%%%%%%%%%%%%%%%%%%%%%%%%%%%%%%%%%%%%%%%%%%%%%%%%%%%%%%%
\begin{figure}[b]
\centering
\begin{tikzpicture}
\begin{axis}[
width=0.75\textwidth,
height = 0.4\textwidth,
xmin=0,xmax=1100,ymin=-15,ymax=15,
xlabel={Numer próbki},
ylabel={Wyjście},
xtick={0, 200, 400, 600, 800, 1000},
ytick={-15, -10, -5, 0, 5, 10, 15},
legend pos=north east,
/pgf/number format/.cd,
use comma,
1000 sep={}
]

\addplot[blue,semithick] file {wykresy/z3_yzad.txt};
\addplot[red,semithick] file {wykresy/pid2_fmincon_y1.txt};
\legend{Wartość zadana, Wyjście $y_1$}

\end{axis}
\end{tikzpicture}
\caption{Trajektoria wyjścia $y_1$, dla drugiego zestawu regulatorów PID, dostrojonych funkcją optymalizacyjną}
\label{fig:pid2_fmincon_y1}
\end{figure}
%%%%%%%%%%%%%%%%%%%%%%%%%%%%%%%%%%%%%%%%%%%%%%%%%%%%%%%%%%%%%%%%%%%%%%%%%%%%%%%%%%%%%%%%%%%%

%%%%%%%%%%%%%%%%%%%%%%%%%%%%%%%%%%%%%%%%%%%%%%%%%%%%%%%%%%%%%%%%%%%%%%%%%%%%%%%%%%%%%%%%%%%%
\begin{figure}[b]
\centering
\begin{tikzpicture}
\begin{axis}[
width=0.75\textwidth,
height = 0.4\textwidth,
xmin=0,xmax=1100,ymin=-5,ymax=5,
xlabel={Numer próbki},
ylabel={Wyjście},
xtick={0, 200, 400, 600, 800, 1000},
ytick={-5, -4, -2, 0, 2, 4, 5},
legend pos=north east,
/pgf/number format/.cd,
use comma,
1000 sep={}
]

\addplot[blue,semithick] file {wykresy/z3_yzad.txt};
\addplot[red,semithick] file {wykresy/pid2_fmincon_y2.txt};
\legend{Wartość zadana, Wyjście $y_2$}

\end{axis}
\end{tikzpicture}
\caption{Trajektoria wyjścia $y_2$, dla drugiego zestawu regulatorów PID, dostrojonych funkcją optymalizacyjną}
\label{fig:pid2_fmincon_y2}
\end{figure}
%%%%%%%%%%%%%%%%%%%%%%%%%%%%%%%%%%%%%%%%%%%%%%%%%%%%%%%%%%%%%%%%%%%%%%%%%%%%%%%%%%%%%%%%%%%%

%%%%%%%%%%%%%%%%%%%%%%%%%%%%%%%%%%%%%%%%%%%%%%%%%%%%%%%%%%%%%%%%%%%%%%%%%%%%%%%%%%%%%%%%%%%%
\begin{figure}[b]
\centering
\begin{tikzpicture}
\begin{axis}[
width=0.75\textwidth,
height = 0.4\textwidth,
xmin=0,xmax=1100,ymin=-2.5,ymax=2.5,
xlabel={Numer próbki},
ylabel={Wyjście},
xtick={0, 200, 400, 600, 800, 1000},
ytick={-2.5, -2, -1.5, -1, -.5, 0, .5, 1, 1.5, 2, 2.5},
legend pos=north east,
/pgf/number format/.cd,
use comma,
1000 sep={}
]

\addplot[blue,semithick] file {wykresy/z3_yzad.txt};
\addplot[red,semithick] file {wykresy/pid2_fmincon_y3.txt};
\legend{Wartość zadana, Wyjście $y_3$}

\end{axis}
\end{tikzpicture}
\caption{Trajektoria wyjścia $y_3$, dla drugiego zestawu regulatorów PID, dostrojonych funkcją optymalizacyjną}
\label{fig:pid2_fmincon_y3}
\end{figure}
%%%%%%%%%%%%%%%%%%%%%%%%%%%%%%%%%%%%%%%%%%%%%%%%%%%%%%%%%%%%%%%%%%%%%%%%%%%%%%%%%%%%%%%%%%%%

%%%%%%%%%%%%%%%%%%%%%%%%%%%%%%%%%%%%%%%%%%%%%%%%%%%%%%%%%%%%%%%%%%%%%%%%%%%%%%%%%%%%%%%%%%%%
\begin{figure}[b]
\centering
\begin{tikzpicture}
\begin{axis}[
width=0.75\textwidth,
height = 0.4\textwidth,
xmin=0,xmax=1100,ymin=-15,ymax=15,
xlabel={Numer próbki},
ylabel={Wyjście},
xtick={0, 200, 400, 600, 800, 1000},
ytick={-15, -10, -5, 0, 5, 10, 15},
legend pos=north east,
/pgf/number format/.cd,
use comma,
1000 sep={}
]

\addplot[blue,semithick] file {wykresy/z3_yzad.txt};
\addplot[red,semithick] file {wykresy/pid3_fmincon_y1.txt};
\legend{Wartość zadana, Wyjście $y_1$}

\end{axis}
\end{tikzpicture}
\caption{Trajektoria wyjścia $y_1$, dla trzeciego zestawu regulatorów PID, dostrojonych funkcją optymalizacyjną}
\label{fig:pid3_fmincon_y1}
\end{figure}
%%%%%%%%%%%%%%%%%%%%%%%%%%%%%%%%%%%%%%%%%%%%%%%%%%%%%%%%%%%%%%%%%%%%%%%%%%%%%%%%%%%%%%%%%%%%

%%%%%%%%%%%%%%%%%%%%%%%%%%%%%%%%%%%%%%%%%%%%%%%%%%%%%%%%%%%%%%%%%%%%%%%%%%%%%%%%%%%%%%%%%%%%
\begin{figure}[b]
\centering
\begin{tikzpicture}
\begin{axis}[
width=0.75\textwidth,
height = 0.4\textwidth,
xmin=0,xmax=1100,ymin=-5,ymax=5,
xlabel={Numer próbki},
ylabel={Wyjście},
xtick={0, 200, 400, 600, 800, 1000},
ytick={-5, -4, -2, 0, 2, 4, 5},
legend pos=north east,
/pgf/number format/.cd,
use comma,
1000 sep={}
]

\addplot[blue,semithick] file {wykresy/z3_yzad.txt};
\addplot[red,semithick] file {wykresy/pid3_fmincon_y2.txt};
\legend{Wartość zadana, Wyjście $y_2$}

\end{axis}
\end{tikzpicture}
\caption{Trajektoria wyjścia $y_2$, dla trzeciego zestawu regulatorów PID, dostrojonych funkcją optymalizacyjną}
\label{fig:pid3_fmincon_y2}
\end{figure}
%%%%%%%%%%%%%%%%%%%%%%%%%%%%%%%%%%%%%%%%%%%%%%%%%%%%%%%%%%%%%%%%%%%%%%%%%%%%%%%%%%%%%%%%%%%%

%%%%%%%%%%%%%%%%%%%%%%%%%%%%%%%%%%%%%%%%%%%%%%%%%%%%%%%%%%%%%%%%%%%%%%%%%%%%%%%%%%%%%%%%%%%%
\begin{figure}[b]
\centering
\begin{tikzpicture}
\begin{axis}[
width=0.75\textwidth,
height = 0.4\textwidth,
xmin=0,xmax=1100,ymin=-2.5,ymax=2.5,
xlabel={Numer próbki},
ylabel={Wyjście},
xtick={0, 200, 400, 600, 800, 1000},
ytick={-2.5, -2, -1.5, -1, -.5, 0, .5, 1, 1.5, 2, 2.5},
legend pos=north east,
/pgf/number format/.cd,
use comma,
1000 sep={}
]

\addplot[blue,semithick] file {wykresy/z3_yzad.txt};
\addplot[red,semithick] file {wykresy/pid3_fmincon_y3.txt};
\legend{Wartość zadana, Wyjście $y_3$}

\end{axis}
\end{tikzpicture}
\caption{Trajektoria wyjścia $y_3$, dla trzeciego zestawu regulatorów PID, dostrojonych funkcją optymalizacyjną}
\label{fig:pid3_fmincon_y3}
\end{figure}
%%%%%%%%%%%%%%%%%%%%%%%%%%%%%%%%%%%%%%%%%%%%%%%%%%%%%%%%%%%%%%%%%%%%%%%%%%%%%%%%%%%%%%%%%%%%

%%%%%%%%%%%%%%%%%%%%%%%%%%%%%%%%%%%%%%%%%%%%%%%%%%%%%%%%%%%%%%%%%%%%%%%%%%%%%%%%%%%%%%%%%%%%
\begin{figure}[b]
\centering
\begin{tikzpicture}
\begin{axis}[
width=0.75\textwidth,
height = 0.4\textwidth,
xmin=0,xmax=1100,ymin=-15,ymax=15,
xlabel={Numer próbki},
ylabel={Wyjście},
xtick={0, 200, 400, 600, 800, 1000},
ytick={-15, -10, -5, 0, 5, 10, 15},
legend pos=north east,
/pgf/number format/.cd,
use comma,
1000 sep={}
]

\addplot[blue,semithick] file {wykresy/z3_yzad.txt};
\addplot[red,semithick] file {wykresy/pid4_fmincon_y1.txt};
\legend{Wartość zadana, Wyjście $y_1$}

\end{axis}
\end{tikzpicture}
\caption{Trajektoria wyjścia $y_1$, dla czwartego zestawu regulatorów PID, dostrojonych funkcją optymalizacyjną}
\label{fig:pid4_fmincon_y1}
\end{figure}
%%%%%%%%%%%%%%%%%%%%%%%%%%%%%%%%%%%%%%%%%%%%%%%%%%%%%%%%%%%%%%%%%%%%%%%%%%%%%%%%%%%%%%%%%%%%

%%%%%%%%%%%%%%%%%%%%%%%%%%%%%%%%%%%%%%%%%%%%%%%%%%%%%%%%%%%%%%%%%%%%%%%%%%%%%%%%%%%%%%%%%%%%
\begin{figure}[b]
\centering
\begin{tikzpicture}
\begin{axis}[
width=0.75\textwidth,
height = 0.4\textwidth,
xmin=0,xmax=1100,ymin=-5,ymax=5,
xlabel={Numer próbki},
ylabel={Wyjście},
xtick={0, 200, 400, 600, 800, 1000},
ytick={-5, -4, -2, 0, 2, 4, 5},
legend pos=north east,
/pgf/number format/.cd,
use comma,
1000 sep={}
]

\addplot[blue,semithick] file {wykresy/z3_yzad.txt};
\addplot[red,semithick] file {wykresy/pid4_fmincon_y2.txt};
\legend{Wartość zadana, Wyjście $y_2$}

\end{axis}
\end{tikzpicture}
\caption{Trajektoria wyjścia $y_2$, dla czwartego zestawu regulatorów PID, dostrojonych funkcją optymalizacyjną}
\label{fig:pid4_fmincon_y2}
\end{figure}
%%%%%%%%%%%%%%%%%%%%%%%%%%%%%%%%%%%%%%%%%%%%%%%%%%%%%%%%%%%%%%%%%%%%%%%%%%%%%%%%%%%%%%%%%%%%

%%%%%%%%%%%%%%%%%%%%%%%%%%%%%%%%%%%%%%%%%%%%%%%%%%%%%%%%%%%%%%%%%%%%%%%%%%%%%%%%%%%%%%%%%%%%
\begin{figure}[b]
\centering
\begin{tikzpicture}
\begin{axis}[
width=0.75\textwidth,
height = 0.4\textwidth,
xmin=0,xmax=1100,ymin=-2.5,ymax=2.5,
xlabel={Numer próbki},
ylabel={Wyjście},
xtick={0, 200, 400, 600, 800, 1000},
ytick={-2.5, -2, -1.5, -1, -.5, 0, .5, 1, 1.5, 2, 2.5},
legend pos=north east,
/pgf/number format/.cd,
use comma,
1000 sep={}
]

\addplot[blue,semithick] file {wykresy/z3_yzad.txt};
\addplot[red,semithick] file {wykresy/pid4_fmincon_y3.txt};
\legend{Wartość zadana, Wyjście $y_3$}

\end{axis}
\end{tikzpicture}
\caption{Trajektoria wyjścia $y_3$, dla czwartego zestawu regulatorów PID, dostrojonych funkcją optymalizacyjną}
\label{fig:pid4_fmincon_y3}
\end{figure}
%%%%%%%%%%%%%%%%%%%%%%%%%%%%%%%%%%%%%%%%%%%%%%%%%%%%%%%%%%%%%%%%%%%%%%%%%%%%%%%%%%%%%%%%%%%%

% \chapter{Zakłócenie sinusoidalne}
Regulator DMC o parametrach:
\begin{itemize}
	\item $D=100$
	\item $D^z=40$
	\item $N=20$
	\item $N_u=5$
	\item $\lambda=0,5$
\end{itemize}
zostanie użyty do regulacji obiektu ze skokiem wartości zadanej $Y_{zad}=1$ oraz zakłóceniem sinusoidalnym, którego przebieg przedstawia wykres \ref{fig:z6_sine}. Regulator nie jest w stanie całkowicie zniwelować zakłócenia sinusoidalnego, co widać na wykresie \ref{fig:z6_y}. Wersja z uwzględnianiem zakłóceń tłumi jednak zakłócenie znacznie lepiej, o czym świadczą mniejsze uchyby niż w wersji regulatora bez uwzględniania zakłóceń.

Obie wersje regulatora mają podobne przebiegi sygnału sterowania (wykres \ref{fig:z6_u}). Reakcja na sygnał zakłócenia następuje jednak szybciej dla regulatora z uwzględnianiem zakłóceń niż bez.

Wskaźnik jakości regulacji dla regulatora DMC z uwzględnianiem zakłóceń ma wartość $E=13,4614$, a dla regulatora bez uwzględniania zakłóceń $E=24,5018$. Jak widać, zastosowanie uwzględniania zakłóceń pozwala znacznie ograniczyć błędy regulacji. Trzeba jednak mieć na uwadze, że nie jest możliwa całkowita niwelacja zakłócenia sinusoidalnego.

\begin{figure}[!htb]
	\centering
	\begin{tikzpicture}
	\begin{axis}[
	width=0.9\textwidth,
	height=0.9\textwidth,
	xmin=0,xmax=500,ymin=0,ymax=1.6,
	xlabel={$k$},
	ylabel={$Y(k)$},
	xtick={0,50,100,150,200,250,300,350,400,450,500},
	ytick={0,0.2,0.4,0.6,0.8,1,1.2,1.4,1.6},
	/pgf/number format/.cd,
	use comma,
	1000 sep={}
	]
	\addplot[blue,semithick] file {wykresy/zad6_z_y.txt};
	\addplot[red,semithick] file {wykresy/zad6_bez_y.txt};
	\addplot[magenta,dashed] file {wykresy/zad6_yzad.txt};
	\legend{$Y^z(k)$,$Y(k)$}
	\end{axis}
	\end{tikzpicture}
	\caption{Wyjście obiektu przy zakłóceniu sinusoidalnym}
\label{fig:z6_y}
\end{figure}

\begin{figure}[!htb]
	\centering
	\begin{tikzpicture}
	\begin{axis}[
	width=0.9\textwidth,
	height=0.9\textwidth,
	xmin=0,xmax=500,ymin=0,ymax=2.5,
	xlabel={$k$},
	ylabel={$U(k)$},
	xtick={0,50,100,150,200,250,300,350,400,450,500},
	ytick={0,0.5,1,1.5,2,2.5},
	/pgf/number format/.cd,
	use comma,
	1000 sep={}
	]
	\addplot[blue,semithick] file {wykresy/zad6_z_u.txt};
	\addplot[red,semithick] file {wykresy/zad6_bez_u.txt};
	\legend{$U^z(k)$,$U(k)$}

	\end{axis}
	\end{tikzpicture}
	\caption{Sterowanie obiektu przy zakłóceniu sinusoidalnym}
\label{fig:z6_u}
\end{figure}

\begin{figure}[!htb]
	\centering
	\begin{tikzpicture}
	\begin{axis}[
	width=0.9\textwidth,
	height=0.9\textwidth,
	xmin=0,xmax=500,ymin=-1.5,ymax=1.5,
	xlabel={$k$},
	ylabel={$Z(k)$},
	xtick={0,50,100,150,200,250,300,350,400,450,500},
	ytick={-1.5,-1,-0.5,0,0.5,1,1.5},
	/pgf/number format/.cd,
	use comma,
	1000 sep={}
	]
	\addplot[blue,semithick] file {wykresy/zad6_sine.txt};
	\end{axis}
	\end{tikzpicture}
	\caption{Przebieg sinusoidalnego sygnału zakłóceń}
\label{fig:z6_sine}
\end{figure}

% \chapter{Zadanie 7}
Od tego rozdziału wszystkie zadania będą dotyczyć obiektu Inteco,
opisanego w rozdziale \ref{sec:inteco}. Na początku został opracowany sposób
komunikacji z obiektem. Zarówno wejścia jak i wyjścia obiektu są sterowane
przy pomocy PWM. Do generacji sygnałów wejściowych do obiektu, użyta
została specjalna funkcja PWM sterownika. Generuje ona potrzebny sygnał
o zadanej częstotliwości, oraz poziomie wypełnienia fali. Tak wygenerowana
fala wysyłana jest na wyjście sterownika. Odczyt pomiarów został
wykonany przy pomocy, również wbudowanej w sterownik, opcji szybkiego licznika
na wejściu. Przy tak skonfigurowanym programie sprawdzona została możliwość
sterowania obiektem, jak i odczytywania wejść. Aby ułatwić ocenę działania
jakiegokolwiek regulatora, pojdęliśmy próbę przeskalowania wartości mierzonych
tak, aby skala była taka sama jak skala umieszczona bezpośrednio na zbiornikach.
Zastosowaliśmy przy tym skalę liniową opartą na pomiarze przy niskim stanie
wody, oraz na pomiarze w górnych granicach skali. Na podstawie dwóch pomiarów
stworzyliśmy skalę liniową, która dobrze przybliżała poziom wody w odniesieniu
do skali na zbiornikach.

\end{document}
