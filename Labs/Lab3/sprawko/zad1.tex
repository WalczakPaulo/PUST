\chapter{Zadanie 1}

W pierwszym zadaniu sprawdziliśmy możliwośc interakcji ze stanowiskiem. Sprawdziliśmy to wykorzystyjąc gotowe metody \texttt{sendControls} oraz \texttt{readMeasurements}. Stanowisko reagowało
prawidłowo. Następnie sprawdziliśmy wyjścia obiektu w punkcie pracy. Punkt pracy zdefiniowany był wzorem w treści zadania laboratoryjnego i u nas wynosił $U_{pp1} = 27$ oraz $U_{pp2} = 32$.
Po odczekaniu odpowiedniej ilości czasu (kilku minut) wnioskujemy, że dla zadanych wartości sterowania, wyjścia obiektu stabilizują się na poziomach $Y_{pp1} = 35.5$, oraz $Y_{pp2} = 37.06$.
