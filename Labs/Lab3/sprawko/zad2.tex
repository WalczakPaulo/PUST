\chapter{Zadanie 2}
Celem zadania drugiego było wyznaczenie skrośnych odpowiedzi skokowych procesu dla trzech różnych zmian sygnału sterującego G1 rozpoczynając z punktu pracy – pomiar na T3 (lub odwrotnie: sygnał
sterujacy G2 – pomiar T1). Wykonanie zadania drugiego okazało się wyjątkowo czasochłonne ponieważ trzeba było pozyskać dużą ilość odpowiedzi skokowych. Wiązało się to z naprzemiennymi skokami
wartości sterowania oraz sprowadzaniem obiektu do punktu pracy. Ze względu na wolny czas stabilizacji obiektu trwało to dużą ilość czasu, lecz wyciągając wnioski z poprzedniego laboratorium
postanowiliśmy szczególną uwagę zwrócić własnie na pozyskanie odpowiedzi skokowych wysokiej jakości, mając na celu stworzenie dobrego regulatora DMC. Wykonaliśmy skoki sterorowań z punktu pracy
o odpowiednio 10\%, 15\% oraz 20\%. Polegało to na skoku sterowania na jednej grzałce, podczas gdy druga zachowywała wartość punktu pracy. Analogicznie przeprowadzono potem eksperyment odwrotny:
pierwsza grzałka zachowywała wartość punktu pracy, podczas gdy druga grzałka "otrzymała" skok sterowania. Badaliśmy po skokach sterowania wartości wyjśc obiektu. Wzorując się jedynie na
otrzymanych wykresach nie moglibyśmy stwierdzić, że obiekt charakteryzuje wzmocnienie statyczne. Musimy jednakże wziąć po uwagę, że odpowiedzi skrośne charakteryzują się
małym wzmocnieniem. Z tego względu duży wpływ na odpowiedź mają wówczas zakłócenia. Jak wiemy z doświadczenia obiekt jest bardzo wrażliwy na zakłócenia, co powoduje że wyniki
mogą nie być kompletnie zgodne z wynikami które uzyskalibyśmy w hermetycznym środowisku. Z doświadczeń z obiektem i przeprowadzonych wcześniej licznych eksperymentów na stanowisku,
wynikało że obiekt jest liniowy i posiada właściwości statyczne. Przeanalizujmy już same przebiegi. Najpierw na warsztat weźmiemy przebieg $Y2(U1)$. Po skoku sterowania o 10\%
natepuje skok temperatury o około $1^\circ$C. Po skoku o 15\% nastepuje skok temperatury o około $1.38^\circ$C. Po skoku o 20\% wyjście rośnie około $1.88^\circ$C. Jest to zależność prawie liniowa,
o wzmocnieniu statycznym około $1,88/20 = 0.094$. Zajmijmy się teraz drugim torem $Y2(U1)$. Po skoku sterowania o 10\%
natepuje skok temperatury o około $0.82^\circ$C. Niestety zgubiliśmy dane na temat skoku sterowania o 15\%. Po skoku o 20\% wyjscie rosnie około $1.63^\circ$C. Jest to zależność praktycznie
liniowa,
o wzmocnieniu statycznym około 1,63/20 = 0.0815.  Biorąc to pod uwagę, możemy z pewną małą dozą niepewności ustalić
że obiekt ma właściwości statyczne (proporcjonalny skok wartości wyjścia do skoku wartości sterowania). Wzmocnienie statyczne toru $Y2(U1) = 0.094$, zaś toru $Y1(U2) = 0.0815$.

\begin{figure}[tb]
\centering
\begin{tikzpicture}
\begin{groupplot}[group style={group size=1 by 2}, width=0.9\textwidth, height=0.4\textwidth]
\nextgroupplot
[
xmin=0,xmax=350,ymin=36,ymax=46,
xlabel={$k$},
ylabel={$y_1$},
xtick={0, 50, 100, 150, 200, 300, 350},
ytick={36, 38, 40, 42, 44, 46},
legend pos=north west,
/pgf/number format/.cd,
use comma,
1000 sep={}
]
\addplot[blue,semithick] file {wykresy/z2_y1_u1_10.txt};
\addplot[red,semithick] file {wykresy/z2_y1_u1_15.txt};
\addplot[green,semithick] file {wykresy/z2_y1_u1_20.txt};
\legend{$\delta u_1 = 10$, $\delta u_1 = 15$, $\delta u_1 = 20$}

\nextgroupplot
[
xmin=0,xmax=350,ymin=36,ymax=46,
xlabel={$k$},
ylabel={$y_2$},
xtick={0, 50, 100, 150, 200, 300, 350},
ytick={36, 38, 40, 42, 44, 46},
legend pos=north west,
/pgf/number format/.cd,
use comma,
1000 sep={}
]
\addplot[blue,semithick] file {wykresy/z2_y2_u1_10.txt};
\addplot[red,semithick] file {wykresy/z2_y2_u1_15.txt};
\addplot[green,semithick] file {wykresy/z2_y2_u1_20.txt};
\legend{$\delta u_1 = 10$, $\delta u_1 = 15$, $\delta u_1 = 20$}

\end{groupplot}
\end{tikzpicture}
\caption{Wpływ skoków sterowania $u_1$ na wyjścia.}
\label{fig:skoki_u1}
\end{figure}

\begin{figure}[tb]
\centering
\begin{tikzpicture}
\begin{groupplot}[group style={group size=1 by 2}, width=0.9\textwidth, height=0.4\textwidth]
\nextgroupplot
[
xmin=0,xmax=350,ymin=36,ymax=46,
xlabel={$k$},
ylabel={$y_1$},
xtick={0, 50, 100, 150, 200, 300, 350},
ytick={36, 38, 40, 42, 44, 46},
legend pos=north west,
/pgf/number format/.cd,
use comma,
1000 sep={}
]
\addplot[blue,semithick] file {wykresy/z2_y1_u2_10.txt};
\addplot[red,semithick] file {wykresy/z2_y1_u2_15.txt};
\addplot[green,semithick] file {wykresy/z2_y1_u2_20.txt};
\legend{$\delta u_1 = 10$, $\delta u_1 = 15$, $\delta u_1 = 20$}

\nextgroupplot
[
xmin=0,xmax=350,ymin=36,ymax=46,
xlabel={$k$},
ylabel={$y_2$},
xtick={0, 50, 100, 150, 200, 300, 350},
ytick={36, 38, 40, 42, 44, 46},
legend pos=north west,
/pgf/number format/.cd,
use comma,
1000 sep={}
]
\addplot[blue,semithick] file {wykresy/z2_y2_u2_10.txt};
\addplot[red,semithick] file {wykresy/z2_y2_u2_15.txt};
\addplot[green,semithick] file {wykresy/z2_y2_u2_20.txt};
\legend{$\delta u_2 = 10$, $\delta u_2 = 15$, $\delta u_2 = 20$}

\end{groupplot}
\end{tikzpicture}
\caption{Wpływ skoków sterowania $u_2$ na wyjścia.}
\label{fig:skoki_u1}
\end{figure}
