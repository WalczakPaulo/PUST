\chapter{DMC z uwzględnieniem zakłóceń}
Celem zadania piątego było porównanie działania regulatora DMC z uwzględnieniem
zakłóceń oraz bez zakłóceń. W celu przeprowadzenia testów należało zasymulować
zakłócenie mierzalne. Po osiągnięciu przez regulator wartości zadanej
przeprowadzaliśmy skok wartości zakłócenia. Zgodnie z tym co opisaliśmy w
poprzednich podpunktach mieliśmy bardzo duże problemy z doprowadzeniem regulatora
do stanu gdzie zostanie osiągnięta i ustabilizuje się wartość zadana. Było to
spowodowane wrażliwością obiektu na zakłócenia zewnętrzne, takie jak otwarte okno,
poruszające się po sali osoby. Powodowało to cyrkulację powietrza, która
uniemożliwiała dokładne sterowanie obiektem. Z tego powodu, ciężko również ocenić
jakość regulacji regulatora DMC z uwzględnieniem zakłóceń oraz tego który zakłóceń
nie uwzględnia. Zgodnie z teorią, jaką przedstawiliśmy w sprawozdaniu z
projektu nr 2, lepszy powinien się okazać regulator z uwzględnieniem zakłóceń.
Jakość regulacji oceniamy obserwując wykresy oraz wyliczając tzw. błąd kwadratowy
(między wyjściem obiektu a wartością zadaną). Wykres \ref{fig:zakl} przedstawia
wyjście obiektu ze skokiem zakłóceń z w chwili 400.

\begin{figure}[tb]
\centering
\begin{tikzpicture}
\begin{axis}[
width=0.75\textwidth,
xmin=0,xmax=600,ymin=31.5,ymax=36,
xlabel={Czas (s)},
ylabel={Wartość wyjścia},
xtick={0, 100, 200, 300, 400, 500, 600},
ytick={31.5, 32, 32.5, 33, 33.5, 34, 34.5, 35, 35.5, 36},
legend pos=south east,
/pgf/number format/.cd,
use comma,
1000 sep={}
]
\addplot[blue,semithick] file {Skrypty/y_zakl.txt};
\addplot[red,semithick] file {Skrypty/y_zad_zakl.txt};
\legend{Wyjście obiektu, Wartość zadana}
\end{axis}
\end{tikzpicture}
\caption{Skok zakłócenia w chwili 400.}
\label{fig:zakl}
\end{figure}

\begin{figure}[tb]
\centering
\begin{tikzpicture}
\begin{axis}[
width=0.75\textwidth,
xmin=0,xmax=600,ymin=25,ymax=65,
xlabel={Czas (s)},
ylabel={Wartość wyjścia},
xtick={0, 100, 200, 300, 400, 500, 600},
ytick={25, 30, 35, 40, 45, 50, 55, 60, 65},
legend pos=north west,
/pgf/number format/.cd,
use comma,
1000 sep={}
]
\addplot[blue,semithick] file {Skrypty/u_zakl.txt};
\end{axis}
\end{tikzpicture}
\caption{Sterowanie.}
\label{fig:ster_zakl}
\end{figure}
