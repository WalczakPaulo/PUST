\chapter{Odpowiedź skokowa do DMC}
Celem zadania trzeciego było przygotowanie odpowiedzi skokowych wykorzystywanych
w algorytmie DMC. W naszym przypadku konieczne było zebranie odpowiedzi na skok
sygnału sterującego oraz zakłócającego. Skok wykonujemy z obranego wcześniej punktu
pracy ($Y_{pp} = 32,20$, $U_{pp} = 27$, $Z = 0$). Z powodu nieuniknionych zakłóceń
w pomieszczeniu, szczególnie trudne było ustabilizowanie obiektu w punkcie pracy,
po wykonaniu wcześniejszych eksperymentów. Z tego powodu otrzymane przez nas odpowiedzi
były prawdopodbnie niedokładne. W naszych głowach zrodziła się z początku myśl,
aby skorzystać z odpowiedzi skokowej ( na skok sterowania ) z poprzedniego laboratoium
(pracowaliśmy również na tym stanowisku ), jednakże jak się później okazało
( problemy z algorytmem DMC) zdecydowaliśmy się na ponowne pozyskanie owej odpowiedzi skokowej.
Wykonując skok sterowania, wartość zakłócenia ustawiliśmy na 0 i odpowiednio
wykonując skok zakłócenia, wartość sterowania pozostawiliśmy na taką jak w punkcie
pracy ( $U_{pp} = 27$). Z powodu zakłóceń i problemów ze stabilizacją obiektu uzyskane
przez nas odpowiedzi skokowe nie były dla nas satysfakcjonujące, lecz z powodu braku
czasu nie zdecydowaliśmy się na powtarzanie eksperymentu. Normalizacji dokonaliśmy
poprzez przesunięcie wszystkich wartości wyjścia obiektu o wartość wyjścia w punkcie
pracy oraz podzielenie przez wartość odpowiednio skoku sterowania lub zakłócenia.
Aproksymacja ma postać członu inercyjnego drugiego rzędu z opóźnieniem.
Aproksymacji dokonaliśmy przy użyciu opracowanego przez nas wcześniej skryptu.
Użyliśmy do tego między innymi wbudowanej funkcji GA ( Genetic Algorithm ) w celu
pozyskania aproksymacji minimalizującej funkcję błędu kwadratowego.

\begin{figure}[tb]
\centering
\begin{tikzpicture}
\begin{axis}[
width=0.75\textwidth,
xmin=0,xmax=400,ymin=0,ymax=0.35,
xlabel={Czas (s)},
ylabel={Wartość wyjścia},
xtick={0, 100, 200, 300, 400},
ytick={0, 0.05, 0.1, 0.15, 0.2, 0.25, 0.3, 0.35},
legend pos=north west,
/pgf/number format/.cd,
use comma,
1000 sep={}
]
\addplot[blue,semithick] file {Skrypty/s_approx.txt};
\end{axis}
\end{tikzpicture}
\caption{Przybliżona odpowiedź obiektu na skok sterowania.}
\label{fig:skok_zak}
\end{figure}


\begin{figure}[tb]
\centering
\begin{tikzpicture}
\begin{axis}[
width=0.75\textwidth,
xmin=0,xmax=400,ymin=0,ymax=0.08,
xlabel={Czas (s)},
ylabel={Wartość wyjścia},
xtick={0, 100, 200, 300, 400},
ytick={0, 0.01, 0.02, 0.03, 0.04, 0.05, 0.06, 0.07, 0.08},
legend pos=north west,
/pgf/number format/.cd,
use comma,
1000 sep={}
]
\addplot[blue,semithick] file {Skrypty/odp_zakl.txt};
\end{axis}
\end{tikzpicture}
\caption{Przybliżona odpowiedź obiektu na skok zakłócenia.}
\label{fig:skok_zak}
\end{figure}
