\chapter{DMC}
Celem zadania 4 była implementacja algorytmu DMC oraz doboru nastaw $\Lambda$, $N$ oraz $N_u$. Po testach wybraliśmy nastawy $N=Nu=D=200$ . Jako lambdę wybraliśmy wartość 1. Taka wartość sprawdziła się idealnie
na poprzednim laboratorium. Jakość regulacji nie była na wysokim poziomie. W porównaniu do ostatniego laboratorium znacznie się pogorszyła, co uzasadniamy głównie zakłóceniami takimi jak
otwarte okno oraz poruszające się w pobliżu stanowiska osoby, które prawdopodobnie
pogorszyły poprawność naszej odpowiedzi skokowej. Główną wadą były oscylacje w pobliżu wartośći zadanej.Prawdopodobnie wybranie większej wartości lambdy spowodowałoby mniejsze oscylacje w pobliżu wartości zadanej,
jednakże równocześnie pogarszając szybkość regulacji. Niestety z powodu braku czasu nie zdołaliśmy przetestować innych wartości nastaw. Zakładamy jednak, że zmienianie nastaw nie wpłynęłoby
znacząco na poprawę jakości regulacji, gdyż głównym czynnikiem zaburzającym jakość regulacji była silna wrażliwość obiektu na zakłócenia z otoczenia.
