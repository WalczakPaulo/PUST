\chapter{DMC}
Celem zadania 4 była implementacja algorytmu DMC oraz doboru nastaw $D$, $\Lambda$,
$N$ oraz $N_u$. Po testach wybraliśmy nastawy $N=N_u=D=200$. Jako lambdę wybraliśmy wartość $1$.
Taka wartość sprawdziła się idealnie na poprzednim laboratorium.
Jakość regulacji nie była na wysokim poziomie. W porównaniu do ostatniego
laboratorium znacznie się pogorszyła, co uzasadniamy głównie zakłóceniami,
które prawdopodobnie pogorszyły poprawność naszej odpowiedzi skokowej.
Główną wadą były oscylacje w pobliżu wartośći zadanej. Prawdopodobnie wybranie
większej wartości lambdy spowodowałoby mniejsze oscylacje w pobliżu wartości zadanej,
jednakże równocześnie pogarszając szybkość regulacji.
