\chapter{Zadanie 2}
Celem zadania drugiego było wyznaczenie odpowiedzi skokowych toru wejście-wyjście dla pięciu
różnych zmian sygnału sterującego G1 rozpoczynając z punktu pracy oraz określenie czy
obiekt ma właściwości liniowe. Skoki sterowania przeprowadziliśmy do
kilku różnych wartości U: 30, 40, 50, 60, 70, oraz 80. Wykres dla wymienionych odpowiedzi skokowych
znajduje się na rysunku \ref{fig:skoki}. Ustaliliśmy, iż po 400 sekundach wyjście obiektu jest
stabilne. Na tej podstawie stworzyliśmy charakterystykę statyczną obiektu przedstawiającą
wyjście w 400. chwili, w zależności od wartości sterowania. Wspomniany wykres jest
przedstawiony na rysunku \ref{fig:char_stat}.
Zauważamy, że charakterystykę statyczną może podzielić na dwa przebiegi (dwie linie proste
o różnych nachyleniach). Stąd, wskazane będzie użycie tak zwanej regulacji rozmytej. Użyjemy
do tego dwóch regulatorów. Pierwszy będzie odpowiadał za sterowanie dolnego przedziału, kolejny zaś
będzie sterował drugim przedziałem. Drugi regulator musi charakteryzować się większa "agresywnością"
ze względu na to, że w tym przedziale obiekt słabiej reaguje na zmiany sterowania.

\begin{figure}[tb]
\centering
\begin{tikzpicture}
\begin{axis}[
width=0.75\textwidth,
height = 0.6\textwidth,
xmin= 0,xmax=400,ymin=30,ymax=50,
xlabel={$k$},
ylabel={$y(k)$},
xtick={0, 100, 200, 300, 400},
ytick={30, 32, 34, 36, 38, 40, 42, 44, 46, 48, 50},
legend pos=north west,
/pgf/number format/.cd,
use comma,
1000 sep={}
]

\addplot[blue,semithick] file {wykresy/skok30.txt};
\addplot[red,semithick] file {wykresy/skok40.txt};
\addplot[green,semithick] file {wykresy/skok50.txt};
\addplot[orange,semithick] file {wykresy/skok60.txt};
\addplot[violet,semithick] file {wykresy/skok80.txt};
\legend{$u = 30$, $u = 40$, $u = 50$, $u = 60$, $u = 80$}

\end{axis}
\end{tikzpicture}
\caption{Odpowiedzi skokowe obiektu.}
\label{fig:skoki}
\end{figure}

\begin{figure}[tb]
\centering
\begin{tikzpicture}
\begin{axis}[
width=0.75\textwidth,
height = 0.6\textwidth,
xmin=20,xmax=80,ymin=30,ymax=50,
xlabel={$k$},
ylabel={$y(400)$},
xtick={20, 30, 40, 50, 60, 70, 80},
ytick={30, 32, 34, 36, 38, 40, 42, 44, 46, 48, 50},
legend pos=north west,
/pgf/number format/.cd,
use comma,
1000 sep={}
]

\addplot[blue,semithick] file {wykresy/char_stat.txt};
% \legend{$u = 30$, $u = 40$, $u = 50$, $u = 60$, $u = 80$}

\end{axis}
\end{tikzpicture}
\caption{Charakterystyka statyczna obiektu.}
\label{fig:char_stat}
\end{figure}
