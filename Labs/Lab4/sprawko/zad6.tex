\chapter{Zadanie 6}
Celem zadania szóstego było dla zaproponowanej trajektorii zmian wartości zadanych, zaimplementowanie
oraz dobranie parametrów rozmytego regulatora PID. Trajektoria wartości zadanej
Jest taka sama jak w zadniu 5, aby jak najłatwiej porównywać regulatory. Zdecydowaliśmy się na przeprowadzenie eksperymentu
jedynie dla 2 regulatorów lokalnych, ze względu na to, że charaketerystyka statyczna odzwierciedla, że
uzasadnione jest zastosowanie tylko dwóch regulatorów (jeden punkt przegięcia, dwie proste linie o różnych nachyleniach).
Teza ta jest poparta wiedzą ekspreta, w postaci prowadzącego laboratoria.
Nastawy regulatorów PID dobieraliśmy metodą eksperymentalną, wraz ze zwróceniem uwagi na wyniki
regulacji przeprowadznych dla zwykłego regulatora PID z zadania 5. Stwierdziliśmy, że
dla górnego przedziału wartości wyjść potrzebny będzie regulator bardziej agresywny, tak
aby rekompensował słabą reaktywność obiektu na zmiany sterowania w górnym przedziale wartości wyjść.
Jako funkcję przynależności wybraliśmy funkcję sigmoidalną. Ze względu na użycie dwóch
regulatorów lokalnych, jako parametry tej funkcji wybraliśmy c = 50 (punkt przecięć funkcji dla obu regulatorów),
oraz d = 10 ("stromość" funkcji), jako podaną przez prowadzącego odpowiednią wartość.

\section{Rozmyty PID}
Pierwszym regulatorem rozmytym przetestowanym przez nas był regulator PID.
Dla dolnej części sterowań, przed punktem przełamania charakterystyki obiektu,
użyliśmy nastaw regulatora z poprzedniego zadania. Drugi regulator został dobrany
na podstawie charakterystyki statycznej, mając pierwszy regulator jako punkt odniesienia.
Wiedzieliśmy, iż drugi regulator powinien być bardziej agresywny.
Na tej podstawie dostaliśmy następujące nastawy
\begin{equation}
  K_1 = 5, \qquad T_{i1} = 65, \qquad T_{d1} = 1,25,
\end{equation}
\begin{equation}
  K_2 = 6,5, \qquad T_{i2} = 45, \qquad T_{d2} = 1,25.
\end{equation}
Działanie tak dobranych regulatorów zostało przedstawione na rysunku \ref{fig:z6_pid_1}.
Sterowanie generowane przez taki zestaw regulatorów przedstawia rysunek \ref{fig:z6_pid_1_u}. Błąd regulacji wynosi $E=\num{2,101e+3}$.
Niestety, nie byliśmy zadowoleni działaniem tego regulatora, dlatego postanowiliśmy
przetestować zestaw regulatorów o lekko poprawionych nastawach.
Drugie nastawy miały lekko niższe wzmocnienie, tak aby zmniejszyć przesterowanie.
Nastawy prezentowały się następująco
\begin{equation}
  K_1 = 4, \qquad T_{i1} = 65, \qquad T_{d1} = 1,25,
\end{equation}
\begin{equation}
  K_2 = 6, \qquad T_{i2} = 45, \qquad T_{d2} = 1,25.
\end{equation}
Wyniki działania tych regulatorów przedstawiają wykresy \ref{fig:z6_pid_2}, oraz
\ref{fig:z6_pid_1_u}. Przeregulowania są zdecydowanie niższe. Błąd regulacji wynosi $E=\num{1,961e+3}$. Wydaje nam się
że można uzyskać lepsze efekty. Jednak po pierwsze nie mieliśmy czasu na takie eksperymenty,
a po drugie celem laboratorium nie było stworzenie idealnego regulatora, a
poznanie i porównanie działania regulatorów rozmytych, co zdecydowanie nam się udało.

\begin{figure}[tb]
\centering
\begin{tikzpicture}
\begin{axis}[
width=0.75\textwidth,
height = 0.6\textwidth,
xmin=0,xmax=1200,ymin=30,ymax=48,
xlabel={$k$},
ylabel={$y$},
xtick={0, 200, 400, 600, 800, 1000, 1200},
ytick={30, 32, 34, 36, 38, 40, 42, 44, 46, 48},
legend pos=north west,
/pgf/number format/.cd,
use comma,
1000 sep={}
]

\addplot[blue,semithick] file {wykresy/z6_pid_1.txt};
\addplot[red,semithick] file {wykresy/z5_yzad.txt};
\legend{Wyjście obiektu, Wartość zadana}

\end{axis}
\end{tikzpicture}
\caption{Regulacja pierwszym rozmytym regulatorem PID.}
\label{fig:z6_pid_1}
\end{figure}

\begin{figure}[tb]
\centering
\begin{tikzpicture}
\begin{axis}[
width=0.75\textwidth,
height = 0.6\textwidth,
xmin=0,xmax=1200,ymin=10,ymax=90,
xlabel={$k$},
ylabel={$u$},
xtick={0, 200, 400, 600, 800, 1000, 1200},
ytick={10, 20, 30, 40, 50, 60, 70, 80, 90},
legend pos=north west,
/pgf/number format/.cd,
use comma,
1000 sep={}
]

\addplot[blue,semithick] file {wykresy/z6_pid_1_u.txt};

% \legend{Wyjście obiektu, Wartość zadana}

\end{axis}
\end{tikzpicture}
\caption{Sterowanie pierwszego rozmytego regulatora PID.}
\label{fig:z6_pid_1_u}
\end{figure}


%%%%%%%%%%%%%%%%%%%%%%%%%%%%%%%%%%%%%%%%%%%%%%%%%%%%%%%%%%

\begin{figure}[tb]
\centering
\begin{tikzpicture}
\begin{axis}[
width=0.75\textwidth,
height = 0.6\textwidth,
xmin=0,xmax=1200,ymin=30,ymax=48,
xlabel={$k$},
ylabel={$y$},
xtick={0, 200, 400, 600, 800, 1000, 1200},
ytick={30, 32, 34, 36, 38, 40, 42, 44, 46, 48},
legend pos=north west,
/pgf/number format/.cd,
use comma,
1000 sep={}
]

\addplot[blue,semithick] file {wykresy/z6_pid_2.txt};
\addplot[red,semithick] file {wykresy/z5_yzad.txt};
\legend{Wyjście obiektu, Wartość zadana}

\end{axis}
\end{tikzpicture}
\caption{Regulacja drugim rozmytym regulatorem PID.}
\label{fig:z6_pid_2}
\end{figure}

\begin{figure}[tb]
\centering
\begin{tikzpicture}
\begin{axis}[
width=0.75\textwidth,
height = 0.6\textwidth,
xmin=0,xmax=1200,ymin=10,ymax=90,
xlabel={$k$},
ylabel={$u$},
xtick={0, 200, 400, 600, 800, 1000, 1200},
ytick={10, 20, 30, 40, 50, 60, 70, 80, 90},
legend pos=north west,
/pgf/number format/.cd,
use comma,
1000 sep={}
]

\addplot[blue,semithick] file {wykresy/z6_pid_2_u.txt};

% \legend{Wyjście obiektu, Wartość zadana}

\end{axis}
\end{tikzpicture}
\caption{Sterowanie drugiego rozmytego regulatora PID.}
\label{fig:z6_pid_2_u}
\end{figure}

\section{Rozmyty DMC}
Nastawy regulatora DMC wybrane przez nas eskperymentalnie wynoszą:
\begin{equation}
  D = 500, \qquad N = 500, \qquad N_u = 500, \qquad \Lambda_1 = 2, \qquad \Lambda_2 = 1.
\end{equation}
Oczywiście bazą dla działania tego regulatora jest odpowiednia odpowiedź skokowa.
Zostały tutaj użyte aproksymacje znormalizowanych odpowiedzi skokowych przy skoku
sterowania do $U = 40$, oraz $U = 80$. Pierwsza z tych aproksymacji była użyta już
wcześniej i przedstawiona na wykresie \ref{fig:approx}. Wynik działania tego
algorytmu przedstawia rysunek \ref{fig:z6_dmc}. Sterowanie generowane przezeń przedstawia
wykres \ref{fig:z6_dmc_u}. Wynik działania tego regulatora nie jest najlepszy. Błąd regulacji wynosi $E=\num{2.479e+3}$.
Na początku widać ewidentny brak stabilizacji obiektu na wartości zadanej, co
prawdopodobnie było wynikiem pośpiechu. Następne złe wyniki były prawdopodobnie
spowodowane zakłóceniami i ludźmi chodzącymi dookoła, gdyż było to wykonywane
na koniec zajęć. Niestety z tego samego powodu nie udało nam się przetestować
innych nastaw, ani tego algorytmu w lepszych warunkach.

%%%%%%%%%%%%%%%%%%%%%%%%%%%%%%%%%%%%%%%%%%%%%%%%%%%%%%%%%%%%%5

\begin{figure}[tb]
\centering
\begin{tikzpicture}
\begin{axis}[
width=0.75\textwidth,
height = 0.6\textwidth,
xmin=0,xmax=1200,ymin=30,ymax=48,
xlabel={$k$},
ylabel={$y$},
xtick={0, 200, 400, 600, 800, 1000, 1200},
ytick={30, 32, 34, 36, 38, 40, 42, 44, 46, 48},
legend pos=north west,
/pgf/number format/.cd,
use comma,
1000 sep={}
]

\addplot[blue,semithick] file {wykresy/z6_dmc.txt};
\addplot[red,semithick] file {wykresy/z5_yzad.txt};
\legend{Wyjście obiektu, Wartość zadana}

\end{axis}
\end{tikzpicture}
\caption{Regulacja rozmytym regulatorem DMC.}
\label{fig:z6_dmc}
\end{figure}

\begin{figure}[tb]
\centering
\begin{tikzpicture}
\begin{axis}[
width=0.75\textwidth,
height = 0.6\textwidth,
xmin=0,xmax=1200,ymin=10,ymax=90,
xlabel={$k$},
ylabel={$u$},
xtick={0, 200, 400, 600, 800, 1000, 1200},
ytick={10, 20, 30, 40, 50, 60, 70, 80, 90},
legend pos=north west,
/pgf/number format/.cd,
use comma,
1000 sep={}
]

\addplot[blue,semithick] file {wykresy/z6_dmc_u.txt};

% \legend{Wyjście obiektu, Wartość zadana}

\end{axis}
\end{tikzpicture}
\caption{Sterowanie rozmytego regulatora DMC.}
\label{fig:z6_dmc_u}
\end{figure}
