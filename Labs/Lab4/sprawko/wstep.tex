\chapter{Wstęp}
Tematem laboratorium była imlplementacja, weryfikacja działania i dobór parametrów algorytmów
regulacji jednowymiarowego procesu laboratoryjnego o nieliniowych właściwościach.
Podczas pracy w laboratorium korzystaliśmy z pakietu \texttt{MATLAB}.
Badania podczas laboratorium dotyczyły procesu o jednym wejściu i jednym wyjściu, W tym celu wykorzystane
zostało stanowisko z następującymi elementami:
\begin{itemize}
  \item sterowanie --- grzałka G1,
  \item pomiar --- czujnik temperatury T1,
  \item stałe zakłócenie --- wentylator W1.
\end{itemize}
Sygnał sterujący mógł zmieniać się w zakresie 0 - 100\%.
Sygnał wyjsciowy to pomiar wykonany przez czujnik temperatury T1 (w stopniach celsjusza),
natomiast wentylator W1 należy traktować jako cechę otoczenia- jego użycie pozwala przyspieszyć opadanie
temperatury zmierzonej na czujniku T1. Sterowanie wentylatora W1 miało stałą wartość 50\%.
Czas próbkowania wynosi 1s. Ze względu na występujące w pobliżu stanowiska zakłócenia,
w postaci otwartych drzwi (powodującej mocne cyrkulacje powietrza),
przechodzące w pomieszczeniu osoby itp., przeprowadzone przez nas pomiary nie oddawały
idealnie zachowania obiektu.
