\chapter{Zadanie 5}
Celem zadania piątego było dla zaproponowanej trajektorii zmian sygnałów zadanych (6 skoków o różnej amplituudzie)
dobranie nastaw dla regulatora PID i patametrów dla algorytmu DMC metodą eksperymentalną.
Jakość regulacji ocenialiśmy wizualnie (na podstawie przebiegów wartości wyjścia obiektu), oraz
ilościowo za pomocą wkaźnika $E$, liczonego jako suma kwadratów uchybów regulacji.
Dla obu regulatorów ustaliliśmy taką samą trajektorię wartości zadanej, aby umożliwić
jak najrzetelniejsze porównanie regulatorów. Trajektoria polegała na ustawieniu
stałej wartości zadanej, kolejno: 34, 37, 40, 43, 46, 41, na 200 sekund.
Pomiary rozpoczynaliśmy z punktu pracy wyznaczonego w zadaniu pierwszym.
\section{PID}
Jako parametry regulatora PID dobraliśmy wstępnie parametry, które okazały się
dobre na poprzednim laboratorium. Byliśmy świadomi tego, iż obiekt jest lekko
zmodyfikowany, więc użyliśmy ich jako wstępnego zestawu parametrów, który potem
poprawiliśmy. Regulator był tak zestawiany, aby działał jak najlepiej w pobliżu punktu pracy,
czyli w dolnej części charakterystyki obiektu.
Finalnie parametry regulatora PID wynosiły
\begin{equation}
  K = 5, \qquad T_i = 65, \qquad T_d = 1,25.
\end{equation}
Działanie tego regulatora przedstawia rysunek \ref{fig:z5_pid}, natomiast sterowanie
generowane przez niego rysunek \ref{fig:z5_pid_u}. Widać na nich, iż regulator działa
wyraźnie lepiej dla temperatur zdecydowanie poniżej punktu, w którym charakterystyka
statyczna się ,,przełamuje", czyli przy pierwszych dwóch skokach. Przy pracy na wyższych
temperaturach regulator radzi sobie dużo gorzej. Wpływ na to miał również fakt
działania z dala od punktu pracy. Na podstawie tylko tych wykresów nie jesteśmy
w stanie stwierdzić co miało większy wpływ na takie a nie inne zachowanie obiektu.

\begin{figure}[tb]
\centering
\begin{tikzpicture}
\begin{axis}[
width=0.75\textwidth,
height = 0.6\textwidth,
xmin=0,xmax=1200,ymin=30,ymax=48,
xlabel={$k$},
ylabel={$y$},
xtick={0, 200, 400, 600, 800, 1000, 1200},
ytick={30, 32, 34, 36, 38, 40, 42, 44, 46, 48},
legend pos=north west,
/pgf/number format/.cd,
use comma,
1000 sep={}
]

\addplot[blue,semithick] file {wykresy/z5_pid_y.txt};
\addplot[red,semithick] file {wykresy/z5_yzad.txt};
\legend{Wyjście obiektu, Wartość zadana}

\end{axis}
\end{tikzpicture}
\caption{Regulacja regulatorem PID.}
\label{fig:z5_pid}
\end{figure}

\begin{figure}[tb]
\centering
\begin{tikzpicture}
\begin{axis}[
width=0.75\textwidth,
height = 0.6\textwidth,
xmin=0,xmax=1200,ymin=10,ymax=90,
xlabel={$k$},
ylabel={$u$},
xtick={0, 200, 400, 600, 800, 1000, 1200},
ytick={10, 20, 30, 40, 50, 60, 70, 80, 90},
legend pos=north west,
/pgf/number format/.cd,
use comma,
1000 sep={}
]

\addplot[blue,semithick] file {wykresy/z5_pid_u.txt};

% \legend{Wyjście obiektu, Wartość zadana}

\end{axis}
\end{tikzpicture}
\caption{Sterowanie regulatora PID.}
\label{fig:z5_pid_u}
\end{figure}

\section{DMC}
Kolejnym testowanym regulatorem był regulator DMC. Jego parametry zostały dobrane
podobnie jak poprzednio. Finalnie wynosiły one
\begin{equation}
  D = 500, \qquad N = 500, \qquad N_u = 500, \qquad \Lambda = 1.
\end{equation}
Jednak w przeciwieństwie do regulatora PID, tutaj większy wpływ na dobre działanie
mają dobre odpowiedzi skokowe obiektu. Została tutaj użyta odpowiedź przy skoku
sterowania do $U = 40$, więc, podobnie jak w przypadku regulatora PID, w dolnej
części charakterystyki obiektu. Działanie regulatora przedstawia rysunek \ref{fig:z5_dmc},
natomiast generowane przez niego sterowanie rysunek \ref{fig:z5_dmc_u}. Na tym wykresie
idealnie widać, iż regulator jest nie przystosowany do pracy przy wyższych temperaturach,
gdzie charakterystyka statyczna obiektu jest inna. Widać dokładnie, iż regulator ma problemy
z dojściem do wartości zadanej.


\begin{figure}[tb]
\centering
\begin{tikzpicture}
\begin{axis}[
width=0.75\textwidth,
height = 0.6\textwidth,
xmin=0,xmax=1200,ymin=30,ymax=48,
xlabel={$k$},
ylabel={$y$},
xtick={0, 200, 400, 600, 800, 1000, 1200},
ytick={30, 32, 34, 36, 38, 40, 42, 44, 46, 48},
legend pos=north west,
/pgf/number format/.cd,
use comma,
1000 sep={}
]

\addplot[blue,semithick] file {wykresy/z5_dmc_y.txt};
\addplot[red,semithick] file {wykresy/z5_yzad.txt};
\legend{Wyjście obiektu, Wartość zadana}

\end{axis}
\end{tikzpicture}
\caption{Regulacja regulatorem DMC.}
\label{fig:z5_dmc}
\end{figure}

\begin{figure}[tb]
\centering
\begin{tikzpicture}
\begin{axis}[
width=0.75\textwidth,
height = 0.6\textwidth,
xmin=0,xmax=1200,ymin=10,ymax=90,
xlabel={$k$},
ylabel={$u$},
xtick={0, 200, 400, 600, 800, 1000, 1200},
ytick={10, 20, 30, 40, 50, 60, 70, 80, 90},
legend pos=north west,
/pgf/number format/.cd,
use comma,
1000 sep={}
]

\addplot[blue,semithick] file {wykresy/z5_dmc_u.txt};

% \legend{Wyjście obiektu, Wartość zadana}

\end{axis}
\end{tikzpicture}
\caption{Sterowanie regulatora DMC.}
\label{fig:z5_dmc_u}
\end{figure}
