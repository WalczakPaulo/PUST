\chapter{Zadanie 3}
Celem zadania trzeciego było przeksztalcenie jednej z otrzymanych odpowiedzi skokowych używając w tym celu
członu inercyjnego drugiego rzędu z opóźnieniem. Wzór ogólny takiego obiektu przedstawia równanie \ref{eq:2in_op}.
\begin{equation}
  \label{eq:2in_op}
  y(k) = b_1u(k-T_D-1) + b_2u(k-T_D-2)-a_1y(k-1)-a_2y(k-2),
\end{equation}
gdzie
\begin{align}
  a_1 &= -e^{-\frac{1}{T_1}}-e^{-\frac{1}{T_2}} \nonumber \\
  a_2 &= e^{-\frac{1}{T_1}}e^{-\frac{1}{T_2}} \nonumber \\
  b_1 &= \frac{K}{T_1 - T_2}[T_1(1 - e^{-\frac{1}{T_1}}) - T_2(1 - e^{-\frac{1}{T_2}})] \nonumber \\
  b_2 &= \frac{K}{T_1 - T_2}[e^{-\frac{1}{T_1}}T_2(1 - e^{-\frac{1}{T_2}}) - e^{-\frac{1}{T_2}}T_1(1 - e^{-\frac{1}{T_1}})]
\end{align}
Przybliżenie polega na takim dobraniu parametrów $T_1$, $T_2$, $K$, oraz $T_D$ tak,
aby różnica między faktycznym wyjściem obiektu i przybliżeniem była jak najmniejsza.
Do doboru parametrów użyliśmy funkcji \texttt{GA}.  Do znormalizowania wykorzysaliśmy odpowiedź
skokową, gdzie skok sterowania wynosił $\Delta U = 13$ (do wartości sterowania $U = 40$). Nasz wybór
motywujemy, tym że otrzymana odpowiedź była najmniej zaszumiona. Procesu normalizacji odpowiedzi skokowej, przed poddaniem
jej aproksymacji, nie będziemy tutaj opisywać, ze względu na dokładny opis owego procesu
w poprzednich sprawozdaniach.
% Ciężko jest szacować wspomniane wartości same w sobie. Mając jednak wiedzę, że wynikają
% one z innych parametrów charakterystycznych dla członu dwuinercyjnego z opóźnieniem,
% można już próbować je szacować. Owe zależności wyglądają następująco:
% Jako wartości początkowe parametrów $T_1$, $T_2$, $K$, oraz $T_D$ wybraliśmy odpowiednio
% 5, 10, 2, oraz 9. Wybralismy 2 dla K (zebrane odpowiedzi nie wskazywały dużego wzmocnienia),
% TD = 9 (na tyle oceniamy opóźnienie obiektu), oraz T1 i T2  jako 5 i 10 (wartości z grubsza losowe).
%PIERDOLENIE
Do przybliżenia użyliśmy algorytm ewolucyjny, który nie używa punktu początkowego,
a losowej populacji początkowej. Dlatego jedyne parametry algorytmu to ogarniczenia
Kierując się logiką ograniczyliśmy od dołu parametry zerami, zaś od góry wartościami 1000, 1000, 10, 50,
dla zmiennych odpowiednio $T_1$, $T_2$, $K$, oraz $T_D$.
Aproksymacja odpowiedzi skokowej została pokazana na rysunku \ref{fig:approx}.

\begin{figure}[tb]
\centering
\begin{tikzpicture}
\begin{axis}[
width=0.75\textwidth,
height = 0.6\textwidth,
xmin=0,xmax=400,ymin=0,ymax=0.45,
xlabel={$k$},
ylabel={$y$},
xtick={0, 50, 100, 150, 200, 250, 300, 350, 400},
ytick={0, 0.05, 0.1, 0.15, 0.2, 0.25, 0.3, 0.35, 0.4, 0.45},
legend pos=north west,
/pgf/number format/.cd,
use comma,
1000 sep={}
]

\addplot[blue,semithick] file {wykresy/skok40_norm.txt};
\addplot[red,semithick] file {wykresy/skok40_approx.txt};
\legend{Odpowiedź obiektu, Aproksymacja}

\end{axis}
\end{tikzpicture}
\caption{Aproksymacja odpowiedzi skokowej obiektu.}
\label{fig:approx}
\end{figure}
