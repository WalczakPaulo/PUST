\chapter{Przygotowanie do algorytmu DMC}
W następnym kroku poczyniliśmy przygotowania do stworzenia alogrytmu regulatora
DMC dla naszego obiektu. Do tego znormalizowaliśmy odpowiedź skokową otrzymaną
w sekcji \ref{sec:zad12}. Aby mieć jak najmniejszy wpływ zakłóceń na odpowiedź
wykorzystaliśmy do tego odpowiedź z najwyższym skokiem sterowania, czyli
skokiem do wartości $U = 60$. Został on przesunięty o wartość w chwili $k = 0$,
oraz znormalizowany poprzez podzielenie przez wartość skoku ($60 - 27$).
Efektem była odpowiedź zaprezentowana na wykresie \ref{fig:odp_dmc}.
Następnie została ona przybliżona funkcją dwuinercyjną z opóźnieniem.
Użyty został do tego algorytm \texttt{GA}.
Przybliżenie polegało na dobraniu parametrów równania ogólnego obiektu
dwuinercyjnego z opóźnieniem --- $T_D$, $T_1$, $T_2$, oraz $K$. Wspomniane
równanie, w dziedzinie czasu dyskretnego, wygląda następująco:
\begin{equation}
  y(k) = b_1u(k - T_D - 1) + b_2u(k-T_D-2)-a_1y(k-1)-a_2y(k-2)
\end{equation}
gdzie
\begin{align}
  a_1 &= -e^{-\frac{1}{T_1}}-e^{-\frac{1}{T_2}} \nonumber \\
  a_2 &= e^{-\frac{1}{T_1}}e^{-\frac{1}{T_2}} \nonumber \\
  b_1 &= \frac{K}{T_1 - T_2}[T_1(1-e^{-\frac{1}{T_1}}) - T_2(1 - e^{-\frac{1}{T_2}})] \nonumber \\
  b_2 &= \frac{K}{T_1 - T_2}[e^{-\frac{1}{T_1}}T_2(1-e^{-\frac{1}{T_2}}) - e^{-\frac{1}{T_2}}T_1(1-e^{-\frac{1}{T_1}})] \nonumber \\
\end{align}

\begin{figure}[tb]
\centering
\begin{tikzpicture}
\begin{axis}[
width=0.75\textwidth,
xmin=0,xmax=300,ymin=0,ymax=0.35,
xlabel={Czas (s)},
ylabel={$s_i$},
xtick={0, 50, 100, 150, 200, 250, 300},
ytick={0, 0.05, 0.1, 0.15, 0.2, 0.25, 0.3, 0.35},
legend pos=south east,
/pgf/number format/.cd,
use comma,
1000 sep={}
]
\addplot[blue,semithick] file {odpowiedz.txt};
\addplot[red,semithick] file {s_approx.txt};
\legend{Odpowiedz skokowa, Przybliżenie odpowiedzi}
\end{axis}
\end{tikzpicture}
\caption{Odpowiedź znormalizowana, oraz jej przybliżenie.}
\label{fig:odp_dmc}
\end{figure}
