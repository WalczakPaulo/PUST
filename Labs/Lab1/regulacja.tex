\chapter{Regulacja DMC i PID}
Na koniec laboratorium przygotowaliśmy dwa alogrytmy regulacji. Algorytm PID,
oraz algorytm DMC w wersji analitycznej, a następnie sprawdziliśmy
zachowanie obiektu dla tych algorytmów. Przy algorytmach uwzględnione
zostało ograniczenie na wartość sterowania w granicach $0 \leq u \leq 100$.
Nastawy dla regulatora PID zostały dobrane eksperymentalnie, na podstawie
naszych własnych doświadczeń z regulatorem i obiektem. Pierwsze wybrane
nastawy były dla nas zadowalające, więc pozostaliśmy przy nich.
\begin{equation}
  K = 9 \quad T_i = 30 \quad T_d = 2,5
\end{equation}
Dla regulatora DMC, z braku odpowiedniej wiedzy o obiekcie,
przyjęliśmy opcję bezpieczną, czyli długości horyzontów predykcji,
oraz sterowania równe horyzontowi dynamiki obiektu, oraz parametr $\lambda = 1$.
Wyniki działania algorytmów ukazuje wykres \ref{fig:reg}, natomiast sterowanie
generowane przez regulatory wykres \ref{fig:ster}. Na początku regulator
DMC zdaje się działać dużo wolniej niż PID, jednak przy następnym skoku
różnica ta już nie jest tak drastyczna, więc prawdopodobnie przy tym pomiarze
wystąpiły jakieś zakłócenia, na które nie jest odporny żaden z regulatorów.
Uwzględniając taki scenariusz stwierdzamy, iż jakość regulacji regulatora
DMC jest lepsza niż regulatora PID. Niestety pomiary nie były dość długie aby
dobrze określić jak szybko regulacja kończy oscylacje po drugim kroku, jednak
w tym przypadku zdaje się, iż regulator DMC ustabilizuje obiekt szybciej.
W momencie przed skokiem możemy zauważyć, że utrzymywanie stałej temperatury
obydu regulatorom wychodzi tak samo dobrze. Ocena jakości sterowania wypada
zdecydowanie na korzyść regulatora DMC, które jest dużo spokojniejsze, zmiany
są mniej drastyczne i, w przeciwieństwie do regulatora PID, nie dochodzi do
ograniczeń. Przy wyliczaniu jakości regulacji jako druga norma odległości
między wyjściem obiektu, a odpowiadającą mu wartością zadaną otrzymujemy
wyniki: $E_{PID} = 26.4061$, $E_{DMC} = 27.2064$. Widać, że jakość regulacji
regulatora PID zdaje się być nieznacznie lepsza. Moim zdaniem nie jest to
różnica, która skłoniła by mnie do wybrania regulatora PID nad DMC, biorąc
pod uwagę wcześniejsze rozważania.

\begin{figure}[tb]
\centering
\begin{tikzpicture}
\begin{axis}[
width=0.75\textwidth,
xmin=0,xmax=800,ymin=33,ymax=42,
xlabel={Czas (s)},
ylabel={Wartość temperatury (${}^\circ$C)},
xtick={0, 100, 200, 300, 400, 500, 600, 700, 800},
ytick={33, 34, 35, 36, 37, 38, 39, 40, 41, 42},
legend pos=south east,
/pgf/number format/.cd,
use comma,
1000 sep={}
]
\addplot[blue,semithick] file {PID_Y.txt};
\addplot[red,semithick] file {DMC_Y.txt};
\addplot[violet,semithick] file {Yzad.txt};
\legend{Regulator PID, Regulator DMC, Wartość zadana}
\end{axis}
\end{tikzpicture}
\caption{Działanie alogrytmów regulatorów.}
\label{fig:reg}
\end{figure}

\begin{figure}[tb]
\centering
\begin{tikzpicture}
\begin{axis}[
width=0.75\textwidth,
xmin=0,xmax=800,ymin=20,ymax=100,
xlabel={Czas (s)},
ylabel={Wartość temperatury (${}^\circ$C)},
xtick={0, 100, 200, 300, 400, 500, 600, 700, 800},
ytick={20, 30, 40, 50, 60, 70, 80, 90, 100},
legend pos=south east,
/pgf/number format/.cd,
use comma,
1000 sep={}
]
\addplot[blue,semithick] file {PID_U.txt};
\addplot[red,semithick] file {DMC_U.txt};
\legend{Regulator PID, Regulator DMC}
\end{axis}
\end{tikzpicture}
\caption{Sterowanie generowane przez regulatory.}
\label{fig:ster}
\end{figure}
