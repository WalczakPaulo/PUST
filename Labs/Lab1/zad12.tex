\chapter{Charakterystyka obiektu}
\label{sec:zad12}
Pierwszym krokiem tego laboratorium było określenie punktu pracy naszego
stanowiska dla sterowania $U_{pp} = 27$. Zadanie to było o tyle trudne, że
w trakcie jego wykonywania zmieniało się nasze stanowisko pracy, z powodu
problemów z jego użytkowaniem. Stąd punkt pracy został przeprowadzony dla
dwóch różnych stanowisk, jednak nie różnił on się zbytnio. Dla pierwszego
stanowiska wynosił $Y_{pp} \approx 33,8$, podczas gdy dla drugiego stanowiska
przyjęliśmy $Y_{pp} = 33,43$. Następnie wykonane zostały trzy skoki, jednak
znów problem tutaj był ze zmianą stanowiska. Pierwsze dwa skoki zostały
wykonane dla pierwszego obiektu, a ostatni dla trzeciego. Skoki te nie
różnią się drastycznie, jednak biorąc pod uwagę, iż dalsze pracę będą wykonywane
na drugim obiekcie, to nasze obliczenia zostały opartę na tym ostatnim pomiarze.
Oczywiście zostały tu zamieszczone wszystkie one na wykresie \ref{fig:skoki}.
Odpowiedź skokowa przy skoku do wartości sterowania $40$ w zupełności nie
odpowiada zachowaniu obiektu. Uważamy, że wynika to z tego iż obiekt w trakcie
skoku nie był w stanie ustalonym a jedynie miał temperaturę początkową taką
jak w stanie początkowym. Jednak średnia temperatura obiektu była wyższa
niż w punkcie pomiaru, stąd nieregularny wzrost, oraz wyniki dużo wyższe niż
wynikałoby ze wzmocnienia statycznego obiektu. Pozostałe dwa skoki zostały
przeprowadzone staranniej i widać niemal podręcznikową odpowiedż obiektu
dwuinercyjnego. Warto przy tym zauważyć, że owe dwa skoki zostały przeprowadzone
na dwóch różnych obiektach, więc możemy wykluczyć lepszą charakterystykę
jednego obiektu w porównaniu z drugim.

\begin{figure}[tb]
\centering
\begin{tikzpicture}
\begin{axis}[
width=0.75\textwidth,
xmin=0,xmax=350,ymin=32,ymax=44,
xlabel={Czas (s)},
ylabel={Wartość wyjścia},
xtick={0, 50, 100, 150, 200, 250, 300, 350},
ytick={32, 34, 36, 38, 40, 42, 44},
legend pos=south east,
/pgf/number format/.cd,
use comma,
1000 sep={}
]
\addplot[blue,semithick] file {skok40.txt};
\addplot[red,semithick] file {skok50.txt};
\addplot[violet,semithick] file {skok60.txt};
\legend{Skok do 40, Skok do 50, Skok do 60}
\end{axis}
\end{tikzpicture}
\caption{Odpowiedzi skokowe przy skokach sterowania do wartości opisanych w
legendzie.}
\label{fig:skoki}
\end{figure}
