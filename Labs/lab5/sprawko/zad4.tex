\chapter{Zadanie 4}
Kolejne zadanie polegało na przygotowaniu regulatora DMC dla stanowiska
grzejącego. Aby to osiągnąć należało zebrać odpowiedzi skokowe czterech torów.
Zmierzony został wpływ skoku każdej z dwóch grzałek, na każdy z dwóch pomiarów,
a następnie wykresy zaproksymowano obiektem dwuinercyjnym z opóźnieniem.
Wyniki zostały zaprezentowane na wykresach \ref{fig:s11}, \ref{fig:s12}, \ref{fig:s21},
oraz \ref{fig:s22}. Zebrane odpowiedzi zdają się nie być najwyższej jakości,
jednak pod presją czasu postanowiliśmy zostawić je takimi jakie są. Patrząc
na odpowiedzi skokowe widać, iż horyzont dynamiki obiektu wynosi 49 próbek.
Horyzont predykcji, oraz horyzont sterowania przyjęliśmy taki sam jak horyzont
dynamiki. Natomiast jako wartość wyjściową parametru $\Lambda$ przyjęliśmy
1. Na podstawie tych próbek wyznaczone zostały cztery parametry, używane
w algorytmie - wektory $k^u_1$ i $k^u_2$, oraz wartości $k^e_1$ i $k^e_2$.
Są one definiowane jako:
\begin{equation}
  k^e_1 = \sum_{j=1}^Nk_{1,j}, \qquad k^e_2 = \sum_{j=1}^Nk_{2,j}, \qquad k^u_1 = \overline{K}_1M^P, \qquad k^u_2 = \overline{K}_2M^P
\end{equation}
gdzie $k_{1,j}$ jest $j$-tym elementem pierwszego rzędu macierzy $\bm{K}$,
$k_{2,j}$ $j$-tym elementem drugiego rzędu macierzy $\bm{K}$, $\overline{K}_1$
jest pierwszym rzędem macierzy $\bm{K}$, a $\overline{K}_2$ drugim rzędem owej
macierzy. Sam algorytm wykorzystujący owe parametry został zaprezentowany w
listingu \ref{lst:dmc}.

\begin{lstlisting}[style=customlatex,frame=single, caption=Kod regulatora DMC, label=lst:dmc]
//Wejscia oraz wartosci zadane
SV_DMC3 := Zadana_DMC3;
PV_DMC3 := D100 / 100;
SV_DMC2 := Zadana_DMC2;
PV_DMC2 := D101 / 100;

//Uchyby
ek1 := SV_DMC3 - PV_DMC3;
ek2 := SV_DMC2 - PV_DMC2;

//Wyliczenie zmiany sterowania
dU1 := ke1 * ek1;
dU2 := ke2 * ek2;
FOR cnt:= 0 TO 95 BY 1 DO
	dU1 := dU1 - ku1[cnt] * dup[cnt];
	dU2 := dU2 - ku2[cnt] * dup[cnt];
END_FOR;

//Przesuniecie wektora poprzednich zmian sterowania
FOR cnt := 93 TO 0 BY -1 DO
	dup[cnt + 2] := dup[cnt];
END_FOR;
dup[0] := dU1;
dup[1] := dU2;

//Wyliczenie nowej wartosci sterowania
U_DMC2 := U_DMC2 + dU1;
U_DMC3 := U_DMC3 + dU2;

//Ograniczenia
IF (U_DMC2 > 100.0) THEN
	U_DMC2 := 100.0;
ELSIF (U_DMC2 < 0.0) THEN
	U_DMC2 := 0.0;
END_IF;

IF (U_DMC3 > 100.0) THEN
	U_DMC3 := 100.0;
	ELSIF (U_DMC3 < 0.0) THEN
	U_DMC3 := 0.0;
END_IF;

//Wyslanie obliczonych wartosci do obiektu
D114 := REAL_TO_INT(U_DMC2 * 10);
D115 := REAL_TO_INT(U_DMC3 * 10);
\end{lstlisting}

% 1
\begin{figure}[tb]
\centering
\begin{tikzpicture}
\begin{axis}[
width=0.75\textwidth,
height = 0.4\textwidth,
xmin=0,xmax=50,ymin=0,ymax=0.35,
xlabel={Numer próbki},
ylabel={Wyjście znormalizowane},
xtick={0, 10, 20, 30, 40, 50},
ytick={0, 0.05, 0.1, 0.15, 0.2, 0.25, 0.3, 0.35},
legend pos=south east,
/pgf/number format/.cd,
use comma,
1000 sep={}
]

\addplot[blue,semithick] file {wykresy/s11_real.txt};
\addplot[red,semithick] file {wykresy/s11_approx.txt};
\legend{Znormalizowane wyjście obiektu, Aproksymacja odpowiedzi skokowej}

\end{axis}
\end{tikzpicture}
\caption{Odpowiedź skokowa, pomiaru T1, na skok sygnału G1.}
\label{fig:s11}
\end{figure}

% 2
\begin{figure}[tb]
\centering
\begin{tikzpicture}
\begin{axis}[
width=0.75\textwidth,
height = 0.4\textwidth,
xmin=0,xmax=50,ymin=0,ymax=0.35,
xlabel={Numer próbki},
ylabel={Wyjście znormalizowane},
xtick={0, 10, 20, 30, 40, 50},
ytick={0, 0.05, 0.1, 0.15, 0.2, 0.25, 0.3, 0.35},
legend pos=north east,
/pgf/number format/.cd,
use comma,
1000 sep={}
]

\addplot[blue,semithick] file {wykresy/s12_real.txt};
\addplot[red,semithick] file {wykresy/s12_approx.txt};
\legend{Znormalizowane wyjście obiektu, Aproksymacja odpowiedzi skokowej}

\end{axis}
\end{tikzpicture}
\caption{Odpowiedź skokowa, pomiaru T1, na skok sygnału G2.}
\label{fig:s12}
\end{figure}

% 3
\begin{figure}[tb]
\centering
\begin{tikzpicture}
\begin{axis}[
width=0.75\textwidth,
height = 0.4\textwidth,
xmin=0,xmax=50,ymin=0,ymax=0.35,
xlabel={Numer próbki},
ylabel={Wyjście znormalizowane},
xtick={0, 10, 20, 30, 40, 50},
ytick={0, 0.05, 0.1, 0.15, 0.2, 0.25, 0.3, 0.35},
legend pos=south east,
/pgf/number format/.cd,
use comma,
1000 sep={}
]

\addplot[blue,semithick] file {wykresy/s21_real.txt};
\addplot[red,semithick] file {wykresy/s21_approx.txt};
\legend{Znormalizowane wyjście obiektu, Aproksymacja odpowiedzi skokowej}

\end{axis}
\end{tikzpicture}
\caption{Odpowiedź skokowa, pomiaru T2, na skok sygnału G1.}
\label{fig:s21}
\end{figure}

% 4
\begin{figure}[tb]
\centering
\begin{tikzpicture}
\begin{axis}[
width=0.75\textwidth,
height = 0.4\textwidth,
xmin=0,xmax=50,ymin=0,ymax=0.35,
xlabel={Numer próbki},
ylabel={Wyjście znormalizowane},
xtick={0, 10, 20, 30, 40, 50},
ytick={0, 0.05, 0.1, 0.15, 0.2, 0.25, 0.3, 0.35},
legend pos=north east,
/pgf/number format/.cd,
use comma,
1000 sep={}
]

\addplot[blue,semithick] file {wykresy/s22_real.txt};
\addplot[red,semithick] file {wykresy/s22_approx.txt};
\legend{Znormalizowane wyjście obiektu, Aproksymacja odpowiedzi skokowej}

\end{axis}
\end{tikzpicture}
\caption{Odpowiedź skokowa, pomiaru T2, na skok sygnału G2.}
\label{fig:s22}
\end{figure}

\begin{figure}[tb]
\centering
\begin{tikzpicture}
\begin{axis}[
width=0.75\textwidth,
height = 0.6\textwidth,
xmin=0,xmax=528.5,ymin=34,ymax=48,
xlabel={Czas [s]},
ylabel={Temperatura [$^\circ$C]},
xtick={0, 100, 200, 300, 400, 500},
ytick={34, 36, 38, 40, 42, 44, 46, 48},
legend pos=north east,
/pgf/number format/.cd,
use comma,
1000 sep={}
]

\addplot[blue,semithick] file {wykresy/dmc1_y1.txt};
\addplot[red,semithick] file {wykresy/dmc1_yz1.txt};
\addplot[violet,semithick] file {wykresy/dmc1_y2.txt};
\addplot[orange,semithick] file {wykresy/dmc1_yz2.txt};
\legend{Temperatura na czujniku T1, Wartość zadana temperatury T1, Temperatura na czujniku T2, Wartość zadana temperatury T2}

\end{axis}
\end{tikzpicture}
\caption{Regulacja regulatorem DMC.}
\label{fig:dmc1_y}
\end{figure}
