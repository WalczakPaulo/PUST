\chapter{Zadanie 2}
W tym zadaniu należało stworzyć mechanizm zabezpieczający stanowisko przed przegrzaniem.
Miał on polegać na tym, iż wyłączał on grzałkę przy czujniku, jeżeli
pomiar na tym ostatnim przekroczył 150$^\circ$C. Osiągnęliśmy ten rezultat,
poprzez sprawdzenie temperatury na każdym z czujników z osobna oraz,
jeżeli któryś z nich spełniał wcześniej wymieniony warunek, wyzerowanie
wartości odpowiedniego rejestru, z którego sterowanie jest wysyłane do
obiektu. Przy okzji była podnoszona flaga, oznajmiająca, iż jesteśmy w stanie
awaryjnym. Służyła ona do tego, aby po powrocie temperatury do stanu akceptowalnego
wrócić do punktu pracy. W późniejszym czasie, gdy obiekt będzie regulowany
owa sekwencja powrotu jest zbędna, gdyż sterowanie będzie wyznaczał regulator.
Wspomniany kod znajduje się w listingu \ref{lst:overheat_lock}. Ów fragment
kodu miał w zamyśle znaleźć się zaraz za kodem regulatora, tak aby rejestry
były ustawione na odpowiednie wartości zanim zostaną wysłane do obiektu.
Zamysł ten był oparty na założeniu, iż każdy z plików kodu w sterowniku
wykonuje się sekwencyjnie w całości, bez wykonywania kodu wysyłania do sterownika w połowie.
Gdyby okazało się, iż tak nie jest należałoby zapisywać wartość sterowania, nie bezpośrednio
do rejestrów, ale do zmiennych tymczasowych, w których byłoby sterowanie wyliczone,
i w razie wystąpienia sytuacji awaryjnej zerowane, a dopiero po tej sekwencji
zapisywane w odpowiednim rejestrze.

\begin{lstlisting}[style=customlatex,frame=single, caption=Kod mechanizmu zabezpieczającego przed przegrzaniem, label=lst:overheat_lock]
IF (D100 > 15000) THEN
	ERROR_FLAG1 := TRUE;
	D114 := 0;
END_IF;

IF (D101 > 15000) THEN
	ERROR_FLAG2 := TRUE;
	D115 := 0;
END_IF;

IF (ERROR_FLAG1 AND (D100 < 15000)) THEN
	D114 := 270;
	ERROR_FLAG1 := FALSE;
END_IF;

IF (ERROR_FLAG2 AND (D101 < 15000)) THEN
	D115 := 320;
	ERROR_FLAG2 := FALSE;
END_IF;
\end{lstlisting}
