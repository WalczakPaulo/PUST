\chapter{Zadanie 2}
W tym zadaniu należało stworzyć mechanizm zabezpieczający stanowisko przed przegrzaniem.
Miał on polegać na tym, iż miał on wyłączyć grzałkę przy czujniku, jeżeli
pomiar na tym ostatnim przekroczył 150$^\circ$C. Osiągnęliśmy ten rezultat,
poprzez sprawdzenie temperatury na każdym z czujników z osobna oraz,
jeżeli któryś z nich spełniał wcześniej wymieniony warunek, wyzerowanie
odpowiedniej wartości tymczasowej, w której zapisywaliśmy wyliczone sterowanie.
Było to ostatnią czynnością w programie przed skopiowaniem wartości sterowania
do rejestru z którego miała być ona wysyłana do obiektu.
