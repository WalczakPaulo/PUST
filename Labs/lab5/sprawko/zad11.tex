\chapter{Zadanie 11}
Ostatecznie do sterowania obiektem zostały użyte trzy regulatory PID.
Użyta została gotowa funkcja PID, która jest częścią środowiska developerskiego
dla sterownika. Nastawy dla każdego z regulatorów były następujące:
\begin{equation}
  K = 200, \qquad T_i = 20, \qquad T_d = 0.
\end{equation}
Nastawy były dobierane metodą eksperymentalną. Zaczęliśmy od regulatora
P i dobraliśmy wzmocnienie, które zdawało się działać dobrze, a następnie,
dodaliśmy całkowanie. Byliśmy zadowoleni z efektów i uznaliśmy różniczkę za
zbędną dla tego obiektu. Proces ten został wykonany dla najwyższego ze
zbiorników, dla pozostałych zbiorników przyjęliśmy takie same nastawy,
gdyż ich charakterystyka jest nieliniowa i musielibyśmy spędzić więcej
czasu nad nastawami. Napięte ramy czasowe nie pozwoliły nam też odpowiednio
skonfigurować komunikacji między sterownikiem i komputerem, tak aby utworzyć
wykresy, dlatego, aby obiekt był jakkolwiek udokumentowany nagraliśmy krótką
prezentację jego działania w formie video. Plik wideo dostępny jest w folderze
z tym sprawozdaniem pod nazwą \texttt{prezentacja.mp4}.
