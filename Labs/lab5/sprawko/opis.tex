\chapter{Opis problemu}
\section{Stanowisko grzejące}
\label{sec:grzjace}
Pierwszym z celów piątego laboratorium było zbadanie obiektu, a następnie
implementacja, weryfikacja poprawności działania i dobór parametrów algorytmów
regulacji obiektu na sterowniku firmy Mitsubishi, przy pomocy oprogramowania
GX Works.
Badania dotyczyły obiektu wielowymiarowego, o dwóch wejściach
i dwóch wyjściach. Obiekt składał się z:
\begin{itemize}
  \item sygnał wejściowy pierwszy -- grzałka G1
  \item sygnał wejściowy drugi -- grzałka G2
  \item sygnał wyjściowy pierwszy -- czujnik temperatury T1
  \item sygnał wyjściowy drugi -- czujnik temperatury T2
  \item zakłócenie pierwsze -- wentylator W1
  \item zakłócenie drugie -- wentylator W2.
\end{itemize}
Sygnały wejściowe G1 i G2 mogą przyjmować wartości w zakresie 0 - 100.
Sygnały wyjściowe, zwracają pomiary temperatury w $^\circ$C. Wentylatory
W1 oraz W2 należy traktować jako cechę otoczenia. Mają one stałe wysterowanie
na 50\% swojej mocy. Ich użycie pozwala na bardziej dynamiczne schładzanie
obiektu. Czas próbkowania obiektu wynosi 4s.
\section{Obiekt Inteco}
\label{sec:inteco}
Drugim celem laboratorium było stworzenie regulatora dla obiektu firmy Inteco.
W naszym przypadku obiekt składał się z trzech zbiorników umieszczonych,
jeden nad drugim, wypełnianych wodą. Każda kolejna para zbiorników była
połączona ze sobą zaworami, przez które woda mogła przepływać z jednego
zbiornika do drugiego. Zawór z najniższego zbiornika był połączony ze zbiornikiem,
będącym buforem nadmiarowej wody. Z owego bufora woda mogła być pompowana do
najwyższego zbiornika. Z punktu widzenia zadania, które mieliśmy wykonać
obiekt prezentował się następująco:
\begin{itemize}
  \item pompa P1 -- stale wysterowane wejście wody do zbiornika pierwszego
  \item zawór V1 -- ujście wody z pierwszego zbiornika i
  jednocześnie wejście wody do zbiornika drugiego (pierwszy sygnał sterujący)
  \item zawór V2 -- ujście wody z drugiego zbiornika i jednocześnie wejście
  wody do zbornika trzeciego (drugi sygnał sterujący)
  \item zawór V3 -- ujście wody z trzeciego zbiornika (trzeci sygnał sterujący)
  \item pomiar M1 -- stan wody w pierwszym zbiorniku (pierwszy sygnał wyjściowy)
  \item pomiar M2 -- stan wody w drugim zbiorniku (drugi sygnał wyjściowy)
  \item pomiar M3 -- stan wody w trzecim zbiorniku (trzeci sygnał wyjściowy)
\end{itemize}
Wszelkie sygnały wejściowe sterowane były za pomocą PWM. Czas próbkowania obiektu
wynosił 100ms.
