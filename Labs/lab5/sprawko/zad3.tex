\chapter{Zadanie 3}
Celem zadania trzeciego było zaprojektowanie regulatora PID. Kod zaprojektowanego
przez nas regulatora znajduje się w listingu \ref{lst:pid}. Przy odczycie aktualnych
wartości rejestry są normalizowane do zakresów wartości opisanych we wstępie.
Przy wysyłaniu wartości są wymnażane razy 10, aby zakres wartości był zgodny
z zakresem używanym przez obiekt.
Biorąc pod uwagę fakt, iż obiekt jest symetryczny, postanowiliśmy użych dwóch regulatorów
o takich samych nastawach. Wyszliśmy od nastaw używanych do tego obiektu w
projekcie trzecim, gdzie zadanie było bardzo podobne, czyli:
\begin{equation}
  K = \num{5} \qquad T_i = \num{30} \qquad T_d = \num{2.5}
\end{equation}
Okazało się jednak, iż te nastawy nie zdają egzaminu. Uważamy, iż w dużej
mierze wpływ na to ma okres próbkowania obiektu, który wydłużył się z 1 sekundy
w projekcie trzecim do 4 sekund w aktualnym projekcie. Działanie regulatorów
o takich nastawach przedstawia wykres \ref{fig:pid2_y}. Widząc słabe działanie
regulatora postanowiliśmy nie prowadzić dalej tego eksperymentu. Następnie
testowane były inne nastawy. Aby zdążyć z wykonaniem laboratorium w czasie
kolejne nastawy były oceniane dosyć szybko i w oparciu o znajomość chwilowego
stanu obiektu, dlatego aby nie zmylić czytelnika wykresy przejściowe nie zostały
zamieszczone. Finalne nastawy są następujące:
\begin{equation}
  K = \num{4} \qquad T_i = \num{45} \qquad T_d = \num{0}
\end{equation}
Ich działanie przedstawia wykres \ref{fig:pid1_y}. Początkowo obiekt nie był
w stanie równowagi, stąd zakłócenie na początku. Widać na wykresie jednak, iż
udało mu się dosyć szybko ustabilizować na wartości zadanej oraz wykonać skok
owej wartości.

\begin{lstlisting}[style=customlatex,frame=single, caption=Kod regulatora PID, label=lst:pid]
//Regulator PID na podstawie rownania roznicowego
SV_PID3 := Zadana_PID3;
PV_PID3 := D100 / 100;

//Nastawy regulatorow
K_PID3 := 4;
Ti_PID3 := 45;
Td_PID3 := 0;
//Wyliczenie parametrow
R0_PID3 := K_PID3*( 1+(4/(2*Ti_PID3))+Td_PID3/4 );
R1_PID3 := K_PID3*( (4/(2*Ti_PID3))-(2*Td_PID3/4)-1 );
R2_PID3 := K_PID3*Td_PID3/4;
//Wyliczenie uchybu regulacji i przesuniecie historii
E2_PID3 := E1_PID3;
E1_PID3 := E0_PID3;
E0_PID3 := SV_PID3 - PV_PID3;
//Obliczenie sterowania
U_PID3 := R2_PID3*E2_PID3 + R1_PID3*E1_PID3 + R0_PID3*E0_PID3 + U_PID3;

//Ograniczenia
IF (U_PID3 > 100.0) THEN
	U_PID3 := 100.0;
ELSIF (U_PID3 < 0.0) THEN
	U_PID3 := 0.0;
END_IF;

D114 := REAL_TO_INT(U_PID3 * 10);

// PID NUMMER ZWEI, SEHR STARK
SV_PID2 := Zadana_PID2;
PV_PID2 := D101 / 100;

//Wyliczenie parametrow
R0_PID3 := K_PID3*( 1+(4/(2*Ti_PID3))+Td_PID3/4 );
R1_PID3 := K_PID3*( (4/(2*Ti_PID3))-(2*Td_PID3/4)-1 );
R2_PID3 := K_PID3*Td_PID3/4;
//Wyliczenie uchybu regulacji i przesuniecie historii
E2_PID2 := E1_PID2;
E1_PID2 := E0_PID2;
E0_PID2 := SV_PID2 - PV_PID2;
//Obliczenie sterowania
U_PID2 := R2_PID3*E2_PID2 + R1_PID3*E1_PID2 + R0_PID3*E0_PID2 + U_PID2;

//Ograniczenia
IF (U_PID2 > 100.0) THEN
	U_PID2 := 100.0;
	ELSIF (U_PID2 < 0.0) THEN
	U_PID2 := 0.0;
END_IF;

D115 := REAL_TO_INT(U_PID2 * 10);
\end{lstlisting}

\begin{figure}[tb]
\centering
\begin{tikzpicture}
\begin{axis}[
width=0.75\textwidth,
height = 0.6\textwidth,
xmin=0,xmax=170.8,ymin=32,ymax=44,
xlabel={Czas [s]},
ylabel={Temperatura [$^\circ$C]},
xtick={0, 50, 100, 150},
ytick={32, 34, 36, 38, 40, 42, 44},
legend pos=north east,
/pgf/number format/.cd,
use comma,
1000 sep={}
]

\addplot[blue,semithick] file {wykresy/pid2_Y1.txt};
\addplot[red,semithick] file {wykresy/pid2_Yz1.txt};
\addplot[violet,semithick] file {wykresy/pid2_Y2.txt};
\addplot[orange,semithick] file {wykresy/pid2_Yz2.txt};
\legend{Temperatura na czujniku T1, Wartość zadana temperatury T1, Temperatura na czujniku T2, Wartość zadana temperatury T2}

\end{axis}
\end{tikzpicture}
\caption{Regulacja regulatorem PID, nastawy pierwsze.}
\label{fig:pid2_y}
\end{figure}

\begin{figure}[tb]
\centering
\begin{tikzpicture}
\begin{axis}[
width=0.75\textwidth,
height = 0.6\textwidth,
xmin=0,xmax=213.2,ymin=32,ymax=50,
xlabel={Czas [s]},
ylabel={Temperatura [$^\circ$C]},
xtick={0, 50, 100, 150, 200},
ytick={32, 34, 36, 38, 40, 42, 44, 46, 48, 50},
legend pos=north east,
/pgf/number format/.cd,
use comma,
1000 sep={}
]

\addplot[blue,semithick] file {wykresy/pid1_Y1.txt};
\addplot[red,semithick] file {wykresy/pid1_Yz1.txt};
\addplot[violet,semithick] file {wykresy/pid1_Y2.txt};
\addplot[orange,semithick] file {wykresy/pid1_Yz2.txt};
\legend{Temperatura na czujniku T1, Wartość zadana temperatury T1, Temperatura na czujniku T2, Wartość zadana temperatury T2}

\end{axis}
\end{tikzpicture}
\caption{Regulacja regulatorem PID, nastawy drugie.}
\label{fig:pid1_y}
\end{figure}
