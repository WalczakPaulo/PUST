\chapter{Zadanie 1}
\section{Sprawdzenie połączenia}
Zadanie pierwsze polegało na sprawdzeniu możliwości komunikacji ze stanowiskiem.
Sprawdziliśmy to, modyfikując wybrane wejścia obiektu, oraz obserwując zmiany
na obiekcie. Przy modyfikacji wejścia wiatraków widać i słychać było, iż wiatraki kręcą
się wolniej lub szybciej, w zależności od wejścia. Przy sprawdzeniu działania
grzałki polegaliśmy na diodach elektroluminescencyjnych, które świeciły mocniej
lub słabiej w zależności od mocy grzania odpowiedniej grzałki. Nasza pewność
w tej sprawie jest oparta na zaufaniu do konstruktora obiektu. Możliwość
pomiaru wyjśc obiektu zoastała sprawdzona w oparciu o trendy pomiaru,
w zależności od wysterowania wcześniej wspomnianych wejść. Otóż, przy
zwiększeniu mocy grzałki lub zmniejszeniu mocy wiatraka temperatura
na czujniku bliższym danej grzałce i wiatrakowi rośnie szybko, a na dalszym
rośnie wolniej. Przy zwiększaniu mocy wiatraka lub zmniejszaniu mocy grzałki pomiar
temperatur spadał. Podobnie jak wcześniej działo się to z szybkością odwrotnie proporcjonalną
do odległości czujnika od wiatraka i grzałki których sterowanie jest zmieniane.
\section{Punkt pracy}
Punkt pracy naszego stanowiska wynosił $G1 = 27$, $G2 = 32$, $W1 = 50$, $W2 = 50$.
Przy tak ustawionych wejściach pozwoliliśmy obiektowi się ustabilizować. Z powodów
różnych zakłóćeń pomiary temperatury nieustannie się wahały. Oszacowaliśmy
arbitralnie iż, pomiary ustabilizowały się na wartościach $T1 = 35$, $T2 = 36.68$.
Pomiary wyjść przy takich ustawieniach sygnałów wejściowych przedstawia wykres
\ref{fig:pkt_pracy}.

\begin{figure}[tb]
\centering
\begin{tikzpicture}
\begin{axis}[
width=0.75\textwidth,
height = 0.6\textwidth,
xmin=0,xmax=150,ymin=32.5,ymax=37,
xlabel={czas [s]},
ylabel={temperatura [$^\circ$C]},
xtick={0, 50, 100, 150},
ytick={32.5, 33, 33.5, 34, 34.5, 35, 35.5, 36, 36.5, 37},
legend pos=south east,
/pgf/number format/.cd,
use comma,
1000 sep={}
]

\addplot[blue,semithick] file {wykresy/pkt_pracy_y1.txt};
\addplot[red,semithick] file {wykresy/pkt_pracy_y2.txt};
\legend{Temperatura T1, Temperatura T2}

\end{axis}
\end{tikzpicture}
\caption{Punkt pracy obiektu.}
\label{fig:pkt_pracy}
\end{figure}
