\chapter{Zadanie 7}
Od tego rozdziału wszystkie zadania będą dotyczyć obiektu Inteco,
opisanego w rozdziale \ref{sec:inteco}. Na początku został opracowany sposób
komunikacji z obiektem. Zarówno wejścia jak i wyjścia obiektu są sterowane
przy pomocy PWM. Do generacji sygnałów wejściowych do obiektu, użyta
została specjalna funkcja PWM sterownika. Generuje ona potrzebny sygnał
o zadanej częstotliwości, oraz poziomie wypełnienia fali. Tak wygenerowana
fala wysyłana jest na wyjście sterownika. Odczyt pomiarów został
wykonany przy pomocy, również wbudowanej w sterownik, opcji szybkiego licznika
na wejściu. Przy tak skonfigurowanym programie sprawdzona została możliwość
sterowania obiektem, jak i odczytywania wejść. Aby ułatwić ocenę działania
jakiegokolwiek regulatora, pojdęliśmy próbę przeskalowania wartości mierzonych
tak, aby skala była taka sama jak skala umieszczona bezpośrednio na zbiornikach.
Zastosowaliśmy przy tym skalę liniową opartą na pomiarze przy niskim stanie
wody, oraz na pomiarze w górnych granicach skali. Na podstawie dwóch pomiarów
stworzyliśmy skalę liniową, która dobrze przybliżała poziom wody w odniesieniu
do skali na zbiornikach.
